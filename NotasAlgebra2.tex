\documentclass[12pt]{book}

% This first part of the file is called the PREAMBLE. It includes
% customizations and command definitions. The preamble is everything
% between \documentclass and \begin{document}.

\usepackage[margin=1in]{geometry}  % set the margins to 1in on all sides
\usepackage{graphicx}              % to include figures
\usepackage{amsmath}               % great math stuff
\usepackage{amsfonts}              % for blackboard bold, etc
\usepackage{amsthm}                % better theorem environments
\usepackage{amssymb}
\usepackage[activeacute, spanish]{babel}
\usepackage[utf8]{inputenc}
\usepackage{hyperref}
\usepackage{multicol}
\usepackage{tikz-cd}
\usetikzlibrary{calc}
\usetikzlibrary{matrix}

\setcounter{tocdepth}{3}% to get subsubsections in toc

\let\oldtocsection=\tocsection

\let\oldtocsubsection=\tocsubsection

\let\oldtocsubsubsection=\tocsubsubsection

% various theorems, numbered by section

\newtheorem{teo}{Teorema}[section]
\newtheorem{lem}[teo]{Lema}
\newtheorem{prop}[teo]{Proposición}
\newtheorem{cor}[teo]{Corolario}

\theoremstyle{definition}
\newtheorem{conj}[teo]{Conjetura}
\newtheorem{obs}[teo]{Observación}
\newtheorem{defn}[teo]{Definición}
\newtheorem{ax}[teo]{Axioma}
\newtheorem{ex}[teo]{Ejemplo}

\newcommand{\bd}[1]{\mathbf{#1}}  % for bolding symbols
\newcommand{\CC}{\mathbb{C}}
\newcommand{\RR}{\mathbb{R}}      % for Real numbers
\newcommand{\ZZ}{\mathbb{Z}}      % for Integers
\newcommand{\NN}{\mathbb{N}}
\newcommand{\QQ}{\mathbb{Q}}
\newcommand{\col}[1]{\left[\begin{matrix} #1 \end{matrix} \right]}
\newcommand{\comb}[2]{\binom{#1^2 + #2^2}{#1+#2}}
\newcommand{\eps}{\varepsilon}
\renewcommand{\hom}{\mathrm{Hom}}
\let\oldemptyset\emptyset
\let\emptyset\varnothing
\DeclareMathOperator{\id}{id}
\DeclareMathOperator{\mcm}{mcm}
\DeclareMathOperator{\mcd}{mcd}
\DeclareMathOperator{\ord}{ord}
\DeclareMathOperator{\im}{im}
\DeclareMathOperator{\End}{End}
\DeclareMathOperator{\Aut}{Aut}
\DeclareMathOperator{\sg}{sg}
\DeclareMathOperator{\coker}{coker}
\DeclareMathOperator{\Obj}{Obj}
\DeclareMathOperator{\rank}{rk}
\DeclareMathOperator{\gr}{gr}
\DeclareMathOperator{\car}{car}
\DeclareMathOperator{\Nil}{Nil}
\DeclareMathOperator{\spec}{Spec}
\DeclareMathOperator{\ev}{ev}
\DeclareMathOperator{\ann}{Ann}
\def\acts{\curvearrowright}
\def\stca{\curvearrowleft}



\begin{document}
\pagenumbering{roman}\clearpage\tableofcontents\newpage\pagenumbering{arabic}

\chapter{Grupos}

\section{Definiciones Básicas}

\begin{defn} \begin{itemize}
\item Un \textbf{semigrupo} $(G,*)$ consiste de un conjunto $G\neq \emptyset$ y de una operación $* :G\times G\to G, \qquad (a,b)\mapsto a*b$ que cumple \textbf{asociatividad} (es decir, $\forall a,b,c\in G, a*(b* c) = (a* b)* c$).
\item Un \textbf{monoide} es un semigrupo con \textbf{elemento neutro} (es decir, $\exists e \in G$ tal que $a* e = e = e*a$).
\item Un \textbf{grupo} es un monoide donde todo elemento tiene inverso (es decir, $\forall a\in G, \text{ } \exists a'\in G$ tal que $a* a' = a' * a = e$).
\end{itemize}
\end{defn}

\begin{obs}
El elemento neutro en un mononoide es único. En efecto, sean $e,e'$ neutros, entonces $e = e * e' = e'$. Además, $e$ es inverso de sí mismo pues $e*e = e$.
\end{obs}

\begin{defn}
Un grupo se dice \textbf{abeliano} si $\forall a,b\in G$ tenemos que $a* b = b* a$.
\end{defn}

\textbf{Notación:} en general la operación $*$ se nota como $\cdot$ y la llamaremos producto. Si el grupo es abeliano, se suele notar como $+$. Al neutro $e$ lo notaremos $1$ cuando la estructura es multiplicativa o $0$ cuando es aditiva.

Sea $(G,\cdot)$ un semigrupo y sean $a,b,c\in G$. Por la asociatividad tenemos que $(a\cdot b)\cdot c = a \cdot (b\cdot c)$ y entonces lo notaremos $a\cdot b\cdot c$. Por inducción, si tenemos $a_1,\ldots , a_n\in G$ queda definido $\prod_{i=1}^{n}a_i = a_1\cdots a_n = (a_1\cdots a_{n-1})\cdot a_n$.

\begin{prop}
Sea $G$ un monoide y sean $a,b,c\in G$ tales que $b\cdot a = e = a\cdot c$. Entonces $b=c$.
\begin{proof}
Es sencillo: $b=b\cdot e = b\cdot (a\cdot c) = (b\cdot a)\cdot c = e\cdot c = c$.
\end{proof}
\end{prop}
\begin{cor}
Si $G$ es un grupo, $\forall a\in G$ existe un \textit{único} $a'\in G$ tal que $a\cdot a' = a'\cdot a = e$. Como es único, al inverso de $a$ lo denotamos $a^{-1}$ en el caso multiplicativo y $-a$ en el aditivo.
Además, si $b\cdot a = e \Longrightarrow a=b^{-1}$.
\end{cor}

\begin{prop}[Ley de cancelación en grupos] $G$ grupo, $a,b,c\in G$. Si $a\cdot b = a\cdot c \Longrightarrow b=c$ (análogamente, si $b\cdot a = c\cdot a \Longrightarrow b=c$).
\begin{proof}
Como $a\cdot b=a\cdot c$ y $G$ es un grupo, multiplicando a izquierda por $a^{-1}$ a ambos lados obtenemos $a^{-1}\cdot (a\cdot b) = a^{-1}\cdot (a\cdot c)$ y así $b=c$ por la asociatividad.
\end{proof}
\end{prop}

\begin{defn}
El \textbf{orden} de un grupo $G$ es su cardinal y se denota $|G|$. Si $|G|<\infty$ se dice que $G$ es un \textbf{grupo finito}. En otro caso se dice que es un \textbf{grupo infinito}.
\end{defn}

Veamos un par de ejemplos de estos conceptos. La verificación queda como ejercicio para el lector. $(\NN,+)$ es un semigrupo, $(\NN_0,+)$ es un monoide, $(\NN,\cdot)$ es un monoide y $(\NN_0,\cdot)$ también. $(\ZZ,+)$ es un grupo y $(\RR,+)$ también. $(\RR - \{0\},\cdot)$ y $(\CC - \{0\},\cdot)$ son grupos también.

Si $K$ es un cuerpo, $K[x]$ el conjunto de polinomios con coeficientes en $K$, tenemos que $(K[x],+)$ es un grupo y $(K[x],\cdot)$ es un monoide. Si $\ZZ_n = \{\overline{0},\ldots , \overline{n-1} \mod{n} \}$ entonces $(\ZZ_n ,+)$ es un grupo y $(\ZZ_n,\cdot)$ es un monoide. Si llamamos $G_n=\{a\in \CC :a^n = 1 \}$ entonces $(G_n,\cdot)$ es un grupo y $|G_n|=n$. Finalmente si $S^1 = \{x\in\CC : |x|=1\}$ entonces con el producto que hereda de $\CC$ forma un grupo.

Si $(M,\cdot)$ es un monoide llamamos $\mathcal{U}(M) = \{a\in M : \exists a'\in M \text{ tal que } a\cdot a' = e = a'\cdot a\}$. Con el producto heredado de $M$, $\mathcal{U}(M)$ forma un grupo y se llama el \textbf{grupo de unidades} de $M$. También se nota $\mathcal{U}(M) = M^\times$.

Si $V$ es un $K$-espacio vectorial entonces $(\End_K(V),\circ)$ es un monoide y $(\Aut_K(V),\circ)$ es un grupo. Más aún, $(\Aut_K(V),\circ)$ es el grupo de unidades de $(\End_K(V),\circ)$.

Ahora dos ejemplos importantes:

Si $X$ es un conjunto no vacío, $S(X) = \{\varphi:X\to X \text{ biyectivas}\}$ entonces $(S(x),\circ)$ es un grupo. Sea $n\in\NN$. Llamamos $X_n=\{1,\ldots , n\}$. Entonces $S_n = S(X_n)$ es el grupo de las permutaciones de $n$ elementos. Trivialmente $|S_n|=n!$. Este grupo se llama \textbf{grupo simétrico}.

Sea $P_n$ un polígono regular de $n$ lados. $r$ es la rotación de $P_n$ en $\alpha = \frac{2\pi}{n}$ en sentido antihorario y $s$ la simetría de $P_n$ con respecto al eje $x$. Es claro entonces que $r^n = 1$ y $s^2 = 1$. Sea $D_n$ el conjunto de movimientos rígidos que deja fijo a $P_n$. Entonces $D_n$ es composición de rotaciones y simetrías. Así $(D_n,\circ)$ es un grupo. Notemos que $r\circ s \neq s\circ r$ así que $D_n$ no es abeliano. Además, es fácil notar que $s\circ r \circ s = r^{-1}$ y así $s\circ r = r^{-1}\circ s$. Notemos además que $r^{n-1}=r^{-1}$. Finalmente, notemos que gracias a esta identidad, cualquier expresión de la forma $s^{\alpha_1}\circ r^{\alpha_2}\circ \ldots \circ s^{\alpha_k}\circ r^{\alpha_{k+1}}$ se puede llevar a algo de la forma $r^\alpha \circ s^\beta$. Entonces $D_n = \{1,r,r^2 ,\ldots , r^{n-1}, s , rs,\ldots ,r^{n-1}s\}$ y así $|D_n| = 2n$ (los índices de $r$ se miran módulo $n$ y los de $s$ módulo $2$). Este grupo $D_n$ se llama el \textbf{grupo diedral}.

\section{Orden de elementos}
Sea $G$ un grupo y $a\in G$. Si $n\in\NN$ notamos $a^n = \underbrace{a\cdots a}_{n \text{ veces}}$. Por convención, $a^0 = e$ y $a^{-n}=(a^{n})^{-1}$. Notar que $a^{-n} = (a^{-n})^{-1}$ (simplemente se van tachando todos los términos al escribir $a^{-n}\cdot a^n$). Es fácil notar mediante inducción sobre $|m+n|$ que $a^{m+n} = a^n \cdot a^m$ y $(a^n)^m = a^{n\cdot m}$.

\begin{defn}
Sea $G$ grupo y $a\in G$. Decimos que $a$ tiene \textbf{orden finito} en $G$ si existe un $n\in\NN$ tal que $a^n = e$. En ese caso, el orden de $a$ es $\ord(a) = \min\{n\in\NN : a^n = e\}$. Se suele notar también $\ord(a)=o(a)=|a|$. Si $a$ no tiene orden finito en $G$, se dice que tiene \textbf{orden infinito}.
\end{defn}

\begin{prop} Supongamos que $\ord(a)=n$. Entonces $a^0, a^1 , \ldots , a^{n-1}$ son todos distintos en $G$.
\begin{proof}
Supongamos que existen $r,s\in\{0,1,\ldots , n-1\}$ tales que $a^r = a^s$. Supongamos sin pérdida de la generalidad que $r>s$. Entonces tenemos que $a^{r-s} = e$. Pero $0<r-s<n$. ¡Absurdo!
\end{proof}
\end{prop}
\begin{cor}
$\ord(a) = \sharp\{a^m : m\in\ZZ\}$ \begin{proof}Esto es pues si $m\in\ZZ$ entonces existen $q,r$ con $0\leq r \leq n-1$ tales que $m=nq+r$ y así $a^m = a^{nq+r} = (a^n)^q \cdot a^r = a^r$ y el corolario sigue.\end{proof}
\end{cor}

\begin{obs}
Si $\ord(a)=n$ entonces $a^{-1} = a^{n-1}$ pues $a\cdot a^{n-1} = a^n = e$. Además, si $a^m = e$ entonces $\ord(a)\mid m$.
\end{obs}

\begin{defn}
Un grupo $G$ se dice \textbf{cíclico} si existe un $a\in G$ tal que $\forall g\in G, \text{ } g=a^n, n\in\ZZ$. Se dice que $a$ es un \textbf{generador} de $G$.
\end{defn}

Ejemplos de grupos cíclicos son $(G_n,\cdot)$, $(\ZZ_n,+)$ y $(\ZZ,+)$.

\section{Subgrupos}

\begin{defn}
Sea $G$ un grupo. Un \textbf{subgrupo} de $G$ es un subconjunto no vacío $H\subseteq G$ tal que es cerrado por productos y por inversos. Es decir, $\forall a,b\in H \Longrightarrow a\cdot b \in H$ y además $\forall a\in H\Longrightarrow a^{-1}\in H$.
\end{defn}

\begin{obs}
$e\in H$ pues al ser $H\neq \emptyset$ existe un $a\in H$ y así $a^{-1}\in H$. Entonces $e = a\cdot a^{-1} \in H$.
\end{obs}
\begin{prop}
Si $H$ es finito y no vacío, cerrado por productos implica cerrado por inversos.
\begin{proof}
Sea $a\in H$. Considero $\{a^n : n\in\NN\}\subseteq H$ (esta inclusión es porque $H$ es cerrado por productos e inducción). Como $H$ es finito entonces $\{a^n : n\in\NN\}$ debe ser finito, y así deben existir $r>s$ tales que $a^r = a^s$, lo que implica que $a^{r-s}=e$. Si $r-s=k$ entonces $a^{k-1}=a^{-1}\in H$ y así $e= a\cdot a^{-1}\in H$.
\end{proof}
\end{prop}
\begin{obs}
De ser $H$ infinito esto es mentira: es fácil ver que con $G=(\NN,\cdot)$ y $H=\{2^n : n\in NN_0\}$ tenemos un contraejemplo.
\end{obs}

\begin{defn}
Sea $G$ un grupo. $\mathcal{Z}(G) = \{a\in G : a\cdot g = g\cdot a \text{ }\forall g\in G\}$ se llama el \textbf{centro} de $G$. Es fácil ver que $\mathcal{Z}(G)$ es un subgrupo pues si $a,b\in \mathcal{Z}(G)$ entonces $a\cdot g = g\cdot a$ y $b\cdot g = g\cdot b$ para todo $g\in G$, entonces $(a\cdot b) \cdot g = a\cdot (b\cdot g) = a\cdot (g\cdot b) = (a\cdot g)\cdot b = (g\cdot a)\cdot b = g\cdot (a\cdot b)$ y así $a\cdot b\in \mathcal{Z}(G)$ y además si $a\in \mathcal{Z}(G)$ entonces multiplicando por $a^{-1}$ a izquierda y derecha en $a\cdot g = g\cdot a$ obtenemos $a^{-1} \cdot g = g\cdot a^{-1}$.
\end{defn}

\begin{obs}
Sea $G$ un grupo y $\{H_\alpha\}_{\alpha \in \Lambda}$ una familia arbitraria de subgrupos de $G$ (no vacía). Entonces $\bigcap_{\alpha\in\Lambda}H_{\alpha}$ es un subgrupo de $G$.
\end{obs}

\begin{defn}
Sea $G$ un grupo y $X\subseteq G$ un subconjunto. Se define el \textbf{subgrupo generado por} $X$ como $\left\langle X\right\rangle = \bigcap_{H\subseteq G \text{ subgrupo tal que } X\subseteq H} H$. Esto es un subgrupo pues la intersección es no vacía porque $G$ está ahí. Si $X=\{x_1,\ldots ,x_n\}$ es un conjunto finito notamos $\langle X\rangle =\langle x_1,\ldots ,x_n\rangle$.
\end{defn}

\begin{obs}
$\langle X\rangle$ es un subgrupo de $X$, $X\subseteq \langle X\rangle$ y es el mínimo subgrupo (respecto de la inclusión) que contiene a $X$. Concretamente, $\langle X\rangle$ consiste de los elementos de $G$ de la forma $a_1\cdots a_m$ con $m\in\NN$, $a_i \in X$ ó $a_i^{-1}\in X$. Esto se ve pues vale la doble inclusión: estos elementos deben estar en $\langle X\rangle$ y como forman un grupo que contiene a $X$ debe estar contenido en $\langle X\rangle$.
\end{obs}
\begin{obs}
Si $X=\emptyset$ entonces $\langle \emptyset\rangle = 1$, el subgrupo trivial.
\end{obs}

Si $G$ es un grupo y $a\in G$ entonces $\langle a\rangle = \{a^n : n\in\ZZ\}\subseteq G$. Entonces $|\langle a\rangle| = \ord(a)$. Además, tenemos una definición equivalente de grupo cíclico: $G$ es cíclico si y sólo si existe un $a\in G$ tal que $\langle a\rangle = G$.

\begin{defn}
Sea $G$ un grupo. Decimos que $G$ tiene \textbf{exponente finito} si existe un $n\in\NN$ tal que $a^n = 1\text{ }\forall a\in G$. Si $G$ tiene exponente finito, se define $\exp(G) = \min \{n\in \NN : a^n = 1 \text{ }\forall a\in G\}$.
\end{defn}
\begin{obs}
Si $G$ es finito tenemos que $\exp(G) = \mcm\{\ord(a) : a\in G\}$. En efecto, si $a\in G$ con $G$ finito, tenemos que $\ord(a)<\infty$ pues $\langle a\rangle \subseteq G$ y $|\langle a\rangle | = \ord(a)$. Además, sabíamos que si $a^m = 1$ entonces $\ord(a)\mid m$. Estas dos cosas implican la observación.
\end{obs}

\section{Conjugados y Subgrupos Normales}

Sea $G$ un grupo, $H\subseteq G$ un subgrupo y $a\in G$. Se define $aHa^{-1}=\{a\cdot h\cdot a^{-1} : h\in H\}$. Notemos que $aHa^{-1}$ es un subgrupo. En efecto, $1\in aHa^{-1}$ pues $1 = a\cdot 1 \cdot a^{-1} \in H$. Si $h,h'\in H$ entonces $a\cdot h\cdot^{-1} \cdot a \cdot h'\cdot a^{-1} = a \cdot (h\cdot h')\cdot a^{-1}$ gracias a la asociatividad y así como $h\cdot h' \in H$ el producto es cerrado en $H$. Finalmente, es trivial corroborar que $(a\cdot h\cdot a^{-1})^{-1} = a \cdot h^{-1} a^{-1}$.

\begin{defn}
Sea $G$ un grupo y $H,H'\subseteq G$ subgrupos de $G$. Decimos que $H$ y $H'$ son \textbf{conjugados} si existe un $a\in G$ tal que $H' = aHa^{-1}$. Es fácil chequear que ser conjugados es una relación de equivalencia.
\end{defn}

\begin{obs}
Si $G$ es abeliano, entonces $aHa^{-1}=H$ y así dos subgrupos son conjugados si y sólo si son iguales.
\end{obs}

\begin{defn}
Sea $G$ un grupo. $H\subseteq G$ un subgrupo se dice \textbf{normal} o \textbf{invariante} en $G$ si $aHa^{-1}=H$ para todo $a\in G$. Esto lo notaremos $H\triangleleft G$.
\end{defn}

\begin{prop}
Sea $G$ grupo y $H\subseteq G$ subgrupo. $H\triangleleft G$ si y sólo si $\forall a\in G$ $aHa^{-1}\subseteq H$.
\begin{proof}
$(\Longrightarrow)$ No hay nada que probar.
$(\Longleftarrow)$ Sea $g\in G$. Quiero ver que $H\subseteq gHg^{-1}$, pues ahí tendría la doble contención. Como vale que $aHa^{-1}\subseteq H$ para todo $a\in G$, en particular vale para $a=g^{-1}$. Entonces $g^{-1}Hg\subseteq H$. Esto implica que $g(g^{-1}Hg)g^{-1}\subseteq gHg^{-1}$ y así $H\subseteq gHg^{-1}$, como queríamos ver.
\end{proof}
\end{prop}

Veamos un ejemplo: si $D_n$ es el $n$-ésimo grupo diedral y $n\neq 2$ entonces $D_n$ no es abeliano. Veamos que $\langle r\rangle \triangleleft D_n$. Quiero ver que para todo $a\in D_n$ tenemos $aHa^{-1}\subseteq H$. Como todo $a\in D_n$ es producto de $r^i$ y $s^j$ basta chequearlo para $r$ y $s$. Para $r$ es trivial pues $r\langle r\rangle r^{-1} = \langle r\rangle$ pues para cualquier $r^k\in \langle r\rangle$ entonces $r\cdot r^k \cdot r^{-1} = r^k$. Ahora para $s$, tenemos que $sr^ks = s sr^{-k} = r^{-k}$. Entonces $sHs^{-1} \subseteq H$.

Pero $\langle s\rangle$ no resulta normal en $D_n$ ya que $rsr^{-1} = r^2 s$ y $r^2$ no es ni $1$ ni $s$.

\begin{prop}\begin{enumerate}
\item Sea $\{H_\alpha\}_{\alpha\in \Lambda}$ una familia de subgrupos normales en $G$. Entonces $\bigcap_{\alpha\in\Lambda}H_\alpha \triangleleft G$.
\item Sea $H\subseteq G$ subgrupo. Entonces $\bigcap_{a\in G}aHa^{-1} \triangleleft G$.
\item $\mathcal{Z}(G)\triangleleft G$.
\end{enumerate}
\begin{proof}
\begin{enumerate}
\item Vamos a probar que para todo $g\in G$ y todo $h\in \bigcap_{\alpha\in\Lambda}H_\alpha$ tenemos que $ghg^{-1}\in \bigcap_{\alpha\in\Lambda}$. En efecto, si $h$ está en la intersección, $h\in H_\alpha$ para todo $\alpha\in\Lambda$. Pero $H_\alpha\triangleleft G$ para todo $\alpha\in\Lambda$. Esto implica que $ghg^{-1}\in H_\alpha$, para todo $\alpha\in\Lambda$, y así debe estar en la intersección, como queríamos.
\item El plan es el mismo que para el item anterior. Sea $g\in G$ y $x\in \bigcap_{a\in G}aHa^{-1}$. Entonces $x\in aHa^{-1}$ para todo $a\in G$. Esto implica que $gxg^{-1} \in gaHa^{-1}g^{-1} = (ga)H(ga)^{-1}$ para todo $a\in G$. Pero al variar $a\in G$, $ga$ recorre todos los elementos de $G$. Es decir, $a\mapsto ga$ es una biyección de $G$ en $G$. En efecto, si $ga=gb$ por la propiedad cancelativa $a=b$, o sea, es inyectiva, y además, $a = g(g^{-1}a)$ y así es sobreyectiva. Por lo tanto $gxg^{-1}\in aHa^{-1}$ para todo $a\in G$ y estamos.
\item Nuevamente, sea $x\in \mathcal{Z}(G)$. Quiero ver que $gxg^{-1}\in \mathcal{Z}(G)$ para todo $g\in G$. En efecto, al ser $x\in\mathcal{Z}(G)$ tenemos que $gx = xg$ para todo $g\in G$ y así $gxg^{-1} = x \in \mathcal{Z}(G)$, como queríamos.
\end{enumerate}
\end{proof}
\end{prop}

\section{Conmutador}
\begin{defn} Sea $G$ grupo, $a,b\in G$. Definimos $[a,b]=aba^{-1}b^{-1}$ el \textbf{conmutador} entre $a$ y $b$.
\end{defn}
\begin{obs}
$[a,b]=1$ si y sólo si $ab=ba$. Además, $[a,b]^{-1}=[b,a]$. 
\end{obs}
\begin{defn}
El \textbf{conmutador} de $G$ es $[G,G]=\left\langle [a,b] \right\rangle_{a,b\in G}$.
\end{defn}
\begin{prop}
El conmutador $[G,G]$ es invariante en $G$.
\begin{proof}
Notemos que $g[a,b]g^{-1}=[gag^{-1},gbg^{-1}]$. Como el inverso de un conmutador es un conmutador, tenemos que $[G,G]$ es el producto finito de conmutadores. Entonces, si $[a_1,b_1]\cdots [a_k,b_k]$ es un elemento de $[G,G]$ tenemos que $$g[a_1,b_1]\cdots [a_k,b_k]g^{-1} = g[a_1,b_1]g^{-1}g[a_2,b_2]\ldots g[a_k,b_k]g^{-1} = [ga_1g^{-1},gb_1g^{-1}]\cdots [ga_kg^{-1},gb_kg^{-1}]$$ que es un elemento de $[G,G]$. Como queríamos.
\end{proof}
\end{prop}

\section{Morfismos de Grupos}

\begin{defn}
Sean $G,H$ grupos. Un \textbf{morfismo de grupos} $\varphi:G\to H$ es una función que cumple que $\varphi(a\cdot b) = \varphi(a)\cdot \varphi(b)$ para todos $a,b\in G$. Es decir, respeta la estructura.
\end{defn}

\begin{obs}
Tenemos que $\varphi(1)=1$ pues $\varphi(1) = \varphi(1\cdot 1) = \varphi(1)\cdot \varphi(1)$. Además, $1=\varphi(1) = \varphi(a\cdot a^{-1})=\varphi(a)\cdot \varphi(a^{-1})$, entonces $\varphi(a^{-1}) = (\varphi(a))^{-1}$.
\end{obs}

\begin{defn}
Un morfismo de grupos $\varphi:G\to H$ se dice \textbf{monomorfismo} si es inyectivo, \textbf{epimorfismo} si es sobreyectivo, e \textbf{isomorfismo} si es biyectivo. Podemos definir además $\ker \varphi = \{x\in G : \varphi(x)=1\}\subseteq G$ e $\im \varphi = \{\varphi(x) : x\in G\}\subseteq H$.
\end{defn}

\begin{obs}
$\ker \varphi \subseteq G$ e $\im \varphi\subseteq H$ son subgrupos. Más aún $\ker \varphi \triangleleft G$ pues si $x\in\ker \varphi$ entonces $\varphi(gxg^{-1}) = \varphi(g)\varphi(x)\varphi(g^{-1}) = \varphi(g)\varphi(g)^{-1}= 1$ y así $gxg^{-1}\in\ker \varphi$ para todo $g\in G$.
\end{obs}

\begin{prop}
$\varphi:G\to H$ es monomorfismo si y sólo si $\ker \varphi = 1$.
\begin{proof}
$(\Longrightarrow)$ Sea $a\in\ker\varphi$. Entonces $\varphi(a)=1$. Pero $\varphi(1)=1$, y como $\varphi$ es inyectiva, tenemos que $a=1$.

$(\Longleftarrow)$ Sean $a,b\in G$ tales que $\varphi(a)=\varphi(b)$. Entonces $1 = \varphi(a)\varphi(b)^{-1} = \varphi(a)\varphi(b^{-1})=\varphi(ab^{-1})$. Como $\ker\varphi$ es trivial, debemos tener $ab^{-1}=1$ y así $a=b$ y $\varphi$ es inyectiva.
\end{proof}
\end{prop}

\begin{obs}
Para cualesquiera $G,H$ grupos, siempre tenemos un morfismo entre ellos: el morfismo trivial/nulo/constante $\varphi:G\to H$, $\varphi(a)=1 \text{ }\forall a\in G$.
\end{obs}

\begin{ex} Para halla los morfismos de grupos de $\ZZ_n$ a $\ZZ$, consideremos $\varphi:\ZZ_n\to\ZZ$ morfismo de grupos. Entonces $\varphi(0)=0$. 
Además $0=\varphi(0)=\varphi(\underbrace{1+\ldots+1}_{n \text{ veces}}) = n\varphi(1)$. Esto implica que $\varphi(1)=0$ y así $\varphi(a) = \varphi(\underbrace{1+\ldots+1}_{a\text{ veces}})= a\varphi(1) = 0$ para todo $a\in\ZZ_n$. Es decir, el único morfismo que existe es el trivial.\end{ex}

\begin{defn}
Dos grupos $G$ y $H$ se dicen \textbf{isomorfos} si existe $\varphi:G\to H$ isomorfismo de grupos. Lo notamos $G\simeq H$.
\end{defn}

\begin{prop}\label{prop::isociclico}
Sea $G$ grupo. $G$ es cíclico si y sólo si $G\simeq \ZZ$ o $G\simeq \ZZ_n$ para algún $n\in\NN$.
\begin{proof}
$(\Longrightarrow)$ Supongamos que $|G|<\infty$. Esto implica que $G = \langle a \rangle$ para algún $a\in G$. Sea $\ord(a)=m$. Afirmo que $\varphi:G\to \ZZ_{m}$ definido por $\varphi(a^k)=k$ es un isomorfismo de grupos. Notemos que es un morfismo de grupos pues $\varphi(a^k \cdot a^\ell) = \varphi(a^{k+\ell})= k+\ell = \varphi(a^k)+\varphi(a^\ell)$. Además, es inyectivo pues si $\varphi(a^k) = \varphi(a^\ell)$ entonces $k\equiv \ell \pmod{m}$. Pero si $\ell = mq + k$ tenemos que $a^{\ell} = (a^{m})^q a^k = a^k$ pues $a$ tiene orden $m$. Por la definición es obvio que es sobreyectiva pues para cualquier $k\in\ZZ_{m}$ tenemos que $k=\varphi(a^k)$. Entonces es un isomorfismo como afirmamos y $G\simeq \ZZ_{m}$.

Ahora si $|G|$ es infinito, tenemos de igual forma que $G = \langle a\rangle$ para algún $a\in G$. Afirmamos que $\varphi:G\to \ZZ$ definido por $\varphi(k) = k$. En efecto, es muy similar a lo anterior: es morfismo de grupos pues $\varphi(a^k\cdot a^\ell) = \varphi(a^{k+\ell}) = k+\ell = \varphi(a^k)+\varphi(a^\ell)$. Además, es inyectiva pues si $\varphi(a^k) = \varphi(a^\ell)$ entonces $k=\ell$. Es sobreyectiva pues para cualquier $k\in\ZZ$ tenemos que $k=\varphi(a^k)$. Entonces es isomorfismo como afirmamos y $G\simeq \ZZ$.

$(\Longleftarrow)$ Supongamos que $G\simeq \ZZ_n$. Entonces existe un isomorfismo $\varphi:\ZZ_n\to G$. Afirmo que $G = \langle \varphi(1)\rangle$. La inclusión $\langle \varphi(1)\rangle \subseteq G$ es trivial. Ahora, como $\varphi$ es sobreyectiva por ser isomorfismo, tenemos que para cualquier $g\in G$ existe un $\ell$ tal que $\varphi(\ell)=g$. Pero $g=\varphi(\ell)=\varphi(\underbrace{1+\ldots +1}_{\ell \text{ veces}}) = \varphi(1)^{\ell}$. Es decir, $\varphi(1)^\ell$ recorre todos los valores de $G$ y así $G\subseteq \langle \varphi(1) \rangle$.

Ahora supongamos que $G\simeq \ZZ$. De la misma forma, hay un isomorfismo $\varphi:\ZZ\to G$ y resulta, procediendo de manera análoga al caso anterior, que $G = \langle \varphi(1)\rangle$. Y estamos.
\end{proof}
\end{prop}

\section{Coclases a derecha e izquierda}

Sea $G$ un grupo, $H\subseteq G$ subgrupo y $a\in G$. Definimos $aH=\{ah : h\in H\}$ la \textbf{coclase a izquierda} de $a$ en $H$. Análogamente, $Ha= \{ha : h\in H\}$ es la \textbf{coclase a derecha} de $a$ en $H$. Estos dos conjuntos no suelen ser subgrupos de $G$ en general.

\begin{prop} \label{prop::igualdaddecoclases}
$aH=H$ si y sólo si $a\in H$ (idem para coclases a derecha)
\begin{proof}
$(\Longrightarrow)$ $a=a\cdot 1 \in aH = H \Longrightarrow a\in H$.
$(\Longleftarrow)$ Supongamos que $a\in H$. Entonces $aH\subseteq H$ pues si $ah\in aH$ y $a,h\in H$, entonces $ah\in H$ al ser $H$ un subgrupo. La otra contención sale porque $h=a(a^{-1}h)\in aH$ (pues $a^{-1}\in H$ por ser subgrupo y así $a^{-1}h\in H$ por la misma razón).
\end{proof}
\end{prop}

\begin{obs}
Sea $G$ un grupo y $H\subseteq G$ un subgrupo.
\begin{itemize}
\item Sean $a,b\in G$ Si $b\in aH$ entonces $aH=bH$.
\begin{proof} Probemos la doble inclusión:

$(aH\subseteq bH)$: Sea $ah\in aH$. Como $b\in aH$ tenemos que $b=ah'$. Entonces $ah = (bh'^{-1})h = b(h'^{-1}h)\in bH$.

$(bH\subseteq aH)$: Como $b\in aH$ tenemos que $b=ah'$ y así $bh = (ah')h = a(h'h)\in aH$ para todo $h\in H$.
\end{proof}
\item $aH\cap bH\neq \emptyset \Longrightarrow aH=bH$.
\begin{proof}
Si $aH\cap bH\neq \emptyset$ entonces existe un $z\in aH\cap bH$. Por el item anterior, $aH=zH$ y $bH=zH$. Entonces $aH=bH$ como queríamos.
\end{proof}
\item $aH=bH \Longleftrightarrow a^{-1}b\in H$.
\begin{proof}
Notemos que $aH=bH$ si y sólo si $a^{-1}(aH) = a^{-1}(bH)$. Es decir, $H=(a^{-1}b)H$. Pero por la proposición ~\ref{prop::igualdaddecoclases} tenemos que eso pasa si y sólo si $a^{-1}b \in H$. Y listo.
\end{proof}
\item Para todos $a,b\in G$ tenemos que $\sharp aH = \sharp bH$.
\begin{proof}
Tenemos una biyección $\varphi:aH\to bH$ dada por $\varphi(ah)=bh$. Esto prueba la afirmación.
\end{proof}
\end{itemize}
\end{obs}

De esta observación, se desprende que podemos a escribir a $G$ como la unión disjunta de las coclases de $a$ en $H$, pues $a\in aH$ y dos coclases o son iguales o son disjuntas. Además, como dos coclases cualesquiera en $H$ tienen igual cardinal podemos hacer la siguiente definición:

\begin{defn}
Sea $G$ un grupo y $H$ subgrupo. El \textbf{índice} de $H$ en $G$ es la cantidad de coclases a izquierda de $H$ que hay en $G$, es decir $(G:H)=\sharp\{aH : a\in G\}$.
\end{defn}

\begin{obs} Es trivial notar que $(G:G)=1$ y $(G:1)=|G|$.
\end{obs}

Estamos en condiciones de probar el siguiente teorema
\begin{teo}[Lagrange] \label{teo::lagrange} Sea $G$ un grupo finito y $H\subseteq G$ subgrupo. Entonces $$|G|=(G:H)|H|$$ En particular, $|H|\mid |G|$ y $\forall a\in G$ tenemos que $\ord(a)\mid |G|$.
\begin{proof}
Como $G$ es la unión disjunta de las coclases $aH$, $\sharp aH = |H|$ para todo $a\in G$ y hay $(G:H)$ coclases, tenemos que $|G|=(G:H)|H|$. El resto del teorema es un simple corolario pues ya probamos que $|\langle a\rangle| = \ord(a)$. Y estamos.
\end{proof}
\end{teo}

\begin{prop}
Sea $G$ un grupo de orden $p$ con $p$ primo. Entonces $G\simeq \ZZ_p$.
\begin{proof}
Si $x\in G$ entonces $\ord(x)\mid |G|=p$ por el teorema de Lagrange ~\ref{teo::lagrange}. Esto implica que $\ord(x)=1 \text{ ó } p$. Ahora si tomamos $x\in G$ con $x\neq 1$ (existe pues $|G|=p>1$) debemos tener que $\ord(x)=p$, pues de lo contrario sería el neutro. Entonces $\{1,x,\ldots , x^{p-1}\}$ es un conjunto con sus elementos todos distintos entre sí. Pero $\{1,x,\ldots , x^{p-1}\}\subseteq G$ y ambos tienen $p$ elementos. Entonces $G= \langle x\rangle$. Esto quiere decir que $G$ es cíclico y como probamos en ~\ref{prop::isociclico} debemos tener $G\simeq \ZZ_p$. Y estamos.
\end{proof}
\end{prop}

Los elementos de $a$ determinan unívocamente a la clase $aH$ pero la clase $aH$ no determina unívocamente a un representante $a$.

Notemos que todas estas definiciones que hicimos para coclases a izquierda las podríamos haber hecho para coclases a derecha y haber obtenido resultados análogos. En efecto, hay una biyección entre las coclases a izquierda y las coclases a derecha, que está dada por $aH\longleftrightarrow Ha^{-1}$. En la observación habíamos notado que $aH=bH \Longleftrightarrow a^{-1}b\in H$. Vale el resultado análogo para coclases a derecha $Ha = Hb \Longleftrightarrow ab^{-1}\in H$. Entonces, si $aH = bH$ tenemos que $Ha^{-1} = Hb^{-1}$, pues ambos son equivalentes a $a^{-1}b$. Es decir, la asignación $aH \mapsto Ha^{-1}$ \textbf{no} depende del representante elegido. Podemos definir entonces al índice como la cantidad de coclases, sin importar si es a izquierda o derecha, pues están en biyección.

\begin{prop}
Sea $G$ grupo y sean $K\subseteq H\subseteq G$ subgrupos de $G$. Entonces $$(G:K) = (G:H)(H:K)$$
\begin{proof}
Podemos escribir a $G$ como la unión disjunta de las coclases a izquierda en $H$: $G = \displaystyle\coprod_{i\in I}a_iH$, con $\sharp I = (G:H)$. Análogamente, hacemos $H = \displaystyle\coprod_{j\in J}b_j K$ con $\sharp J = (H:K)$. Entonces, $G = \displaystyle\coprod_{i\in I}a_i \left( \displaystyle\coprod_{j\in J}b_j K\right) = \displaystyle\coprod_{i\in I, j\in J}(a_ib_j)K$. 

Esto implica que $(G:K) = \sharp I\cdot \sharp J = (G:H)(H:K)$, como queríamos probar.
\end{proof}
\end{prop}

\begin{obs}
$H\triangleleft G$ si y sólo si $\forall a\in G$ $aHa^{-1}= H \Longleftrightarrow \forall a\in G \text{ } aH=Ha$.
\end{obs}

\begin{prop}
Si $H\subseteq G$ es un subgrupo tal que $(G:H)=2$, entonces $H\triangleleft G$.
\begin{proof}
Sea $a\in G$. Queremos ver que $aH=Ha$. Si $a\in H$ esto es trivialmente cierto pues $aH=H=Ha$. Ahora, si $a\notin H$, como sólo hay dos clases por ser el índice $2$, debemos tener que $aH=G-H = Ha$ pues $G = H\coprod aH$. Entonces $aH=Ha$ para todo $a\in G$ y así $H\triangleleft G$, como queríamos.
\end{proof}
\end{prop}

Un ejemplo para esta última proposición es que $\langle r\rangle \triangleleft D_n$, donde $D_n$ es el $n$-ésimo grupo diedral y $\langle r\rangle$ es el subgrupo generado por las rotaciones. En efecto, $|D_n|=2n$ y $|\langle r\rangle|=n$, y entonces por Lagrange, $(D_n : \langle r\rangle) = |D_n|/|\langle r\rangle| = 2$.

\begin{defn}
Sea $G$ grupo. $G$ se dice \textbf{simple} si los únicos subgrupos $H\subseteq G$ tales que $H\triangleleft G$ son $1$ y $G$.
\end{defn}

\begin{prop}\label{prop::productonormal}
Sea $G$ grupo y $H,N$ subgrupos de $G$. Consideremos $HN=\{hn:h\in H, n\in N\}$. Si $N\triangleleft G$ entonces $HN$ es un subgrupo de $G$. Más aún, si también $H\triangleleft G$ entonces $HN\triangleleft G$.

\begin{proof}
Primero probemos que si $N\triangleleft G$ entonces $HN$ es un subgrupo de $G$. En efecto, solo debemos probar que es cerrado por productos y por inversos pues es trivial que $e=e\cdot e \in HN$. Si $h_1n_1\in HN$ y $h_2n_2\in HN$ entonces, usando que $h_2^{-1}n_1h_2 = n\in N$ por la normalidad de $N$ en $G$, tenemos que $h_1n_1h_2n_2 = h_1h_2(h_2^{-1}n_1h_2)n_2 = h_1h_2 nn_2 \in HN$. Análogamente, $(h_1n_1h_2n_2)^{-1} = n_2^{-1}h_2^{-1}n_1^{-1}h_1^{-1} = h_2^{-1}h_2 n_2^{-1}h_2^{-1}n_1^{-1}h_1^{-1} = (h_2^{-1}n'n_1^{-1}) h_1^{-1}$, que es un producto de dos elementos de $HN$ pues $h_2^{-1} n'n_1^{-1}$ está y $h_1^{-1} = h_1^{-1} 1$ también y ya probamos que es cerrado por productos, y así por inversos.

Para ver que si también $H\triangleleft G$ entonces $HN\triangleleft G$ simplemente debemos conjugar. En efecto $gHNg^{-1} = gHg^{-1}gNg^{-1} = HN$. Y listo.

\end{proof}
\end{prop}

\section{Cocientes}

\begin{defn}
Sea $G$ un grupo y $H\triangleleft G$ un subgrupo normal. Se define el \textbf{grupo cociente} de $G$ por $H$, $G/H = \{aH:a\in G\}$, con el producto $(aH)\cdot(bH)=((ab)H)$.
\end{defn}

Notemos que este producto está bien definido, en gran parte gracias al hecho de que $H\triangleleft G$. Si $aH=a'H$ y $bH=b'H$ entonces: $$(ab)H = a(bH) = a(b'H) = a(Hb') = (aH)b = (a'H)b' = a'(Hb') = a'(b'H) = (a'b')H$$

\begin{prop}
$(G/H,\cdot)$ es un grupo.
\begin{proof}
La asociatividad sigue pues $$((aH)(bH))(cH) = ((ab)H)(cH) = ((ab)c)H = (a(bc))H = (aH)((bc)H) = (aH)((bH)(cH))$$

El elemento neutro es $1H = H$ pues $(aH)(1H)=((a1)H) = (aH)$ para todo $a\in G$.

Finalmente, el inverso de $aH$ es $a^{-1}H$ pues $$(aH)(a^{-1}H) = ((a a^{-1})H) = 1H = ((a^{-1}a) H) = (a^{-1}H)(aH)$$ Y la proposición sigue.
\end{proof}
\end{prop}

\begin{defn}
Dados $G$ grupo y $H\triangleleft G$ subgrupo normal, definimos $\pi:G\to G/H$, $a\mapsto aH$, la proyección de $G$ al cociente. Es un morfismo de grupos por cómo definimos al producto en $G/H$.
\end{defn}

\textbf{Notación:} Vamos a escribir $\overline{a} = aH$. Entonces, $G/H = \{\overline{a}:a\in G\}$, con $\overline{a} = \overline{b} \Longleftrightarrow a^{-1}b\in H$. Con esta notación, el producto definido queda $\overline{ab} = \overline{a}\overline{b}$, con neutro $\overline{1}$ e inverso $\overline{a^{-1}}$. Además, la proyección al cociente es $\pi:G\to G/H$, $a\mapsto \overline{a}$.

\begin{obs}
Es trivial notar que $\ker \pi = H$.
\end{obs}

\section{Propiedad Universal del Cociente y Teoremas de Isomorfismo}

\begin{prop}Sea $G$ un grupo, $H\triangleleft G$ y $\pi:G\to G/H$ la proyección al cociente. Entonces, $\pi:G\to G/H$ tiene la siguiente propiedad (universal): \begin{itemize}\item $H\subseteq \ker \pi$, es decir $\pi(H)=1$ \item Para todo subgrupo $G'$ y morfismo $\varphi:G\to G'$tal que $H\subseteq \ker \varphi$ existe un único morfismo $\overline{\varphi}:G/H\to G'$ tal que $\overline{\varphi}\circ \pi = \varphi$. Es decir, el siguiente diagrama conmuta:\end{itemize}

\begin{center}
\begin{tikzcd}[row sep=3.3em,column sep=4em,minimum width=2em]
 G\arrow[]{d}[left,font=\normalsize]{\pi}\arrow[]{r}[above, font=\normalsize]{\varphi}& G' \\
G/H\arrow[dashed]{ur}[right, font=\normalsize]{\overline{\varphi}} & \\
\end{tikzcd}
\end{center}

\begin{proof}
Veamos primero la existencia de $\overline{\varphi}$. En efecto, definamos $\overline{\varphi}:G/H\to G'$ como $\overline{\varphi}(\overline{a}) = \varphi(a)$. Veamos que esto está bien definido: si $\overline{a}=\overline{b}$ entonces $a^{-1}b\in H$. Entonces $\varphi(a^{-1}b) = 1$. Como $\varphi$ es morfismo, tenemos que $1=\varphi(a^{-1}b) = \varphi(a^{-1})\varphi(b) = \varphi(a)^{-1}\varphi(b)$. Esto implica que $\varphi(a)=\varphi(b)$. Con esta definición, $\overline{\varphi}$ resulta morfismo trivialmente pues $\overline{\varphi}(\overline{ab})=\varphi(ab)=\varphi(a)\varphi(b) = \overline{\varphi}(\overline{a})\overline{\varphi}(\overline{b})$.

Además, esta $\overline{\varphi}$ es única, pues tiene que cumplir que $\overline{\varphi}\circ \pi = \varphi$. Es decir, para todo $a\in G$, queremos que $\overline{\varphi}(\overline{a})=\varphi(a)$. 

En otras palabras, si tuvieramos otra $\psi$ que cumple eso, tendríamos $\psi(\overline{a})=\varphi(a) = \overline{\varphi}(\overline{a})$. 

Y estamos.
\end{proof}
\end{prop}

De hecho, se cumple algo más fuerte y bastante útil: el cociente queda completamente caracterizado (bajo isomorfismo) por esta propiedad (universal). Esto se ve reflejado en la siguiente proposición:

\begin{prop}
Sean $G$ y $T$ grupos. Supongamos que $\psi:G\to T$ morfismo tal que $H \subseteq \ker \psi$ y para todo grupo $G'$ y morfismo $\varphi:G\to G'$ con $H\subseteq \ker \varphi$ existe un único $\overline{\varphi}:T\to G'$ tal que $\overline{\varphi}\circ \psi = \varphi$. Entonces $T\simeq G/H$. Es decir, si el siguiente diagrama conmuta, debemos tener que $T\simeq G/H$. 

\begin{center}
\begin{tikzcd}[row sep=3.3em,column sep=4em,minimum width=2em]
 G\arrow[]{d}[left,font=\normalsize]{\psi}\arrow[]{r}[above, font=\normalsize]{\varphi}& G' \\
T\arrow[dashed]{ur}[right, font=\normalsize]{\overline{\varphi}} & \\
\end{tikzcd}
\end{center}

\begin{proof}
Tenemos el siguiente diagrama:

\begin{center}
\begin{tikzcd}[row sep=3.3em,column sep=4em,minimum width=2em]
 G\arrow[]{dd}[left, font=\normalsize]{\pi}\arrow[]{rr}[font=\normalsize]{\psi}& & T\arrow[start anchor=-115, end anchor=25, dashed]{ddll}[font=\normalsize]{\overline{\pi}} \\
 \\
G/H\arrow[start anchor=65, end anchor=-155, dashed]{uurr}[font=\normalsize]{\overline{\psi}} & &G\arrow[]{ll}[font=\normalsize]{\pi}\arrow[]{uu}[right, font=\normalsize]{\psi} \\
\end{tikzcd}
\end{center}

Sabemos que el triángulo superior conmuta por la proposición anterior (pues tomando $G'=T$ con $\psi$ el morfismo, estamos bajo las hipótesis ya que $\ker \psi \supset H$), y el triángulo inferior conmuta por la hipótesis de esta proposición, tomando $G'=G/H$ con el morfismo proyección $\pi$.

Probemos que $\overline{\pi}$ y $\overline{\psi}$ son inversas. Como son morfismos de grupos, habremos construido un isomorfismo entre $G/H$ y $T$. En efecto, por la proposición anterior tenemos que $\overline{\psi}\circ \pi = \psi$. Además, por las hipótesis tenemos que $\overline{\pi}\circ \psi = \pi$. Por lo tanto $\overline{\psi}\circ (\overline{\pi}\circ \psi) = \psi$. Entonces, $(\overline{\psi}\circ \overline{\pi})\circ \psi = \psi$. Pero como $\id\circ \psi = \psi$ y tenemos propiedad universal, tenemos que tener $\overline{\psi}\circ\overline{\pi}=\id$. Análogamente, tenemos que $(\overline{\pi}\circ\overline{\psi})\circ \pi = \pi$, y nuevamente, gracias a la propiedad universal, tenemos que $\overline{\pi}\circ \overline{\psi}$ es la identidad y entonces $\overline{\pi}$ y $\overline{\psi}$ son inversas. Como son morfismos de grupos y son inversos, tenemos que son isomorfismos y así $G/H\simeq T$, como queríamos probar.

\end{proof}

\end{prop}
\begin{obs}
Si tenemos $G$ grupo, $H\triangleleft G$ y tenemos $\varphi:G\to G'$ morfismo de grupos, entonces existe un único $\overline{\varphi}:G/H\to G'$ tal que $\overline{\varphi}\circ \pi = \varphi$. Si $\varphi$ es epimorfismo, entonces $\overline{\varphi}$ lo es y si $\ker \varphi = H$ entonces $\overline{\varphi}$ es monomorfismo. Esto se ve bien claro en el próximo teorema.
\end{obs}

\begin{teo}[Primer Teorema de Isomorfismo]
Sea $\varphi:G\to H$ un morfismo de grupos. Entonces $\im\varphi \simeq G/\ker\varphi$.
\begin{proof}

Por la Propiedad Universal del cociente, sabemos que existe un único morfismo $\overline{\varphi}:G/\ker\varphi \to\im\varphi$ tal que $\overline{\varphi}\circ \pi = \varphi$. Como $\varphi$ es trivialmente sobreyectiva, tenemos que $\overline{\varphi}$ debe serlo también. Resta ver que es inyectiva. Para eso, notemos que $\ker \overline{\varphi} = \{\overline{a} : \overline{\varphi}(\overline{a}) = 1\} = \{\overline{a} : \varphi(a)=1\}$. O sea, $\ker \overline{\varphi} = \{\overline{a} : a\in \ker \varphi\}$. Esto implica que $\ker \overline{\varphi} = 1$ y así es inyectiva. Y estamos.

\end{proof}
\end{teo}

Podemos ilustrar el Primer Teorema de Isomorfismo con este ejemplo:

\begin{ex}
$\RR/\ZZ \simeq S^1$. En efecto, consideremos $\varphi:\RR \to S^1$ dado por $t\mapsto e^{2\pi i t}$. Notemos que de hecho es un morfismo pues $\varphi(t+s) = e^{2\pi i (t+s)} = e^{2\pi i t}e^{2\pi i s} = \varphi(t)\varphi(s)$. Además, notemos que $e^{2\pi i t} = 1$ si y sólo si $t\in \ZZ$. Es decir, $\ker \varphi = \ZZ$. Por el Primer Teorema de Isomorfismo, debemos tener entonces que $\RR/\ZZ\simeq S^1$.
\end{ex}

La siguiente proposición está también intimamente relacionada con este Teorema:

\begin{prop}
Sean $G,H$ grupos y $G\stackrel{\varphi}{\longrightarrow}H$ morfismo. Entonces, podemos factorizar a $\varphi = g\circ f$ con $f$ sobreyectiva y $g$ inyectiva. Más aún, esto se hace de forma única (salvo isomorfismo).
\begin{proof}

La existencia de una tal factorización es obvia: Sea $\pi : G\to G/\ker \varphi$ la proyección al cociente y sea $\overline{\varphi}$ el único morfismo tal que $\overline{\varphi}\circ \pi = \varphi$. Es claro que $\pi$ es sobreyectiva y si $\overline{\varphi}(\overline{a}) = \varphi(a) = \varphi(b) = \overline{\varphi}(\overline{b})$ entonces $\varphi(a^{-1}b)=1$ y así $a^{-1}b\in \ker\varphi$, lo que implica $\overline{a}=\overline{b}$. Es decir, $\overline{\varphi}$ es inyectiva.

\begin{center}
\begin{tikzcd}[row sep=3.3em,column sep=4em,minimum width=2em]
 G\arrow[bend right]{ddr}[font=\normalsize]{f}\arrow[]{dr}[font=\normalsize]{\pi}\arrow[]{rr}[font=\normalsize]{\varphi}& & H \\
 &  G/\ker\varphi \arrow[]{ur}[font=\normalsize]{\overline{\varphi}}\arrow[leftrightarrow, dashed]{d}[font=\normalsize]{} &   \\
  & M \arrow[bend right]{uur}[font=\normalsize]{g}&  \\
\end{tikzcd}
\end{center}


Ahora, supongamos que tenemos $G\stackrel{f}{\longrightarrow} M \stackrel{g}{\longrightarrow}H$, con $f$ sobreyectiva y $g$ inyectiva, con $\varphi = g\circ f$. Como $g$ es inyectiva, tenemos que $\ker g = 1$. 

Esto implica que $\ker \varphi = \ker g\circ f = \{x\in G : f(x) = 1\} = \ker f$. Entonces, $\ker \varphi = \ker f$. Entonces, tengo un morfismo $f:G\to M$ cuyo núcleo es igual al de $\varphi$. Esto quiere decir que, por el primer teorema de isomorfismo, $M\simeq G/\ker\varphi$. Y estamos.

\end{proof}
\end{prop}

\begin{teo}[Segundo Teorema de Isomorfismo]
Sean $K\subseteq H\subseteq G$ subgrupos de $G$ y $K\triangleleft G$, $H\triangleleft G$. Entonces $K\triangleleft H$ y $(G/K)/(G/H)\simeq G/H$.
\begin{proof}
Vamos a construir un epimorfismo $\varphi:G/K\to G/H$ con $\ker \varphi = H/K$. Usamos la propiedad universal del cociente: como $K\subseteq H=\ker\pi_H$, existe un único $\overline{\pi}_H:G/K\to G/H$ que hace conmutar al diagrama. 
\begin{center}
\begin{tikzcd}[row sep=3em,column sep=3em,minimum width=2em]
 G\arrow[]{d}[left,font=\normalsize]{\pi_K}\arrow[]{r}[above, font=\normalsize]{\pi_H}& G/H \\
G/K\arrow[dashed]{ur}[right, font=\normalsize]{\overline{\pi}_H} & \\
\end{tikzcd}
\end{center}
Como $\pi_H$ es epimorfismo, entonces $\overline{\pi}_H$ es epimorfismo. Pero ahora, $\ker{\overline{\pi}_H}=\{\overline{g}\in G/K : \overline{\pi}_H(\overline{g}) = \overline{1}\} = \{\overline{g}\in G/K : \pi_H(g)=1\}=\{\overline{g}\in G/K : g\in H\} = H/K$.

Por el primer teorema de isomorfismo, $\overline{\pi}_H$ es el isomorfismo que queríamos. Y estamos.
\end{proof}
\end{teo}

\begin{defn}
$H\subseteq G$ subgrupo, $N(H) = \{g\in G : gHg^{-1} = H\}$ es el \textbf{normalizador} de $H$ en $G$.
\end{defn}
\begin{obs}
Es trivial ver que $H\triangleleft G \Longleftrightarrow N(H)=G$.

Notar además que $N(H)$ es un subgrupo de $G$ y más aún, es normal. En forma similar a ~\ref{prop::productonormal}, se puede probar que si $K\subseteq G$ es un subgrupo con $K\subseteq N(H)$ entonces $HK$ es un subgrupo de $G$ y $H\triangleleft HK$.

Finalmente, si $K\subseteq N(H)$ entonces $K\cap H$ es normal en $K$.

\end{obs}
Estamos en condiciones de probar el tercer teorema de isomorfismo:

\begin{teo}[Tercer Teorema de Isomorfismo]
Sean $H,K\subseteq G$ subgrupos y $K\subseteq N(H)$. Entonces $K/(K\cap H)\simeq HK/H$
\begin{proof}
Buscamos un epimorfismo de $K$ en $HK/H$ con núcleo $K\cap H$. Si construyeramos eso, por el primer teorema de isomorfismo el resultado sigue.

En efecto, definamos $\varphi : K\to HK/H$, $\varphi (k) = \overline{1\cdot k} = \overline{k}$. Es claro que es un morfismo de grupos. Quiero ver que $\varphi$ es epimorfismo. En efecto, notemos que $\overline{hk} = \overline{k}$ pues $k^{-1} (hk)\in H$ por tener $K\subseteq N(H)$. Esto implica que $\varphi$ es epimorfismo como queríamos. Resta ver que su núcleo es $K\cap H$. Pero $\ker \varphi = \{k\in K : \overline{k} = \overline{1} \in KH/H\} = \{k\in K : k\in H\} = H\cap K$. Entonces $\varphi$ es el morfismo que queríamos construir y estamos.

\end{proof}
\end{teo}
\newpage 

\section{Algunas Propiedades Universales Más}

\begin{defn}
Una \textbf{Categoría} $\mathcal{C}$ es una clase $\Obj(\mathcal{C})$ de \textbf{objetos} y para cada par de objetos $M,N\in\Obj(\mathcal{C})$ un conjunto de \textbf{morfismos} $\hom_{\mathcal{C}}(M,N)$, con una ley de composición asociativa $\circ:\hom(N,S)\times \hom(M,N)\to \hom(M,S)$ y un neutro para esta composición (es decir, dado $M\in\Obj(\mathcal{C})$ existe $\id_M\in\hom(M,M)$ tal que es un neutro para $\circ$). A los morfismos se los suele denominar \textbf{flechas}.
\end{defn}
\begin{ex}
Ejemplos de categorías son $\text{Grp}$, cuyos objetos son grupos y los morfismos son morfismos de grupos. $\text{Sets}$, cuyos objetos son conjuntos y morfismos son funciones entre los conjuntos y $\text{Vect}_K$ cuyos objetos son $K$-espacios vectoriales y morfismos son las transformaciones $K$-lineales.
\end{ex}

Vamos a definir un producto categóricamente. Es decir, lo vamos a definir como un objeto que cumple cierta propiedad universal.
\begin{defn}
Sea $\mathcal{C}$ una categoría y sean $G,H\in\Obj(\mathcal{C})$. Decimos que $K$ es un \textbf{producto} de $G$ y $H$ si existen dos morfismos $p_1\in \hom(K,G)$ y $p_2\in \hom(K,H)$ y tal que para todo $K'\in\Obj(\mathcal{C})$ con $f_1\in\hom(K',G)$ y $f_2\in\hom(K',H)$ tenemos que existe una única flecha $\overline{f}\in\hom(K',K)$ tal que $p_1\circ \overline{f}=f_1$ y $p_2\circ \overline{f}=f_2$. Es decir, existe una única flecha tal que el siguiente diagrama conmuta.
\begin{center}
\begin{tikzcd}[row sep=2em,column sep=2em,minimum width=2em]
 & K'\arrow[bend right]{ldd}[font=\normalsize]{f_1}\arrow[bend left]{rdd}[font=\normalsize]{f_2}\arrow[dashed]{d}[font=\normalsize]{\overline{f}} &  \\
 &  K\arrow{ld}[font=\normalsize]{p_1} \arrow{rd}[font=\normalsize]{p_2}&   \\
  G& & H \\
\end{tikzcd}
\end{center}
\end{defn}
Podemos definir entonces la \textit{noción dual}, es decir, dar vuelta cada flecha:
\begin{defn}
Sea $\mathcal{C}$ una categoría y sean $G,H\in\Obj(\mathcal{C})$. Decimos que $K$ es el \textbf{coproducto} de $G$ y $H$ si existen dos morfismos $p_1\in \hom(G,K)$ y $p_2\in \hom(H,K)$ y tal que para todo $K'\in\Obj(\mathcal{C})$ con $f_1\in\hom(G,K')$ y $f_2\in\hom(H,K')$ tenemos que existe una única flecha $\overline{f}\in\hom(K,K')$ tal que $\overline{f}\circ p_1=f_1$ y $\overline{f}\circ p_2=f_2$. Es decir, existe una única flecha tal que el siguiente diagrama conmuta.
\begin{center}
\begin{tikzcd}[row sep=2em,column sep=2em,minimum width=2em]
 & K' &  \\
 &  K\arrow[dashed]{u}[font=\normalsize]{\overline{f}} &   \\
  G\arrow[bend left]{ruu}[font=\normalsize]{f_1}\arrow{ru}[font=\normalsize]{p_1}& & H\arrow[bend right]{luu}[font=\normalsize]{f_2}\arrow{lu}[font=\normalsize]{p_2} \\
\end{tikzcd}
\end{center}
\end{defn}
También podemos definir categóricamente al núcleo (y en consiguiente su noción dual, conúcleo). Antes necesitamos definir una categoría con morfismos triviales.
\begin{defn}
Una categoría $\mathcal{C}$ se dice que tiene \textbf{morfismos triviales} si para objetos $X,Y,Z\in\Obj(\mathcal{C})$ y morfismos $f\in \hom(Y,Z)$, $g\in\hom(X,Y)$ el siguiente diagrama conmuta
\begin{center}
\begin{tikzcd}[row sep = 2.5em, column sep = 2.5em, minimum width = 2em]
X\arrow{dr}[font=\normalsize]{e_{X,Z}}\arrow{d}[font=\normalsize]{g}\arrow{r}[font=\normalsize]{e_{X,Y}} & Y \arrow{d}[font=\normalsize]{f}\\
Y\arrow{r}[font=\normalsize]{e_{Y,Z}} & Z
\end{tikzcd}
\end{center}
Es decir $e_{X,Y}\in\hom(X,Y)$ actúa como el morfismo que envía a todo elemento de $X$ al neutro de $Y$.
\end{defn}

\begin{defn}
Sea $\mathcal{C}$ una categoría con morfismos triviales. Sean $G,H\in\Obj(\mathcal{C})$ y $f\in\hom(G,H)$. Decimos que $K$ es el \textbf{núcleo} de $f$ si existe $\iota\in\hom(K,G)$ tal que $f\circ \iota = e_{K,H}$ tal que para todo $K'\in\Obj(\mathcal{C})$ con $\iota' \in \hom(K',G)$ que cumple $f\circ \iota' = e_{K',H}$ existe una única flecha $g\in\hom(K',K)$ tal que $\iota \circ g = \iota'$. Es decir, existe una única flecha que hace conmutar al siguiente diagrama:
\begin{center}
\begin{tikzcd}[row sep=3em,column sep=2em,minimum width=2em]
K\arrow[]{rr}[font=\normalsize]{\iota}\arrow[bend left]{rrrr}[font=\normalsize]{e_{K,H}} & & G\arrow[]{rr}[font = \normalsize]{f} & & H \\
& K' \arrow[dashed]{lu}[font=\normalsize]{g}\arrow[]{ru}[font=\normalsize]{\iota'}\arrow[]{rrru}[font=\normalsize]{e_{K',H}} & &
\end{tikzcd}
\end{center}
\end{defn}
\begin{defn}
Sea $\mathcal{C}$ una categoría con morfismos triviales. Sean $G,H\in\Obj(\mathcal{C})$ y $f\in\hom(H,G)$. Decimos que $K$ es el \textbf{conúcleo} de $f$ si existe $\pi\in\hom(G,K)$ tal que $\pi\circ f = e_{H,K}$ tal que para todo $K'\in\Obj(\mathcal{C})$ con $\pi' \in \hom(G,K')$ que cumple $\pi' \circ f = e_{H,K'}$ existe una única flecha $g\in\hom(K,K')$ tal que $g\circ \pi= \pi'$. Es decir, existe una única flecha que hace conmutar al siguiente diagrama:
\begin{center}
\begin{tikzcd}[row sep=3em,column sep=2em,minimum width=2em]
K\arrow[dashed]{dr}[font=\normalsize]{g} & & G\arrow[]{ll}[font=\normalsize]{\pi}\arrow[]{ld}[font=\normalsize]{\pi'} & & H\arrow[]{ll}[font = \normalsize]{f}\arrow[bend right]{llll}[font=\normalsize]{e_{H,K}}\arrow[]{llld}[font=\normalsize]{e_{H,K'}} \\
& K'  & &
\end{tikzcd}
\end{center}
\end{defn}

Para que esta sección no sea simplemente definiciones aparentemente vacías, fijémosnos qué son ciertos objetos en ciertas categorías.

\begin{prop}
Si estamos en la categoría de grupos, el producto de $G$ y $H$ es isomorfo a $G\times H$.
\begin{proof}
Es fácil ver que $G\times H$ tiene la propiedad universal del producto. Ahora, supongamos que tenemos un $K$ que también tiene la propiedad universal. Tenemos el siguiente diagrama:
\begin{center}
\begin{tikzcd}[row sep=2em,column sep=2em,minimum width=2em]
 & K\arrow[bend right]{ldd}[font=\normalsize]{f_1}\arrow[bend left]{rdd}[font=\normalsize]{f_2}\arrow[dashed]{d}[font=\normalsize]{v} &  \\
 &  G\times H\arrow[dashed]{u}[font=\normalsize]{u}\arrow{ld}[font=\normalsize]{\pi_1} \arrow{rd}[font=\normalsize]{\pi_2}&   \\
  G& & H \\
\end{tikzcd}
\end{center}
Queremos ver que $u$ y $v$ son inversas, de donde seguirá que $G\times H$ y $K$ son isomorfos. Por la conmutatividad de los diagramas, tenemos que $\pi_1\circ vu = f_1 \circ u = \pi_1$ y que $\pi_2 \circ vu = f_2\circ vu = \pi_2$. Es decir, tenemos que el siguiente diagrama conmuta:
\begin{center}
\begin{tikzcd}[row sep=2em,column sep=2em,minimum width=2em]
 & G\times H\arrow[bend right]{ldd}[font=\normalsize]{\pi_1}\arrow[bend left]{rdd}[font=\normalsize]{\pi_2}\arrow[dashed]{d}[font=\normalsize]{vu} &  \\
 &  G\times H\arrow{ld}[font=\normalsize]{\pi_1} \arrow{rd}[font=\normalsize]{\pi_2}&   \\
  G& & H \\
\end{tikzcd}
\end{center}
Pero $\id_{G\times H}$ hace conmutar a ese diagrama, y $vu$ es la única flecha que lo hace conmutar. Debemos tener entonces que $vu=\id_{G\times H}$. Análogamente, probamos que $uv = \id_{K}$. Entonces $K\simeq G\times H$ y estamos.
\end{proof}
\end{prop}

Es fácil ver que el coproducto en Sets es la unión disjunta. También, en grupos abelianos, el coproducto es igual al producto, pero en grupos, sin pedir abelianidad, no es tan fácil definirlo. En la categoría de grupos, el núcleo es el núcleo de siempre. En la categoría de grupos abelianos, se puede ver fácil (pues $\im f\triangleleft G$ por ser abeliano) que $\coker f = G/\im f$.

\section{Producto Directo y Semidirecto}

\begin{defn}[Producto Directo Externo]
Sean $G$ y $H$ grupos. Definimos el \textbf{producto directo externo} de $G$ y $H$ como $G\times H = \{(g,h) : g\in G, h\in H\}$ con la operación $(g,h)\cdot (g',h') = (gg', hh')$. Si $G$ y $H$ son abelianos esto se suele notar $G \oplus H$.
\end{defn}

\begin{obs}
Notar que el producto directo externo $G\times H$ cumple la propiedad universal del producto. En efecto, tenemos dos morfismos proyecciones $\pi_1:G\times H\to G$ dado por $(g,h)\mapsto g$ y $\pi_2 :G\times H\to H$ dado por $(g,h)\mapsto h$. Además, si tenemos un $K$ tal que existen $f_1:K\to G\times H$ y $f_2:K\times H$ tenemos una flecha $f:K\to G\times H$ $k\mapsto (f_1(k),f_2(k))$ que hace conmutar trivialmente al diagrama.
\end{obs}

\begin{defn}[Producto Directo Interno]
Sea $G$ un grupo. Decimos que $G$ es el \textbf{producto directo interno} de $H$ y $T$ si $H$ y $T$ subgrupos de $G$ tales que \begin{itemize}\item $G=HT= \{ht : h\in H, t\in T\}$ (como conjunto). \item Si $ht=h't'$ con $h,h'\in H$ y $t,t'\in T$, entonces $h=h'$ y $t=t'$. \item $(ht)(h't') = (hh')(tt')$ \end{itemize}
\end{defn}

\begin{prop}
Sean $H$ y $T$ subgrupos de $G$. Son equivalentes \begin{enumerate}\item Si $ht=h't'$ con $h,h'\in H$ y $t,t'\in T$, entonces $h=h'$ y $t=t'$. \item $H\cap T = 1$. \end{enumerate}
\begin{proof}
(\textit{1}$\Longrightarrow$\textit{2}): Sea $x\in H\cap T$. Entonces $1x = x = x1$. Pero esto implica que $x=1$ pues escribimos a un mismo elemento de dos formas distintas (aprovechando que $x\in H\cap T$).

(\textit{2}$\Longrightarrow$\textit{1}): Supongamos que $ht=h't'$. Entonces $(h'^{-1}h)(tt'^{-1})=1$. Esto quiere decir que $(tt'^{-1})$ es el inverso de $(h'^{-1}h)\in H$ y así $tt'^{-1}\in H\cap T = 1$. Entonces $t=t'$. Análogamente, $h=h'$.
\end{proof}
\end{prop}

\begin{prop}
Si $G$ es el producto directo interno de dos subgrupos $H$ y $T$ entonces $G\simeq H\times T$.
\begin{proof}
Hay un claro isomorfismo dado por $\varphi:G\to H\times T$, $\varphi(h\cdot t)=(h,t)$. Está bien definido pues al ser un producto directo interno hay una única forma de escribir a cada elemento de $G$ como $h\cdot t$. 

Es morfismo pues $\varphi ((ht)(h't')) = \varphi((hh')(tt')) = (hh',tt') = (h,t)(h',t') = \varphi (ht)\varphi(h't')$. Además, es inyectivo pues si $\varphi(ht)=\varphi(h't')$ entonces $(h,t)=(h',t')$ y así $h=h'$, $t=t'$. Y es sobreyectivo pues para cada $(h,t)\in H\times T$ tenemos que $\varphi(ht) = (h,t)$.
\end{proof}
\end{prop}

\begin{prop}
Si $H,T\subset G$ son subgrupos, $G$ es el producto directo interno de $H$ y $T$ si y sólo si se verifican: \begin{itemize}\item $G=HT= \{ht : h\in H, t\in T\}$ (como conjunto). \item Si $ht=h't'$ con $h,h'\in H$ y $t,t'\in T$, entonces $h=h'$ y $t=t'$. \item $H\triangleleft G$ y $T\triangleleft G$. \end{itemize}
\begin{proof}
($\Longrightarrow$) Si $H\triangleleft G$ y $T\triangleleft G$ entonces $tht^{-1} = h'$, lo que implica $th = h't$ y así $h^{-1}th = (h^{-1}h')t$. Pero nuevamente, $t' = h^{-1}th = (h^{-1}h')t$. Como la expresión es única, debemos tener $h=h'$ y $t=t'$. Entonces $tht^{-1}=h$ y así $ht=th$.

Entonces, $hth't' = h(th')t' = h(h't)t' = (hh')(tt')$.

($\Longleftarrow$) Conjuguemos un elemento cualquiera de $H$ y veamos que cae en $H$. Para $T$ será análogo. En efecto, como $G$ es el producto directo interno de $H$ y $T$ tenemos que todo $g\in G$ se escribe como $g=ht$ con $h\in H$, $t\in T$. Entonces, $gxg^{-1} = (ht)x(t^{-1}h^{-1}) = (ht)(xt^{-1})(h^{-1}1) = (hxh^{-1})(tt^{-1}1) = hxh^{-1}$. Pero como $x\in H$, tenemos que $gxg^{-1}\in H$, y listo.

\end{proof}
\end{prop}

\begin{obs}
Sean $H,N\subseteq G$ subgrupos y $N\triangleleft G$ tales que $G=NH$ y $N\cap H = 1$. Entonces, si tenemos $g=nh$ y $g'=n'h'$ es claro que el producto se puede escribir como $gg'= nhn'h' = \underbrace{(nhn'h^{-1})}_{\in N}\underbrace{(hh')}_{\in H}$. Sea $\varphi : H\to \Aut (N)$, $h\mapsto (n\mapsto hnh^{-1})$. Entonces podemos escribir al producto como $(nh)(n'h') = (n\varphi(h)(n')) (hh')$. Esto motiva la siguiente definición:
\end{obs}

\begin{defn}[Producto Semidirecto Interno]
Sean $N,H\subseteq G$ subgrupos con $N\triangleleft G$, $G=NH$ y $N\cap H = 1$. Decimos que $G$ es un \textbf{producto semidirecto interno} de $N$ y $H$ y lo denotamos $G= N \rtimes H$.
\end{defn}
\begin{defn}[Producto Semidirecto Externo]
Dados $G$ y $H$ grupos con un morfismo de grupos $\varphi:H\to \Aut (G)$, el \textbf{producto semidirecto externo} $G\rtimes_{\varphi} H$ es el grupo que tiene como conjunto subyacente a $G\times H$ y como producto $(g,h)(g',h') = (g\varphi (h)(g') , hh')$.
\end{defn}


\section{El grupo \texorpdfstring{$S_n$}{Sn}}

Ya vimos que si $I_n = \{1,2,\ldots ,n\}$ con $n\in \NN$ entonces el conjunto $S_n = \{\sigma:I_n\to I_n \text{ biyectiva}\}$ junto con la composición forma un grupo: el grupo de las permutaciones o $n$-ésimo grupo simétrico. Es claro que $|S_n|=n!$.

\begin{obs}
Este grupo no es abeliano. Si $\sigma_{i,j}$ es la permutación que intercambia a $i$ con $j$, tenemos que $\sigma_{i,j}\circ \sigma_{j,k} \neq \sigma_{j,k}\circ \sigma_{i,j}$, pues en ambas $k$ va a parar a distintos lados.
\end{obs}

\begin{defn}
Sea $i\in I_n, \sigma\in S_n$. Decimos que $i$ \textbf{queda fijo} por $\sigma$ si $\sigma (i)=i$ y $\sigma$ \textbf{mueve} a $i$ si $\sigma (i)\neq i$.
\end{defn}

\begin{obs}
Dado $\sigma\in S_n, i\in I_n$, si $\sigma$ mueve a $i$ entonces $\sigma$ mueve a $\sigma(i)$ (en efecto, si $\sigma(i)$ quedara fijo, tendríamos que $\sigma(i) = \sigma(\sigma(i))$ con $i\neq \sigma(i)$ y $\sigma$ no sería inyectiva).
\end{obs}

\begin{defn}
Si $\sigma,\tau \in S_n$ son dos permutaciones, decimos que son \textbf{disjuntas} si los elementos movidos por alguna de ellas quedan fijos por la otra (es decir, $\not\exists i\in I_n$ tal que $\sigma(i)\neq i$ y $\tau(i)\neq i$).
\end{defn}

\begin{prop}\label{prop::disjuntosconmutan}
Si $\sigma,\tau \in S_n$ son dos permutaciones disjuntas entonces $\sigma \circ \tau = \tau \circ \sigma$.

\begin{proof}
Sea $i\in I_n$. Entonces, $\sigma\circ\tau (i)=\begin{cases}\sigma(i) \text{ si } \tau \text{ deja fijo a } i \\ \tau(i) \text{ si } \tau \text{ mueve a } i\end{cases}$ En efecto, el primer renglón es por definición y el segundo pues si $\tau$ mueve a $i$, debe mover a $\tau(i)$ por la observación anterior, y como son disjuntas esto implica que $\sigma$ fija a $\tau(i)$. Análogamente, se tiene que $\tau\circ\sigma (i)=\begin{cases}\tau(i) \text{ si } \sigma \text{ deja fijo a } i \\ \sigma(i) \text{ si } \sigma \text{ mueve a } i\end{cases}$ Por lo tanto, si $i$ queda fijo tanto por $\sigma$ como por $\tau$, tenemos $\tau\circ\sigma (i)=\sigma\circ\tau (i) = i$. Si $\tau$ mueve a $i$ entonces $\sigma$ lo deja fijo pues son disjuntos y así $\tau\circ\sigma (i) = \tau(i) = \sigma\circ\tau(i)$. Si $\sigma$ mueve a $i$, entonces $\tau$ lo deja fijo pues son disjuntos y así $\tau\circ\sigma(i)=\sigma(i)=\sigma\circ\tau(i)$. Y la proposición sigue.
\end{proof}
\end{prop}

Ahora, nos va a interesar analizar la estructura cíclica de $S_n$. Es decir, vamos a probar que toda permutación se puede factorizar como producto de ciclos disjuntos de forma única.

\begin{defn}
Sea $\sigma\in S_n$. Se dice que $\sigma$ es un $r$-ciclo ($2\leq r\leq n$) si existen $r$ elementos $j_0,j_1\ldots j_{r-1}\in I_n$ distintos de modo tal que \begin{itemize}\item $\sigma$ deja fijo a $I_n - \{j_0,j_1,\ldots ,j_{r-1}\}$ \item $\sigma(j_{k})=j_{k+1}$ con los índices módulo $r$\end{itemize}
En este caso notamos $\sigma = (j_0, j_1, \ldots , j_{r-1})$.
\end{defn}

\begin{defn}
Una \textbf{transposición} es un $2$-ciclo.
\end{defn}

\begin{obs}
Sea $\sigma = (j_0, j_1 , \ldots , j_r{-1})$. 

Entonces $\sigma = (j_1, j_2, \ldots , j_{r-1},j_0) = (j_k, j_{k+1}, \ldots j_{r-1},j_0,\ldots , j_{k-1})$. Notemos entonces que $\sigma^{k}(j_i) = j_{i+k}$ con los índices módulo $r$, y $\sigma(x)=x$ si $x\neq j_i \text{ }\forall i$. Esto implica que $\ord(\sigma)=r$.

Además, si escribimos $\sigma = \sigma_1\cdots \sigma_s$ con $\sigma_i$ ciclos disjuntos dos a dos, por la proposición ~\ref{prop::disjuntosconmutan} tenemos que $\sigma_i \sigma_j = \sigma_j \sigma_i$. Entonces, $\sigma^k = \sigma_1^k \cdots \sigma_s^k$. Estamos en condiciones de probar el siguiente teorema:
\end{obs}

\begin{teo}
Todo $\sigma\in S_n$ se factoriza como producto de ciclos disjuntos. Más aún, la factorización es única salvo orden de los ciclos. Además, si $\sigma = \sigma_1\cdots \sigma_s$, entonces podemos calcular el orden como $\ord(\sigma) = \displaystyle\mcm_{i=1,\ldots , s} \{\ord(\sigma_i)\}$.

\begin{proof}
Primero veamos la existencia de una factorización. Procederemos por inducción en $k$, la cantidad de elementos movidos por $\sigma$. Con $k=0$, $\sigma = \id$ y es trivial. Supongamos que vale para todo $\tau\in S_n$ que mueva menos de $k$ elementos. Sea $\sigma\in S_n$ que mueven $k$ elementos. Tomo $j_1$ tal que $\sigma$ mueve a $j_1$ (pues $k>0$) y considero $\sigma^k (j_1) = j_k$. Notemos por finitud que existe un $j_r$ tal que $\sigma (j_r) \in \{j_1,\ldots , j_r\}$. Pero en efecto debe ser $\sigma (j_r)=j_1$ pues si no $\sigma$ no sería inyectiva y mucho menos biyectiva. Entonces, tenemos un ciclo $\sigma_1 = (j_1, \ldots , j_r)$. Consideremos $\sigma_{1}^{-1}\circ \sigma$. Notemos que los $j_1, \ldots , j_r$ quedan fijos, y además los que quedaban fijos por $\sigma$ también quedan fijos. Entonces, $\sigma_1^{-1}\circ \sigma$ mueve a menos elementos que $\sigma$. Podemos aplicar la hipótesis inductiva entonces: $\sigma_{1}^{-1}\circ \sigma = \sigma_2 \circ \ldots \circ \sigma_s$. Además, $\sigma_1$ y $\sigma_2\circ\ldots\circ\sigma_s$ son disjuntos pues $j_1,\ldots ,j_r$ quedan fijos en $\sigma_2\circ\ldots\circ\sigma_s$ pero en $\sigma_1$ se mueven (y $\sigma_1$ no mueve a ningún otro elemento). Entonces $\sigma = \sigma_1\circ\ldots\circ \sigma_s$ es una tal factorización.

Ahora, veamos que es única. En efecto, supongamos que $\sigma = \sigma_1\cdots \sigma_s = \sigma_1' \cdots \sigma_t'$. Supongamos que $\sigma_1 = (j_1 ,\ldots , j_r)$. Entonces $\sigma (j_1) = j_2$ pues las $\sigma_i$ son disjuntas. Esto implica que $\sigma^k(j_1)=\sigma_1^k(j_1) = j_{k+1}$ (índices módulo $r$) . Ahora, exactamente algún $\sigma_i'$ mueve a $j_1$ (pues los ciclos son disjuntos y $\sigma$ lo mueve). Entonces, podemos suponer sin pérdida de la generalidad que es $\sigma_1'$. Entonces $\sigma_1'(j_1)=j_2$. Pero como $\sigma$ mueve a $j_2$ y los ciclos son disjuntos debemos tener que $\sigma_1'(j_2)=j_3$. Así sucesivamente probamos que $\sigma_1'^k(j_1) = \sigma^k(j_1)=\sigma_1^k(j_1)$ para todo $k$ y así $\sigma_1 = \sigma_1'$. Por propiedad cancelativa obtenemos $\sigma_2 \cdots \sigma_s = \sigma_2'\cdots \sigma_t'$. Procediendo así vamos a obtener $\sigma_i = \sigma_i'$ (post-reordenamiento) y además $s=t$. Con lo que la unicidad sigue.

Finalmente, notemos que $\sigma^k = (\sigma_1\cdots \sigma_s)^k = \sigma_1^k \cdots \sigma_s^k$ pues los ciclos disjuntos conmutan. Sea $r=\mcm_{i=1,\ldots ,s}\{\ord(\sigma_i)\}$. Entonces, $\sigma^r = \sigma_1^r \cdots \sigma_s^r = \id$, entonces $\ord\sigma \mid r$.

Análogamente, notemos que $\ord \sigma_i \mid \ord\sigma$ para todo $i$, pues $\sigma_i^{\ord\sigma} = \id$. En efecto, sea $j\in\{1,2,\ldots, n\}$. Si $\sigma_i$ fija a $j$ entonces $\sigma_i^{\ord\sigma}(j)=j$. Si no lo fija, entonces todos los otros ciclos sí la fijan (pues son disjuntos) y así $\sigma^k(j) = \sigma_i^k(j)$ para todo $k$. Esto implica que $j=\sigma^{\ord\sigma}(j)=\sigma_i^{\ord\sigma}(j)$. Y estamos.

\end{proof}

\end{teo}

\begin{obs}\label{obs::conjugarciclos}
Notemos que si $\sigma = (j_1, \ldots , j_r)\in S_n$ y $\tau\in S_n$, entonces $\tau \sigma \tau^{-1} = (\tau(j_1), \ldots , \tau (j_r))$. En efecto, si $\tau^{-1}(j)\neq j_i$ para todo $i$, entonces $\sigma(\tau^{-1}(j))=\tau^{-1}(j)$ y así $\tau(\sigma(\tau^{-1}(j))) = \tau(\tau^{-1}(j)) = j$. Pero si $\tau^{-1}(j) = j_i$ entonces $\sigma(\tau^{-1}(j)) = \sigma(j_i) = j_{i+1}$ y así $\tau(\sigma(\tau^{-1}(j))) = \tau(j_{i+1})$. Y listo.

Esto nos da una forma fácil de conjugar cualquier permutación. Sea $\sigma = \sigma_1\cdots \sigma_s$ la factorización en ciclos de $\sigma$. Entonces $\tau\sigma\tau^{-1} = (\tau\sigma_1\tau^{-1}) (\tau \sigma_2 \tau^{-1}) \cdots (\tau \sigma_s \tau^{-1})$ y conjugamos cada ciclo como sabemos.

\end{obs}

\begin{defn}
Dados $\sigma,\mu\in S_n$, decimos que tienen la misma \textbf{estructura cíclica} si puedo factorizar como ciclos disjuntos $\sigma = \sigma_1\cdots \sigma_s$, $\mu = \mu_1\cdots \mu_s$ y $\ord \sigma_i = \ord \mu_i$ para todo $i$ (ordenando los factores adecuadamente).
\end{defn}

\begin{teo}
$\sigma,\mu\in S_n$ tienen la misma estructura cíclica si y sólo si son conjugados (es decri $\exists \tau\in S_n$ tal que $\mu = \tau\sigma\tau^{-1}$.
\begin{proof}
($\Longrightarrow$) Sigue trivialmente de la observación ~\ref{obs::conjugarciclos}.

($\Longleftarrow$) Supongamos que tenemos las factorizaciones $\sigma = (j_1,\ldots , j_{r_1})\cdots (j_{r_{s-1}+1},\ldots , j_{r_s})$ y $\mu = (j_1',\ldots , j_{r_1}')\cdots (j_{r_{s-1}+1}',\ldots , j_{r_s}')$. Queremos encontrar un $\tau\in S_n$ tal que $\mu = \tau\sigma\tau^{-1}$.

Sea $\varphi: I_n - \{j_1,\ldots ,j_{r_s}\}\to I_n-\{j_1',\ldots , j_{r_s}'\}$ una biyección cualquiera (existe pues ambos son conjuntos finitos con la misma cantidad de elementos). Entonces definimos $$\tau(j) = \begin{cases}j_k' \text{ si } j=j_k \\ \varphi(j) \text{ si } j\neq j_k \text{ }\forall k\in\NN\end{cases}$$

$\tau$ es una biyección y es trivial verificar que $\mu = \tau \sigma\tau^{-1}$. El teorema sigue.
\end{proof}
\end{teo}

Ahora vamos a querer estudiar los generadores de $S_n$. Notemos que por el teorema de factorización, los ciclos generan a $S_n$.

\begin{obs}
Sea podemos factorizar a todo $r$-ciclo como producto de $r-1$ transposiciones $(i_1,\ldots , i_r) = (i_1,i_r)\cdots (i_1,i_3)(i_1,i_2)$. Como los ciclos generan a $S_n$, y todo ciclo se puede escribir como producto de transposiciones, las transposiciones generan a $S_n$. Pero además, notemos que $(i,j) = (1,i)(1,j)(1,i)$. Entonces $S_n = \langle (1,j) : j\in\NN \rangle$.
\end{obs}

\begin{obs}
También tenemos que $S_n = \langle (1,2), (2,3),\ldots , (n-1,n)\rangle$. Esto vale pues $(1,j)=(j-1,j)(j-2,j-1)\ldots (1,2)$. Análogamente, $S_n = \langle (1,2),(1,2,\ldots , n)\rangle$ pues $\sigma^{\ell}(\tau(\sigma^{n-\ell}(j))) = (1+\ell,2+\ell)$.
\end{obs}

Sabemos entonces que podemos escribir a $\sigma \in S_n$ como producto de trasposiciones (no en forma única) $\sigma = \tau_1\cdots \tau_m = \tau_1'\cdots \tau_{\ell}'$. ¿Qué podemos decir sobre esta escritura?

\begin{teo}
Si $\sigma = \tau_1\cdots \tau_m = \tau_1'\cdots \tau_\ell'$, con $\tau_i,\tau_i'$ trasposiciones, entonces $m\equiv \ell\pmod{2}$.

\begin{proof}
Dado $\sigma\in S_n$ defino $N(\sigma) \in \NN_0$ como $N(\id)=0$ y si $\sigma =\sigma_1\cdots \sigma_s$ con $\sigma_i$, ciclos disjuntos, entonces $N(\sigma)=\sum_{i=1}^{s}(\ord(\sigma_i)-1) = -s + \sum_{i=1}^s \ord(\sigma_i)$.

Afirmo que si $\tau$ es una trasposición y $\sigma\in S_n$ entonces $N(\tau\sigma) = N(\sigma)\pm 1$, es decir $N(\tau\sigma)\not\equiv N(\sigma)\pmod{2}$. Vamos a cubrir cuatro casos:

\textbf{Caso A:} $\sigma, \tau$ disjuntos. $\sigma = \sigma_1\cdots \sigma_s$. Entonces $\tau\sigma = \tau\sigma_1\cdots \sigma_s$ ya es la factorización en ciclos disjuntos. Entonces $N(\tau\sigma) = N(\sigma)+1$.

\textbf{Caso B:} $\sigma$ y $\tau$ mueven exactamente un elemento en común.

Si $\tau = (k, h)$ y $\sigma = \sigma_1\cdots \sigma_s$ con $\sigma_1 = (k,j_2,\ldots , j_r)$ (sin pérdida de la generalidad), entonces $\tau\sigma = (k,h)(k,j_2,\ldots , j_r)\sigma_2\cdots \sigma_s = (k,j_2,\ldots , j_r,h)\sigma_2\cdots \sigma_s$ y eso es la factorización en ciclos disjuntos. Entonces $N(\tau\sigma) = N(\sigma)+1$.

\textbf{Caso C:} $\sigma$ y $\tau$ mueven dos elementos y en ciclos disjuntos.

$\tau = (k,h)$ y $\sigma = \sigma_1\cdots \sigma_s$ con $\sigma_1 = (h,j_1,\ldots , j_\ell)$ y $\sigma_2=(k,i_1,\ldots, i_r)$ (sin pérdida de la generalidad). Entonces $$\tau\sigma = (k,h)(h,j_1,\ldots , j_\ell)(k,i_1,\ldots ,i_r)\sigma_3\cdots\sigma_s = (k,i_1,\ldots , i_r,h,j_1,\ldots ,j_\ell)\sigma_3\cdots \sigma_s$$ es una factorización en ciclos disjuntos y así $N(\tau\sigma)=N(\sigma)+1$.

\textbf{Caso D:} $\sigma$ y $\tau$ comparten dos elementos que están en el mismo ciclo de $\sigma$.

$\tau = (k,h)$, $\sigma = \sigma_1\cdots \sigma_s$ con $\sigma_1 = (k,i_1,\ldots , i_r,h,j_1,\ldots , j_\ell)$ (sin pérdida de la generalidad). Como en el caso anterior, $(k,i_1,\ldots , i_r,h,j_1,\ldots , j_\ell) = (k,h)(h,j_1,\ldots j_\ell)(k,i_1,\ldots , i_r)$. Entonces $$\tau\sigma = (k,h)(k,h)(h,j_1,\ldots j_\ell)(k,i_1,\ldots , i_r)\sigma_2\cdots\sigma_s = (h,j_1,\ldots j_\ell)(k,i_1,\ldots , i_r)\sigma_2\cdots \sigma_s$$ Esto implica que $N(\tau\sigma) = N(\sigma)-1$, y la afirmación queda probada.

Ahora, sean $\tau_1,\ldots ,\tau_m$ trasposiciones. $N(\tau_m)=1$ por ser una trasposición. Entonces $N(\tau_1\cdots \tau_m) \equiv m \pmod{2}$, por inducción. Esto implica que si $\sigma = \tau_1\cdots\tau_m = \tau_1'\cdots\tau_\ell'$ entonces $N(\sigma)=N(\tau_1\cdots \tau_m)=N(\tau_1'\cdots \tau_\ell')$. 

Pero esto quiere decir que $m \equiv N(\tau_1\cdots \tau_m)=N(\tau_1'\cdots \tau_\ell') \equiv \ell \pmod{2}$. Y estamos.
\end{proof}
\end{teo}

\begin{defn}
Sea $\sigma\in S_n$. Decimos que $\sigma$ es par si se escribe como un producto de una cantidad par de trasposiciones. En ese caso $\sg(\sigma)=1$. Si no, se dice que es impar y $\sg(\sigma)=-1$.
\end{defn}

\begin{obs}
$\sg(\sigma\mu)=\sg(\sigma)\sg(\mu)$. En efecto, si $\sigma=\tau_1\cdots\tau_m$ y $\mu=\tau_1'\cdots \tau_\ell'$ entonces $\sg(\sigma)=(-1)^m$ y $\sg(\mu) = (-1)^\ell$. Pero a su vez, $\sigma\mu = \tau_1\cdots \tau_m \tau_1'\cdots \tau_\ell'$ y entonces $\sg(\sigma\mu)=(-1)^{m+\ell} = \sg(\sigma)\sg(\mu)$.

Esto quiere decir que $\sg:S_n\to G_2 = \{-1,1\}$ es un morfismo de grupos. Más aún, es un epimorfismo pues siempre hay permutaciones pares e impares (si $n\geq 3$).
\end{obs}

\begin{defn}
Definimos el $n$-ésimo grupo alternado $A_n = \ker \sg = \{\sigma\in S_n : \sigma \text{ es par}\}$.
\end{defn}

\begin{prop}
$|A_n| = \dfrac{|S_n|}{2}$. Es decir, hay tantas permutaciones pares como impares.
\begin{proof}
Tenemos el siguiente diagrama conmutativo:
\begin{center}
\begin{tikzcd}[row sep=3em,column sep=3em,minimum width=2em]
 S_n\arrow[]{d}[left,font=\normalsize]{\pi}\arrow[]{r}[above, font=\normalsize]{\sg}& G_2 \\
S_n/A_n\arrow[dashed]{ur}[right, font=\normalsize]{\overline{\varphi}} & \\
\end{tikzcd}
\end{center}
Como $A_n = \ker\sg$, tenemos que $A_n\triangleleft S_n$, y así podemos considerar el cociente $S_n/A_n$. Por el primer teorema de isomorfismo, tenemos que $S_n/A_n \simeq G_2$. 

En particular, $|S_n/A_n|= \dfrac{|S_n|}{|A_n|} = |G_2|=2$. Entonces $|A_n|=\dfrac{|S_n|}{2}$, como queríamos probar.

\end{proof}
\end{prop}

\begin{obs}
Cualquier $3$-ciclo es par, es decir $(i,j,k)\in A_n$ pues $\sg(i,j,k)=1$ (Sabemos que $(i,j,k)=(i,k)(i,j)$).
\end{obs}

\begin{prop}
Los $3$-ciclos generan $A_n$ (para $n\geq 3$).
\begin{proof}
Sabemos que $S_n = \langle (1,k) : k\in\NN\rangle$. Entonces, cualquier $\sigma\in A_n$ se puede escribir como $\sigma = (1,a_1)\cdots (1,a_m)$. Pero $m$ debe ser par, pues $\sigma$ debe ser una permutación par. Entonces tenemos $\sigma = ((1,a_1)(1,a_2))\cdots ((1,a_{2\ell-1})(1,a_{2\ell})) = (1,a_2,a_1)\cdots (1,a_{2\ell},a_{2\ell-1})$ y así cualquier elemento de $A_n$ se puede escribir como producto de $3$-ciclos. Trivialmente los $3$-ciclos son permutaciones pares y así están en $A_n$. Y listo.
\end{proof}
\end{prop}

\begin{prop}
Para $n\geq 3$ tenemos que $[S_n,S_n]=A_n$
\begin{proof}
($\subseteq$) Es trivial pues $\sg(\sigma\mu\sigma^{-1}\mu^{-1}) = \sg(\sigma)\sg(\mu)\sg(\sigma)\sg(\mu) = 1$.

($\supseteq$) Como los $3$-ciclos generan a $A_n$ basta escribir a los $3$-ciclos como productos de conmutadores. Pero $(i,j,k) = (i,j)(i,k)(i,j)(i,k) = [(i,j),(i,k)]$. Y estamos.
\end{proof}
\end{prop}

\section{Acciones de Grupos y la Ecuación de Clases}

\begin{defn}
Sea $G$ grupo y $X\neq \emptyset$ un conjunto. Una \textbf{acción a izquierda} de $G$ en $X$ es una función $\mu:G\times X\to X$, $(g,x)\mapsto g\cdot x$ que cumple: \begin{enumerate}\item $1\cdot x = x \text{ }\forall x\in X$ \item $(a\cdot a')\cdot x = a\cdot (a'\cdot x) \text{ } \forall a,a'\in G, x\in X$\end{enumerate}

Cuando $G$ actúa a izquierda en $X$ lo notamos $G\acts X$.
\end{defn}

\begin{ex}
$G=S_n$, $X=\{1,2,\ldots , n\}$. $G\acts X$ vía $\mu: S_n\times X\to X$, $\sigma\cdot x = \sigma(x)$. Es decir, las permutaciones actúan en $\{1,2,\ldots , n\}$ a través de evaluar a una permutación en $x\in \{1,2,\ldots , n\}$.
\end{ex}
\begin{ex}
$H\subseteq G$ subgrupo, $H\acts G$ vía $\mu :H\times G\to G$, $\mu(h,g) = h\cdot g$.
\end{ex}
\begin{ex}
$G$ grupo, $H\subseteq G$ subgrupo, $X=\{aH:a\in G\}$, es decir, el conjunto de las coclases de $H$ a izquierda (como $H$ no es normal esto no tiene por qué ser un grupo). $G\acts X$ vía $g\cdot (aH) = (ga)H$.
\end{ex}
\begin{ex}
$G$ grupo. $G$ actúa en sí mismo por conjugación $G\acts G$. Es decir, $g*h = ghg^{-1}$.
\end{ex}
\begin{obs}
Sea $\mu:G\times X\to X$ una acción. Para cada $g\in G$ se tiene una función $\ell_g:X\to X$ definida por $\ell_g(x)=g\cdot x$. Esto es una biyección, con inversa $\ell_{g^{-1}}$. Se tiene entonces $\ell^{\mu}:G\to S(X)$, $g\to \ell_g$. Notemos que es un morfismo de grupos pues $\ell^\mu (g\cdot g') (x) = \ell_{gg'}(x) = (gg')x = g(g'x) = \ell_g\circ \ell_{g'}(x) = \ell^\mu (g)\circ \ell^\mu (g')(x)$ para todo $x\in X$. Esto implica que $\ell^\mu (gg') = \ell^\mu (g)\circ \ell^\mu (g')$, y es morfismo de grupos.

Pero de hecho vale algo más interesante: dado un morfismo $\varphi:G\to S(X)$ determina unívocamente una acción a izquierda de $G$ en $X$ dada por $\mu_{\varphi}:G\times X\to X$, $\mu_{\varphi}(g,x) = \varphi(g)(x)$. La siguiente proposición prueba bien este hecho.
\end{obs}
\begin{prop}
Hay una correspondencia uno a uno entre las acciones a izquierda de $G$ en $X$ y los morfismos de grupos de $G$ en $S(X)$.
\begin{proof}
Después de la observación anterior, es claro que los candidatos a biyección que tenemos son $\mu \mapsto \ell^\mu$ y $\varphi\mapsto \mu_\varphi$. Probemos que son inversas una de otra y así la proposición seguirá.

En efecto, $\varphi(g)(x) = \mu_\varphi(g,x) = \ell_{g}(x) = \ell^{\mu_{\varphi}}(g)(x)$, para cualquier $x\in X, g\in G$. Entonces $\varphi = \ell^{\mu_\varphi}$.

Además, $\mu_{\ell^\mu}(g,x) = \ell^\mu (g)(x) = \ell_g (x) = \mu(g,x)$, para cualquier $x\in X, g\in G$ y así $\mu = \mu_{\ell^\mu}$. Entonces son inversas como queríamos.

\end{proof}
\end{prop}

\begin{defn}
Una acción $G\acts X$ se dice \textbf{trivial} si $g\cdot x = x$ para todos $x\in X, g\in G$. Esto es equivalente a que $\ell^\mu : G\to S(X)$ sea el morfismo trivial.
\end{defn}
\begin{defn}
Una acción se dice \textbf{fiel} si cumple que dado $g\in G$ tal que $g\cdot x = x$ para todo $x\in X$ entonces $g=1$. Es decir, el $1$ es el único elemento de $G$ que fija a todos los elementos de $X$. Es equivalente a pedir que $\ell^\mu :G\to S(X)$ tenga núcleo trivial, es decir, sea monomorfismo.
\end{defn}

Todo esto lo desarrollamos para acciones a izquierda. Podemos definir de forma análoga acciones a derecha.
\begin{defn}
Sea $G$ grupo y $X\neq \emptyset$ un conjunto. Una \textbf{acción a derecha} de $G$ en $X$ es una función $\mu:X\times G\to X$, $(x,g)\mapsto x\cdot g$ que cumple: \begin{enumerate}\item $x\cdot 1 = x \text{ }\forall x\in X$ \item $(x\cdot a)\cdot a' = x\cdot (a\cdot a') \text{ } \forall a,a'\in G, x\in X$\end{enumerate}

Cuando $G$ actúa a derecha en $X$ lo notamos $G\stca X$.
\end{defn}
\begin{obs}
Existe una correspondencia entre acciones a izquierda y a derecha de $G$ en $X$. Si tengo una acción $\mu:G\times X\to X$ a izquierda entonces defino $\tilde{\mu}:X\times G\to X$ como $\tilde{\mu}(x,g) = \mu (g^{-1},x)$. Del mismo modo, si tengo una acción a derecha puedo conseguir una acción a izquierda.
\end{obs}
De ahora en más trabajaremos con acciones a izquierda.
\begin{defn}
Sea $G\acts X$ una acción a izquierda de $G$ en $X$. Dado $x\in X$ se define la \textbf{órbita} de $x$ como el subconjunto $$\mathcal{O}_x=\{g\cdot x : g\in G\}\subseteq X$$ Denotamos por $G\backslash X$ al conjunto de órbitas de la acción a izquierda de $G$ en $X$.
\end{defn}
\begin{obs}
De forma similar a las coclases, si dos órbitas se intersecan, deben ser iguales. En efecto, supongamos que $\mathcal{O}_x\cap \mathcal{O}_y \neq \emptyset$, entonces existe un $z\in \mathcal{O}_x\cap\mathcal{O}_y$ y así $gx=z=hy$ para ciertos $g,h\in G$. Esto implica que $x=(g^{-1}h)y$ e $y=(gh^{-1})x$. Entonces $\mathcal{O}_x = \mathcal{O}_y$.

Esto implica que podemos escribir a $X$ como la unión disjunta sobre las órbitas, $X=\coprod{\mathcal{O}_x}$. Esto define una relación de equivalencia sobre $X$: $x\sim y \Longleftrightarrow \mathcal{O}_x = \mathcal{O}_y$.

\end{obs}

\begin{ex}
Sea $G$ grupo y $H\subseteq G$ subgrupo. Entonces $H\acts G$ de la forma obvia. $\mathcal{O}_a = \{ha : h\in H\} = Ha$, es decir, la coclase a derecha de $a$ en $H$. Entonces $H\backslash G$ es el conjunto de coclases a derecha.
\end{ex}
\begin{ex}
Sea $G$ grupo y $G\acts G$ por conjugación. Entonces $\mathcal{O}_a = \{gag^{-1} : g\in G\}$. Es decir, es el conjunto de conjugados de $a$.
\end{ex}

\begin{defn}
La acción de $G$ en $X$ se dice \textbf{transitiva} si existe una única órbita. Es decir, $\forall x,y\in X \text{ } \exists g\in G$ tal que $x=gy$.
\end{defn}

\begin{ex}
$G=S_n$, $X=\{1,2,\ldots , n\}$. Entonces $G\acts X$ actúa transitivamente pues dados $x,y\in X$ existe $\sigma\in S_n$ tal que $\sigma\cdot x = y$, es decir, $\sigma(x)=y$. (O sea, existe una permutación que cambie a $x$ con $y$ para cualquier $x$ e $y$).
\end{ex}

\begin{obs}
$G\acts X$. Puede pasar que $\sharp\mathcal{O}_x \neq \sharp \mathcal{O}_y$. Por ejemplo, con $G=G_2=\{-1,1\}$ actuando sobre $\{a,b,c\}$ como $1\cdot a = a$, $1\cdot b = b$, $1\cdot c = c$, $(-1)\cdot a = b$, $(-1)\cdot b = a$ y $(-1)\cdot c = c$ (se verifica trivialmente que es una acción) tenemos que $\mathcal{O}_a = \{a,b\}$ y $\mathcal{O}_c = \{c\}$.
\end{obs}

\begin{defn}
Tenemos una acción a izquierda $G\acts X$ y $x\in X$. Definimos el \textbf{grupo de isotropía} o \textbf{estabilizador} de $x$ como $$G_x = \{g\in G : gx=x\}\subseteq X$$ Es trivial verificar que $G_x$ es un subgrupo de $G$.
\end{defn}

\begin{prop}
Sea $G$ un grupo, $G\acts X$ una acción y $x\in X$. Entonces $(G:G_x) = \sharp \mathcal{O}_x$.
\begin{proof}
Sea $G/G_x$ el conjunto de las coclases a izquierda de $G_x$ (no tiene por qué ser un grupo pues $G_x$ no necesariamente es un subgrupo normal de $G$). Vamos a construir una biyección $\varphi :G/G_x \to \mathcal{O}_x$ definida por $\varphi(aG_x) = a\cdot x$. Veamos que está bien definida. Notemos que $aG_x = bG_x \Longleftrightarrow a^{-1}b\in G_x\Longleftrightarrow (a^{-1}b)x=x \Longleftrightarrow ax = bx$. Siguiendo las implicaciones para el otro lado obtenemos que $\varphi$ es inyectiva. Pero también es sobreyectiva pues dado $ax\in\mathcal{O}_x$ tenemos $\varphi (aG_x) = ax$. Entonces es una biyección y la proposición sigue.
\end{proof}
\end{prop}
\begin{cor}
Si el grupo $G$ es finito, entonces para todo $x\in X$ tenemos que $\sharp\mathcal{O}_x\mid |G|$.
\begin{proof}
Usamos el Teorema de Lagrange y la proposición anterior.
\end{proof}
\end{cor}
\begin{prop}
Supongamos que $G\acts X$, $x\in X, g\in G$. Entonces $G_{gx} = g G_x g^{-1}$.
\begin{proof}
Vamos a probar la doble inclusión:

($\subseteq$) Notemos que si $h\in G_{gx}$ entonces $h(gx) = gx$, lo que implica que $(g^{-1}hg) x = x$. Pero esto quiere decir que $g^{-1}hg\in G_x$. Y así $h\in gG_xg^{-1}$.

($\supseteq$) Ahora, sea $h\in g G_x g^{-1}$. Esto quiere decir que $g^{-1}hg\in G_x$. O sea, $(g^{-1}hg)x = x$ y así $h(gx) = gx$. Entonces $h\in G_{gx}$. Y la proposición sigue.

\end{proof}
\end{prop}

\begin{obs}
Supongamos que $G\acts X$ y consideremos $$\displaystyle\bigcap_{x\in X}G_x = \{g\in G : gx = x \forall x \in X\} = \ker \ell^{\mu}$$ Esto prueba que $\displaystyle\bigcap_{x\in X} G_x \triangleleft G$.

Pero qué pasa cuándo analizamos esto para $G\acts G$ por conjugación. Para un $a\in G$, $G_a = \{g\in G : gag^{-1}=a\} = \{g\in G: ga=ag\}$. Pero esto por definición es el \textbf{centralizador} de $a$, $Z(a)$.
Esto implica entonces que $\displaystyle\bigcap_{a\in G} Z(a) = Z(G)$. Esto da otra demostración de que el centro de $G$ es un subgrupo normal.
\end{obs}
Finalmente, estamos en condiciones de dar la ecuación de clases:
\begin{teo}[Ecuación de Clases]
Dado un grupo finito $G$, existe un conjunto (eventualmente vacío) $\{x_1,\ldots , x_r\}\subseteq G$ tal que $$|G| = |Z(G)| + \displaystyle\sum_{i=1}^{r}(G:Z(x_i))$$ con $(G:Z(x_i))\geq 2$ para todo $1\leq i\leq r$, o equivalentemente, $x_i\notin Z(G) \text{ } \forall 1\leq i\leq r$.
\begin{proof}
Sabemos que podemos escribir a $X$ como unión disjunta de órbitas, y como es finito, va a ser una cantidad finita de órbitas las que unamos. Entonces $X = \displaystyle\bigcup_{i=1}^{k} \mathcal{O}_{x_i}$ y así $|X| = \displaystyle\sum_{i=1}^{k}\sharp \mathcal{O}_{x_i} = \displaystyle\sum_{i=1}^{k} (G:G_{x_i})$. Pero por la observación anterior, en el caso que $G$ actúa sobre sí mismo por conjugación esto se escribe como $|G| = \displaystyle\sum_{i=1}^{k}(G:Z(x_i))$. Pero $(G:Z(g))=1$ si y sólo si $g=Z(G)$. Además, si $x\in Z(G)$ tenemos que $\mathcal{O}_x=\{x\}$ y así $x=x_i$ para algún $i$. Separando entonces los que tienen índice $1$ obtenemos $$|G| = |Z(G)| + \displaystyle\sum_{i=1}^{r}(G:Z(x_i))$$ con $(G:Z(x_i))\geq 2 \text{ } \forall 1\leq i\leq r$, como queríamos probar.
\end{proof}
\end{teo}

Ahora estamos en condiciones de probar el Teorema de Cauchy:
\begin{teo}[Cauchy]
Sea $G$ un grupo finito y $p$ un primo tal que $p\mid |G|$. Entonces, existe un $x\in G$ tal que $\ord (x)= p$.
\begin{proof}
Vamos a separar en dos casos, según $G$ es abeliano o no. En ambos procederemos por inducción sobre $|G|$.

\textbf{Caso A:} $G$ es abeliano. Si $|G|=1$ no hay nada que probar. Sea $|G|>1$. Tomemos un $x\in G$ con $\ord (x) > 1$. Si $p\mid \ord (x)$ entonces $\ord (x) = px$ y así tenemos que $(x^s)^p = 1$ y además como $x^s \neq 1$ pues $x$ tendría orden $s$, debemos tener que $\ord (x^s)=p$. Ahora, si $p\nmid \ord (x)$ entonces consideramos $\langle x\rangle \subseteq G$. Este subgrupo es trivialmente normal por ser $G$ abeliano. Consideremos entonces $G/\langle x\rangle$. Es abeliano, pues $G$ lo es, y tenemos que $|G/\langle x\rangle|<|G|$. Pero por el Teorema de Lagrange, tenemos que $|G| = |G/\langle x\rangle | |\langle x\rangle |$, y como $p\mid |G|$ pero $p\nmid |\langle x\rangle |$ debemos tener que $p\mid |G/\langle x\rangle |$. Por hipótesis inductiva existe un $\overline{z}\in G/\langle x\rangle$ tal que $\ord (\overline{z}) = p$. Esto quiere decir que $\overline{z}^p = \overline{1}$. Es decir, $z^p \in \langle x \rangle$. Sea $m$ el orden de $z^p$. Trivialmente, $m\mid \langle x\rangle$, lo que implica que $m$ es coprimo con $p$. Tenemos entonces que $(z^m)^p = 1$. Si fuera $z^m = 1$ entonces mirando la proyección a $G/\langle x\rangle$ obtenemos que $\overline{z}^m = \overline{1}$. Entonces $p\mid m$. Absurdo. Entonces $z^m$ tiene orden $p$ en $G$ y la inducción está completa.

\textbf{Caso B:} $G$ es no abeliano. Si $|G|=1$, nuevamente, no hay nada que probar. Sea $|G|>1$. Si $p\mid |Z(G)|$ como $Z(G)$ es un subgrupo abeliano, existe un $x\in Z(G)\subseteq G$ de orden $p$. Ahora, supongamos que $p\nmid |Z(G)|$. Por la ecuación de clases, $|G| = |Z(G)| + \displaystyle\sum_{i=1}^{r}(G:Z(x_i))$. Pero $p\mid |G|$ y $p\nmid |Z(G)|$. Esto implica que existe algún $1\leq k\leq r$ tal que $p\nmid |Z(x_k)|$ pues si no, no dividiría al lado derecho pero sí al izquierdo. Pero como $(G:Z(x_i))\geq 2 \text{ }\forall 1\leq i\leq r$, tenemos que $|Z(x_i)| < |G|$. Por Lagrange, $|G| = |Z(x_k)| (G:Z(x_k))$, y como $p\mid |G|$ y $p\nmid (G:Z(x_k))$ entonces $p\mid |Z(x_k)|$. Pero aplicamos hipótesis inductiva, y tenemos un $x\in Z(x_k)$ con $\ord (x) = p$. Y estamos.

\end{proof}
\end{teo}

\begin{prop}
Sea $G$ un grupo con $|G|=2p$, $p$ primo impar. Entonces $G\simeq \ZZ_{2p}$ o $G\simeq D_p$.
\begin{proof}
Por el Teorema de Cauchy, existe $r\in G$ con $\ord (r) = p$ y $s\in G$ con $\ord (s) = 2$. Sea $H=\langle r\rangle$. Notemos que como $(G:H)=2$, tenemos que $H\triangleleft G$. Notemos que $s\notin H$ pues $2\nmid p = |H|$. Como hay exactamente dos coclases, estas deben ser $H$ y $Hs$. Entonces $G = H\coprod H_s = \{1,r,\ldots , r^{p-1},s,rs,\ldots , r^{p-1}s\}$.

Ahora, notemos que $srs = r^i$ para algún $0 \leq i\leq p-1$ pues $srs=srs^{-1}$ y $H=\langle r\rangle \triangleleft G$. Entonces, tenemos que $$ r = s^2 r s^2 = s(srs)s = sr^i s = (srs)^i = (r^i)^i = r^{i^2}$$Esto implica que $r^{i^2 -1} = 1$ y así $p\mid i^2 - 1$. Por lo tanto $i=1$ o $i=p-1$.

Si $i=1$ obtenemos $rs=sr$ y así el grupo es abeliano y $\ord(rs)=2p$. Entonces $G$ es cíclico y $G=\langle rs\rangle$. Esto implica $G\simeq \ZZ_{2p}$.

Si $i=p-1$ obtenemos $rs=sr^{p-1}$ y tienen la misma estructura que el diedral. Entonces $G\simeq D_p$. La proposición sigue.

\end{proof}
\end{prop}

Finalmente, veamos un aspecto más combinatorial de las acciones:

\begin{lem}[Lema de Burnside]
Sean $X$ un conjunto y $G$ un grupo finito tal que $G\acts X$. Entonces, si $X/G$ son las órbitas de X bajo la acción de $G$, $$ |X/G| = \dfrac{1}{|G|}\sum_{g\in G} |X^g|$$
\begin{proof}
El argumento va a consistir en contar de dos formas distintas la cantidad de elementos de un conjunto. Notemos que $$\displaystyle\sum_{g\in G} |X^g| = \sharp \{(x,g)\in X\times G : gx=x\} = \displaystyle\sum_{x\in X}|G_x|$$ Pero, usando que $(G:G_x)=\sharp \mathcal{O}_x$ y escribiendo a $X$ como unión de sus órbitas, $$\displaystyle\sum_{x\in X}|G_x| = \displaystyle\sum_{x\in X}\dfrac{|G|}{\sharp \mathcal{O}_x} = |G| \displaystyle\sum_{x\in X}\dfrac{1}{\sharp\mathcal{O}_x} = |G|\displaystyle\sum_{A\in X/G}\sum_{x\in A}\dfrac{1}{\sharp\mathcal{O}_x} = \displaystyle\sum_{A\in X/G} 1 = |X/G|$$ Y la conclusión sigue.
\end{proof}
\end{lem}

\begin{ex}
Se tienen perlas de $k$ colores. ¿Cuántos collares se pueden hacer si se considera que dos collares rotados son los mismos?
\begin{proof}[Solución]
Primero consideremos sin mirar las rotaciones. Un collar es una función $f:I_n\to I_k$ (pues $f(i)$ nos dice el color de la $i$-ésima perla). Sea $X=\{f:I_n\to I_k\}$. Consideremos la acción $\ZZ_n\acts X$ vía $(j\cdot f)(i) = f(i+j)$ (con los índices vistos módulo $n$). Como dos collares son iguales por rotación, queremos contar cuántas órbitas tiene esta acción, y eso será la cantidad de collares que tenemos. Notemos además que $X^j$, es decir, los elementos que quedan fijos en la acción por $j$ deben cumplir que $f(i+j) = f(i)$, y aplicando la acción sucesivamente, $f(i')=f(i)$ si y sólo si $j\mid i'-i$. Entonces, es claro que $|X^j| = \mcd (n:j)$.

Por el lema de Burnside, $\sharp \text{collares} = |X/\ZZ_n| = \dfrac{1}{n}\displaystyle\sum_{j=0}^{n-1} k^{\mcd(n:j)}$. En particular, si $n=p$ es primo, tenemos que $\sharp \text{collares} = \dfrac{1}{p} (k^p + (p-1)k)$. Es decir, el Pequeño Teorema de Fermat.
\end{proof}
\end{ex}
\begin{obs}
Si en el ejemplo anterior, dos collares simétricos se consideran iguales, podemos hacer actuar al grupo diedral $D_n$ en vez del grupo cíclico $\ZZ_n$, y obtenemos un resultado similar.
\end{obs}
\section{Teoremas de Sylow}

Antes de comenzar con las definiciones y poder encaminarnos a probar los Teoremas de Sylow, necesitamos del Teorema de Correspondencia.

\begin{teo}
Sea $\varphi :G\to \overline{G}$ un epimorfismo de grupos. Entonces existe una correspondencia unívoca entre los subgrupos de $G$ que contienen a $\ker \varphi$ y los subgrupos de $\overline{G}$. Esta correspondencia está dada por $H\mapsto \varphi(H)=\overline{H}$ y $\overline{H}\mapsto \varphi^{-1}(\overline{H})=H$. Además, $H\subseteq H'$ si y sólo si $\overline{H}\subseteq \overline{H'}$. Los índices se preservan $(H':H) = (\overline{H'}:\overline{H})$. También, $H\triangleleft G$ si y sólo si $\overline{H}\triangleleft\overline{G}$. Finalmente $G/H \simeq \overline{G}/\overline{H}$.
\begin{proof}

Notemos que si $H$ y $H'$ contienen a $\ker \varphi$, entonces $H\subseteq H'$ si y sólo si $\varphi(H)\subseteq \varphi(H')$. Una implicación es trivial. Ahora, si $\varphi (H)\subseteq \varphi(H')$, supongamos que existe un $x\in H$ tal que $x\notin H'$. Como $\varphi (x)\in \varphi(H)\subseteq \varphi (H')$, existe un $\ell\in H'$ tal que $\varphi (x) = \varphi(\ell)$. Entonces $\varphi(x\ell^{-1})=1$ y así $x\ell^{-1}\in \ker\varphi \subseteq H'$. Es decir, existe un $h'\in H'$ tal que $x\ell^{-1}=h'$ y así $x= h' \ell\in H'$. Absurdo.

Esto nos prueba que $\varphi(H)=\varphi(H')$ si y sólo si $H=H'$ (la doble inclusión). Como $\varphi$ es epimorfismo, es claro que la correspondencia entre los subgrupos dada por $H\mapsto\varphi(H)=\overline{H}$ y $\overline{H}\mapsto \varphi^{-1}(\overline{H}) = H$ es biyectiva.

Ahora, notemos que $(G:H)$ es la cantidad de coclases a izquierda de $H$ en $G$. Veamos que la cantidad de coclases a izquierda de $\varphi (H)$ en $\varphi (G)$ es la misma. En efecto, si $aH$ es una coclase de $H$ en $G$, entonces $\varphi(a)\varphi(H)$ es una coclase de $\varphi(G)$ en $\varphi (H)$. Notemos que si $a$ y $b$ son representantes de la misma coclase, $a^{-1}b\in H$ y así $\varphi(a^{-1}b)\in \varphi(H)$. Esto implica que $\varphi(a)$ y $\varphi(b)$ son representantes de la misma coclase. Análogamente, si $\varphi(a)$ y $\varphi(b)$ son representantes de la misma coclase, entonces $\varphi(a^{-1}b)\in \varphi(H)$ y así $a^{-1}b\in H$, lo que implica que $a$ y $b$ son representantes de la misma coclase. Entonces hay una biyección entre las coclases y listo.

Notemos que $H\triangleleft G$ si y sólo si $\varphi (H)\triangleleft \varphi (G)$. En efecto, si $H\triangleleft G$, tenemos que $gHg^{-1} = H$ para todo $g\in G$. Ahora, tomemos $x\in \varphi(G)$. Por definición, debe existir $g\in G$ tal que $x=\varphi(g)$. Entonces $x\varphi(H)x^{-1} = \varphi (g)\varphi(H) \varphi(g^{-1}) = \varphi(gHg^{-1}) = \varphi(H)$. Esto quiere decir que $\varphi (H) \triangleleft \varphi (G)$. De forma análoga, si $\varphi(H)\triangleleft \varphi(G)$, tenemos que $\varphi(g)\varphi(H)\varphi(g^{-1}) = \varphi(H)$ para todo $g\in G$. Es decir $\varphi (gHg^{-1})=\varphi(H)$. Pero habíamos probado que $\varphi(H)=\varphi(H')$ sii $H=H'$. Entonces $gHg^{-1}=H$ para todo $g\in G$ y así $H\triangleleft G$.

Finalmente, $G/H \simeq \varphi(G)/\varphi(H)$ pues son los conjuntos de coclases a izquierda y ya vimos que están en biyección. Más aún, esta biyección está dada por $\gamma:G/H\to \varphi(G)/\varphi(H)$, $aH\mapsto \varphi(a)\varphi(H)$. Es claro que $\gamma$ es un morfismo de grupos pues $$\gamma(abH) = \varphi(ab)\varphi(H) = \varphi(a)\varphi(b)\varphi(H) = \varphi(a)\varphi(H)\varphi(a^{-1})\varphi(a)\varphi(b)\varphi(H)= \varphi(H)\gamma(aH)\gamma(bH)$$ Esto quiere decir que $\gamma(abH)=\gamma(aH)\gamma(bH)$ y así es un isomorfismo. Y el teorema sigue.

\end{proof}
\end{teo}

Ahora comencemos con las definiciones de $p$-grupos.

\begin{defn}
Sea $p\in\NN$ primo. Un grupo $G$ se dice un $p$-grupo si todo elemento de $G$ tiene orden una potencia de $p$.
\end{defn}
\begin{prop}
Si $G$ es finito entonces $G$ es un $p$-grupo si y sólo si $|G|=p^r$ para algún $r$.
\begin{proof}
($\Longrightarrow$) Si $q\mid |G|$ para algún $q$ primo, por Cauchy, hay un elemento de orden $q$ y así $q=p$. Entonces $|G|=p^r$.

($\Longleftarrow$) Por Lagrange, $\ord (x) \mid |G|\text{ }\forall x\in G$. Entonces $\ord (x)\mid p^r$ y así $\ord (x) = p^{a_x}$.
\end{proof}
\end{prop}

\begin{prop}
Si $G$ es un $p$-grupo finito no trivial entonces $Z(G)\neq 1$.
\begin{proof}
Por la ecuación de clases tenemos que $|G| = |Z(G)| + \displaystyle\sum_{i=1}^{r}(G:Z(x_i))$ con $(G:Z(x_i))\geq 2$.

Por Lagrange, $p^r = |G| = |Z(x_i)|(G:Z(x_i))$. Entonces $|Z(x_i)|$ y $(G:Z(x_i))$ son ambos potencias de $p$. Además, como $(G:Z(x_i))\geq 2$ son potencias con exponente no trivial. Como $p\mid |G|$ y $p\mid (G:Z(x_i))$, mirando la ecuación de clases módulo $p$ debemos tener que $p\mid |Z(G)|$. Pero como $|Z(G)|\geq 1$ tenemos $|Z(G)|\geq p$. Y la proposición sigue.
\end{proof}
\end{prop}

\begin{prop}
Si $|G|=p^n$ con $p$ primo, entonces para todo $m\leq n$ existe $H\triangleleft G$ con $|H| = p^m$.
\begin{proof}
Procederemos por inducción en $n$. Para $n=1$ es trivial. Consideremos $Z(G)\subseteq G$. Como $|Z(G)|\mid |G|$ y $|Z(G)|\neq 1$, debemos tener que $|Z(G)| = p^r$. Por el teorema de Cauchy, existe un $x\in Z(G)$ de orden $p$. Como $x\in Z(G)$, tenemos que $\langle x\rangle \triangleleft G$. Considero $G/\langle x\rangle$. Es claro que $|G/\langle x\rangle| = p^{n-1}$. Por hipótesis inductiva, para todo $1\leq m\leq n-1$ existe un $\overline{H}\triangleleft G/\langle x\rangle$ tal que $|\overline{H}|=p^m$. Pero por el Teorema de Correspondencia, tenemos que existe $H\triangleleft G$ tal que $|H| = p^{m+1}$ para todo $2\leq m+1\leq n$ (pues $(G:H)=(\overline{G}:\overline{H})$). Pero agregamos un subgrupo dado por Cauchy y cubrimos $|H|=p^1$. La proposición sigue.
\end{proof}
\end{prop}

\begin{defn}
Sea $p$ primo y $G$ un grupo con $|G|=p^r m$ (con $\mcd (p:m)=1$). Un subgrupo $H\subseteq G$ se dice que es un $p$-subgrupo de Sylow o $p$-Sylow si $|H|=p^r$.	
\end{defn}

\begin{defn}
Supongamos que $G\acts X$. Entonces definimos los $G$-invariantes por $X$ como $X^G = \{x\in X : gx = x \text{ }\forall g\in G\}$.
\end{defn}

\begin{lem}
Sea $H$ un $p$-grupo finito y $H\acts X$ con $\sharp X < +\infty$. Entonces $\sharp X\equiv \sharp X^H \pmod{p}$.
\begin{proof}
Partimos a $X$ como unión disjunta de las órbitas. Separemos además las órbitas unipuntuales del resto. Esto se traduce en $X = X^H \coprod \mathcal{O}_{x_i}$, con $\sharp \mathcal{O}_{x_i} \geq 2 \text{ }\forall i$. Entonces $\sharp X = \sharp X^H + \displaystyle\sum_{i=1}^{k}\sharp \mathcal{O}_{x_i}$. Pero como $\sharp \mathcal{O}_{x_i}\mid |H| = p^r$ y $\sharp \mathcal{O}_{x_i}\geq 2$, tenemos que $p\mid \sharp \mathcal{O}_{x_i}$. Esto implica que $\sharp X \equiv \sharp X^H \pmod{p}$. Y estamos.
\end{proof}
\end{lem}
\begin{teo}[Primer Teorema de Sylow]
Sea $G$ grupo finito con $|G|=p^r m$ (con $\mcd (p:m)=1$). Entonces existe $H\subseteq G$ subgrupo con $|H|=p^r$. Es decir, existe un $p$-subgrupo de Sylow.
\begin{proof}
Sea $X=\{A\subseteq G \text{ subconjuntos tales que } \sharp A = p^r\}$. Es claro que $G\acts X$ vía $g\cdot A = \{ga : a\in A\}\subseteq G$. Además, es claro que $\sharp g\cdot A = p^r$ pues multiplicar por $g$ es una biyección en $G$.

Sea $A\in X$. Consideremos el estabilizador $G_A = \{g\in G : gA = A\}$. Afirmo que $|G_A| \leq \sharp A = p^r$. En efecto, fijo $a_0\in A$ y defino $\gamma :G_A\to A$ por $g\mapsto g a_0 \in A$. Es claro que $\gamma$ es inyectiva pues si $ga_0 = ha_0$, por la propiedad cancelativa, $g=h$. Como tengo una función inyectiva de $G_A$ a $A$, debemos tener $\sharp G_A\leq \sharp A$.

Pero por otra parte, si $p\mid \sharp\mathcal{O}_A$ para todo $A\in X$, entonces $p\mid \displaystyle\sum \sharp \mathcal{O}_{A_i} = \sharp X$. Pero $\sharp X$ es la cantidad de formas de tomar $p^r$ elementos entre $p^r m$ y así $\sharp X = \displaystyle\binom{p^r m}{p^r}$. Pero $p\nmid\displaystyle\binom{p^r m}{p^r}$. Esto se ve pues $\nu_p(n!) = \displaystyle\sum_{i=1}^{\infty} \left\lfloor\dfrac{n}{p^i}\right\rfloor$ es la mayor potencia de $p$ que divide a $n!$ y calculamos eso para el coeficiente binomial. Esto nos lleva a concluir que $p\nmid \sharp\mathcal{O}_A$ para algún $A\in X$. Pero por Lagrange, $|G| = |G_A| (G:G_A) = |G_A| \sharp\mathcal{O}_A$. Como $|G|=p^r m$ y $\sharp\mathcal{O}_A$ no es divisible por $p$, todos los factores de $p$ deben estar en $|G_A|$ y así $|G_A|\geq p^r$. Pero habíamos probado que $|G_A| \leq p^r$. Entonces debemos tener para este $A$ que $|G_A|=p^r$ y el teorema sigue.

\end{proof}
\end{teo}

\begin{obs}
Sea $H\subseteq G$ un subgrupo. Entonces $(G:N(H)) = \sharp \{gHg^{-1} : g\in G\}$. En efecto, $G\acts X$ con $X=\{A\subseteq G : A \text{ es un subgrupo de } G\}$ por conjugación. Entonces $N(H) = G_H$, y esto quiere decir que $(G:N(H))=(G:G_H)=\sharp \mathcal{O}_H = \sharp \{gHg^{-1} : g\in G\}$.
\end{obs}

\begin{lem}
Sea $G$ grupo con $|G| = p^r m $ con $p$ primo y $\mcd (p:m) = 1$. Sea $P\subseteq G$ un $p$-Sylow y $H\subseteq G$ un $p$-subgrupo. Si $H\subseteq N(P)$ entonces $H\subseteq P$. En particular, esto implica que si $P'$ es otro $p$-Sylow distinto de $P$, debemos tener que $P'\not\subseteq N(P)$.
\begin{proof}
Sabemos que $|P|=p^r$ y que $|H| = p^s$ con $s\leq r$. Como $H\subseteq N(P)$, tenemos que $HP$ es un subgrupo de $G$ y $P\triangleleft HP$ con $HP/P\simeq H/H\cap P$ por el Tercer Teorema de Isomorfismo. Entonces $(HP:P) = (H:H\cap P) \mid |H| = p^s$. Entonces, $(HP:P)=p^t$ para algún $t\leq s$. Pero $|HP| = |P|(HP:P) = p^r p^t = p^{r+t}$. Pero como $|HP|$ es un subgrupo de $G$, $|HP| \mid |G|$ y así la potencia máxima de $p$ que divide a $|HP|$ debe ser menor o igual que la potencia máxima de $p$ que divide a $|G|$. Es decir, $r+t\leq r$ y así $t=0$. Pero entonces $(HP:P)=1$ y así $H\subseteq HP = P$. Como queríamos probar.
\end{proof}
\end{lem}

\begin{obs}
Sea $G$ un grupo y $X$ un conjunto tal que $G\acts X$. Si $x\in X$ entonces puedo restringir la acción a la órbita de $x$: $G\acts \mathcal{O}_x$. A su vez, si $H$ es un subgrupo de $G$, entonces puedo restringirme a $H$ y $H\acts X$. Para dejar claro, escribimos $\mathcal{O}_x^G$ como la órbita de $x$ bajo la acción por $G$.
\end{obs}

\begin{teo}[Teoremas 2 y 3 de Sylow]
Sea $G$ un grupo con $|G|=p^r m$, con $p$ primo y $\mcd (p:m)=1$. Entonces \begin{enumerate}\item Si $P$ y $P'$ son $p$-Sylows entonces son conjugados \item Si $n_p = \sharp \{H\subseteq G : H \text{ }p\text{-subgrupo de Sylow}\}$, entonces $n_p\equiv 1\pmod{p}$ y $n_p\mid m$. \item Todo $p$-subgrupo de $G$ está incluído dentro de algún $p$-Sylow. \end{enumerate}
\begin{proof}

(\textit{1}): Sea $X=\{p\text{-subgrupos de Sylow de }G\}$. Por el Teorema 1 de Sylow, $X$ es no-vacío. Consideremos $G\acts X$ por conjugación. Sea $P\in X$. Veamos que $\mathcal{O}_P = X$, es decir, que la acción es transitiva. Restrinjamos la acción de $G\acts X$ a $P\acts \mathcal{O}_P$. Sea $Q \in \mathcal{O}_P$. Si $Q=P$, entonces $\mathcal{O}^P_P = \{aPa^{-1} : a\in P\} = \{P\}$. Es decir, $P$ tiene órbita unipuntual en la acción de $P$ sobre $\mathcal{O}_P$. Ahora, si $Q\neq P$, entonces $\mathcal{O}_Q^P \neq \{Q\}$. En efecto, si $\mathcal{O}_Q^P = \{Q\}$ entonces, para todo $a\in P$, $aQa^{-1} = Q$, lo que implica que $P\subseteq N(Q)$ y por el lema anterior $P\subseteq Q$. Absurdo, pues $Q\neq P$. Por otra parte, como $\sharp \mathcal{O}_Q^P \mid |P| = p^r$ y $\sharp \mathcal{O}_Q^P >1$, tenemos que $\sharp \mathcal{O}_Q^P \equiv 0 \pmod{p}$. Esto quiere decir que como $\mathcal{O}_P$ se puede escribir como la unión disjunta de las órbitas de la acción de $P$ sobre $\mathcal{O}_P$, debemos tener que $\sharp \mathcal{O}_P = \displaystyle\sum_{i=1}^{r} \sharp \mathcal{O}_{x_i}^{P}$. Pero, mirando módulo $p$, todas esas órbitas son $0$ salvo la de $P$, que es $1$. Entonces $\sharp  \mathcal{O}_P \equiv 1 \pmod{p}$.

Ahora, si $\mathcal{O}_P\neq X$ existiría un $P'\in X$ tal que $P'\in \mathcal{O}_P$. Consideremos $P'\acts \mathcal{O}_P$. Ahora, para cualquier $Q\in \mathcal{O}_P$, tenemos que $Q\neq P'$. Esto implica que ninguna órbita es unipuntual, pues si la órbita de $Q$ lo fuera entonces $Q\subseteq N(P')$ y por el lema es absurdo. Esto quiere decir que $p\mid \sharp\mathcal{O}_Q^{P'}$ para todo $Q\in\mathcal{O}_P$. Entonces, al partir a $\mathcal{O}_P$ como unión de las órbitas, tendríamos que $\sharp \mathcal{O}_P \equiv 0 \pmod{p}$. Absurdo.

(\textit{2}): Ya probamos que $\sharp \mathcal{O}_P\equiv 1\pmod{p}$. Sea ahora $n_p$ la cantidad de $p$-subgrupos de Sylow de $G$. Pero por (\textit{1}) sabemos que $n_p = \sharp X = \sharp \mathcal{O}_P \equiv 1\pmod{p}$. Además, tenemos $n_p = \sharp \mathcal{O}_P = (G:N(P)) = |G|/|N(P)| = |G|/(N(P):P)|P| = m/(N(P):P)$. Entonces $n_p (N(P):P) = n_p$ y así $n_p\mid m$.

(\textit{3}): Sea $H$ un $p$-subgrupo. Consideremos $H\acts X$ por conjugación. Como $H$ es un $p$-grupo y $X$ un conjunto, habíamos probado que $\sharp X \equiv \sharp X^H \pmod{p}$. Pero $\sharp X \equiv 1\pmod{p}$. Entonces, $\sharp X^H \equiv 1 \pmod{p}$ y en particular es no-vacío. Entonces existe un $P\in X$ tal que $P\in X^H$. O sea, $hPh^{-1} = P $ para todo $h\in H$ y así $H\subseteq N(P)$. Pero por el lema anterior, $H\subseteq P$. Y estamos.

\end{proof}
\end{teo}

\begin{obs}
Sea $G$ un grupo finito y $p\mid |G|$ con $p$ primo. Si $n_p=1$ entonces existe un único $p$-subgrupo de Sylow, $P$. Pero bajo conjugación obtenemos también subgrupos de Sylow. Entonces $gPg^{-1} = P$ para todo $g\in G$ y así $P\triangleleft G$.
\end{obs}

Veamos algunos ejemplos del uso de los Teoremas de Sylow. Pero antes, probemos estas proposiciones:

\begin{prop}
Sea $G$ un grupo tal que $G/Z(G)$ es cíclico. Entonces $G$ es abeliano.
\begin{proof}
Supongamos que $G$ no es abeliano. Entonces existen $g_1,g_2\in G$ tales que $g_1g_2 \neq g_2g_1$. Notamos por $\overline{g}$ a la clase de $g$ en $G/Z(G)$.

Como $G/Z(G)$ es cíclico, existe un $g\in G$ tal que $G/Z(G) = \left\langle \overline{g}\right\rangle$. Entonces, $\overline{g_1} = \overline{g}^{r_1} = \overline{g^{r_1}}$ y $\overline{g_2} = \overline{g}^{r_2} = \overline{g^{r_2}}$. Pero esto quiere decir que $g_1 = g^{r_1}z_1$ y $g_2 = g^{r_2}$ con $z_1,z_2\in Z(G)$. Entonces $g_1g_2 = g^{r_1}z_1 g^{r_2}z_2 = g^{r_1}z_1 g^{r_2}z_2 = g^{r_1}g^{r_2}z_1 z_2 = g^{r_2}g^{r_1}z_2z_1 = g^{r_2}z_2 g^{r_1}z_1 = g_2g_1$, donde podemos conmutar las cosas porque $z_1$ y $z_2$ están en el centro y $g^{r_1}$ y $g^{r_2}$ claramente conmutan. Esto es absurdo pues supusimos que $g_1$ y $g_2$ no conmutan. Entonces $G$ es abeliano y estamos.
\end{proof}
\end{prop}

\begin{prop}
Sea $G$ un grupo con $|G| = p^2$ donde $p$ es primo. Entonces $G$ es abeliano. Más aún, $G\simeq \mathbb{Z}_{p^2}$ o $G\simeq \mathbb{Z}_p\oplus \mathbb{Z}_p$.
\begin{proof}
Como $Z(G)$ es un subgrupo de $G$, por Lagrange, $|Z(G)|\mid |G|=p^2$. Pero como $G$ es un $p$-grupo sabemos que $Z(G)\neq 1$ (esto era por la ecuación de clases). Entonces $|Z(G)|=p$ ó $p^2$. Si $|Z(G)|=p^2$, entonces $G$ es abeliano por definición. Si $|Z(G)|=p$, entonces existe un $g\in G-Z(G)$ con $\ord (g) = p$, pues $\ord (g) \mid |G| = p^2$. Si $\ord(g) = p^2$ entonces $G$ es cíclico y como es finito debe ser isomorfo a $\ZZ_{p^2}$ y así abeliano. Y si $\ord(g)=1$ entonces $g=1$, pero esto es absurdo pues $Z(G)$ es un subgrupo propio de $G$. Veamos entonces que $G/Z(G)$ resultará cíclico, y así $G$ abeliano por la proposición anterior.

Como $g^p = 1$, tomando clase en $Z(G)$ tenemos que $\overline{g}^p = \overline{1}$ y así $\ord (\overline{g}) \mid p$. Pero si $\ord (\overline{g}) = 1$ entonces $\overline{g} =\overline{1}$ y así $g\in Z(G)$. Pero esto es absurdo por la construcción de $g$. Entonces $\ord (\overline{g})=p$ y como $|G/Z(G)| = |G|/|Z(G)| = p^2 / p = p$, tenemos que $G/Z(G) = \langle \overline{g} \rangle$. Entonces $G$ debe ser abeliano.

Ahora, si existe un $g\in G$ tal que $\ord (g)=p^2$ entonces $G$ es cíclico y $G\simeq \ZZ_{p^2}$. Si no, debe existir un $g\in G$ con $\ord (g)=p$ (pues si no sería el grupo trivial). Entonces $\langle g\rangle \triangleleft G$ por ser $G$ abeliano. Y sea ahora $h\in G-\langle g\rangle$. 

Es claro que $\langle g\rangle \cap \langle h\rangle = 1$ pues notemos que $h\in \langle g\rangle$ si y sólo si $h^m \in \langle g\rangle $ para todo $1\leq m\leq p-1$. En efecto, si $h\in \langle g\rangle$, por ser un subgrupo, $h^m \in \langle g\rangle$. Y si $h^m \in \langle g\rangle$, tomando $a$ el inverso modular de $m$ módulo $p$, tenemos que $h^{am} \in \langle g\rangle$, con $h^{am} = h$. 

Por lo tanto, $\langle g\rangle \langle h\rangle\subseteq G$ trivialmente, pero además, como tienen intersección trivial, hay una única forma de escribir un producto de elementos de $\langle g\rangle$ y $\langle h\rangle$. Entonces $|\langle g\rangle \langle h\rangle | = p^2$, y así $\langle g\rangle \langle h\rangle = G$.

Entonces, $G$ es el producto directo interno de $\langle g\rangle$ y $\langle h\rangle$, y así $G\simeq \langle g\rangle \times \langle h\rangle \simeq \ZZ_p\oplus\ZZ_p$. 

Y estamos.

\end{proof}
\end{prop}

Estamos en condiciones de ver la primera consecuencia de Sylow:

\begin{prop}
Sean $p,q$ primos con $p<q$ y $p\nmid q-1$. Si $|G|=pq$ entonces $G\simeq \ZZ_{pq}$.
\begin{proof}
Sabemos que $n_p\equiv 1\pmod{p}$ y $n_p\mid q$. Como $q$ es primo, debemos tener que $n_p = 1$ ó $q$. Si $n_p=q$ entonces $n_p \equiv 1 \pmod{p}$ es lo mismo que decir $p\mid q-1$. Absurdo por el enunciado. Entonces $n_p=1$ y así hay un único $p$-subgrupo de Sylow $P$, y debemos tener $P\triangleleft G$.

De forma similar, $n_q\equiv 1\pmod{q}$ y $n_q\mid p$. Si $n_q = p$ entonces $q\mid n_q - 1 = p-1$, pero $q>p$, lo que es absurdo. Entonces $n_q=1$ y nuevamente hay un único $q$-subgrupo de Sylow $Q$ y debemos tener $Q\triangleleft G$.

Es fácil ver que $PQ\subseteq G$. Además, como $P\cap Q=1$ pues si $x\in P\cap Q$, $\ord x \mid |P|=p$ y $\ord x \mid |Q|=q$. Como $P\cap Q=1$, no hay dos escrituras $p_1q_1 = p_2q_2$ de un mismo elemento. Entonces, debemos tener que $|PQ| = pq$, y por lo tanto $PQ=G$.

Esto quiere decir que $G$ es el producto directo interno de $P$ y $Q$ y esto implica que $G\simeq P\times Q \simeq \ZZ_p\oplus \ZZ_q \simeq \ZZ_{pq}$ pues $p$ y $q$ son coprimos.

\end{proof}
\end{prop}
Ahora, veamos una segunda aplicación:
\begin{obs}
Sea $G$ un grupo y $H\triangleleft G$ un subgrupo normal. Si $G/H$ es abeliano entonces $[G:G]\subseteq H$. En efecto, si $\pi:G\to G/H$ es la proyección canónica, entonces $\pi (xyx^{-1}y^{-1}) = \pi (x)\pi(y)\pi(x)^{-1}\pi(y)^{-1} = 1$ por ser $G/H$ abeliano. Entonces, todo elemento de $[G:G]$ está en $\ker \pi = H$. Y listo.
\end{obs}

\begin{prop}
Sea $G$ un grupo no abeliano con $|G|=p^3$. Entonces, $Z(G)=[G:G]\simeq \ZZ_p$ y $G/Z(G) \simeq \ZZ_p\oplus \ZZ_p$. Más aún, todos los subgrupos de orden $p^2$ contienen a $Z(G)$ y son normales. Finalmente, si $G$ tiene exponente $p$ entonces existe $H\triangleleft G$ con $H\simeq \ZZ_p\oplus\ZZ_p$.
\begin{proof}
Notemos que por Lagrange, $|Z(G)|\mid p^3$, y así $|Z(G)|=p$ ó $p^2$ pues no es $1$ por ser $G$ un $p$-grupo y no es $p^3$ al $G$ no ser abeliano. Si $|Z(G)| = p^2$ entonces $|G/Z(G)|=p$ y entonces $G/Z(G)$ es cíclico, lo que implica por una de las proposiciones anteriores que $G$ es abeliano. Entonces $|Z(G)|=p$ y entonces $Z(G)\simeq \ZZ_p$.

Notemos que $G/Z(G)\simeq \ZZ_p\oplus\ZZ_p$ pues tiene orden $p^2$ y no es cíclico (pues si no $G$ sería abeliano). Entonces, por la observación anterior, como $G/Z(G)$ es abeliano, esto quiere decir que $[G:G]\subseteq Z(G)$. Pero como $[G:G]$ es un subgrupo de $G$, tiene $1,p,p^2$ o $p^3$ elementos. Como $[G:G]\subseteq Z(G)$, $|[G:G]|\leq p$. Pero $|[G:G]|\neq 1$ pues $G$ no es abeliano. Entonces $[G:G] = p$ y así $[G:G]=Z(G)$.

Ahora, sea $H$ un subgrupo de orden $p^2$. Como $(G:H)=p$, que es el menor primo que divide a $|G|$, tenemos que $H\triangleleft G$. Entonces, $|G/H|=p$ y así $G/H\simeq \ZZ_p$. Pero como $\ZZ_p$ es abeliano, tenemos que $[G:G]\subseteq H$. Es decir, $Z(G)\subseteq H$.

Sabemos que $G/Z(G)\simeq \ZZ_p\oplus\ZZ_p$. Sea $\varphi:G/Z(G)\to \ZZ_p\oplus\ZZ_p$ un isomorfismo. Tomemos $H=\varphi^{-1}(\ZZ_p\oplus \{ 0\})$. Entonces, $H/Z(G) \simeq \ZZ_p\oplus \{0\}$ vía $\varphi$. Por lo tanto $|H|=p^2$. Pero $H$ no es cíclico, pues $|H|=p^2$ y los elementos de $G$ tienen exponente $p$. Esto quiere decir que $H\simeq \ZZ_p\oplus\ZZ_p$. Y la proposición sigue.

\end{proof}
\end{prop}

\section{Grupos Libres y Presentaciones}

El objetivo de esta sección va a ser generalizar el concepto de bases y generadores que tenemos presente en el Álgebra Lineal a la Teoría de Grupos.

\begin{obs}
Sea $V$ un $K$-espacio vectorial y $\beta\subseteq V$ un conjunto. Se dice que $\beta$ es una base si es un conjunto de generadores de $V$ linealmente independiente. ¿Pero cómo se puede escribir esto en términos de las relaciones entre los objetos? Del Álgebra Lineal sabemos que fijar el valor de una transformación lineal en una base, entonces ya queda fija en todo el espacio. Entonces, si tenemos $W$ un $K$-espacio vectorial y una función $\varphi:\beta \to W$ (que es definir a la transformación en la base), tenemos una única $\overline{\varphi}:V\to W$ transformación lineal tal que $\left.\overline{\varphi}\right |_{\beta} = \varphi$. Es decir, el siguiente diagrama conmuta:
\begin{center}
\begin{tikzcd}[row sep=3em,column sep=2.5em,minimum width=3em]
\beta \arrow[hookrightarrow]{d}[font=\normalsize, left]{\iota} \arrow{r}[font=\normalsize]{\varphi} & W \\
V\arrow[dashed]{ru}[font=\normalsize, right]{\overline{\varphi}}
\end{tikzcd}
\end{center}

Es más, es fácil ver que vale la vuelta: es decir, si para cualquier $W$ y $\varphi :\beta\to W$ existe una única transformación lineal $\overline{\varphi}$ tal que $\left. \overline{\varphi} \right |_{\beta} = \varphi$, entonces $\beta$ es una base. Entonces caracterizamos a una base por una propiedad universal. Estamos en condiciones de hacer la siguiente definición:
\end{obs}

\begin{defn}[Grupo Libre] Un grupo $F$ se dice \textbf{libre} si existe un subconjunto $S\subseteq F$ con la siguiente propiedad universal: Para todo grupo $G$ y función $\varphi : S\to G$ existe un único morfismo de grupos $\overline{\varphi}:F\to G$ tal que $\left.\overline{\varphi}\right |_{S}=\varphi$. Es decir, el siguiente diagrama conmuta:
\begin{center}
\begin{tikzcd}[row sep=3em,column sep=2.5em,minimum width=3em]
S \arrow[hookrightarrow]{d}[font=\normalsize, left]{\iota} \arrow{r}[font=\normalsize]{\varphi} & G \\
F\arrow[dashed]{ru}[font=\normalsize, right]{\overline{\varphi}}
\end{tikzcd}
\end{center}
En estas condiciones, se dice que $S$ es una \textbf{base} de $F$.
\end{defn}
\begin{obs}
Por definición de base, $1\notin S$ pues para una función que lo mande a algo distinto de $1$, no la podemos extender a un morfismo de grupos.

Además, es claro ver que $F=1$ es un grupo libre con base $S=\emptyset$.
\end{obs}

\begin{prop}
Sean $F_1$ y $F_2$ grupos libres con bases $S_1$ y $S_2$ respectivamente. Entonces $F_1\simeq F_2$ si y sólo si $\sharp S_1 =\sharp S_2$. En particular, todas las bases de un grupo libre tienen el mismo cardinal.
\begin{proof}
($\Longleftarrow$) Supongamos que $\sharp S_1 = \sharp S_2$. Entonces existe una biyección $\varphi :S_1\to S_2$. Por las propiedades universales de los grupos libres, tenemos el siguiente diagrama conmutativo:
\begin{center}
\begin{tikzcd}[row sep=3em,column sep=2.5em,minimum width=3em]
S_1 \arrow[hookrightarrow]{d}[font=\normalsize, left]{\iota_1} \arrow{r}[font=\normalsize]{\varphi} & S_2\arrow[hookrightarrow]{r}[font=\normalsize]{\iota_2} & F_2 \arrow[start anchor = -115, end anchor=25, dashed]{lld}[font=\normalsize]{\overline{\psi}}  \\
F_1\arrow[start anchor=65, end anchor=-155, dashed]{rru}[font=\normalsize]{\overline{\varphi}} & S_1\arrow[hookrightarrow]{l}[font=\normalsize]{\iota_1} & S_2\arrow[hookleftarrow]{u}[font=\normalsize,right]{\iota_2} \arrow{l}[font=\normalsize]{\varphi^{-1}}
\end{tikzcd}
\end{center}
Tenemos entonces que $\overline{\varphi}\circ \iota_1 = \iota_2\circ\varphi$ y $\overline{\psi}\circ\iota_2 = \iota_1\circ \varphi^{-1}$. Veamos entonces que $\overline{\varphi}$ y $\overline{\psi}$ son inversas, y así tendremos que $F_1\simeq F_2$.

En efecto, tenemos que $\overline{\psi}\circ (\overline{\varphi}\circ \iota_1) = \overline{\psi}\circ(\iota_2\circ \varphi) = (\overline{\psi}\circ \iota_2)\circ \varphi = \iota_1\circ \varphi^{-1} \circ \varphi = \iota_1$. Es decir, $(\overline{\psi}\circ\overline{\varphi})\circ \iota_1 = \iota_1$. Por lo tanto, $\overline{\psi}\circ \overline{\varphi}$ hace conmutar al siguiente diagrama:

\begin{tikzcd}[row sep = 3em, column sep = 4em, minimum width = 3em]
S_1 \arrow[hookrightarrow]{d}[font=\normalsize, left]{\iota_1}\arrow[hookrightarrow]{r}[font=\normalsize]{\iota_1} & F_1 \\
F_1 \arrow[dashed]{ur}[font=\normalsize, right]{\overline{\psi}\circ\overline{\varphi}} 
\end{tikzcd}
Pero trivialmente $\id_{F_1}$ también lo hace conmutar.

Por la propiedad universal, debemos tener que $\overline{\psi}\circ\overline{\varphi} = \id_{F_1}$. Análogamente, se puede ver que $\overline{\varphi}\circ\overline{\psi} = \id_{F_2}$. Entonces construimos un isomorfismo y la implicación sigue.

($\Longrightarrow$) Supongamos que $F_1\simeq F_2$. Entonces, es claro que $\sharp \hom(F_1,\ZZ_2) = \sharp\hom(F_2,\ZZ_2)$. Pero como cada función $f:S\to\ZZ_2$ se extiende a un único morfismo (y cada morfismo es una función) tenemos que $\sharp \{f:S_1\to \ZZ_2\} = \sharp \{f:S_2\to\ZZ_2\}$ y así $2^{\sharp S_1} = 2^{\sharp S_2}$. Si son finitos entonces $\sharp S_1 =\sharp S_2$. Si no, suponiendo la hipótesis generalizada del continuo, es claro que $\sharp S_1 = \sharp S_2$. Y estamos.
\end{proof}
\end{prop}

\begin{defn}
Si $F$ es un grupo libre, se define el \textbf{rango} de $F$ como $\rank (F) = \sharp S$ con $S$ base de $F$. Está bien definido por la proposición anterior.
\end{defn}

Vamos a probar que no es necesario asumir la hipótesis generalizada del continuo para demostrar la proposición. En efecto, vamos a probar que para cualquier grupo $S$ hay un grupo libre con $S$ como base. Pero antes, veamos un par de ejemplos.

\begin{ex}
$\ZZ$ es un grupo libre de rango $1$. En efecto, es generado por $S=\{1\}$. Intuitivamente, esto es porque definir a un morfismo en el $1$ ya nos fija al morfismo.

$\ZZ_2$ no es libre. En efecto, si $S$ fuera una base, entonces $0\notin S$ por una observación anterior y $S\neq \emptyset$ pues $1\not\simeq \ZZ_2$. Entonces $S=\{1\}$. Pero por la proposición anterior, resultaría que $\ZZ_2\simeq \ZZ$. Absurdo.

Más generalmente, si $G$ es un grupo abeliano no trivial y $G\not\simeq \ZZ$ entonces $G$ \textbf{no} es libre. En efecto, supongamos que $G$ fuera libre y $S$ una base. Entonces $S\neq \emptyset$ y $\sharp S\neq 1$ (pues esas condiciones implicarían que $G$ es trivial o isomorfo a $\ZZ$). Entonces $\sharp S\geq 2$ y así tenemos $a,b\in S$ con $a\neq b$. Consideremos $\varphi : S\to D_3$ dado por $\varphi(a)=r$, $\varphi(b)=s$ y $\varphi(x) = 1$ para todo $x\in S-\{a,b\}$. Entonces, existe una única extensión de la función, $\overline{\varphi}:G\to D_3$ con $\left. \overline{\varphi}\right |_{S} = \varphi$. Pero $1 = \overline{\varphi}(1) = \overline{\varphi}(aba^{-1}b^{-1}) = rsr^{-1}s = r^2 $. Absurdo. Entonces $G$ no es libre.
\end{ex}

Ahora, dado un conjunto $S$ queremos construir un grupo libre $F(S)$ que tenga a $S$ como base. Si $S=\emptyset$ entonces $F(S)=1$. Supongamos ahora $S$ no vacío. Llamamos generadores a los elementos de $S$. Construimos $S^- =\{a^{-1}:a\in S\}$, un conjunto de símboles formales. Ahora, llamamos \textbf{letras} a los elementos de $S^\pm = S \coprod S^-$.

Entonces, una \textbf{palabra} en $S$ es una sucesión finita $x_1\ldots x_n$ con $x_i\in S^\pm$ y $n\in\NN_0$. Cuando $n=0$ tenemos la palabra vacía, que denotaremos $1$. Finalmente, definimos $W(S)$ como el conjunto de palabras en $S$. La longitud de una palabra $x_1\ldots x_n$ es $n$.

Es fácil notar que $W(S)$ es un monoide con la yuxtaposición. Es decir, $(x_1\ldots x_n)\cdot (y_1\ldots y_m) = x_1\ldots x_n y_1\ldots y_m$, con el neutro la palabra vacía $1$. ¿Podremos extender esto a un grupo de alguna forma?

Una subpalabra de una palabra $x_1\ldots x_n$ es una palabra $x_ix_{i+1}\ldots x_{i+k}$. Decimos que una palabra es \textit{reducida} si no tiene subpalabras de la forma $aa^{-1}$ o  $a^{-1}a$. Una reducción elemental de una palabra es la palabra que se obtiene de cancelar una subpalabra de la forma $aa^{-1}$ o $a^{-1}a$.

Se puede probar (aunque no es trivial) que toda palabra se puede llevar, haciendo reducciones elementales, a una palabra reducida y que siempre es la misma palabra independientemente de cómo se hagan las reducciones. Si $w$ es una palabra, denotamos $\rho (w)$ a su palabra reducida.

Definimos entonces $F(S) = \{\rho (w) : w\in W(S)\}$, y la operación que definimos es $u\cdot v = \rho (uv)$. Es decir, reduzco la yuxtaposición de las dos palabras.

\begin{prop}
$F(S)$ es un grupo libre con base $S$.
\end{prop}

Notemos que $\sharp F(S) = \sharp S$ si $S$ es infinito pues en ese caso $\sharp S^{\aleph_0} = \sharp S$. Entonces, como si $G$ es libre con base $S$ tenemos que $G\simeq F(S)$ (cumplen la misma propiedad universal), es claro que si $F_1$ y $F_2$ son dos grupos libres isomorfos entre sí con bases $S_1$ y $S_2$ respectivamente (con cardinal infinito), $F(S_1)\simeq F_1\simeq F_2\simeq F(S_2)$ y así $\sharp F(S_1) = \sharp F(S_2)$. Pero esto quiere decir que $\sharp S_1 = \sharp S_2$ , y no hizo falta la hipótesis del continuo generalizada.

\begin{prop}
Todo grupo $G$ es cociente de un grupo libre. Es decir, existe un $F$ libre y $H\triangleleft F$ tal que $F/H\simeq G$.
\begin{proof}
Tomemos $F=F(G)$. Por la propiedad universal de grupos libres, debemos tener el siguiente diagrama conmutativo: 

\begin{tikzcd}[row sep=3em,column sep=2.5em,minimum width=3em]
G \arrow[hookrightarrow]{d}[font=\normalsize, left]{\iota} \arrow{r}[font=\normalsize]{\id} & G \\
F(G)\arrow[dashed]{ru}[font=\normalsize, right]{\overline{\varphi}}
\end{tikzcd}
Por lo tanto, tenemos $\overline{\varphi}:F(G)\to G$ que es un epimorfismo pues al restringirnos a $G$ es la identidad. Tomando $H=\ker\overline{\varphi}$ y por el Primer Teorema de Isomorfismo, debemos tener que $F(G)/\ker\overline{\varphi}\simeq \im\overline{\varphi}=G$. Y estamos.
\end{proof}
\end{prop}
\begin{obs}
El $F$ de la proposición anterior no tiene por qué ser único. Por ejemplo, podríamos haber tomado $F=F(S)$ con $S$ tal que $G=\langle S\rangle$.
\end{obs}

\begin{defn}
Sea $G$ un grupo y $R\subseteq G$ un subconjunto. La \textbf{clausura normal} de $R$ en $G$ es el menor subgrupo normal (en el sentido de inclusión) que contiene a $R$. Es decir, $N_R = \displaystyle\bigcap_{\stackrel{N\triangleleft G}{R\subseteq N}} N$.
\end{defn}

\begin{defn}
Una \textbf{presentación} $P=\langle S | R\rangle$ consiste de un conjunto $S$ (que se llaman generadores) y un conjunto $R\subseteq F(S)$ (que se llaman relaciones). El grupo que presenta a $G(P) = F(S)/N_R$. 
\end{defn}

\begin{obs}
Todo grupo admite una presentación, pues tomamos $S=F(G)$ y $R= H\triangleleft G$ como en la proposición anterior.
\end{obs}

\begin{defn}
Se dice que $P=\langle S |R\rangle $ es una presentación finita si $S$ y $R$ son finitos. Se escribe $\langle x_1,\ldots , x_n | r_1,\ldots , r_t\rangle$.
\end{defn}

\begin{ex}
\begin{itemize}
\item $\ZZ=\langle a| \rangle  = \langle a,b | b\rangle = \langle a,b | ab\rangle$
\item $\ZZ_n = \langle a | a^n \rangle$
\item $\ZZ\oplus\ZZ = \langle a,b | aba^{-1}b^{-1}\rangle$
\item $D_n = \langle r,s|r^n ,s^2 , srsr\rangle$
\end{itemize}
\end{ex}

\chapter{Anillos}

\section{Definiciones Básicas}

\begin{defn}
Un anillo $(A,+,\cdot)$ es un conjunto no vacío $A$ provisto de dos operaciones, la suma $+$ y el producto $\cdot$ que cumplen que \begin{itemize}\item $(A,+)$ es un grupo abeliano \item $(A,\cdot)$ es un monoide \item Propiedad distributiva: $a(b+c)=ab+ac$ y $(b+c)a=ba+ca$\end{itemize}
\end{defn}

\begin{defn}
Un anillo $A$ se dice \textbf{conmutativo} si $ab=ba$ para todos $a,b\in A$. Es decir, si $(A,\cdot)$ es un monoide abeliano.
\end{defn}

\begin{obs}
Notemos que $0\cdot a = 0 = a\cdot 0$ pues $0\cdot a = (0+0)\cdot a = 0\cdot a + 0\cdot a$ y así $0\cdot a = 0$.

Además, si $1=0$, tenemos que debe ser el anillo trivial, pues $a=a\cdot 1 = a\cdot 0 = 0$.

También es fácil ver que valen $-a = (-1)a$, $(-a)b=-(ab) = a(-b)$ y $(-a)(-b)=ab$.
\end{obs}

\begin{defn}
Sea $A$ un anillo. Decimos que $a$ tiene \textbf{inverso a izquierda} si existe un $b\in A$ tal que $ba=1$. Análogamente, decimos que $a$ tiene \textbf{inverso a derecha} si existe un $c\in A$ tal que $ac=1$.
\end{defn}
\begin{obs}
Es fácil ver que si $a$ tiene inverso a izquierda $b$ e inverso a derecha $c$ entonces $b=c$ pues $b = b\cdot 1 = b(ac) = (ba)c = 1\cdot c = c $.
\end{obs}

\begin{defn}
Se dice que $a$ es una \textbf{unidad} o \textbf{inversible} si existe un $b$ inverso a derecha e izquierda. En ese caso, por la observación anterior, el inverso es único y se denota $a^{-1}$. Notamos $\mathcal{U}(A)$ ó $A^\times$ al conjunto de las unidades de $A$. Es fácil ver que $(A^{\times},\cdot)$ es un grupo.
\end{defn}

\begin{obs}
Si $A\neq 0$ entonces $0\notin \mathcal{U}(A)$ (es claro que no tiene inverso pues si $u$ fuera inverso de $0$ entonces $1 = u0 = 0$).
\end{obs}

\begin{defn}
Un \textbf{anillo de división} es un anillo $A\neq 0$ tal que $\mathcal{U}(A) = A - \{0\}$. Un \textbf{cuerpo} es un anillo de división conmutativo.
\end{defn}

\begin{ex}
\begin{itemize}
\item $(\ZZ,+,\cdot)$ es un anillo conmutativo y $(\mathbb{Q} , +,\cdot), (\RR,+,\cdot), (\mathbb{C},+,\cdot)$ son cuerpos.
\item $(\ZZ_n,+,\cdot)$ es un anillo conmutativo y es un cuerpo si y sólo si $n$ es primo.
\item  $A$ anillo, $n\in\NN$, entonces $M_n(A)$ las matrices de $n\times n$ con coeficientes en $A$ forman un anillo con respecto a la suma y el producto de matrices usuales.
\item $A$ anillo, entonces $A[x]$ el conjunto de los polinomios en una variable con coeficientes en $A$ forman un anillo con respecto a la suma y producto de polinomios usuales.
\item $A$ anillo, $X$ conjunto, $A^X = \{f:X\to A\}$ forma un anillo con las operaciones definidas punto a punto.
\item $\mathbb{Z}[\sqrt{2}]$ es un anillo y $\mathbb{Q}[\sqrt{2}]$ es un cuerpo.
\end{itemize}
\end{ex}

\section{Divisores de cero}

\begin{defn}
Sea $A$ un anillo, $a,b\in A$, $a,b\neq 0$. Si $ab=0$ entonces $a$ se llama un \textbf{divisor de cero a izquierda} y $b$ \textbf{divisor de cero a derecha}.
\end{defn}

\begin{obs}
Sea $a\in A$ y $a\neq 0$. Entonces, $a$ no es divisor de cero a izquierda si y sólo si para todo $b\in A$ tal que $ab=0$ implica que $b=0$. Esto es claramente equivalente a que para todos $b,c\in A$ tales que $ab=ac$ entonces $b=c$. En efecto $ab=ac$ si y sólo si $a(b-c)=0$ de donde $b-c=0$ pues $a$ no es divisor de cero a izquierda. Lo mismo vale con divisores de cero a derecha.
\end{obs}

\begin{defn}
Sea $a\in A$, $a\neq 0$. Se dice que $a$ es un divisor de cero si es un divisor de cero a izquierda o derecha.
\end{defn}

\begin{defn}
Un anillo $A$ se dice un \textbf{anillo íntegro} si $A\neq 0$ y no tiene divisores de cero (es decir, $\forall a,b\in A$, $ab=0\Longrightarrow a=0$ o $b=0$). Más aún, un anillo $A$ se dice \textbf{dominio íntegro} si es un anillo íntegro conmutativo.
\end{defn}

\begin{obs}
Sea un anillo $A$, $a\in A$, $a\neq 0$. Si $a$ tiene inverso a izquierda, entonces $a$ no es un divisor de cero a izquierda (ídem a derecha). En efecto, supongamos que $ab=0$ y $a'$ el inverso a izquierda de $a$. Entonces $a'ab = a'0 = 0$, es decir, $b=0$. En particular, si $a\in A^\times$ entonces $a$ no es un divisor de cero.

De esto sigue que si $A\neq 0$ es un anillo de división entonces es un anillo íntegro, y que si $A\neq 0$ es un cuerpo entonces $A$ es un dominio íntegro.
\end{obs}

\begin{ex}
$\ZZ$ y $k[x]$ son dominios íntegros que no son cuerpos. $M_n(\ZZ)$ y $M_n(\RR)$ no son íntegros.
\end{ex}

\begin{prop}
Si $A$ es un dominio íntegro entonces $A[x]$ lo es.
\begin{proof}
Sean $f,g\in A[x]$. Entonces $f=\displaystyle\sum_{k=0}^{n}a_k x^k$ y $g=\displaystyle\sum_{k=0}^{m}b_k x^k$ (con $a_n,b_m \neq 0$). Pero entonces el coeficiente de $x^{n+m}$ en $fg$ es $a_nb_m$. Pero como ambos son distintos de cero y $A$ es íntegro, $a_nb_m\neq 0$. Esto quiere decir que $fg\neq 0$ y así $A[x]$ es íntegro.
\end{proof}
\end{prop}
\begin{obs}
De la demostración de la proposición anterior se desprende que si $A$ es íntegro, $\gr (fg) = \gr (f) + \gr (g)$. Pero si $A$ no es íntegro esto es todo mentira. Por ejemplo, en $A= \ZZ_6$, $f=2x$, $g=3x$ tenemos que $fg = 0$. La fórmula de grados tampoco vale, $(2x+1)(3x)=3x$.
\end{obs}

\begin{prop}
Un anillo íntegro $A$ finito es un anillo de división. En particular, si $A$ es un dominio íntegro finito entonces es un cuerpo.
\begin{proof}
Sea $a\in A$, $a\neq 0$. Debo ver que $a$ es inversible. Sea $\ell_a : A \to A$ la aplicación definida por $\ell_a(b)=ab$, la multiplicación a izquierda por $a$. Como $a$ no es un divisor de cero, resulta que $\ell_a$ es inyectiva. Como $A$ es un conjunto finito, ser inyectiva es lo mismo que ser sobreyectiva. Entonces existe un $b\in A$ tal que $\ell_a(b)=1$. Entonces $ab=1$ y tiene inverso a izquierda. Pero definiendo análogamente la multiplicación a derecha por $a$, $r_a:A\to A, b\mapsto ba$, obtenemos que tiene inverso a derecha. Entonces $a$ es inversible y estamos.
\end{proof}
\end{prop}

\section{Subanillos y morfismos}

\begin{defn}
Sea $A$ un anillo. Un subanillo de $A$ es un subconjunto $B\subseteq A$ tal que \begin{itemize} \item $(B,+)\subseteq (A,+)$ es un subgrupo. \item $1\in B$. \item $x,y\in B \Longrightarrow xy\in B$. \end{itemize}
\end{defn}

\begin{ex}
$\ZZ\subseteq \QQ \subseteq \RR\subseteq \CC$ son subanillos. $\ZZ [\sqrt{2}]\subseteq \RR$ es un subanillo. $\ZZ[i]\subseteq \CC$ es un subanillo.
\end{ex}

\begin{obs}
Si $A$ es conmutativo y $B\subseteq A$ es un subanillo, entonces $B$ es un anillo conmutativo. Si $A$ es íntegro y $B\subseteq A$ subanillo, entonces $B$ es íntegro.
\end{obs}

\begin{defn}
Sean $A,B$ anillos. Un \textbf{morfismo de anillos} $f:A\to B$ es una función que cumple que \begin{itemize}\item $f(a+a')=f(a)+f(a')$ \item $f(aa') = f(a)f(a')$ \item $f(1)=1$ \end{itemize}
\end{defn}

\begin{obs}
Si $f:A\to B$ es un morfismo de anillos, entonces $f(0)=0$. En efecto, $f(0)+f(0)=f(0+0)= f(0)$. En cambio, la condición de $f(1)=1$ no se desprende de las otras dos (es decir, hay que pedirla).
\end{obs}

\begin{obs}
Sea $f:A\to B$ morfismo de anillos y $a\in A^\times$. Entonces $f(a)\in B^\times$. Esto es obvio pues si $a\in A^\times$ existe $a'\in A$ tal que $aa'=a'a=1$. Entonces $1=f(1)=f(aa') = f(a)f(a') $ y de forma análoga $f(a')f(a)=1$.
\end{obs}

\begin{prop}
Sea $A$ un anillo cualquiera. Entonces existe un único morfismo de anillos $\lambda: \ZZ \to A$.
\begin{proof}
Notemos que $\lambda(1)=1$ por definición de morfismo de anillos. Ahora, para cualquier $n\in \NN$, $\lambda(n)=\lambda(\underbrace{1+\ldots + 1}_{n \text{ veces}}) = \underbrace{1 +\ldots + 1}_{n\text{ veces}} = \underbar{n} = n\cdot 1$. Pero además $\lambda(-1) = -1$ (el inverso aditivo de $1$ en $A$). Entonces $\lambda(-n) = \underbrace{(-1)+\ldots + (-1)}_{n \text{ veces}} = - \underbar{n} = n\cdot (-1)$. Esto quiere decir que si hay un morfismo, tiene que estar definido así. Es trivial chequear que esto es un morfismo.
\end{proof}
\end{prop}

\begin{ex}
Sea $f:\QQ\to \RR$ morfismo de anillos. Entonces $f(1)=1$. Si $n\in \ZZ$, $f(n)=n$. Ahora, $1= f(1)=f\left(\underbrace{\dfrac{1}{n} + \ldots + \dfrac{1}{n}}_{n \text{ veces}}\right) = nf\left(\dfrac{1}{n}\right)$, y así $f\left(\dfrac{1}{n}\right) = \dfrac{1}{n}$. Esto implica que $f\left( \dfrac{m}{n}\right) = \dfrac{m}{n}$, y así $f$ sólo puede ser la inclusión de $\QQ$ en $\RR$.
\end{ex}

\begin{defn}
Sea $f:A\to B$ morfismo de anillos. Definimos el núcleo $\ker f = \{a\in A : f(a)=0\}$ y la imagen $\im f = \{f(a) : a\in A\}\subseteq B$.
\end{defn}

\begin{obs}
Es fácil ver que $\ker f = \{0\}$ si y sólo si $f$ es inyectiva. En efecto, si $\ker f = 0$ y $f(a)=f(a')$, $f(a-a')=0$ y así $a=a'$. Y si $f$ es inyectiva, como $f(0)=0$, $\ker f = \{0 \}$.
\end{obs}

\begin{ex}
Veamos que no hay ningún morfismo de anillos de $\RR$ en $\QQ$. En efecto, si $f:\RR\to \QQ$ fuera un morfismo de anillos, tendríamos que $f(1)=1$. Para todo $x\in \RR$, $x\neq 0$, existe $x^{-1}\in \RR$ y así $1=f(1)=f(xx^{-1})=f(x)f(x^{-1})$. Entonces, si $x\neq 0$, $f(x)\neq 0$. Esto quiere decir que $\ker f = \{0 \}$ y así $f$ inyectiva. Pero esto implica que $\mathfrak{c}=\sharp \RR \leq \sharp \QQ = \aleph_0$. Absurdo.
\end{ex}

\section{Ideales y Cocientes}

\begin{defn}
Sea $A$ un anillo. Un \textbf{ideal a izquierda} (resp. a derecha) de $A$ es un subconjunto $I\subseteq A$ tal que \begin{itemize} \item $(I,+)\subseteq (A,+)$ es un subgrupo. \item $\forall a\in A$, $\forall x\in I$, $ax\in I$ (resp. $xa\in I$).\end{itemize} Equivalentemente, \begin{itemize} \item $I\neq \emptyset$. \item $x,y\in I \Longrightarrow x+y\in I$. \item $a\in A$, $x\in I \Longrightarrow ax\in I$ (resp. $xa\in I$).\end{itemize}

Un ideal $I\subseteq A$ se dice \textbf{ideal bilátero} si es un ideal a izquierda y derecha.
\end{defn}

\begin{obs}
Si $A$ es un anillo conmutativo, entonces $I\subseteq A$ es un ideal a izquierda si y sólo si es un ideal a derecha. Es decir, todos los ideales son biláteros.
\end{obs}

\begin{obs}
Es claro que $0,A\subseteq A$ son ideales biláteros de $A$.
\end{obs}

\begin{defn}
Un ideal $I\subseteq A$ se dice \textbf{propio} si $I\neq 0$, $I\neq A$.
\end{defn}

\begin{obs}
Si $I\subseteq A$ es un ideal y $1\in I$, entonces $I=A$. En efecto, para cualquier $a\in A$, $1a\in I$.
\end{obs}

\begin{obs}
Si $f:A\to B$ es un morfismo de anillos, entonces $\ker f\subseteq A$ es un ideal bilátero. En efecto, $\ker f \neq \emptyset$ pues $0\in \ker f$. Además, si $x,y\in \ker f$ entonces $f(x)=f(y)=0$ y así $f(x+y)=f(x)+f(y)=0+0=0$, entonces $x+y\in \ker f$. Finalmente, si $a\in A$, $x\in \ker f$, entonces $f(ax)=f(a)f(x)=f(a)\cdot 0 = 0$, y así $ax\in \ker f$. Análogamente vale que $xa \in \ker f$.
\end{obs}

\begin{obs}
Sea $A$ un anillo y $\{I_j\}_{j\in J}$ una familia de ideales de $A$ (todos a izquierda, a derecha o biláteros). Entonces, $\displaystyle\bigcap_{j\in J} I_j$ es un ideal (Esto es trivial de verificar). Tiene sentido hacer la siguiente definición entonces.
\end{obs}

\begin{defn}
Sea $A$ un anillo y $S\subseteq A$ un subconjunto. Definimos el ideal (a izquierda, a derecha o bilátero) $\langle S\rangle$ generado por $S$ como $\langle S\rangle = \displaystyle\bigcap_{\stackrel{I \text{ ideal}}{S\subseteq I}}I$. 

A izquierda este ideal es $\langle S\rangle =\left\{\displaystyle\sum_{\text{finita}} a_s s : s\in S , a_s\in A\right\}$

A derecha este ideal es $\langle S\rangle = \left\{\displaystyle\sum_{\text{finita}} sa_s : s\in S, a_s\in A\right\}$ 

Y bilátero es $\langle S\rangle = \left\{\displaystyle\sum_{\text{finita}}a_ssb_s : s\in S, a_s,b_s\in A\right\}$.
\end{defn}

\begin{defn}
Sea $A$ un anillo conmutativo. Un ideal $I\subseteq A$ se dice \textbf{principal} si existe un $x\in I$ tal que $I=\langle x\rangle$. Un \textbf{dominio de ideales principales} (DIP) es un dominio íntegro tal que todo ideal es principal.
\end{defn}

\begin{ex}
Veamos que $\ZZ$ es un DIP. En efecto, sea $I\subseteq \ZZ$ un ideal. Si $I=0$, $I=\langle 0\rangle$. Si $I\neq 0$, existe un $m\in\NN$ tal que $m\in I$. Sea $n=\min \{m\in\NN : m\in I\}$. Veamos que $I=\langle n\rangle = n\ZZ = \{kn : k\in\ZZ\}$. La inclusión $\langle n\rangle \subseteq I$ es trivial pues $n\in I$ y así $kn\in I$ por ser $I$ ideal. Ahora, sea $m\in I$. Por algoritmo de división, $m=qn+r$ para algún $q\in \ZZ$ y $0\leq r\leq n-1$. Entonces, $r=m-nq\in I$. Pero $n$ era el más chico. Entonces $r=0$ y así $m\in \langle n\rangle$, con lo que vale la otra inclusión.
\end{ex}

\begin{obs}
Se puede ver de forma similar que si $k$ es un cuerpo, entonces $k[x]$ es un DIP (hay que hacer algoritmo de división también, viendo que se puede escribir $g= qf + r$ con $\gr r < gr f$). Esto es mentira en general: si $A$ es un anillo, $A[x]$ no suele ser un DIP. Por ejemplo $\ZZ[x]$ no es un DIP pues $\langle 2,x\rangle$ no es principal.
\end{obs}

\begin{prop}
Sea $A$ un anillo conmutativo. $A$ no tiene ideales propios si y sólo si $A$ es un cuerpo.
\begin{proof}
($\Longrightarrow$) Sea $x\in A$ y $x\neq 0$. Quiero ver que $x$ es una unidad. Como $x\neq 0$, $\langle x\rangle \subseteq A$. Como $A$ no tiene ideales propios, y ese ideal es no nulo, debemos tener que $\langle x \rangle = A$ y así $1\in \langle x\rangle $. Es decir, existe un $a\in A$ tal que $ax=1=xa$. Entonces $x\in A^\times$.

($\Longleftarrow$) Sea $I\neq 0$. Entonces, existe $x\in I$ tal que $x\neq 0$. Como $A$ es un cuerpo, $x^{-1}\in A$, entonces $1=x^{-1}x \in I$, y así $I=A$. 
\end{proof}
\end{prop}

\begin{defn}
Sea $f:A\to B$ un morfismo de anillos. Decimos que $f$ es un \textbf{isomorfismo de anillos} si y sólo si es un morfismo de anillos biyectivo.
\end{defn}

\begin{obs}
Si $f:A\to B$ es un isomorfismo de anillos entonces $f^{-1}:B\to A$ también lo es.
\end{obs}

\section{Cocientes}

De ahora en más, en esta sección, todos los ideales que consideremos serán biláteros.

\begin{defn}
Sea $A$ un anillo e $I\subseteq A$ un ideal bilátero. Como $(I,+)\subseteq (A,+)$ es un subgrupo, puede considerar el grupo cociente $(A/I,+)$, que es un grupo abeliano.
Definamos un producto en $A/I$, para poder obtener una estructura de anillo. Concretamente, definimos $\overline{a}\cdot\overline{b} = \overline{a\cdot b}$. Hay que ver que esto está bien definido, es decir, que si $\overline{a}=\overline{a'}$ y $\overline{b}=\overline{b'}$ entonces $\overline{ab} = \overline{a'b'}$. En efecto, si $\overline{a}=\overline{a'}$, entonces $a-a'\in I$ y así $(a-a')b\in I$ por ser $I$ un ideal. Esto implica que $ab-a'b\in I$ y así $\overline{ab}=\overline{a'b}$. De forma similar, $b-b'\in I$ y así $a'(b-b')\in I$, que implica $\overline{a'b} = \overline{a'b'}$. Entonces $\overline{ab}=\overline{a'b'}$.

Entonces, $(A/I,+,\cdot)$ tiene una estructura de anillo y lo llamaremos el anillo cociente de $A$ por el ideal $I$.

\end{defn}

De forma similar a la situación de grupos, acá también el cociente cumple una propiedad universal.

\begin{teo}[Propiedad Universal del Cociente]

Sea $A$ un anillo, $I\subseteq A$ un ideal. Entonces el cociente $A/I$ tiene la siguiente propiedad universal: para todo anillo $B$ y para todo morfismo $f:A\to B$ tal que $I\subseteq \ker f$ existe un único morfismo de anillos $\overline{f}:A/I\to B$ tal que $\overline{f}\circ \pi = f$. Es decir, el siguiente diagrama conmuta:

\begin{tikzcd}[row sep=3em,column sep=2.5em,minimum width=3em]
A \arrow{d}[font=\normalsize, left]{\pi} \arrow{r}[font=\normalsize]{f} & B \\
A/I\arrow[dashed]{ru}[font=\normalsize, right]{\overline{f}}
\end{tikzcd}

\begin{proof}
Si consideramos primero sólo los grupos aditivos, por la propiedad universal del cociente para grupos hay un único morfismo de grupos $\overline{f}:(A/I,+) \to (B,+)$. Pero como está definido para todos los elementos de $A$, resta ver que este morfismo preserva la estructura de anillos. Pero es claro que $\overline{f}(\overline{ab}) = f(ab)$. Pero $f$ es morfismo de anillos, entonces $f(ab)=f(a)f(b)$. Nuevamente, $\overline{f}(\overline{a})=f(a)$ y $\overline{f}(\overline{b})=f(b)$. Entonces $\overline{f}(\overline{ab})=\overline{f}(\overline{a})\overline{f}(\overline{b})$ y así es morfismo de anillos y estamos.
\end{proof}
\end{teo}

Al igual que en grupos, sigue como corolario el Primer Teorema de Isomorfismo de Anillos:

\begin{teo}[Primer Teorema de Isomorfismo]
Sea $f:A\to B$ un morfismo de anillos. Entonces $A/\ker f \simeq \im f$.
\begin{proof}
Consideremos $\left.f\right |^{\im f} : A\to \im f$ la correstricción a la imagen. Por la propiedad universal del cociente, tenemos un morfismo $\overline{f}:A/I\to \im f$ tal que $\overline{f}(\overline{a})=f(\overline{a})$. Veamos primero que $\overline{f}$ es inyectiva. En efecto, si $\overline{f}(\overline{a})=\overline{f}(\overline{b})$ entonces $f(a)=f(b)$ y así $f(a-b)=0$. Entonces $a-b\in \ker f$ y así $\overline{a}=\overline{b}$. Además, $\overline{f}$ es claramente sobreyectiva, pues para cualquier $x\in \im f$ existe un $a\in A$ tal que $f(a)=x$. Entonces $\overline{f}(\overline{a})=f(a)=x$. Entonces $\overline{f}$ es un isomorfismo de anillos y listo.
\end{proof}
\end{teo}

\begin{teo}[Segundo Teorema de Isomorfismo]
Sea $A$ anillo y $I\subseteq J\subseteq A$ ideales biláteros de $A$. Entonces $(A/I)/(J/I) \simeq A/J$.
\begin{proof}
Por la propiedad universal del cociente, tenemos el diagrama conmutativo \begin{tikzcd}[row sep=3em,column sep=2.5em,minimum width=3em]
A \arrow{d}[font=\normalsize, left]{\pi_I} \arrow{r}[font=\normalsize]{\pi_J} & A/J \\
A/I\arrow[dashed]{ru}[font=\normalsize, right]{\overline{\pi}}
\end{tikzcd} donde $\pi_I$ y $\pi_J$ son las proyecciones. Es claro que $\overline{\pi}$ es sobreyectiva pues $\pi_J$ lo es. Si vemos que $\ker \overline{\pi} = J/I$, por el Primer Teorema de Isomorfismo ya estaremos.

Pero $\overline{\pi}(\overline{a})=\pi_J(a)$, entonces $\overline{\pi}(\overline{a})=0$ si y sólo si $\pi_J(a)=0$, es decir, $a\in J$. Entonces $\overline{a}\in J/I$ y el teorema sigue.

\end{proof}
\end{teo}

\section{Característica de un anillo}
\begin{defn}
Sea $A$ un anillo, $A\neq 0$. Ya vimos antes que existe un único morfismo $\lambda : \ZZ\to A$. También sabemos que $\ker \lambda\subseteq \ZZ$ es un ideal. Si $\ker \lambda = 0$, esto significa que $\lambda (n) \neq 0$ y en este caso decimos que la característica de $A$ es $0$, y lo notamos $\car (A)=0$. En este caso, $\ZZ/\ker \lambda \simeq \im \lambda\subseteq A$. Como $\ker\lambda = 0$, tenemos que $A$ tiene un subanillo isomorfo a $\ZZ$ (Además, $\im \lambda$ es el subanillo más chico de $A$ pues cualquier subanillo debe contener al $1$ y a sus sumas).

Ahora, si $\ker\lambda\neq 0$, $\ker\lambda = \langle m\rangle$ pues $\ZZ$ es un DIP. Es claro entonces que $m=\min \{n\in \NN : \underline{n}=0\}$. En este caso, se dice que $\car A = m$. De forma análoga, $\ZZ/m\ZZ\simeq \im\lambda\subseteq A$ y así $A$ tiene un subanillo isomorfo a $\ZZ/m\ZZ$ y es el subanillo más chico de $A$.
\end{defn}
\begin{defn}
Si $k$ es un cuerpo, el \textbf{cuerpo primo} de $k$ es el cuerpo más chico contenido en $k$. Es decir, $\displaystyle\bigcap_{\stackrel{F\subseteq k}{F \text{ subcuerpo}}} F$.
\end{defn}

\begin{obs}
Es fácil ver que si $\car k = 0$, entonces al tener un subanillo isomorfo a $\ZZ$, y estar los inversos por ser $k$ cuerpo, tenemos un subanillo isomorfo a $\QQ$. Pero como el subanillo isomorfo a $\ZZ$ era el subanillo más chico, al agregar los inversos vamos a obtener el subcuerpo más chico. Entonces, si $\car k = 0$, el cuerpo primo de $k$ es isomorfo a $\QQ$.

Ahora, si $k$ es un cuerpo con $\car k = m$, entonces hay un subanillo de $A$ isomorfo a $\ZZ/m\ZZ$. Pero este subanillo debe ser un dominio íntegro, pues $k$ lo es. Entonces $\ZZ/m\ZZ$ debe ser un dominio íntegro, y como es finito, un cuerpo. Entonces $m$ es primo. Es decir, si $\car k = p$ tenemos que el cuerpo primo es $\ZZ/p\ZZ$, que denotaremos $\mathbb{F}_p$.
\end{obs}

\begin{prop}
Sea $k$ un cuerpo con $\car k = p$. Entonces  $\varphi:k\to k$ definido por $\varphi(x)=x^p$ es un morfismo de anillos y se lo llama el \textbf{endomorfismo de Frobenius}.
\begin{proof}
Debemos verificar tres condiciones: que $\varphi(1)=1$, que es trivial pues $1^p = 1$; que $\varphi(a+b)=\varphi(a) + \varphi(b)$ y que $\varphi(ab)=\varphi(a)\varphi(b)$. La última condición es trivial pues $(ab)^p = a^p b^p$ pues los elementos conmutan por ser $k$ un cuerpo. Finalmente, notemos que $(a+b)^p = \displaystyle\sum_{k=0}^p \displaystyle\binom{p}{k} a^k b^{p-k}$. Pero como $p\mid \displaystyle\binom{p}{k}$ para $1\leq k\leq p-1$ y $\car k = p$, debemos tener que $(a+b)^p = a^p + b^p$, y estamos.
\end{proof}
\end{prop}

\section{Ideales Primos y Maximales}

\begin{teo}[Teorema de Correspondencia de Anillos]
Sea $A$ un anillo conmutativo, $I$ un ideal bilátero. Entonces, existe una correspondencia biyectiva entre los ideales de $A$ que contienen a $I$ y los ideales de $A/I$, dada por $J\to \overline{J}=\pi (J)=J/I$, y $\overline{J}\to\pi^{-1}(\overline{J})$. Además, $J\subseteq J'$ si y sólo si $\overline{J}\subseteq \overline{J'}$.
\begin{proof}
Es una demostración muy similar a la del Teorema de Correspondencia de Grupos.
\end{proof}
\end{teo}

\begin{defn}
Sea $A$ un anillo conmutativo ($A\neq 0$). Un ideal $\mathfrak{p}\subseteq A$ se dice \textbf{primo} si $\mathfrak{p}\neq A$ ($1\notin \mathfrak{p}$) y cumple que $\forall x,y\in A$ tales que $xy\in \mathfrak{p}$ entonces $x\in\mathfrak{p}$ ó $y\in\mathfrak{p}$. El \textbf{espectro de un anillo} es el conjunto de los ideales primos, es decir, $\spec(A) = \{\mathfrak{p}\subseteq A : \mathfrak{p}\text{ primo}\}$.
\end{defn}

\begin{prop}
Sea $A$ un anillo conmutativo no trivial y $\mathfrak{p}\subseteq A$ un ideal. $\mathfrak{p}$ es primo si y sólo si $A/\mathfrak{p}$ es un dominio íntegro.
\begin{proof}
($\Longrightarrow$) Como $\mathfrak{p}$ es un ideal primo, $\mathfrak{p}\neq A$. Entonces, $A/\mathfrak{p}\neq 0$. Este ideal cociente es conmutativo pues $A$ lo es. Sean $\overline{a},\overline{b}\in A/\mathfrak{p}$ tales que $\overline{a}\overline{b}=\overline{0}$. Esto quiere decir que $\overline{ab}=\overline{0}$, es decir, que $ab\in\mathfrak{p}$. Pero como $\mathfrak{p}$ es un ideal primo, tenemos que $a\in\mathfrak{p}$ o $b\in\mathfrak{p}$, de donde $\overline{a}=\overline{0}$ o $\overline{b}=\overline{0}$, de donde $A/\mathfrak{p}$ es un dominio íntegro.

($\Longleftarrow$) Supongamos que $\overline{ab}=\overline{0}$. Por ser un dominio íntegro, esto implica que $\overline{a}=\overline{0}$ o $\overline{b}=\overline{0}$. Es decir, si $ab\in\mathfrak{p}$ entonces $a\in\mathfrak{p}$ o $b\in\mathfrak{p}$. Y la proposición sigue.
\end{proof}
\end{prop}

\begin{defn}
Un ideal $\mathfrak{m}\subseteq A$ se dice \textbf{maximal} si $\mathfrak{m}\neq A$ y cumple que para todo ideal $\mathfrak{a}\subseteq A$ tal que $\mathfrak{m}\subseteq \mathfrak{a}$ entonces $\mathfrak{a}=\mathfrak{m}$ ó $\mathfrak{a}=A$ (Es decir, es maximal en el sentido de la inclusión).
\end{defn}

\begin{prop}
Sea $A$ un anillo conmutativo y $\mathfrak{m}\subseteq A$ un ideal. $\mathfrak{m}$ es un ideal maximal si y sólo si $A/\mathfrak{m}$ es un cuerpo.
\begin{proof}
Los únicos ideales que contiene un cuerpo son $0$ ó $A$. Acá nos va a servir el Teorema de Correspondencia: los únicos ideales de $A/\mathfrak{m}$ son $\overline{0}$ y $\overline{A}=A/\mathfrak{m}$ si y sólo si los únicos ideales $\mathfrak{a}\supseteq \mathfrak{m}$ son $\mathfrak{m}$ y $A$. Pero esto precisamente nos dice que $\mathfrak{m}$ es maximal si y sólo si $A/\mathfrak{m}$ es un cuerpo.
\end{proof}
\end{prop}

\begin{cor}
Si $\mathfrak{m}$ es un ideal maximal entonces es un ideal primo.
\begin{proof}[Demostración 1]
Si $\mathfrak{m}$ maximal entonces $A/\mathfrak{m}$ es un cuerpo y en particular es un dominio íntegro, y esto implica que $\mathfrak{m}$ es primo.
\end{proof}
\begin{proof}[Demostración 2]
Sean $I,J$ ideales. Definimos $I+J = \{x+y : x\in I, y\in J\}$. Es fácil ver que $I+J$ es un ideal y que $I,J\subseteq I+J$. Sea $\mathfrak{m}$ un ideal maximal. Veamos que es primo. Por definición, $\mathfrak{m}\neq A$. Ahora, sean $x,y\in A$ tales que $xy\in \mathfrak{m}$. Supongamos que $x\notin \mathfrak{m}$. Consideremos el ideal $I=\mathfrak{m}+\langle x\rangle$. Es claro que $\mathfrak{m}\subseteq I$. Además, $x\notin \mathfrak{m}$, entonces $\mathfrak{m}\neq I$. Como $\mathfrak{m}$ es maximal, esto quiere decir que $I=A$. Entonces, $1\in I$ y así $1=z+ax$ con $z\in \mathfrak{m}$ y $a\in A$. Por lo tanto, $y=yz+axy$. Pero $yz\in \mathfrak{m}$ por ser $\mathfrak{m}$ ideal y como $xy\in \mathfrak{m}$ entonces $axy\in \mathfrak{m}$. Esto implica que $yz+axy\in \mathfrak{m}$, entonces $y\in \mathfrak{m}$. Y listo.
\end{proof}
\end{cor}

\begin{obs}
Si $A$ es un anillo conmutativo no trivial, entonces $0$ es un ideal primo de $A$ si y sólo si es un dominio íntegro, y es un ideal maximal de $A$ si y sólo si $A$ es un cuerpo.
\end{obs}

\section{Axioma de Elección y Lema de Zorn}

\begin{defn}
Un \textbf{conjunto parcialmente ordenado} (poset) $(S,\leq)$ es un conjunto $S$ con una relación $\leq$ de orden, es decir, reflexiva, transitiva y antisimétrica.
\end{defn}

\begin{ex}
\begin{itemize}
\item Sea $X$ un conjunto y $S=\mathcal{P}(X)$. Entonces $(S,\subseteq)$ es un poset.
\item Sea $A$ un anillo conmutativo y $S=\{I\subseteq A \text{ ideal} : I\neq A\}$. Entonces $(S,\subseteq)$ es un poset.
\item Sea $A$ un anillo conmutativo no trivial, $I$ un ideal y $S=\{J\subseteq A \text{ ideal} : I\subseteq J, I\neq A\}$. Entonces $(S,\subseteq)$ es un poset.
\end{itemize}
\end{ex}

\begin{defn}
Sea $(S,\leq)$ un poset. Una \textbf{cadena} $C\subseteq S$ es un subconjunto totalmente ordenado (es decir, $\forall x,y\in S, x\leq y$ ó $y\leq x$). Dado $A\subseteq S$ una cota superior de $A$ es un elemento $s\in S$ tal que $a\leq s$ $\forall a\in A$. Finalmente, un elemento $s\in S$ se dice maximal si $\forall s'\in S$ tal que $s\leq s'$ entonces $s=s'$ (notar que no es lo mismo ser máximo que maximal).
\end{defn}

\begin{obs}
Un elemento maximal de $S=\{I\subseteq A \text{ ideal} : I\neq A\}$ con la inclusión es un ideal maximal.
\end{obs}

Se puede probar que los siguientes dos resultados son equivalentes:

\begin{ax}[Axioma de elección]
Sea $X\neq \emptyset$ un conjunto. Existe una función de elección $\varphi:\mathcal{P}(X)-\{\emptyset\}\to X$ tal que $\varphi(A)\in A$ $\forall A\in\mathcal{P}(X)$. 
\end{ax}
\begin{ax}[Lema de Zorn]
Sea $(S,\leq)$ un poset no vacío tal que toda cadena $C\subseteq S$ tiene cota superior. Entonces $S$ tiene elementos maximales.
\end{ax}

Vamos a ver una aplicación importante del Lema de Zorn sobre ideales maximales.

\begin{prop}
Sea $A$ un anillo conmutativo no trivial. Entonces existe $\mathfrak{m}\subseteq A$ ideal maximal.
\begin{proof}

Consideremos $S=\{I\subseteq A \text{ ideal}:I\neq A\}$ con la inclusión. Es claro que $S\neq \emptyset$ pues $0\in S$. Sea $C\subseteq S$ una cadena. Sea $J=\displaystyle\bigcup_{I\in C} I $. Veamos que $J$ es un ideal. En efecto, si $x\in J$ entonces $x\in I$ para algún $I\in C$. Entonces, para cualquier $a\in A$, $ax\in I$ y así $ax\in J$. Ahora, si $x,y\in J$ existen $I_1,I_2\in C$ tales que $x\in I_1$ e $y\in I_2$. Pero como $C$ es una cadena $I_1\subseteq I_2$ o $I_2\subseteq I_1$. Entonces, sin pérdida de la generalidad, $I_1\subseteq I_2$. Esto implica que $x,y\in I_2$, y así $x+y\in I_2\subseteq J$.

Esto quiere decir que cualquier cadena que tomemos en este poset está acotada superiormente. Por el Lema de Zorn existen elementos maximales y por la observación anterior son precisamente ideales maximales.

\end{proof}
\end{prop}

\begin{prop}
Sea $A$ un anillo conmutativo no trivial y sea $I\subseteq A$ un ideal propio. Entonces existe un ideal maximal $\mathfrak{m}$ de $A$ tal que $I\subseteq \mathfrak{m}$.
\begin{proof}
Usamos el Lema de Zorn en $S=\{J\subseteq A\text{ ideal}:I\subseteq J, J\neq A\}$ ó consideramos $A/I$, que tiene ideal maximal por la proposición anterior y usamos el Teorema de Correspondencia.
\end{proof}
\end{prop}

\begin{obs}
Notemos que si $A$ es un anillo conmutativo no trivial y $x\in A^\times$ entonces $\langle x\rangle = A$ pues $x^{-1}\in A$ y así $1=x^{-1}x\in \langle x\rangle$. Entonces, $x\notin \mathfrak{m}$ para todo ideal $\mathfrak{m}$ maximal. Si $x\notin A^\times$ entonces existe un ideal $\mathfrak{m}$ maximal tal que $x\in \mathfrak{m}$, pues tomando $I=\langle x\rangle \neq A$, por la proposición anterior, hay un ideal maximal tal que $I\subseteq \mathfrak{m}$.
\end{obs}

\begin{ex}[Ejercicios]
\begin{itemize}
\item Sean $A$ y $B$ anillos conmutativos, $f:A\to B$ morfismo de anillos y $\mathfrak{p}\in B$ un ideal primo. Entonces $f^{-1}(\mathfrak{p})\subseteq A$ es un ideal primo.
\item Sean $A$ y $B$ anillos conmutativos, $f:A\to B$ morfismo de anillos sobreyectivo y $\mathfrak{m}\in B$ ideal maximal. Entonces $f^{-1}(\mathfrak{m})\subseteq A$ es un ideal maximal.
\item Sea $V$ un $k$-espacio vectorial. Entonces $V$ tiene una base. 
\end{itemize}
\end{ex}

\section{Elementos Primos e Irreducibles}

\begin{defn}
Sea $A$ un dominio íntegro y sean $a,b\in A$. Decimos que $a\mid b$ si existe $c\in A$ tal que $b=ac$. Es decir, si $b\in\langle a\rangle$.
\end{defn}

\begin{prop}
Dados $a,b\in A$ son equivalentes: \begin{enumerate}\item $\langle a\rangle = \langle b\rangle$\item $a\mid b$ y $b\mid a$ \item $a=ub$ con $u\in A^\times$ \end{enumerate} En este caso, se dice que $a$ y $b$ son \textbf{asociados}.
\begin{proof}
(1)$\Longleftrightarrow$(2) es trivial por la definición.

(2)$\Longrightarrow$(3) Si $b\mid a$ entonces $a=bc$, y si $a\mid b$ entonces $b=ad$. Esto implica que $a=adc$ y así $a(1-cd)=0$. Si $a=0$ entonces $b=0$ trivialmente. Si $a\neq 0$, como $A$ es un dominio íntegro, $1-cd=0$ y así $cd=1$, lo que quiere decir que $c\in A^\times$.

(3)$\Longrightarrow$(2) Es trivial pues $a=ub$ implica que $b\mid a$ y como $u\in A^\times$, $b=u^{-1}a$ y así $a\mid b$.
\end{proof}
\end{prop}

\begin{defn}
Sea $A$ dominio íntegro. Decimos que $a\in A$, $a\neq 0$ es \textbf{irreducible} si $a\notin A^\times$ y sólo es divisible por unidades o asociados. Es decir, si $a=bc$ entonces $b\in A^\times$ o $c\in A^\times$.
\end{defn}

\begin{defn}
Sea $A$ dominio íntegro. Decimos que $a\in A$, $a\neq 0$ es \textbf{primo} si $\langle a\rangle$ es un ideal primo. Equivalentemente, para todos $b,c\in A$ tales que $a\mid bc$ entonces $a\mid b$ o $a\mid c$.
\end{defn}

\begin{prop}
Sea $a\in A$ primo. Entonces $a$ es irreducible.
\begin{proof}
Es claro que $a\notin A^\times $ pues $\langle a\rangle$ es primo y por lo tanto distinto de $A$. Supongamos que $a=bc$. En particular, $a\mid bc$ y como $a$ es primo, $a\mid b$ o $a\mid c$. Si $a\mid b$ entonces $b=aq$ y así $a=aqc$ entonces $a(1-qc)=0$. Como en la proposición anterior, al ser $A$ íntegro y $a\neq 0$, tenemos que $qc=1$ y así $c\in A^\times$. De forma análoga, si $a\mid c$ entonces $b\in A^\times$. Y estamos.
\end{proof}
\end{prop}

\begin{obs}
La vuelta es falsa, es decir, existen elementos irreducibles pero que no son primos en ciertos dominios íntegros. Veamos un ejemplo. Consideremos el anillo $\ZZ[\sqrt{-5}] = \{a+b\sqrt{-5}:a,b\in\ZZ\}\subseteq \CC$. Definamos una función $N:\ZZ[\sqrt{-5}]\to\ZZ$ dada por $N(a+b\sqrt{-5})=a^2 + 5b^2$. Es fácil ver que si $x,y\in\ZZ[\sqrt{-5}]$ entonces $N(xy)=N(x)N(y)$. Notemos entonces que $x$ es una unidad si y sólo si $N(x)=1$.

Ahora, $N(3)=N(2+\sqrt{-5})=N(2-\sqrt{-5})=9$. Veamos primero que $3$ es irreducible. En efecto si $xy=3$ entonces $N(x)N(y)=9$. Entonces, sólo pueden ser $9$ y $1$ o $3$ y $3$. Pero $N(x)\neq 3$ pues $a^2 + 5b^2 = 3$ no tiene solución en $\ZZ$. Entonces alguna de las dos normas es $1$ y así uno de los dos entre $x,y$ tiene que ser una unidad. Entonces $3$ es irreducible.

Pero $9=3\cdot 3 = (2+\sqrt{-5})(2-\sqrt{-5})$. Además, es trivial $3\nmid 2+\sqrt{-5}$ y $3\nmid 2-\sqrt{-5}$ con un argumento de las normas similar al anterior. Entonces $3$ es irreducible pero no primo.

\end{obs}

\begin{obs}
Si $a$ y $b$ son asociados, entonces $a$ es primo si y sólo si $b$ es primo, y $a$ es irreducible si y sólo si $b$ es irreducible.
\end{obs}

\begin{defn}
Sea $A$ un dominio íntegro, $a\in A$, $a\neq 0$. Decimos que $a$ se \textbf{factoriza en forma única por irreducibles} si $a=\mu \prod_{i=1}^r p_i$ con $\mu\in A^\times$ y $p_i\in A$ irreducibles ($r\geq 0$) y cada vez que $a=\mu \prod_{i=1}^r p_i = \nu \prod_{j=1}^s q_j$ con $\mu,\nu\in A^\times$ entonces $r=s$ y $p_i$ es asociado a $q_i$ para todo $1\leq i\leq r$ (salvo orden de los factores).

Un \textbf{dominio de factorización única} (DFU) es un dominio íntegro $A$ tal que todo $a\in A$, $a\neq 0$ se factoriza en forma única por irreducibles.
\end{defn}

\begin{ex}
Son ejemplos $\ZZ$, $k[x]$ (con $k$ cuerpo), $k$ cuerpo. Un no-ejemplo es $\ZZ[\sqrt{-5}]$ (pues $9=3\cdot 3 = (2+\sqrt{-5})(2-\sqrt{-5})$.
\end{ex}

\begin{prop}
Si $A$ es un DFU y $a\in A$, entonces $a$ es primo si y sólo si $a$ es irreducible.
\begin{proof}
Ya vimos que primo implica irreducible siempre. Veamos que si $A$ es un DFU entonces irreducible implica primo. Es decir, queremos ver que si $a\mid bc$ entonces $a\mid b$ o $a\mid c$. Como $A$ es un DFU, tenemos que $b=\mu \prod_{i=1}^r p_i$ y $c=\nu\prod_{j=1}^s q_j$. Esto quiere decir que $bc = (\mu\nu) \prod_{i=1}^r p_i \prod_{j=1}^s q_j$. Como $a\mid bc$ entonces existe un $d\in A$ tal que $ad=bc$. Como $a$ es irreducible, y hay factorización única, entonces $a$ es asociado a algún $p_i$ o $q_j$. Si $a$ es asociado a un $p_i$ entonces $a\mid b$ y si $a$ es asociado a algún $q_j$ entonces $a\mid c$. Y la proposición sigue.
\end{proof}
\end{prop}

\begin{defn}
Sea $A$ un dominio íntegro y $a_1,\ldots , a_n\in A$. Decimos que $d\in A$ es un \textbf{máximo común divisor} de $a_1,\ldots ,a_n$ si $d\mid a_i$ para todo $1\leq i\leq n$ y si $d'\mid a_i$ para todo $1\leq i\leq n$ entonces $d'\mid d$. Por otra parte, decimos que $m\in A$ es un \textbf{mínimo común múltiplo} de $a_1,\ldots , a_n\in A$ si $a_i\mid m$ para todo $1\leq i\leq n$ y si $a_i\mid m'$ para todo $1\leq i\leq n$ entonces $m\mid m'$.
\end{defn}

\begin{obs}
Si $d$ y $d'$ son dos máximo común divisor de $a_1,\ldots , a_n$ distintos, entonces $d\mid d'$ y $d'\mid d$, entonces son asociados. Lo mismo pasa para mínimo común múltiplo. Es decir, son únicos salvo asociados.
\end{obs}

\begin{prop}
En un $A$ DFU existen máximos comunes divisores y mínimos comunes múltiplos.
\begin{proof}
La idea va a ser la misma que para $\ZZ$. Lo que hacíamos era factorizar a los números en primos y tomar los comunes con el menor exponente y los comunes y no comunes con el mayor exponente.

Sea $P$ un conjunto tal que tiene exactamente un representante de cada clase de irreducibles. Dado $a\neq 0$, $a\in A$ entonces $a=\mu\prod_{p\in P} p^{n_p}$ con $n_p\geq 0$ casi todos $0$. Llamo $n_p=\ord_p(a)$. Entonces, afirmo que $\mcm (a_1,\ldots , a_n) = \prod_{p\in P}p^{r_p}$ donde $r_p=\max\{\ord_p(a_i)\}$, y $\mcd (a_1,\ldots ,a_n) = \prod_{p\in P}p^{s_p}$ donde $s_p=\min\{\ord_p(a_i)\}$. Es fácil chequear que cumplen las definiciones.
\end{proof}
\end{prop}

\begin{prop}
Sean $a,b,c\in A$ con $A$ dominio íntegro y $a,b,c\neq 0$. Si $\langle a,b\rangle = \langle c\rangle$, entonces $c$ es un $\mcd (a,b)$.
\begin{proof}
Como $\langle a\rangle , \langle b\rangle\subseteq \langle a,b\rangle \subseteq \langle c\rangle$ tenemos que $c\mid a$ y $c\mid b$.

Ahora, como $\langle c\rangle \subseteq \langle a,b\rangle$ existen $r,s\in A$ tal que $c=ra+sb$. Si $m\mid a$ y $m\mid b$, entonces $m\mid ra+sb$. Pero esto quiere decir que $m\mid c$. Entonces $c$ cumple la definición de $\mcd (a,b)$. Y listo.
\end{proof}
\end{prop}

\begin{obs}
Si $A$ es un DIP, es bastante fácil ver que vale la vuelta.
\end{obs}

\begin{prop}
Sea $A$ un DIP, $a\in A$, $a\neq 0$. Probar que $a$ es irreducible si y sólo si $\langle a\rangle$ es maximal.
\begin{proof}
($\Longrightarrow$) Supongamos que $\langle a\rangle \subseteq I$ con $I$ un ideal de $A$. Como $A$ es un DIP, existe un $b\in A$ tal que $I=\langle b\rangle$. Entonces tenemos que $\langle a\rangle \subseteq \langle b\rangle$, que implica claramente que $b\mid a$. O sea, existe un $c\in A$ tal que $a=bc$. Como $a$ es irreducible, entonces $b\in A^\times$ o $c\in A^\times$. Si $b\in A^\times$ entonces $1\in \langle b\rangle$ y así $I=\langle b\rangle = A$. Si $c\in A^\times$ entonces $\langle a\rangle =\langle b\rangle$ (es trivial ver la doble inclusión). Esto quiere decir que $\langle a\rangle$ es maximal, como queríamos.

($\Longleftarrow$) Es muy similar. Supongamos que $b\mid a$, entonces $\langle a\rangle \subseteq \langle b\rangle$. Pero como $\langle a\rangle$ es un ideal maximal, debemos tener que $\langle a\rangle=\langle b\rangle$ o $\langle b\rangle = A$. En el primer caso, esto quiere decir que $a\mid b$ y $b\mid a$. Es decir, $a$ y $b$ son asociados. En el segundo caso, esto quiere decir que $1\in \langle b\rangle$. Es decir, existe un $c$ tal que $bc=1$. Por lo tanto, $b\in A^\times$. Entonces $a$ es irreducible. Y la proposición sigue.
\end{proof}
\end{prop}

\begin{cor}
Sea $A$ DIP, $a\in A$ irreducible, entonces $a$ es primo.

Sea $A$ DIP, $I\subseteq A$ es un ideal maximal si y sólo si es un ideal primo.
\end{cor}

\begin{teo}[DIP$\Longrightarrow$DFU]
Si $A$ es un DIP entonces $A$ es un DFU.
\begin{proof}
Primero veamos la existencia de factorización. Sea $a\in A$ no nulo. Quiero ver que se puede factorizar como $a=\mu \prod p_i$. Puedo suponer que $a\notin A^\times$ pues si no tiene trivialmente factorización.

Supongamos que $a$ no se factoriza. Entonces, no puede ser irreducible pues si no tendría una factorización en irreducibles (sí mismo). Entonces, existen $a_1,b_1\in A$, $a_1,b_1\notin A^\times$ tales que $a=a_1b_1$. Como $a$ no se factoriza, en particular, si $a_1$ y $b_1$ se factorizaran tendríamos que $a$ se factoriza. Entonces $a_1$ o $b_1$ no se factoriza. Supongamos sin pérdida de la generalidad que $a_1$ no se factoriza. Por inducción, puedo construir una sucesión $(a_n)_{n\in \NN}\subseteq A$ y $(b_n)_{n\in NN}\subseteq A$ tales que $a_{n}=a_{n+1}b_{n+1}$ con $a_i,b_i\notin A^\times$. Entonces, tenemos que $\langle a_n\rangle \subseteq \langle a_{n+1}\rangle$ pues $a_{n+1}\mid a_n$. Además, la contención es estricta pues si $\langle a_n\rangle = \langle a_{n+1}\rangle$ tendríamos que $b_{n+1}$ es una unidad por ser $A$ íntegro (es decir, si $a_n c= a_{n+1}$, como $a_n = a_{n+1}b_{n+1}$ tendríamos que $a_n = a_n c b_{n+1}$ y así $a_n(1-cb_{n+1})=0$ y como $a_n\neq 0$, tendríamos que $cb_{n+1}=1$.

Entonces, tenemos una cadena ascendente de ideales $\langle a_1\rangle \subseteq \ldots \subseteq \langle a_n\rangle \subseteq \ldots$. Consideremos entonces $I=\displaystyle\bigcup_{n\in\NN} \langle a_n\rangle$. Es claramente un ideal pues es la unión de una cadena de ideales (para probar que existen los ideales maximales hicimos una cuenta similar para aplicar el Lema de Zorn). Como $A$ es un DIP, entonces $I=\langle c\rangle$ para algún $c\in A$. Entonces $\displaystyle\bigcup_{n\in\NN}\langle a_n\rangle = \langle c\rangle$. Esto quiere decir que $c$ está en alguno de los ideales de la unión, por ejemplo $c\in \langle a_{n_0}\rangle $. Pero esto implica que $\langle c\rangle \subseteq \langle a_{n_0}\rangle$, y así $I\subseteq \langle a_{n_0}\rangle$. Entonces $\langle a_{n_0 +1}\rangle \subseteq \langle a_{n_0}\rangle$. Pero entonces tenemos la doble inclusión y así $\langle a_{n_0}\rangle = \langle a_{n_0+1}\rangle$. Absurdo, que provino de suponer que $a$ no se factorizaba como irreducibles.

Ahora, veamos la unicidad. Supongamos que $a=\mu \prod_{i=1}^r p_i = \nu \prod_{j=1}^s q_j$. Tomemos $q_1$. Por la proposición anterior, vimos que $q_1$ es primo por ser irreducible en un DIP. Entonces, $q_1\mid a = \mu\prod_{i=1}^{r}p_i$. Como es primo, $q_1\mid p_i$ para algún $i$. Reordenando podemos suponer que $i=1$. Entonces, cancelando obtengo $\tilde{\mu}\prod_{i=2}^{r}p_i = \nu \prod_{j=2}^s q_j$. Continuando este proceso inductivamente, la unicidad sigue.

\end{proof}
\end{teo}

\begin{defn}
Un \textbf{dominio euclideano} es un dominio íntegro $A$ que admite alguna función $d:A-\{0\}\to \NN_0$ que cumple que para todos $a,b\in A$, $b\neq 0 $ existen $q,r\in A$ (no necesariamente únicos) tales que $a=bq+r$ con $r=0$ o $d(r)<d(b)$.
\end{defn}

\begin{obs}
$d$ no es parte de la estructura algebraica de $A$ euclideano. Además, en general, $d$ no es única. Algunos autores piden también que si $r\mid s$ entonces $d(r)\leq d(s)$. Sin embargo, si existe una $d$ como la definimos nosotros se puede construir una $\tilde{d}$ que cumpla también esta propiedad.
\end{obs}

\begin{prop}
Sea $A$ un dominio euclídeo. Entonces $A$ es un DIP.
\begin{proof}
Sea $d$ alguna función euclídea y sea $I$ un ideal no trivial en $A$. Es claro que como $d(I)= \{d(a):a\in I\}$ es un subconjunto de los naturales, existe un mínimo, pues es no vacío. Sea $m$ tal que $d(m) = \min\{d(a):a\in I\}$. Afirmo que $I = \langle m\rangle$. En efecto, supongamos que $a\in I$. Entonces existen $q,r\in A$ tales que $a = mq + r$ y $r=0$ ó $d(r)<d(m)$. Pero $d(m)$ era el mínimo. Entonces $r=0$ y así $m\mid a$ para todo $a\in I$. Entonces, $I\subseteq \langle m\rangle$. Pero como $m\in I$, $\langle m\rangle \subseteq I$, y probamos la doble contención.
\end{proof}
\end{prop}

\begin{obs}
Para probar que algo es un DFU entonces, muchas veces conviene simplemente probar que es euclídeo (aunque las implicaciones DFU$\Longrightarrow$DIP$\Longrightarrow$Euclídeo son falsas).
\end{obs}

\begin{ex}
$\ZZ$ con $d=|-|$, $k[x]$ ($k$ cuerpo) con $d=\gr$, $k$ cuerpo con $d=0$, $\ZZ[i]$ con $d(a+bi)= a^2 + b^2$.
El único no-trivial de verificar es el último.

Sean $\alpha = a+bi$ y $\gamma = c+di$, con $\gamma\neq 0$. Entonces, $\alpha / \gamma = s+ri$ con $s,r\in\QQ$. Tomemos $m,n\in\ZZ$ tales que $|m-r|\leq \dfrac{1}{2}$ y $|n-s|\leq \dfrac{1}{2}$. Si tomo $\delta= m+ni$, entonces $d(\alpha/\gamma - \delta) = (m-r)^2 + (n-s)^2 \leq \dfrac{1}{4} + \dfrac{1}{4} = \dfrac{1}{2}$. Entonces, tomando $\rho = \alpha - \delta \gamma$, tenemos que $\rho=0$ o $d(\rho) = d(\gamma (\alpha/\gamma - \delta)) \leq \dfrac{d(\gamma)}{2}<d(\gamma)$.

Si consideramos $\ZZ[\omega]$ con $\omega$ una raíz cúbica de la unidad, se puede hacer una cuenta similar y ver que es euclídeo también.

A $\ZZ[i]$ se los llama \textbf{enteros gaussianos} y a $\ZZ[\omega]$ se los llama \textbf{enteros de Eisenstein}.

\end{ex}

\section{Localización}

¿Cómo se construye a $\QQ$ a partir de $\ZZ$? Consideramos $\ZZ\times (\ZZ-\{0\}) = \{(a,b) : a\in \ZZ , b\in \ZZ-\{0\}\}$ y una relación de equivalencia $\sim$ tal que $(a,b)\sim(c,d)$ si y sólo si $ad=bc$. Es fácil probar que $\sim$ es de equivalencia.

Ahora, denotemos $\dfrac{a}{b}$ a la clase de equivalencia de $(a,b)$, y el conjunto de esas clases va a ser $\QQ$, equipado con la suma y multiplicación obvias.

Esto nos motiva a la siguiente definición:

\begin{defn}
Sea $A$ dominio íntegro. Se define el \textbf{cuerpo de fracciones} de $A$ como el conjunto de clases de equivalencia de $A\times (A-\{0\})=\{(a,b) : a\in A, b\in A-\{0\}\}$ por la relación $(a,b)\sim(c,d)$ si y sólo si $ad=bc$. Como antes, $\dfrac{a}{b}$ es la clase de $(a,b)$ por esa relación. Es decir, el cuerpo de fracciones es $k=\left\{\dfrac{a}{b} : a,b\in A , b\neq 0\right\}$, con la suma $\dfrac{a}{b}+\dfrac{c}{d} = \dfrac{ad+bc}{bd}$ y el producto $\dfrac{a}{b}\dfrac{c}{d} = \dfrac{ac}{bd}$.
\end{defn}

\begin{obs}
Hay que ver la buena definición de casi todas las cosas en la definición anterior.

En efecto, $\sim$ es una relación de equivalencia. Es fácil ver que es simétrica y reflexiva. Ahora, es transitiva pues si $ad=bc$ y $cf=de$ entonces $acf=ade = bce$ y así $c(af-be)=0$. Como $A$ es íntegro, $c=0$ o $af-be=0$. Si $c=0$ entonces $a=0$ o $d=0$. Pero $d\neq 0$ por definición (es el segundo elemento del par), entonces $a=0$. De forma análoga, $de=0$ así que $d=0$ o $e=0$ y así $e=0$. Entonces, vale trivialmente que $af-be=0$. Y listo.

Ahora, falta ver la buena definición de las operaciones. Si $\dfrac{a}{b}=\dfrac{a'}{b'}$ y $\dfrac{c}{d}=\dfrac{c'}{d'}$ entonces, como $ab'=a'b$, tenemos que $ab'd^2 = a'bd^2$, y así $ab'd^2 + bb'cd = a'bd^2 + bb'cd$. Es decir, $bd(a'd + b'c) = b'd(ad+bc)$. Entonces, $\dfrac{ad+bc}{bd} = \dfrac{a'd+b'c}{b'd}$. Ahora, haciendo el mismo truco, $\dfrac{a'd+b'c}{b'd} = \dfrac{a'd'+b'c'}{b'd'}$. Y así $\dfrac{a}{b}+\dfrac{c}{d} = \dfrac{a'}{b'}+\dfrac{c'}{d'}$ como queríamos.

Para la multiplicación, hacemos el mismo truco, $ab' = a'b$ y así $ab'cd=a'bcd$, entonces $(ac)(b'd)=(bd)(a'c)$, que implica $\dfrac{ac}{bd}=\dfrac{a'c}{b'd}$. Y nuevamente, haciendo lo mismo $\dfrac{a'c}{b'd} = \dfrac{a'c'}{b'd'}$. Por lo tanto $\dfrac{a}{b}\cdot \dfrac{c}{d} = \dfrac{a'}{b'}\cdot \dfrac{c'}{d'}$.

Finalmente, es trivial ver que es un cuerpo, pues es conmutativo y $\dfrac{a}{b}\neq 0$ sii $a\neq 0$, entonces tiene un inverso trivial: $\dfrac{b}{a}$.

\end{obs}

Lo más importante para poder generalizar la construcción de este cuerpo es notar que tenemos una propiedad universal:

\begin{prop}[Propiedad Universal del Cuerpo de Fracciones]

Sea $A$ dominio íntegro, $k$ el cuerpo de fracciones de $A$ y $\iota:A \hookrightarrow k$ la inclusión canónica.

Para todo anillo conmutativo $B$ y morfismo de anillos $f:A\to B$ tal que $f(a)\in B^\times$ para todo $a\neq 0$, entonces existe un único morfismo de anillos $\overline{f}:k\to B$ tal que $\overline{f}(\iota(a))=f(a)$. O sea que $\overline{f}\left(\dfrac{a}{1}\right) = f(a)$. Es decir, el siguiente diagrama conmuta:
\begin{tikzcd}[row sep=3em,column sep=2.5em,minimum width=3em]
A \arrow[hookrightarrow]{d}[font=\normalsize, left]{\iota} \arrow{r}[font=\normalsize]{f} & B \\
k\arrow[dashed]{ru}[font=\normalsize, right]{\overline{f}}
\end{tikzcd}
\begin{proof}
Si tuvieramos un morfismo $\overline{f}$ entonces, es fácil ver que tiene que cumplir que $\overline{f}\left(\dfrac{a}{b}\right) = f(a) f(b)^{-1}$. Pero es fácil ver que es un morfismo de anillos. Entonces existe y es único, y la proposición sigue.
\end{proof}
\end{prop}

\begin{cor}
Sea $A$ dominio íntegro, $k$ su cuerpo de fracciones y $E$ un cuerpo. Sea $j:A\to E$ morfismo de anillos inyectivo. Entonces, $j(a)\in E^\times$ para todo $a\neq 0$. Por la Propiedad Universal del Cuerpo de Fracciones, $\exists ! \overline{j}:k\to E$ tal que $\overline{j}(\iota(a)) = j(a)$. Como $k$ es cuerpo, $\overline{j}$ es inyectivo (pues $\ker \overline{j}$ es un ideal, y los únicos ideales de $k$ cuerpo son triviales, y como tenemos que $\overline{j}({\iota}(a))=j(a)$ no es identicamente cero). Entonces $k\subseteq E$.

Eso quiere decir que el cuerpo de fracciones es el cuerpo más chico que contiene a $A$.
\end{cor}

Ahora, generalicemos todo esto.
\begin{obs}
$A$ dominio íntegro. $S=A-\{0\}$ cumple que $1\in S$ y $x,y\in S\Longrightarrow xy\in S$.
\end{obs}
\begin{defn}
Sea $A$ un anillo conmutativo. Un subconjunto $S\subseteq A$ se dice \textbf{multiplicativamente cerrado} si $1\in S$ y $\forall x,y\in S\Longrightarrow xy\in S$.
\end{defn}

Sea $A$ un anillo conmutativo y $S$ un subconjunto multiplicativamente cerrado. Consideremos $A\times S = \{(a,s):a\in A,s\in S\}$ y defino una relación $(a,s)\sim (b,t)$ sii existe $v\in S$ tal que $v(ad-bc)=0$. Es fácil ver que es una relación de equivalencia. En efecto, es trivial que es simétrica y reflexiva. Para ver que es transitiva, supongamos que $(a,s)\sim (b,t)$ y $(b,t)\sim (c,u)$. Entonces, existen $v,w\in S$ tales que $v(at-bs)=0$ y $w(bu-tc)=0$. Es decir, $vat = vbs$ y $wbu=wtc$.
Pero entonces, $wuvat = wuvbs$ (multiplicando a izquierda por $wu$ en la primera). Es decir, $wtvau = wtcvs$ y así $wtv(au-vs)=0$. Claramente $wtc\in S$ pues $w,t,v\in S$ y es multiplicativamente cerrado.

Denotamos por $\dfrac{a}{s}$ a la clase de equivalencia de $(a,s)$. Tenemos la siguiente definición:

\begin{defn}
La \textbf{localización} de $A$ por $S$ se define como $$S^{-1}A = \left\{\dfrac{a}{s} : a\in A, s\in S\right\}$$ y está equipada con la suma $\dfrac{a}{s}+\dfrac{b}{t} = \dfrac{at+bs}{st}$ y el producto $\dfrac{a}{s}\cdot\dfrac{b}{t} = \dfrac{ab}{st}$.
\end{defn}

\begin{obs}
La buena definición de las operaciones sale de forma totalmente análoga a la buena definición de las operaciones del cuerpo de fracciones. Sólo que acá $S^{-1}A$ no resulta necesariamente un cuerpo (pero sí un anillo).
\end{obs}

\begin{obs}
Tenemos un morfismo de anillos $\iota : A\to S^{-1}A$, $a\mapsto \dfrac{a}{1}$. Notemos que $\iota$ no necesariamente es inyectivo pues $\dfrac{a}{1}=\dfrac{b}{1}$ si y sólo si existe un $v$ tal que $v(a-b)=0$, entonces si hay divisores de cero pierdo la inyectividad.

Sea $s\in S$. Entonces $\iota(s) = \dfrac{s}{1}\in (S^{-1}A)^\times$ pues $\dfrac{s}{1}\dfrac{1}{s}=\dfrac{1}{1}=1$. Entonces, si tomo $\dfrac{s}{t}$ con $s,t\in S$, entonces $\dfrac{s}{t}\in (S^{-1}A)^\times$ pues $\dfrac{s}{t}\dfrac{t}{s} = \dfrac{1}{1}=1$.
\end{obs}

\begin{teo}[Propiedad Universal de la Localización]

Sea $A$ un anillo conmutativo no trivial y $S\subseteq A$ multiplicativamente cerrado. Sea $\iota : A\to S^{-1}A$ la localización respecto a $S$. Para todo anillo conmutativo $B$ y morfismo de anillos $f:A\to B$ tal que $f(s)\in B^\times \; \forall s\in S$ existe un único morfismo de anillos $\overline{f}:S^{-1}A\to B$ tal que $\overline{f}\circ \iota = f$. Es decir, el siguiente diagrama conmuta: 
\begin{tikzcd}[row sep=3em,column sep=2.5em,minimum width=3em]
A \arrow{d}[font=\normalsize, left]{\iota} \arrow{r}[font=\normalsize]{f} & B \\
S^{-1}A\arrow[dashed]{ru}[font=\normalsize, right]{\overline{f}}
\end{tikzcd}
\begin{proof}
Es claro que si hay un morfismo que hace conmutar a ese diagrama tiene que cumplir que $\overline{f}\left(\dfrac{a}{s}\right) = f(a)f(s)^{-1}$. Pero es fácil comprobar que es un morfismo usando la conmutatividad de $A$.
\end{proof}
\end{teo}

\begin{obs}
La propiedad universal nos dice que $S^{-1}A$ es el anillo más "`cercano"' a $A$ que cumple esto (pues cualquier otro anillo junto a un morfismo que cumple esta propiedad tiene que factorizarse por $S^{-1}A$).
\end{obs}

\begin{ex}
Sea $A$ un dominio íntegro, $S=A-\{0\}$. Entonces $S^{-1}A$ es el Cuerpo de Fracciones. En el caso en que $A=\ZZ$ ya vimos que $k=\QQ$ es el cuerpo de fracciones. En el caso en que $A=E[x]$, se puede probar que el cuerpo de fracciones es el conjunto de funciones racionales $E(x) = \left\{\dfrac{f}{g}:f,g\in E[x], g\neq 0\right\}$ (es igual al ejemplo anterior en esencia). En la misma vena, si $B$ es un dominio íntegro, entonces $B[x]$ lo es y la localización de $B[x]$ por $B$ es $E(x)$ las funciones racionales sobre $E$ el cuerpo de fracciones de $B$.
\end{ex}

\begin{defn}[Localización por el complemento de un ideal primo]
Sea $A$ un anillo conmutativo y $\mathfrak{p}\subseteq A$ un ideal primo. Tomemos $S=A-\mathfrak{p}$. Notemos que $S$ es multiplicativamente cerrado. En efecto, $1\in S$ pues $1\notin \mathfrak{p}$ pues es un ideal propio. Además, si $x,y\in S$, si tuvieramos $xy\in \mathfrak{p}$, al ser primo, tendríamos $x\in\mathfrak{p}$ o $y\in\mathfrak{p}$, ambas alternativas absurdas pues supusimos $x,y\in S$.

A esta localización $S^{-1}A$ la denotaremos por $A_\mathfrak{p}$.
\end{defn}

\begin{prop}
$\dfrac{a}{s}\in A_\mathfrak{p}^\times$ si y sólo si $a\notin \mathfrak{p}$ (equivalentemente $a\in S$).
\begin{proof}
($\Longleftarrow$) Ya vimos que esta implicación es cierta para cualquier localización.

($\Longrightarrow$) Sea $\dfrac{a}{s}\in A_\mathfrak{p}^\times$. Entonces, existe $\dfrac{b}{t}\in A_\mathfrak{p}$ tal que $\dfrac{b}{t}\dfrac{a}{s}=\dfrac{1}{1}$. Es decir, existe $v\in S$ tal que $v(ba-ts)=0$. Por lo tanto, $vba = vts$. Si $a\in \mathfrak{p}$ entonces $vba\in\mathfrak{p}$ por ser un ideal. Pero como $v,t,s\in S$ y es multiplicativamente cerrado, $vts\in S=A-\mathfrak{p}$. Esto es un absurdo, que provino de suponer que $a\in\mathfrak{p}$.
\end{proof}
\end{prop}

\begin{obs}
Si consideramos $M= \left\{\dfrac{a}{s} : a\in\mathfrak{p}, s\notin\mathfrak{p}\right\}\subseteq A_\mathfrak{p}$. Afirmo que $M$ es un ideal de $A_\mathfrak{p}$. Esto es fácil, pues si $\dfrac{a}{s},\dfrac{b}{t}\in M$ entonces $\dfrac{a}{s}+\dfrac{b}{t} = \dfrac{at+bs}{st}$. Pero como $a,b\in\mathfrak{p}$ entonces $at+bs\in\mathfrak{p}$ y así $\dfrac{at+bs}{st}\in M$. Ahora, si $\dfrac{a}{s}\in M$, $\dfrac{a}{s}\dfrac{b}{t} = \dfrac{ab}{st}$, pero $ab\in\mathfrak{p}$ pues $a\in\mathfrak{p}$ y es un ideal. Entonces el producto está.

Más aún, este ideal $M$ es maximal. En efecto, $M$ consiste de todas las no unidades, asi que si agregaramos cualquier elemento sería una unidad y el ideal generado sería todo $A_\mathfrak{p}$. Además, es el único ideal maximal pues si hubiera otro, sus elementos tendrían que ser no unidades y así estaría contenido en $M$, y por la maximalidad sería $M$.
\end{obs}

\begin{defn}
Un anillo conmutativo se dice \textbf{local} si tiene un único ideal maximal.
\end{defn}

\begin{prop}
Sea $A$ un anillo conmutativo, $\mathfrak{m}\in A$ un ideal maximal tal que para todo $x\in\mathfrak{m}$, $1+x\in A^\times$. Entonces $A$ es local.
\begin{proof}
Supongamos que existe otro $\mathfrak{m}'$ maximal. Entonces, existe un $x\in \mathfrak{m}$ tal que $x\notin \mathfrak{m}'$. Pero entonces, como $\mathfrak{m}'$ es maximal, $\mathfrak{m}'+ \langle x\rangle = A$ pues es un ideal que no es $\mathfrak{m}'$ pues $x\notin \mathfrak{m}'$. Entonces, tenemos que $t + ax = 1$ para ciertos $t\in\mathfrak{m}'$ y $a\in A$. Es decir, $t = 1 + (-a)x$. Pero $(-a)\in A$ y $x\in \mathfrak{m}$, entonces $(-a)x\in \mathfrak{m}$, que implica que $1 + (-a)x\in A^\times$. Pero entonces $t\in A^\times$. Esto quiere decir que $1\in \mathfrak{m}'$. Absurdo, que vino de suponer que existe otro ideal maximal. Entonces $A$ es local, como queríamos ver.
\end{proof}
\end{prop}

\begin{ex}
Este ejemplo es bastante motivador. Sea $X$ un espacio topológico y sea $A=C(X) = \{f:X\to\RR : f\text{ continua}\}$ el anillo de funciones reales continuas sobre $X$, y sean $x_0\in X$, $S= \{f\in A : f(x_0)\neq 0\}$.

Entonces, podemos ver que $S^{-1}A = \left\{\dfrac{f}{g} : f,g\in A,\; g(x_0)\neq 0\right\}$. Este cociente está para indicar la clase de equivalencia como veníamos haciendo y no una división de funciones.

Supongamos que $f_1,f_2\in A$ son tales que $\dfrac{f_1}{1}=\dfrac{f_2}{1}$. Esto pasa si y sólo si existe un $h\in S$ tal que $h(f_1-f_2)=0$. Como $h\in S$, $h(x_0)\neq 0$. Por la continuidad de $h$ existe un entorno $U$ de $x_0$ en $X$ tal que $\left. h \right|_{U}\neq 0$. Esto quiere decir, que si $x\in U$ entonces $h(x)$ es inversible en $\RR$, y como $h(x)(f_1(x)-f_2(x)) = 0$, debemos tener que $f_1(x)=f_2(x)$.

Por lo tanto, dos funciones $f_1,f_2\in A$ tienen la misma imagen bajo $\iota: A\to S^{-1}A$ si y sólo si existe un entorno $U'$ de $x_0$ sobre el cual coinciden. Esto motiva el nombre de localización (pues estamos observando que el comportamiento local de estas funciones alrededor de $x_0$ es el mismo) y además que $\iota : A\to S^{-1}A$ no siempre es inyectiva.

\end{ex}

\section{Teorema Chino del Resto}

\begin{defn}

Sea $A$ un anillo conmutativo y sean $a_1,\ldots ,a_n\subseteq A$ ideales. La \textbf{suma de ideales} se define como $a_1+\ldots + a_n = \left\langle\displaystyle\bigcup_{i=1}^{n} a_i \right\rangle$. Más generalmente, dada una familia de ideales $\{a_j\}_{j\in J}$ definimos la suma $\displaystyle\sum_{j\in J}a_j = \left\langle \bigcup_{j\in J} a_j \right\rangle$.
\end{defn}
\begin{obs}
Sea $\{a_j\}_{j\in J}$ una familia de ideales. La \textbf{intersección} $\displaystyle\bigcap_{j\in J}a_j$ es un ideal.
\end{obs}
\begin{defn}
Sean $I,J\subseteq A$ ideales. Definimos el \textbf{producto} de los ideales $I,J$ como $IJ = \langle \{xy : x\in I, y\in J\}\rangle = \left\{ \displaystyle\sum_{\text{finita}} x_iy_j : x_i\in I, y_j\in J \right\}$.

\end{defn}

\begin{obs}
Si $A=\ZZ$, $I=n\ZZ$, $J=m\ZZ$, entonces $IJ=(nm)\ZZ$ y además $I\cap J = \mcm (m,n)\ZZ$. En este caso $I\cap J=IJ$ si y sólo si $m$ y $n$ son coprimos. Es decir, si y sólo si $I+J=\ZZ$.
\end{obs}

\begin{prop}
Sea $A$ un anillo conmutativo y sean $I,J\subseteq A$ ideales. Si $I+J=A$ entonces $IJ=I\cap J$.
\begin{proof}
La inclusión $IJ\subseteq I\cap J$ siempre vale pues para cualquier producto $x_iy_j$ con $x_i\in I, y_j\in J$ tenemos que $x_i y_j\in I$ por ser $I$ un ideal, y además $x_iy_j\in J$ por serlo $J$, y así $x_iy_j\in I\cap J$.

Ahora, quiero ver que todo elemento de $I\cap J$ está en $IJ$. En efecto, sea $x\in I\cap J$. Como $A=I+J$ existen $i\in I$, $j\in J$ tales que $x = i+j$. Pero es claro que como $x\in I\cap J$ entonces $i,j\in I\cap J$ (simplemente es restar y usar que son ideales). Además, existen $k\in I$ y $\ell\in J$ tales que $k+\ell = 1$. Entonces, $x = x\cdot 1 = (i+j)(k+\ell) = ik + i\ell + jk + j\ell$. Pero cada uno de estos productos está en $IJ$ pues $i,j\in I\cap J$ entonces funcionan como la parte de $I$ o la parte de $J$ del producto según se necesite. Probamos la doble inclusión y la proposición sigue.

\end{proof}
\end{prop}

\begin{obs}
La vuelta es falsa. Tomando $A=\ZZ[x]$, $I=\langle x\rangle$ y $J=\langle x\rangle$ tenemos que $IJ=\langle 2x\rangle = I\cap J$ pero $1\notin \langle 2,x\rangle$.
\end{obs}


\begin{prop}
Sea $\iota:A\to S^{-1}A$ la localización. Si $I$ es un ideal de $A$ entonces $S^{-1}I = \left\{\dfrac{a}{s}:a\in I, s\in S\right\}\subseteq S^{-1}A$ es un ideal. Verificar que $S^{-1}(I+J) = S^{-1}I + S^{-1}J$, $S^{-1}(IJ) = (S^{-1}I)(S^{-1}J)$ y $S^{-1}(I\cap J) = (S^{-1}I)\cap (S^{-1}J)$.
\end{prop}

\begin{defn}
Sea $\{A_j\}_{j\in J}$ una familia de anillos. Definimos el producto de anillos como $\displaystyle\prod_{j\in J}A_j = \{(x_j)_{j\in J} : x_j\in A_j\}$, con las operaciones definidas coordenada a coordenada: $(x_j)_{j\in J} + (y_j)_{j\in J} = (x_j+y_j)_{j\in J}$, $(x_j)_{j\in J}(y_j)_{j\in J} = (x_jy_j)_{j\in J}$, y los neutros $1=(1_j)_{j\in J}$ y $0=(0_j)_{j\in J}$.
\end{defn}

\begin{obs}
Notar que $\left( \displaystyle\prod_{j\in J} A_j\right)^\times = \displaystyle\prod_{j\in J}A_j^\times$ (pues todo vale coordenada a coordenada, así que un elemento es una unidad si y sólo si es una unidad coordenada a coordenada).
\end{obs}

\begin{defn}
Sea $A$ un anillo conmutativo, $I\subseteq A$ un ideal. Sean $a,b\in A$. Decimos que $a\equiv b\pmod{I}$ si y sólo si $a-b\in I$.
\end{defn}

\begin{obs}
Es fácil notar que $\equiv$ es una relación de equivalencia. En efecto, es simétrica pues $0 = a-a\in I$, es reflexiva pues si $b-a\in I$ entonces $(-1)(b-a) = a-b\in I$. Finalemente es transitiva pues si $a-b\in I$ y $b-c\in I$ entonces $(a-b)+(b-c) = a-c\in I$.

Además, notemos que si $a_1\equiv b_1\pmod{I}$ y $a_2\equiv b_2\pmod{I}$ entonces se cumple que $a_1+a_2 \equiv b_1+b_2\pmod{I}$ y $a_1a_2\equiv b_1b_2\pmod{I}$. En efecto, $(a_1+a_2) - (b_1+b_2) = (a_1-b_1) + (a_2-b_2)\in I$ pues $a_1-b_1\in I$ y $a_2-b_2\in I$. Además, $a_1a_2 - b_1b_2 = a_1a_2 + a_1b_2 - a_1b_2 - b_1b_2 = a_1(a_2-b_2) + (a_1-b_1)b_2 \in I$.
\end{obs}

\begin{teo}[Teorema Chino del Resto para Anillos Conmutativos]

Sea $A$ un anillo conmutativo. Dados $a_1,\ldots , a_n\subseteq A$ ideales tales que $a_i+a_j = A \; \forall i\neq j$ y $x_1,\ldots , x_n\in A$ entonces existe un $x\in A$ tal que $x\equiv x_i \pmod{a_i} \; \forall 1\leq i\leq n$.

\begin{proof}

Primero hagamos el caso $n=2$. Si $a_1+a_2 = A$ entonces existen $\alpha_1\in a_1$ y $\alpha_2\in a_2$ tales que $\alpha_1+\alpha_2 = 1$. Tomemos $x = x_2\alpha_1 + x_1\alpha_2$. Es fácil ver que este elemento satisface que $x\equiv x_1\pmod{a_1}$ y $x\equiv x_2\pmod{a_2}$.

Ahora hagamos el caso general. Sean $a_1,\ldots , a_n$ los ideales. Tomemos $a_j$ para un $j$ fijo. Como $a_j + a_k = A \; \forall k\neq j$, existen $\alpha_k\in a_j$ y $\beta_k \in a_k$ tales que $\alpha_k + \beta_k = 1$. Consideremos entonces $\displaystyle\prod_{k\neq j}(\alpha_k + \beta_k) = 1$. Desarrollando ese producto, es claro que está en $a_1 + \displaystyle\prod_{k\neq j}a_k$. Entonces $a_1 + \displaystyle\prod_{k\neq j} a_k = A$. Por el caso $n=2$ existe un $y_j\in A$ que cumple $y_j \equiv 1 \pmod{a_j}$ y $y_j \equiv 0 \pmod{\displaystyle\prod_{k\neq j} a_k}$. La segunda condición es equivalente a $y_j \equiv 0 \pmod{a_k} \; \forall k\neq j$.

Entonces, para cada $1\leq j\leq n$ tenemos estos $y_j$. Consideremos entonces $x= \displaystyle\sum_{i=1}^{n} x_iy_i$. Claramente, mirando módulo $a_j$ tenemos que los $y_k$ con $k\neq j$ se anulan y queda sólo $y_j\equiv 1\pmod{a_j}$. Entonces $x \equiv x_jy_j\equiv x_j\pmod{a_j}$, y se cumple lo pedido. El teorema sigue.
\end{proof}
\end{teo}

\begin{cor}
Sea $A$ un anillo conmutativo, $a_1,\ldots , a_n\subseteq A$ ideales con $a_i+a_j = A \; \forall i\neq j$. Considero las proyecciones a los cocientes $\pi_i :A\to A/a_i$ para cada $1\leq i\leq n$. Definimos $\pi:A\to \displaystyle\prod_{i=1}^n A/a_i$ por $a\mapsto (\pi_j(a))_{1\leq j\leq n}$. Por el Teorema Chino esta función es claramente sobreyectiva. Pero además, $\ker \pi = \{a\in A : \pi_j(a) = \overline{0}\in A/a_j\} = \displaystyle\bigcap_{j=1}^n a_j$. Por el Primer Teorema de Isomorfismo, tenemos que $A/\displaystyle\bigcap_{j=1}^n a_j \simeq \displaystyle\prod_{j=1}^n A/a_j$.

En el caso de $A=\ZZ$, si $m=p_1^{r_1}\cdots p_s^{r_s}$, tomando $a_i = p_i^{r_i}\ZZ$, entonces por la coprimalidad $\displaystyle\bigcap_{i=1}^n a_i = m\ZZ$. Este corolario nos dice que $\ZZ/m\ZZ \simeq \ZZ/p_1^{r_1}\ZZ \times \ldots \times \ZZ/p_s^{r_s}\ZZ$.
\end{cor}

\begin{defn}
Sea $m\in\NN$. Definimos la \textbf{función phi de Euler} como la cantidad de naturales menores a $m$ coprimos con $m$: $\varphi (m) = \sharp \{1\leq r\leq m : \mcd(r:m)=1\} = |\ZZ_m^\times|$.
\end{defn}

\begin{prop}
Sea $m={p_1}^{r_1}\cdots {p_s}^{r_s}$ con $p_i$ primos distintos. Entonces la función phi de Euler es multiplicativa, es decir, $\varphi(m) = \varphi(p_1^{r_1})\cdots \varphi(p_s^{r_s})$.
\begin{proof}
Tenemos el isomorfismo de anillos $\ZZ_m \simeq \displaystyle\prod_{i=1}^s \ZZ_{{p_i}^{r_i}}$ que nos induce el isomorfismo de grupos entre los grupos de unidades $\ZZ_m^\times \simeq \left(\displaystyle\prod_{i=1}^s \ZZ_{{p_i}^{r_i}}\right)^\times = \displaystyle\prod_{i=1}^s \ZZ_{{p_i}^{r_i}}^\times$. Entonces $|\ZZ_m^\times| = \displaystyle\prod_{i=1}^s |\ZZ_{{p_i}^{r_i}}^\times|$, y así $\varphi(m) = \displaystyle\prod_{i=1}^s \varphi({p_i}^{r_i})$.
\end{proof}
\end{prop}

\begin{prop}
Sea $p$ un primo y $r\in\NN$. Entonces $\varphi(p^r) = (p-1)p^{r-1}$.
\begin{proof}
Procederemos por inducción. Para $n=1$ es trivial que $\varphi(p)=p-1$.

Veamos que si vale para $r$ entonces vale para $r+1$. Consideremos $\tilde{q}:\ZZ_{p^{r+1}}\to \ZZ_{p^r}$ el único morfismo de anillos (es único pues el $1$ tiene que ir al $1$ y genera a todo $\ZZ_{p^{r+1}}$). Entonces esto induce un morfismo de grupos entre los grupos de unidades: $q:\ZZ_{p^{r+1}}^\times \to \ZZ_{p^r}^\times$. Este morfismo $q$ es epimorfismo pues si $x\in \ZZ_{p^r}^\times$ entonces $\mcd(x:p^r)=1$ y así $\mcd(x:p^{r+1})=1$, que implica $x\in \ZZ_{p^{r+1}}^\times$. Además, $\ker q = \{x\in \ZZ_{p^{r+1}}^\times : \overline{x}=\overline{1}\in\ZZ_{p^r}^\times\}$. O sea, $x\in \ker q$ si y sólo si satisface que $x=1+ap^r$, con $0\leq a\leq p-1$. Entonces $|\ker q| = p$. Por el Primer Teorema de Isomorfismo tenemos que $|\ZZ_{p^{r+1}}| = p |\ZZ_{p^r}|=p\cdot p^{r-1}(p-1)$ por hipótesis inductiva. Y estamos.
\end{proof}
\end{prop}
\begin{prop}[Teorema de Fermat-Euler]

Sea $x\in\ZZ$, con $\mcd(x:n)=1$. Entonces $x^{\varphi(n)}\equiv 1\pmod{n}$.
\begin{proof}
Como $\mcd(x:n)=1$, entonces $\overline{x}\in\ZZ_n$ la clase del $x$ está en el grupo de unidades de $\ZZ_n$. Pero entonces $\overline{x}^{|\ZZ_m^\times|} = \overline{1}$. Es decir, $x^{\varphi(n)}\equiv 1\pmod{n}$.
\end{proof}
\end{prop}

\section{La estructura de \texorpdfstring{$\ZZ/n\ZZ^\times$}{Z/nZ*}}

En esta sección vamos a ver ciertos resultados útiles en Teoría de Números. Gran parte de esta sección está basada en el capítulo 4 del libro "`A Classical Introduction to Modern Number Theory"' de Ireland y Rosen.

\begin{defn}
Decimos que $g\in\ZZ$ es una raíz primitiva módulo $n$ si su clase es un generador de $\ZZ/n\ZZ^\times$.
\end{defn}

\begin{lem}[Lagrange]
Sea $f\in k[x]$ con $k$ cuerpo. Supongamos que $\gr f= n$. Entonces $f$ tiene como mucho $n$ raíces distintas.
\begin{proof}
Procederemos por inducción sobre $n$. Para $n=1$ esto es trivial. Supongamos que esto es cierto para polinomios de grado $n-1$. Si $f$ no tiene raíces en $k$ ya estamos. Si no, sea $\alpha\in k$ una raíz y podemos escribir $f(x) = q(x)(x-\alpha)$ para algún $q\in k[x]$ con $\gr q = n-1$. Si $\beta$ es otra raíz de $f$, tiene que ser raíz de $q$. Por hipótesis inductiva, $q$ tiene como mucho $n-1$ raíces, y así $f$ tiene como mucho $n$. Y listo.
\end{proof}
\end{lem}

\begin{cor}
Sean $f,g\in k[x]$ con $\gr f = \gr g = n$. Si $f(\alpha_i) = g(\alpha_i)$ para $\alpha_1,\ldots , \alpha_{n+1}$ distintos, entonces $f(x)=g(x)$.
\begin{proof}
Aplicamos el lema anterior a $f(x)-g(x)$.
\end{proof}
\end{cor}

\begin{prop}
Sea $p$ un primo. Si $d\mid p-1$ entonces $x^d\equiv 1\pmod{p}$ tiene exactamente $d$ soluciones.
\begin{proof}
Es claro que $\dfrac{x^{p-1}-1}{x^d -1 } = \displaystyle\sum_{k=0}^{\frac{p-1}{d}-1}(x^d)^{k} = g(x)$. Entonces, $(x^{d-1}-1)g(x) = x^{p-1}-1$. Mirando el morfismo canónico de $\ZZ[x]$ a $\ZZ/p\ZZ[x]$ tenemos que $x^{p-1}-\overline{1} = (x^d - \overline{1})\overline{g}(x)$. Si $x^d-\overline{1}$ tuviera menos que $d$ raíces entonces por el lema, tendríamos que $x^{p-1}-\overline{1}$ tiene menos de $p-1$ raíces. Por el Teorema de Fermat-Euler esto es claramente absurdo. Entonces $x^{d}-\overline{1}$ tiene exactamente $d$ raíces.
\end{proof}
\end{prop}

\begin{prop}[Fórmula de Inversión de Möbius]
Definimos la función de Möbius por $\mu(1)=1$ y $\mu(p_1^{r_1}\cdots p_s^{r_s}) = \begin{cases}(-1)^s \text{ si } r_1=\ldots = r_s= 1 \\ 0 \text{ si no }\end{cases}$. Si $f,g:\NN\to \CC$ son dos funciones tales que $f(n) = \displaystyle\sum_{d\mid n} g(d)$ entonces $g(n) = \displaystyle\sum_{d\mid n}f(d)\mu\left(\dfrac{n}{d}\right)$.
\begin{proof}
Definimos la fórmula de inversión de Möbius como $f*g(n) = \displaystyle\sum_{d\mid n}f(d)g\left(\dfrac{n}{d}\right)$. Notemos que las funciones de $\NN$ en $\CC$ con la suma y la convolución de Dirichlet forman un anillo. Sea $1=1(n)$ la función idénticamente $1$. El enunciado es equivalente a decir que $f = g*1$ entonces $g = f*\mu$. Probaremos que $1*\mu(n) = \begin{cases}1 \text{ si }n=1 \\ 0 \text{ si no}\end{cases}$, con lo que multiplicando a derecha por $\mu$ en ambos lados el teorema seguirá.

En efecto, queremos ver que $1*\mu(n)= \displaystyle\sum_{d\mid n} \mu(d)$. Si $n=p_1^{r_1}\cdots p_s^{r_s}$, en la suma, por la definición de la $\mu$, sólo aportan los términos de la forma $p_1^{e_1}\cdots p_s^{e_s}$ con $e_i\in \{0,1\}$. Hay $\displaystyle\binom{s}{k}$ términos con $k$ unos de los $e_i$. Esto quiere decir que $1*\mu (n) = 1+\displaystyle\sum_{k=0}^s \binom{s}{k}(-1)^k = (1-1)^s = 0$. Si $n=1$ esto da trivialmente $\mu(1)=1$. Entonces probamos lo que queríamos. Y la proposición sigue.

\end{proof}
\end{prop}

\begin{teo}
Sea $p$ primo. El grupo $\ZZ/p\ZZ^\times$ es cíclico.
\begin{proof}
Para cada $d\mid p-1$ sea $\psi(d)=\sharp \{x\in \ZZ/p\ZZ^\times : \ord(x)=d\}$. Por una de las proposiciones anteriores, tenemos que los $x\in\ZZ/p\ZZ^\times$ que cumplen $x^d \equiv \overline{1}$ forman un grupo de orden $d$. Entonces $\displaystyle\sum_{m\mid d}\psi(m)=d$. Pero notemos que $\displaystyle\sum_{m\mid d}\varphi(m) = d$ pues, si consideremos los números $\dfrac{1}{d},\dfrac{2}{d},\ldots , \dfrac{d-1}{d},\dfrac{d}{d}$ y expresamos cada uno como fracciones irreducibles, los denominadores serán todos divisores de $d$ y si $m\mid d$ entonces hay exactamente $\varphi(m)$ números de los que escribimos con denominador $m$ después de reducirlos.

Por la fórmula de inversión de Möbius debemos tener que $\psi(d) =\varphi(d)$ para cada $d\mid n$. En particular $\psi(p-1)=\varphi(p-1)>1$ si $p>2$. El caso $p=2$ es trivial. Como probamos la existencia de elementos de orden $p-1$, el grupo de unidades de $\ZZ/p\ZZ$ debe ser cíclico y listo.
\end{proof}
\end{teo}

Ahora necesitaremos de un par de lemas elementales.

\begin{lem}
Sea $\ell\geq 1$ y $a\equiv b\pmod{p^\ell}$. Entonces $a^p \equiv b^p \pmod{p^{\ell+1}}$. 
\begin{proof}
Escribimos $a=b+cp^\ell$. Entonces $a^p = (b+cp^\ell)^p = \displaystyle\sum_{k=0}^p \binom{p}{k}b^k (cp^\ell)^{p-k}$. Si $k<p-1$ entonces los términos se la suma se mueren módulo $p^{\ell+1}$ pues $2\ell \geq \ell + 1$. Entonces, $a^p \equiv b^p + \displaystyle\binom{p}{1}cp^\ell = b^p + cp^{\ell+1}\equiv b^p \pmod{p^{\ell+1}}$. Y el lema sigue.
\end{proof}
\end{lem}

\begin{cor}
Si $\ell\geq 2$ y $p\neq 2$ entonces $(1+ap)^{p^{\ell-2}}\equiv 1 + ap^{\ell-1}\pmod{p^\ell}$, para todo $a\in\ZZ$.
\begin{proof}
Esto sigue por inducción en $\ell$. Para $\ell=2$ la afirmación es trivial. Supongamos que es cierto para algún $\ell\geq 2$ y probemos que es cierto para $\ell+1$. Por el lema anterior, $(1+ap)^{p^{\ell-1}} \equiv (1+ap^{\ell-1})^p \pmod{p^{\ell+1}}$. Pero por el Teorema del Binomio, tenemos que $(1+ap^{\ell-1})^p = 1 + \displaystyle\binom{p}{1}ap^{\ell-1} + S$ con $S$ divisible por $p^{1+2(\ell-1)}$ salvo capaz el término $a^p p^{p(\ell-1)}$. Pero como $\ell\geq 2$, $1+2(\ell-1)\geq \ell+1$ y como $p\geq 3$, $p(\ell-1)\geq \ell+1$. Entonces $p^{\ell+1}\mid S$ y el corolario sigue.
\end{proof}
\end{cor}

\begin{cor}
Si $p\neq 2$ y $p\nmid a$ entonces $p^{\ell-1}$ es el orden de $1+ap$ módulo $p^\ell$.
\begin{proof}
Por el corolario anterior, $(1+ap)^{p^{\ell-1}}\equiv 1+ap^{\ell}\pmod{p^{\ell+1}}$, que implica que $(1+ap)^{p^{\ell-1}}\equiv 1\pmod{p^{\ell}}$. Esto implica que el orden de $1+ap$ divide a $p^{\ell-1}$. Notemos que $(1+ap)^{p^{\ell-2}}\equiv 1+ap^{\ell-1}\pmod{p^{\ell}}$, que implica que $p^{\ell-2}$ no es el orden de $1+ap$ (pues $p\nmid a$). El resultado sigue.
\end{proof}
\end{cor}

\begin{teo}
Sea $p$ un primo impar y $\ell\in\NN$. Entonces $\ZZ/p^{\ell}\ZZ^\times$ es cíclico.
\begin{proof}
Ya probamos que $\ZZ/p\ZZ^\times$ es cíclico. Por lo tanto, existe un generador $g\in \ZZ$ de los restos no triviales módulo $p$. Si $g$ genera los restos módulo $p$ entonces $g+p$ también lo hace (pues tienen la misma clase). Si $g^{p-1}\equiv 1 \pmod{p^2}$ entonces $(g+p)^{p-1} \equiv g^{p-1} + (p-1)g^{p-2}p \pmod{p^2}$ por el teorema del binomio. Pero como $p\nmid (p-1)g^{p-2}$, tenemos que $g+p$ es una raíz primitiva módulo $p^2$. Entonces podemos suponer de entrada que $g$ es una raíz primitiva módulo $p^2$.

Afirmo que $g$ ya es una raíz primitiva módulo $p^\ell$. Supongamos que $g^{p-1} = 1+ap$. Por el corolario anterior, tenemos que $p^{\ell-1}$ es el orden de $g^{p-1}$. Pero si $g^n\equiv 1 \pmod{p^\ell}$ entonces debemos tener que $g^n\equiv 1\pmod{p}$ y así $p-1\mid n$.
Esto quiere decir que si $g^n \equiv 1 \pmod{p^\ell}$ entonces $p^{\ell-1}(p-1)\mid n$. Esto prueba que es una raíz primitiva módulo $p^{\ell-1}$ y listo.

\end{proof}
\end{teo}

\begin{teo}
Sea $\ell\in\NN$. Existen raíces primitivas módulo $2^\ell$ si y sólo si $\ell=1,2$.
\begin{proof}
Notemos que $1$ es una raíz primitiva módulo $2$ y $3$ es una raíz primitiva módulo $4$. Ahora, supongamos que $\ell\geq 3$.

Afirmo que $5^{2^{\ell-3}}\equiv 1 +2^{\ell-1} \pmod{2^\ell}$. Lo probaremos por inducción. Esto es cierto para $\ell=3$. Supongamos que vale para algún $\ell\geq 3$ y veamos que vale para $\ell+1$. Notemos que $(1+2^{\ell-1})^2 = 1+2^\ell + 2^{2\ell-2}$ y que $2\ell-2\geq \ell+1$ para $\ell\geq 3$. Por el primer lema de esta última serie de lemas, $(5^{2^{\ell-3}})^2 \equiv (1+2^{\ell-1})^2 \equiv 1+2^{\ell}\pmod{2^{\ell+1}}$, y el paso inductivo está completo.

Como $5^{2^{\ell-2}}\equiv 1 \pmod{2^\ell}$ pero $5^{2^{\ell-3}}\not\equiv 1\pmod{2^\ell}$ tenemos que $2^{\ell-2}$ es el orden de $5$ módulo $2^\ell$.

Consideremos el conjunto $\{(-1)^a 5^b : a\in \{1,2\} , 0\leq b<2^{\ell-2}\}$. Afirmamos que estos números tienen distintos restos módulo $2^{\ell}$. Si $(-1)^a5^b \equiv (-1)^{a'}5^{b'}\pmod{2^\ell}$ para $\ell\geq 3$, entonces $(-1)^a\equiv (-1)^{a'}\pmod{4}$. Pero esto implica que $a\equiv a'\pmod{2}$ y así $a=a'$. Más aún, esto implica que $5^b \equiv 5^{b'}\pmod{2^{\ell}}$, y así $5^{b-b'}\equiv 1\pmod{2^\ell}$. Pero por lo que vimos antes, esto quiere decir que $2^{\ell-2}\mid b-b'$, y así $b=b'$.

Finalmente, notemos que $((-1)^a 5^b)^{2^{\ell-2}}\equiv 1\pmod{2^{\ell}}$. Como tenemos $\varphi(2^{\ell})=2^{\ell-1}$ elementos de orden $2^{\ell-2}$, esto quiere decir que no hay raíces primitivas módulo $2^\ell$. Y estamos.
\end{proof}
\end{teo}

Para cerrar la sección, podemos caracterizar todos los $n$ para los cuales existen raíces primitivas módulo $n$.

\begin{teo}
Sea $n\in\NN$. Existen raíces primitivas módulo $n$ si y sólo si $n=2,4,p^k,2p^k$ con $p$ un primo impar.
\begin{proof}
Por el teorema anterior, podemos suponer que $n\neq 2^{\ell}$ para $\ell\geq 3$. Si $n\neq 2,4,p^k,2p^k$ es fácil ver que $n=n_1n_2$ con $n_1,n_2>2$. Entonces, $\varphi(n_1)$ y $\varphi(n_2)$ son ambos pares. Como $\ZZ/n\ZZ^\times \simeq \ZZ/n_1\ZZ^\times \times \ZZ/n_2\ZZ^\times$, por el Teorema de Cauchy para grupos, debemos tener que $\ZZ/n_1\ZZ^\times$ y $\ZZ/n_2\ZZ^\times$ tienen elementos de orden $2$. Pero un grupo cíclico tiene como máximo un elemento de orden $2$. Entonces $n$ no posee raíces primitvas.

Ya sabemos que $n=2,4,p^k$ tienen raíces primitivas. Finalmente, notemos que tenemos $\ZZ/2p^k\ZZ^\times \simeq \ZZ/2\ZZ^\times \times \ZZ/p^k\ZZ^\times \simeq \ZZ/p^k\ZZ^\times$, que es cíclico. El teorema sigue.
\end{proof}
\end{teo}

\section{El Nilradical}

\begin{defn}
Sea $A$ un anillo conmutativo. Decimos que $x\in A$ es \textbf{nilpotente} si existe $n\in\NN$ tal que $x^n = 0$. Se define el \textbf{nilradical} como $\Nil (A) = \{x\in A : x\text{ es nilpotente}\}$.
\end{defn}

\begin{prop}
Sea $A$ anillo conmutativo. Entonces $\Nil (A)$ es un ideal de $A$ y además $\Nil(A/\Nil A) = 0$.
\begin{proof}
Sea $a\in A$, $x\in \Nil(A)$. Luego, existe $n\in\NN$ tal que $x^n=0$. Pero esto quiere decir que $(ax)^n = a^n x^n = 0$ pues $A$ es conmutativo. Además, si $x,y\in\Nil (A)$ entonces $\exists n,m\in\NN$ tales que $x^n = 0 = y^m$. Entonces $(x+y)^{n+m-1} = \displaystyle\sum_{k=0}^{n+m-1}\binom{n+m-1}{k}x^k y^{n+m-1-k}$. Pero alguno de los dos exponentes va a ser o mayor que $n$ o mayor que $m$, y así toda esa suma es $0$ y $x+y\in\Nil(A)$.

Ahora, sea $\overline{x}\in \Nil(A/\Nil(A))$. Por lo tanto, existe un $n\in\NN$ tal que $\overline{x}^n = \overline{0}$. Es decir, $\overline{x^n} = \overline{0}$, que es lo mismo que $x^n \in \Nil(A)$. Entonces, existe un $m\in\NN$ tal que $(x^n)^m = 0$. O sea, $x^{nm}=0$ y así $x\in\Nil(A)$, que implica que $\overline{x}=\overline{0}$, como queríamos ver.
\end{proof}
\end{prop}

\begin{prop}
Sea $A$ un anillo conmutativo. Entonces $\Nil (A) = \displaystyle\bigcap_{\mathfrak{p}\in\spec(A)} \mathfrak{p}$.
\begin{proof}
($\subseteq$) Sea $x\in\Nil(A)$. Luego, existe $n\in\NN$ tal que $x^n=0\in\mathfrak{p} \; \forall \mathfrak{p}\in\spec(A)$. Como estos ideales son primos, sale fácil (inducción si se quiere) que $x\in\mathfrak{p}\;\forall\mathfrak{p}\in\spec(A)$.

($\supseteq$) Sea $x\notin \Nil (A)$. Si podemos conseguir un $\mathfrak{p}\in\spec(A)$ tal que $x\notin\mathfrak{p}$ habremos probado la recíproca y ya estaremos. Consideremos el conjunto de los ideales que no contienen ninguna potencia de $x$, es decir, $S=\{I\subseteq A \text{ ideal}:x^n\notin I \;\forall n\in\NN\}$. Es claro que es no-vacío pues el ideal $0$ está, ya que $x\notin\Nil(A)$ implica que $x^n\neq 0 \;\forall n\in\NN$. Además, si $C$ es una cadena de $(S,\subseteq)$, está acotada superiormente por la unión (ya vimos que si tenemos una cadena de ideales, su unión también es un ideal), entonces podemos aplicar el Lema de Zorn y obtenemos un elemento maximal de $S$. Llamémoslo (sugerentemente) $\mathfrak{p}$. Notemos que $x\notin\mathfrak{p}$ pues $\mathfrak{p}\in S$. Supongamos que $a,b\notin\mathfrak{p}$. Veamos que $ab\notin \mathfrak{p}$, de donde seguirá que $\mathfrak{p}$ es primo y listo. Notemos que como $a\notin\mathfrak{p}$ entonces $\mathfrak{p}\subset \mathfrak{p}+\langle a\rangle$ pero la contención es estricta. Entonces, como $\mathfrak{p}$ es maximal en $S$, $\mathfrak{p}+\langle a\rangle \notin S$. Esto quiere decir que existe un $n$ tal que $x^n \in \mathfrak{p}+\langle a\rangle$. De forma análoga, existe un $m$ tal que $x^m \in \mathfrak{p}+\langle b\rangle$. Entonces $x^{n+m} = x^n x^m \in (\mathfrak{p}+\langle a\rangle)(\mathfrak{p}+\langle b\rangle)\subseteq \mathfrak{p}+\langle ab\rangle$. Entonces $\mathfrak{p}+\langle ab\rangle\notin S$, que quiere decir que $\langle ab\rangle \notin\mathfrak{p}$. Entonces $\mathfrak{p}$ es el ideal que buscabamos al principio, y la proposición sigue.
\end{proof}
\end{prop}

\section{Polinomios}

Sea $A$ un anillo conmutativo y $A[x]=\left\{\displaystyle\sum_{i=0}^{n} a_i x^i : a_i\in A \; \forall 0\leq i\leq n\right\}$ su anillo de polinomios, con la suma, producto y neutros definidos en forma usual. Es importante notar que dado un polinomio $f\in A[x]$ podemos definir una función dada por la evaluación en cada punto: $f:A\to A$, $a\mapsto f(a)=\displaystyle\sum_{i=0}^n a_i a^i$. Pero notemos que la igualdad como funciones no implica la igualdad como polinomios. Por ejemplo, $x^p - x \in \ZZ_p[x]$ y el polinomio $0$. La igualdad de polinomios está dada por la igualdad coeficiente a coeficiente de esas sumas formales.

Notemos además que podemos ver a $A\subseteq A[x]$ como un subanillo vía $a\mapsto \underline{a}$ polinomio de grado $0$.

Dado un polinomio $f=a_n x^n + \ldots + a_1 x + a_0$ con $a_n\neq 0$, llamamos \textbf{coeficiente principal} a $a_n$ y el grado como $\gr f = n$.

Ya habíamos visto que $A$ es dominio íntegro si y sólo si $A$ lo es. Además, notemos que si son dominio íntegro entonces $(A[x])^\times  = A^\times$. Esto es mentira si $A$ no es dominio íntegro, por ejemplo en $\ZZ_4[x]$, $(2x+1)(2x+1) = 1$.

\begin{defn}
Sea $A\subseteq B$ subanillo con $A,B$ conmutativos. Sea $f\in A[x]$ y $b\in B$. Si $f=\displaystyle\sum_{i=0}^n a_i x^i$, definimos $f(b) = \displaystyle\sum_{i=0}^n a_i b^i\in B$. Dado $b\in B$ definimos $\ev_b:A[x]\to B$ $f\mapsto f(b)$, la evaluación en $b$. Es fácil ver que $\ev_b$ es un morfismo de anillos, y eso implica que $\im\ev_b$ es un subanillo de $B$. Denotamos a este subanillo $A[b]=\im\ev_b = \left\{\displaystyle\sum_{i=0}^n a_ib^i : n\in\NN_0 , a_i\in A\right\}$.
\end{defn}

Notemos que por la definición $A[b]$ es el mínimo subanillo de $B$ que contiene al subanillo $A$ y al elemento $b$.

Ejemplos de esto serían $\ZZ[\sqrt{2}],\ZZ[i],\QQ[\sqrt[3]{2}]\subseteq \CC$.

Miremos ahora el caso $A=k$ un cuerpo. Recordemos que $k[x]$ es un DIP, y que eso implica que ser irreducible es equivalente a ser primo, y que ser ideal primo no nulo es equivalente a ser ideal maximal, y que $\langle c\rangle$ es maximal si y sólo si $c$ es irreducible. Además, es claro que $(k[x])^\times = k^\times = k-\{0\}$.

\begin{defn}
Sea $k\subseteq E$ con $k,E$ cuerpos ($E$ cuerpo y $k$ subcuerpo). Llamamos esto una \textbf{extensión de cuerpos}.
\end{defn}

Notemos que si $k\subseteq E$ es una extensión de cuerpos entonces $E$ es un $k$-espacio vectorial.

Ahora, sea $k\subseteq E$ una extensión de cuerpos y $\alpha\in E$. Sabemos que $k[\alpha]=\im\ev_\alpha\subseteq E$. Además, $\ker\ev_\alpha = \{f\in k[x] : f(\alpha)=0\}$. Si $\ker\ev_\alpha = 0$ entonces $\alpha$ no es raíz de ningún polinomio en $k[x]$ (salvo el $0$). En ese caso, se dice que $\alpha$ es \textbf{trascendente} sobre $k$. Además, por el teorema de isomorfismo $k[x]/\ker\ev_\alpha \simeq k[\alpha]$. Entonces, si $\alpha$ es trascendente sobre $k$ tenemos que $k[x]\simeq k[\alpha]$, y no es un cuerpo.

Si $\ker\ev_\alpha\neq 0$ entonces existe un $g\in k[x]$ no nulo tal que $g(\alpha)=0$. En ese caso, se dice que $\alpha$ es \textbf{algebraico} sobre $k$. Como $k[x]$ es un DIP entonces hay un $f\in k[x]$ tal que $\ker\ev_\alpha = \langle f\rangle$. Entonces, nuevamente por el teorema de isomorfismo, $E\supseteq k[\alpha]\simeq k[x]/\langle f\rangle$. Como $E$ es un cuerpo, en particular es un dominio íntegro y así $k[\alpha]$ también lo es. Esto implica que $\langle f\rangle$ debe ser un ideal primo. Pero como $k[x]$ es un DIP, entonces $\langle f\rangle$ es maximal y así $k[\alpha]$ es un cuerpo.

Notemos que si $k\subseteq E$ es una extensión de cuerpos, entonces $\alpha\in E$ algebraico sobre $k$ si y sólo si $k[\alpha]$ es un cuerpo.

\begin{defn}
Dada una extensión de cuerpos $k\subseteq E$ y dado $\alpha\in E$ entonces $k(\alpha)$ es la intersección de los subcuerpos de $E$ y que contienen a $k$ y $\alpha$. O sea, el cuerpo más chico que contiene a $k$ y $\alpha$. Si $\alpha$ es algebraico entonces $k[\alpha]=k(\alpha)$ pues $k[\alpha]$ es el subanillo más chico que los contiene pero además es un cuerpo. Si no, es fácil ver que si $\alpha$ es trascendente sobre $k$ entonces $k(\alpha)$ es el cuerpo de fracciones de $k[\alpha]$ (se puede ver que es la propiedad universal).
\end{defn}

Como $\ker\ev_\alpha$ es un ideal maximal entonce si es generado por $f$, debe ser irreducible. Además, si $g\in k[x]$ es irreducible y $g(\alpha)=0$ entonces $g\in\ker\ev_\alpha$ y así $g=hf$. Pero como $g$ es irreducible y $f$ lo es, debemos tener que $h\in (k[x])^\times = k^\times$. Esto quiere decir que $c\in k^\times$.

Por lo tanto, dado $\alpha\in E$ algebraico sobre $k$ existe un único polinomio irreducible y mónico $f\in k[x]$ tal que $f(\alpha)=0$. A esto lo notamos $\mathrm{Irr}(\alpha,k)$ al único polinomio irreducible y mónico que anula a $\alpha$.

Es fácil ver que si $k\subseteq E$ una extensión de cuerpos y $\alpha\in E$ algebraico sobre $k$, entonces $\mathrm{dim}_k k[\alpha] = \gr (\mathrm{Irr}(\alpha,k))$. Es solo tomar la base $\{1,\alpha,\ldots , \alpha^{n-1}\}$ con $n=\gr(\mathrm{Irr}(\alpha,k))$.

Ya vimos que sobre un cuerpo $k$, un polinomio de grado $n$ en $k[x]$ tiene como máximo $n$ raíces. Decimos que $f\in k[x]$ se factoriza linealmente en $k$ si $f=c(x-\alpha_1)\cdots (x-\alpha_r)$ con $a_i\in k$ (no necesariamente distintos). Notemos que si $k\subseteq E$ es una extensión de cuerpos y $f\in k[x]$ se factoriza linealmente en $k$, entonces las raíces que $f$ tenga en $E$ están en $k$. En efecto, si $0=f(\alpha) = c(\alpha-\alpha_1)^{m_1}\cdots (\alpha-\alpha_r)^{m_r}$, al ser $E$ un cuerpo, es dominio íntegro y así $\alpha = \alpha_i$ para algún $i$.

\textbf{Importante}: esto no quiere decir que un polinomio que se factoriza linealmente sobre $k$ no tenga raíces fuera de $k$. Es decir, puede ser que un polinomio tenga más de $n$ raíces en un lugar que \textbf{no} sea un cuerpo. Veamos el siguiente ejemplo:

Sea $A=\CC\times \CC$ con el producto y suma coordenada a coordenada. No es un cuerpo pues tiene divisores de $0$: $(1,0)(0,1)=(0,0)$.

Ahora, sea $k=\{(a,a):a\in \CC\}\subseteq A$. Es claro que $k$ es un cuerpo (y es algebraicamente cerrado) pues es isomorfo a $\CC$. Pero ahora, notemos que $f = (1,1)x ((1,1)x-(1,1))$ es un polinomio que se factoriza linealmente, y así en $k$ sus raíces son $(0,0)$ y $(1,1)$. Pero en $A$, notemos que hay más raíces, por ejemplo el $(1,0)$.

Finalmente, podemos considerar derivadas formales en los polinomios. Sea $A[x]$ el anillo de polinomios de $A$ conmutativo. Definimos $D:A[x]\to A[x]$ dada por $D(f)=f'$, la derivación formal $D\left(\displaystyle\sum_{k=0}^n a_k x^k \right) = a_n n x^{n-1} + \ldots + a_2 x + a_1$.

Notemos que las propiedades usuales siguen valiendo: $(f+g)'=f'+g'$, $(fg)'=f'g+fg'$ y si $a\in A$ entonces $(af)'=af'$.

Notemos que si $k$ es un cuerpo y $\alpha\in k$ es una raíz de $f\in k[x]$ entonces (por el algoritmo de división) tenemos que $x-\alpha\mid f$. Esto quiere decir que existe un $m\in\NN$ tal que $(x-\alpha)^m \mid f$ pero $(x-\alpha)^{m+1}\nmid f$. Llamamos a $m$ la multiplicidad de la raíz $\alpha$ en $f$. Finalmente, sigue valiendo un resultado que conocíamos antes: $\alpha$ es raíz múltiple de $f\in k[x]$ si y sólo si $f(\alpha)=0$ y $f'(\alpha)=0$.

\chapter{Módulos}

\section{Definiciones Básicas}

\begin{defn}
Sea $A$ un anillo. Un $A$-módulo a izquierda es un grupo abeliano $(M,+)$ que tiene además una acción a izquierda de $A$ (visto $A$ como monoide con multiplicación) y la acción distribuye a la suma de $M$ y $A$. Más concretamente, cumple que \begin{itemize}\item $(M,+)$ es un grupo abeliano. \item $\exists\mu :A\times M \to M, \; (a,m)\mapsto a\cdot m$ tal que $1\cdot m = m$ y $(a\cdot a') \cdot m = a\cdot (a'\cdot m)$. \item $(a+a')\cdot m = a\cdot m + a'\cdot m$ y $a\cdot (m+m') = a\cdot m + a\cdot m'$ para todos $a,a'\in A$, $m,m'\in M$ (propiedad distributiva)\end{itemize}
\end{defn}
\begin{obs}
Es fácil ver que valen $0\cdot m = 0\;\forall m\in M$, $a\cdot 0 = 0 \;\forall a\in A$ y $a(-m) = (-a)m \; \forall a\in A, m\in M$.
\end{obs}

\begin{obs}
Un $\ZZ$-módulo es lo mismo que un grupo abeliano. En efecto, un $\ZZ$-módulo es un grupo abeliano por definición. De forma recíproca, todo grupo abeliano admite una única estructura de $\ZZ$-módulo: Sea $(M,+)$ grupo abeliano. Para cualquier $n\in\NN$, $x\in M$, podemos definir $n\cdot x = (\underbrace{1+\ldots + 1}_{n\text{ veces}})x = \underbrace{x+\ldots + x}_{n\text{ veces}}$ y $(-n)x = (\underbrace{(-1)+\ldots + (-1)}_{n\text{ veces}})x = \underbrace{(-x)+\ldots + (-x)}_{n\text{ veces}}$. Entonces definimos $zx$ para cualquier $z\in\ZZ$, y es trivial ver que esto es un $\ZZ$-módulo, por como lo construimos.
\end{obs}

\begin{ex}
Si $A=k$ es un cuerpo, entonces un $A$-módulo es un $k$-espacio vectorial.

Si $M=0$, entonces $M$ es un $A$-módulo para cualquier anillo $A$.

Sea $A$ un anillo y sea $I\subseteq A$ un ideal a izquierda. Entonces $I$ es un $A$-módulo a izquierda. En particular, $A$ es un $A$-módulo a izquierda.

Sea $V$ un $k$-espacio vectorial, y sea $A=\End_k(V) = \{f:V\to V \text{ } k \text{ lineales}\}$. Es fácil ver que $(A,+,\circ)$ forma un anillo. Entonces, $V$ es un $A$-módulo vía $f\cdot v = f(v)$.

Sea $G$ un grupo y $A=\ZZ[G] = \left\{ \displaystyle\sum_{\text{finita}} \mu_g g : g\in G, \mu_g\in\ZZ\right\}$. Sea $(M,+)$ un grupo abeliano con $G\acts M$ y $g(m+m') = gm + gm'$. Entonces $M$ es un $\ZZ[G]$-módulo a izquierda, con la estructura dada por $\left( \displaystyle\sum_{\text{finita}} \mu_g g\right) (m) = \displaystyle\sum_{\text{finita}} \mu_g (gm) \in M$. O sea, una acción que respeta la suma es un $\ZZ[G]$-módulo.

Sea $A$ un anillo, entonces $A^n = \{(a_1,\ldots , a_n) : a_i\in A\}$ y $A^{n\times m}$ las matrices de $n\times m$ con coeficientes en $A$, son ambos $A$-módulos.
\end{ex}

\begin{prop}
Sea $(M,+)$ un grupo abeliano y $A$ un anillo. Considero $\End(M)=\{f:M\to M \text{ morfismo de grupos}\}$. Entonces, $(\End(M),+,\circ)$ forma un anillo, y es equivalente tener una estructura de $A$-módulo de $M$ que un morfismo de anillos $\varphi : A\to\End(M)$.
\begin{proof}
Si tenemos que $M$ es un $A$-módulo, entonces tengo el morfismo de anillos $\varphi:A\to\End(M), \; a\mapsto \ell_a$ con $\ell_a:M\to M, \; x\mapsto ax$.

Si $\varphi:A\to \End(M)$ es un morfismo de anillos, entonces defino $ax = \varphi(a)(x)$, y listo.
\end{proof}
\end{prop}

\begin{defn}
Un $A$-módulo a derecha es $(M,+)$ grupo abeliano tal que \begin{itemize} \item $ \mu:M\times A\to M \; (x,a)\mapsto x\cdot a$ con $x\cdot 1 = x$ y $x(aa')=(xa)a'$. \item $(x+x')a=xa+x'a$ y $x(a+a')=xa+xa'$ para todos $a,a'\in A, x,x'\in M$. \end{itemize}
\end{defn}

\begin{obs}
Si $A$ es conmutativo y $M$ es un $A$-módulo a izquierda, entonces lo es definiendo $xa=ax$ (basta pedir $A=A^{\mathrm{op}}$).
\end{obs}

\begin{defn}
Sean $A,B$ anillos. $M$ es un $A,B$-bimódulo si es un $A$-módulo a izquierda y $B$-módulo a derecha, y además $(ax)b=a(xb)$ para todos $a\in A, b\in B, x\in M$. Si $A=B$ se dice directamente que $M$ que es un $A$-bimódulo. Notamos esto como $_AM_B$.
\end{defn}

De ahora en más (salvo explícitamente aclarado) trabajaremos con módulos a izquierda.

\begin{defn}
Sea $M$ un $A$-módulo. Un \textbf{submódulo} $N\subseteq M$ es un subgrupo $(N,+)\subseteq (M,+)$ tal que para todo $a\in A$ y para todo $x\in M$ tenemos que $ax\in N$. Equivalentemente \begin{itemize} \item $N\neq \emptyset$. \item $x,y\in N \Longrightarrow x+y\in N$. \item $a\in A,x\in N \Longrightarrow ax\in N$. \end{itemize}
\end{defn}

\begin{obs}
Es trivial que $0$ y $M$ son submódulos de $M$.
\end{obs}

\begin{defn}
Un $A$-módulo $M$ se dice \textbf{simple} si los únicos submódulos que tiene son $0$ y $M$.
\end{defn}

\begin{obs}
Si $A=k$ cuerpo, $V$ un $k$-espacio vectorial es simple si y sólo si $\dim_k(V)\leq 1$ (en efecto, los subespacios son submonoides).
\end{obs}

\begin{ex}
Sea $M$ un $A$-módulo y $S$ un conjunto no vacío. Consideramos $M^S = \{f:S\to M\}$. Es fácil ver que $M^S$ es un $A$-módulo con $(f+g)(s)=f(s)+g(s)$ y $(af)(s)=af(s)$, es decir, definido punto a punto. Podemos denotar a $f$ por $(x_s)_{s\in S}$ con $f(s)=x_s$, es decir, como una tupla. Notar también que si $N\subseteq M$ es un submódulo, entonces $N^S\subseteq M^S$ es un submódulo.

Ahora, sea $M^{(S)}=\{f: S\to M \text{ de soporte finito}\} = \{(x_s)_{s\in S} : x_s = 0 \text{ salvo finitos}\}$. Entonces, $M^{(S)}\subseteq M^S$ es un submódulo (simplemente hay que chequear las tres condiciones anteriores).
\end{ex}

\begin{defn}
Sea $A$ un anillo conmutativo y $M$ un $A$-módulo. Decimos que $x\in M$ es de \textbf{torsión} si existe $a\in A$ ($a\neq 0$) tal que $ax=0$.
\end{defn}
\begin{ex}
$0\in M$ es de torsión. Si miramos a $\ZZ_n$ como un $\ZZ$-módulo entonces todos sus elementos son de torsión. Ahora, en el caso de $\ZZ$ como $\ZZ$-módulo no hay torsión no trivial ni en el caso de $V$ un $k$-espacio vectorial.
\end{ex}

\begin{prop}
Sea $A$ un dominio íntegro y sea $M$ un $A$-módulo. Entonces $$T(M)=\{x\in M : x\text{ es de torsión}\}\subseteq M$$ es un $A$-submódulo.
\begin{proof}
Es claro que $T(M)\neq \emptyset$ pues $0\in T(M)$.

Sean $x,y\in T(M)$, entonces existen $a,b\neq 0$ tales que $ax=0=by$. Como es dominio íntegro, $ab\neq 0$. Entonces, $ab(x+y) = b(ax)+a(by)=0$, y así $x+y\in T(M)$.

Finalmente, sea $a\in A$ y $x\in T(M)$. Entonces $ax\in T(M)$ pues si $bx=0$ entonces $a(bx)= b(ax)=0$. Y la proposición sigue.
\end{proof}
\end{prop}

\begin{obs}
Sea $M$ un $A$-módulo y sea $\{N_i\}_{i\in J}$ una familia de submódulos. Entonces $\displaystyle\bigcap_{i\in J}N_i$ es un $A$-submódulo. Entonces, dado $X\subseteq M$ un subconjunto, podemos definir el submódulo generado por $X$ a izquierda como $\langle X\rangle  = \displaystyle\bigcap_{\stackrel{N\subseteq M \text{ submódulo}}{X\subseteq N}}N$. En el caso finito, lo podemos describir como siempre: $\langle x_1,\ldots , x_n\rangle = \left\{\displaystyle\sum_{i=1}^{n} a_ix_i : a_i\in A \; \forall 1\leq i\leq n\right\}$.
\end{obs}

\begin{defn}
Decimos que $M$ un $A$-módulo es \textbf{finitamente generado} si existe $X\subseteq M$ finito tal que $M=\langle X\rangle$.
\end{defn}
\begin{ex}
Tomo un anillo $A$ y $M=A$. Entonces $M=\langle 1\rangle$, y así es finitamente generado.

Si $A=\ZZ$, entonces $\ZZ = \langle 1\rangle  =\langle 2,3\rangle$ y ambos son dos sistemas minimales de generadores (pues si saco un generador, ya no generan todo $\ZZ$ en ambos casos). Entonces, no puedo definir un concepto de base como "`el sistema minimal de generadores"'.

Sea $A=k$ un cuerpo, entonces $k[x]$ no es finitamente generado (pues cualquier conjunto generado por un conjunto finito de polinomios tiene el grado acotado).

Es fácil ver que $A^n$ es finitamente generado como $A$-módulo. En efecto, $A^n = \langle e_1, \ldots , e_n\rangle$ con $e_i = (0,\ldots , \underbrace{1}_{i\text{-ésimo lugar}} , \ldots , 0)$. Es decir, la base canónica.

El siguiente ejemplo muestra un $A$-módulo finitamente generado que tiene un submódulo $N\subseteq M$ que no es finitamente generado. Sea $A=k[x_i]_{i\in\NN}$ con $k$ un cuerpo. O sea, el anillo de polinomios con coeficientes en $k$ en infinitas variables. Tomemos a $M=A$. Por lo que dijimos antes, $M$ es finitamente generado como $A$-módulo.
Sea $N=\langle x_i\rangle_{i\in\NN}$. Es claro, por la definición, que es un submódulo. Entonces, $N$ son los polinomios sin término constante. Veamos que $N$ no es finitamente generado. Si lo fuese, existirían $f_1,\ldots , f_r$ polinomios tales que $N=\langle f_1,\ldots , f_r\rangle$. Como son finitos polinomios, aparecen finitas variables. Además, por como es $N$, ninguno de los $f_i$ tiene término constante. Si $x_k$ es una variable que no aparece entre las variables de los $f_i$, tenemos que $x_k\notin \langle f_1,\ldots , f_r\rangle$ pues si $x_k=\displaystyle\sum_{i=1}^r g_i f_i$, en esos términos nunca aparece $x_k$ sólo. Absurdo, que provino de suponer que $N$ es finitamente generado.
\end{ex}

\begin{defn}
Sean $M,N$ dos $A$-módulos. Un morfismo de $A$-módulos es una función $f:M\to N$ tal que $f(x+x')=f(x)+f(x')$ y $f(ax)=af(x)$ para todos $x,x'\in M$ y $a\in A$ (también se dice que $f$ es $A$-lineal). Definimos el núcleo y la imagen de un morfismo de módulos como es habitual: $\ker f = \{x\in M : f(x)=0\}$ e $\im f = \{f(x):x\in N\}$. Decimos que $f$ es \textbf{monomorfismo} si $\ker f =0$, decimos que es \textbf{epimorfismo} si $\im f =N$ y decimos que es \textbf{isomorfismo} si es monomorfismo y epimorfismo. Además, es fácil ver que $\ker f$ e $\im f$ son submódulos (de $M$ y $N$ respectivamente). Es fácil ver también que si $f$ es isomorfismo, $f^{-1}$ resulta $A$-lineal también.
\end{defn}

\begin{defn}
Sea $A$ un anillo. Un diagrama de $A$-módulos y morfismos de $A$-módulos de la forma $N\stackrel{f}{\longrightarrow}M\stackrel{g}{\longrightarrow} T$ se dice \textbf{exacto} en $M$ si $\im f = \ker g$.
\end{defn}

\begin{defn}
Una \textbf{sucesión exacta} es un diagrama de $A$-módulos y morfismos $A$-lineales de la forma $\ldots \longrightarrow M_{n+1}\stackrel{f_{n+1}}{\longrightarrow} M_n \stackrel{f_{n}}{\longrightarrow} M_{n-1} \stackrel{f_{n-1}}{\longrightarrow}M_{n-2}\longrightarrow \ldots$ (finita o infinita) que es exacta en cada lugar.
\end{defn}

\begin{obs}
Notemos que $0\longrightarrow M$ es el morfismo $0$ de $0$ a $M$ y $M\longrightarrow 0$ es el morfismo $0$ de $M$ a $0$. Entonces, la sucesión $0\longrightarrow M\stackrel{f}{\longrightarrow} N$ es exacta si y sólo si $f$ es monomorfismo. De forma análoga, $M\stackrel{g}{\longrightarrow} N\longrightarrow 0$ es exacta si y sólo si $g$ es epimorfismo. Entonces, $0\longrightarrow M\stackrel{f}{\longrightarrow} N \longrightarrow 0$ es exacta si y sólo si $f$ es isomorfismo.
\end{obs}

\begin{defn}
Una \textbf{sucesión exacta corta} es una sucesión exacta de la forma $$0\longrightarrow M\stackrel{f}{\longrightarrow} N\stackrel{g}{\longrightarrow} T \longrightarrow 0$$ Es decir, $f$ es monomorfismo, $g$ es epimorfismo y $\ker g = \im f$.
\end{defn}

\begin{ex}
$0\longrightarrow \ZZ_2 \stackrel{\iota_1}{\longrightarrow} \ZZ_2 \oplus \ZZ_2 \stackrel{\pi_2}{\longrightarrow} \ZZ_2\longrightarrow 0$ es una sucesión exacta corta, donde $\iota_1(x) = (x,0)$ y $\pi_2(x,y)=y$.

Más en general, si $M$ y $N$ son dos $A$-módulos, $0\longrightarrow M\stackrel{\iota_1}{\longrightarrow} M\oplus N\stackrel{\pi_2}{\longrightarrow} N\longrightarrow 0$ es una sucesión exacta corta con $\iota_1$ la inclusión y $\pi_2$ la proyección de la segunda coordenada.

No es cierto en general que tenga que tener que para que una sucesión sea exacta corta entonces el del medio sea la suma directa de los otros dos. Por ejemplo, es fácil ver que $0\longrightarrow \ZZ_2\stackrel{\alpha}{\longrightarrow} \ZZ_4 \stackrel{\beta}{\longrightarrow} \ZZ_2\longrightarrow 0$ con $\alpha(\overline{x})=\overline{2x}$ y $\beta(\overline{x})=\overline{x}$ es una sucesión exacta corta y $\ZZ_4\not\simeq \ZZ_2\oplus\ZZ_2$ (el de la izquierda es cíclico, el de la derecha no).
\end{ex}

\section{Cocientes}

\begin{defn}
Sea $M$ un $A$-módulo a izquierda y $N\subseteq M$ un $A$-submódulo. Como $M$ y $N$ son grupos abelianos, tenemos que $(M/N,+)$ es grupo abeliano. Démosle estructura de $A$-módulo vía $a\overline{x} = \overline{ax}$. Esto está bien definido pues si $\overline{x}=\overline{y}$ entonces $x-y\in N$ y así $a(x-y)\in N$ para cualquier $a\in A$. Esto implica que $\overline{ax}=\overline{ay}$. Esta estructura que le dimos de $A$-módulo nos permite llamarlo el módulo cociente.
\end{defn}

\begin{teo}[Propiedad Universal del Cociente]
Sea $M$ un $A$-módulo y $N\subseteq M$ un submódulo. El módulo cociente $M/N$ tiene la siguiente propiedad universal:
Para todo $A$-módulo $T$ y $f:M\to T$ morfismo $A$-lineal tal que $N\subseteq \ker f$ entonces existe un único $\overline{f}:M/N\to T$ morfismo $A$-lineal tal que $\overline{f}\circ \pi = f$. Es decir, el siguiente diagrama conmuta: \begin{tikzcd}[row sep=3em,column sep=2.5em,minimum width=3em]
M \arrow{d}[font=\normalsize, left]{\pi} \arrow{r}[font=\normalsize]{f} & T \\
M/N\arrow[dashed]{ru}[font=\normalsize, right]{\overline{f}}
\end{tikzcd}
\begin{proof}
Es claro que, por la propiedad universal del cociente para grupos, tenemos una única $\overline{f}$ que respeta la estructura aditiva. Simplemente debemos chequear que respeta la acción, pero esto es fácil de ver, y listo.
\end{proof}
\end{teo}

Como siempre, tenemos los teoremas de isomorfismo como corolarios de la propiedad universal del cociente:

\begin{cor}[Teoremas de Isomorfismo]
\begin{enumerate}
\item Sean $M,N$ dos $A$-módulos y tenemos $f:M\to N$ un morfismo $A$-lineal. Entonces $M/\ker f\simeq \im f$.
\item Sean $T\subseteq N\subseteq M$ $A$-submódulos. Entonces $(M/T)/(N/T)\simeq M/N$.
\item Sean $N,T\subseteq M$. Entonces $(N+T)/T\simeq N/N\cap T$.
\end{enumerate}
\end{cor}

\begin{obs}
Si $A=k$ es un cuerpo y $V$ un $k$-espacio vectorial, entonces para cualquier subespacio $S\subseteq V$ existe un subespacio $T\subseteq V$ tal que $V/S\simeq T$. En efecto, sabemos que existe $T$ tal que $S\oplus T = V$. Entonces, por el Tercer Teorema de Isomorfismo, $V/S \simeq S\oplus T / S \simeq T/S\cap T = T$. Es decir, los cocientes no tienen demasiado interés para espacios vectoriales pues dan simplemente subespacios (complementos del subespacio).
\end{obs}

\begin{defn}
Sea $f:M\to N$ un morfismo $A$-lineal. Definimos el conúcleo, $\coker f = N/\im f$.
\end{defn}

\begin{obs}
Es trivial notar que $\coker f = 0$ si y sólo si $f$ es epimorfismo.
\end{obs}

\begin{obs}
Sea $f:M\to N$ un morfismo $A$-lineal. Entonces, induce las siguiente sucesiones exactas:
$$0\longrightarrow \ker f \stackrel{\iota}{\longrightarrow} M\stackrel{f}{\longrightarrow} \im f \longrightarrow 0$$
$$0\longrightarrow \ker f \stackrel{\iota}{\longrightarrow} M\stackrel{f}{\longrightarrow} N\stackrel{\pi}{\longrightarrow} \coker f\longrightarrow 0$$
$$0\longrightarrow M/\ker f \stackrel{\iota}{\longrightarrow} M\stackrel{\overline{f}}{\longrightarrow} N \stackrel{\pi}{\longrightarrow}\coker f\longrightarrow 0 $$

\end{obs}

\section{Los funtores \texorpdfstring{$\hom_A(M,-)$}{homA(M,-)} y \texorpdfstring{$\hom_A(-,N)$}{homA(-,N)}}

\begin{defn}
Sean $\mathcal{C}$ y $\mathcal{D}$ dos categorías. Un \textbf{funtor} $F$ de $\mathcal{C}$ en $\mathcal{D}$ es una asugnación que a cada objeto $X\in\mathcal{C}$ le asigna un objeto $F(X)\in\mathcal{D}$, a cada morfismo $f:X\to Y\in\mathcal{C}$ le asigna un morfismo $F(f):F(X)\to F(Y)\in\mathcal{D}$ de modo tal que $F(1_X)=1_{F(X)}$ y $F(g\circ f)=F(g)\circ F(f)$ para todos morfismos $f:X\to Y$ y $g:Y\to Z$.

A un funtor, también lo llamamos \textbf{funtor covariante} pues preserva el orden de las flechas. Cuando tenemos una asignación $F$ que cumple todas las propiedades de los funtores, salvo que da vuelta las flechas, es decir, $F(g\circ f)=F(f)\circ F(g)$, decimos que es un \textbf{funtor contravariante}. O sea, un funtor contravariante es un funtor covariante entre las categorías opuestas.

\end{defn}

\begin{defn}
Sean $M,N$ dos $A$-módulos a izquierda. Entonces, el conjunto de las funciones $A$-lineales de $M$ en $N$, $\hom_A(M,N)=\{f:M\to N \text{ } A\text{-lineales}\}$ tiene naturalmente definida una suma punto a punto: $(f+g)(x) = f(x)+g(x)$, con un neutro $0(x)=0$.
\end{defn}

\begin{obs}
$(\hom_A(M,N),+)$ es un grupo abeliano.
\end{obs}

\begin{obs}
Notemos que si $A$ es conmutativo, entonces a la estructura natural de grupo abeliano le podemos agregar una acción que lo convierta en un $A$-módulo. En efecto, si definimos para cada $a\in A$, $(af)(x)= af(x)$, es fácil ver que es una acción. Además, $(af)(x+y) = (af)(x)+(af)(y) = af(x)+af(y)$ y $(af)(bx) = af(bx) = abf(x) = b(af)(x)$, por la conmutatividad de $A$. Si $A$ no es conmutativo, no podemos darle de forma natural una estructura de $A$-módulo al $\hom_A(M,N)$.
\end{obs}

Sea $M$ un $A$-módulo. Para cada $A$-módulo $N$, considero el grupo abeliano $\hom_A(M,N)$. Si $f:N\to N'$ es un morfismo $A$-lineal, definimos $f_*:\hom_A(M,N)\to\hom_A(M,N')$ por $\alpha\mapsto f_*(\alpha)=f\circ \alpha$. Es fácil ver que $f_*$ es un morfismo de grupos abelianos: $$f_*(\alpha+\beta)(x) = f\circ(\alpha+\beta)(x) = f(\alpha(x)+\beta(x)) = f\circ \alpha (x) + f\circ\beta (x) = f_*(\alpha)(x)+f_*(\beta)(x)$$
\begin{prop}[Propiedades Funtoriales de $\hom_A(M,-)$]
Sean $N,N'$ y $N''$ $A$-módulos, y $f:N\to N'$, $g:N'\to N''$ morfismos $A$-lineales. Entonces $(g\circ f)_* = g_*\circ f_*$. Además, si $1:N\to N$ es el morfismo identidad en $N$, tenemos que $1_* = 1_{\hom_A(M,N)}$.

Es decir, $\hom_A(M,-):\mathrm{_{A}Mod}\to \mathrm{Ab}$ es un \textbf{funtor covariante}.
\begin{proof}

Notemos que $(g\circ f)_*(\alpha)(x) = g\circ f\circ \alpha (x) = g\circ (f\circ \alpha)(x) = g_* \circ f_*(\alpha) (x) = (g_*\circ f_*)(\alpha)(x)$. Como $\alpha$ y $x$ son arbitrarios, debemos tener $(g\circ f)_* = g_* \circ f_*$.

Ahora, $1_*(\alpha)(x) = 1\circ \alpha (x) = 1$, y debe ser el morfismo trivial. La proposición sigue.
\end{proof}
\end{prop}

\begin{cor}
Si $f:N\to N'$ es un isomorfismo de $A$-módulos, entonces tenemos un isomorfismo de grupos abelianos $f_*:\hom_A(M,N)\to \hom_A(M,N')$ y $(f_*)^{-1} = (f^{-1})_*$.
\begin{proof}
$1 = 1_* = (f^{-1}\circ f)_* = (f^{-1})_*\circ f_*$ y de forma análoga, $f_*\circ (f^{-1})_* = 1$.
\end{proof}
\end{cor}

\begin{prop}
Sea $M$ un $A$-módulo a izquierda. Entonces $\hom_A(M,-)$ es un \textbf{funtor exacto a izquierda}. Es decir, si $0\longrightarrow N\stackrel{f}{\longrightarrow}N'\stackrel{g}{\longrightarrow}N''$ es una sucesión exacta, entonces $0\longrightarrow \hom_A(M,N)\stackrel{f_*}{\longrightarrow}\hom_A(M,N')\stackrel{g_*}{\longrightarrow}\hom_A(M,N'')$ es una sucesión exacta. Es decir, paso de una sucesión exacta de $A$-módulos a una sucesión exacta de grupos abelianos, o sea, $\ZZ$-módulos.
\begin{proof}
Debemos ver que $f_*$ es monomorfismo y que $\im f_* = \ker g_*$. Ver que $f_*$ es monomorfismo es fácil: Sea $\alpha\in\hom_A(M,N)$ tal que $f_*\circ \alpha = 0$. Entonces, para todo $x\in M$, $f_*\circ \alpha (x) = 0$, y así $f(\alpha(x))=0$. Como $f$ es monomorfismo, debemos tener $\alpha(x)=0$, y así $\alpha=0\in\hom_A(M,N)$ pues $x$ era arbitrario.

Primero veamos que $\im f_*\subseteq \ker g_*$. En efecto, equivale a ver que $g_*\circ f_* = 0$. Pero esto es $(g\circ f)_* = g_*\circ f_*$ y como la sucesión original es exacta, $g\circ f=0$ y $0_*=0$ y listo.

Para ver la otra inclusión, sea $\alpha\in\ker g_*\subseteq \hom_A(M,N')$. Entonces $\alpha:M\to N'$ es un morfismo $A$-lineal. Entonces $g_*\circ \alpha=0$, y así $g(\alpha(x))=0$ para todo $x\in M$. Pero esto quiere decir que $\alpha(x)\in \ker g = \im f$. Luego, $\alpha(x)=f(y)$ para algún $y\in M$. Pero más aún, este $y$ debe ser único, pues si $\alpha(x)=f(y')$, como $f$ es monomorfismo, tendríamos $f(y)=f(y')$, y así $y=y'$. Definamos $\beta:M\to N$ tal que $\beta(x) = y$ con $y$ el único elemento de $M$ tal que $f(y)=\alpha(x)$. Entonces, $f\circ \beta(x)=\alpha(x)$ y así $f_*(\beta)=\alpha$. Luego, si $\beta$ fuera un morfismo $A$-lineal ya estaríamos.

Pero veamos que $\beta$ es $A$-lineal. Sabemos que $f\circ \beta(x+x') = \alpha(x+x') = \alpha(x) + \alpha(x') = f\circ\beta(x)+f\circ\beta(x') = f(\beta(x)+\beta(x'))$. Como $f$ es monomorfismo, lo cancelamos a izquierda y obtenemos $\beta(x+x') = \beta(x)+\beta(x')$. Además, $f\circ \beta(ax) = f(\beta(ax)) = f(a\beta(x))$ y nuevamente cancelamos a izquierda a $f$, y obtenemos $\beta(ax) = a\beta(x)$. La proposición sigue.
\end{proof}
\end{prop}

\begin{obs}
Notemos que $\hom_A(M,-)$ no es exacto a derecha. Es decir, si tenemos $N'\stackrel{g}{\longrightarrow} N''\longrightarrow 0$ no implica que $\hom_A(M,N')\stackrel{g_*}{\longrightarrow} \hom_A(M,N'')\longrightarrow 0$. Es decir, que $g$ sea epimorfismo no implica que $g_*$ lo sea. Veamos un ejemplo de esto.

Tomamos $A=\ZZ$ y $M=\ZZ_2$. Tenemos $\underbrace{\ZZ}_{N}\stackrel{g}{\longrightarrow} \underbrace{\ZZ_2}_{N''}\longrightarrow 0$, con $g(x)=\overline{x}$ (módulo $2$). Es claro que $\hom_\ZZ (\ZZ_2,\ZZ)=0$, pues sólo tengo el morfismo trivial y que $\hom_{\ZZ}(\ZZ_2,\ZZ_2)\simeq \ZZ_2$. Entonces, $g_*:0\to\ZZ_2$ no puede ser epimorfismo.
\end{obs}

\begin{obs}
Cuando tomo $M=A$, es fácil ver que $\hom_A(A,N)\simeq N$. El isomorfismo está dado por $\varphi(f)=f(1)$, pues al determinar un morfismo de grupos en su valor en $1$, queda determinado.
\end{obs}

Todo esto que definimos ahora lo podemos hacer de forma muy similar, sólo que en vez de fijar el módulo de salida, fijamos el de llegada.

\begin{defn}
Sea $N$ un $A$-módulo fijo. Para cada $A$-módulo $M$ tenemos el grupo abeliano $\hom_A(M,N)$. Si $f:M\to M'$ es un morfismo de $A$-módulos entonces tenemos $f^*:\hom_A(M',N)\to\hom_A(M,N)$ definido por $\alpha\mapsto \alpha\circ f$ es un morfismo de grupos abelianos, y se cumple que $(g\circ f)^* = f^*\circ g^*$ y si $1:M\to M$ es la identidad en $M$, $1^* = 1_{\hom_A(M,N)}$. Es decir que $\hom_A(-,N)$ es un \textbf{funtor contravariante}.
\end{defn}

De forma similar a como vimos antes, se puede probar una proposición análoga sobre la exactitud del funtor contravariante que tenemos.
\begin{prop}
Si $M\stackrel{f}{\longrightarrow} M'\stackrel{g}{\longrightarrow} M''\longrightarrow 0$ es una sucesión exacta de $A$-módulos, entonces $0\longrightarrow \hom_A(M'',N)\stackrel{g^*}{\longrightarrow}\hom_{A}(M',N)\stackrel{f^*}{\longrightarrow} \hom_A(M,N)$.
\end{prop}

\begin{obs}
Como con la proposición para el funtor covariante, tenemos que $0\longrightarrow M\stackrel{f}{\longrightarrow}M'$ monomorfismo no implica que $f^*:\hom_A(M',N)\to\hom_A(M,N)$ sea epimorfismo.

En efecto, consideremos $A=\ZZ$, $N=\ZZ_2$. Es claro que $0\longrightarrow \underbrace{\ZZ}_{M}\stackrel{\varphi}{\longrightarrow} \underbrace{\ZZ}_{M'}$ con $\varphi(x)=2x$ es un monomorfismo. Entonces, tenemos $\varphi^*:\hom_\ZZ(\ZZ,\ZZ_2)\to \hom_\ZZ(\ZZ,\ZZ_2)$. Pero esto es $\varphi^*:\ZZ_2\to \ZZ_2$. Pero $\varphi^*(\alpha)(x) = (\alpha\circ\varphi)(x) = \alpha(\varphi(x))=\alpha(2x) = 2\alpha(x)=0$. Entonces, $\varphi^*\circ\alpha=0$ para cualquier $\alpha$ y así $\varphi^*=0$, que no puede ser epimorfismo.

\end{obs}

\section{Productos y sumas directas}

\begin{defn}
Sea $A$ un anillo y $\{M_j\}_{j\in J}$ una familia de $A$-módulos a izquierda. Entonces, $\displaystyle\prod_{j\in J}M_j = \{(x_j)_{j\in J} : x_j\in M_j \; \forall j\in J\}$, con la suma y la acción definidas coordenada a coordenada, obtenemos un $A$-módulo, y además los morfismos $A$-lineales proyección $\pi_i:\displaystyle\prod_{j\in J}M_j\to M_i$, definidos por $(x_j)_{j\in J}\mapsto x_i$.
\end{defn}

\begin{prop}[Propiedad Universal del Producto]

Dada una familia $\{M_j\}_{j\in J}$ de $A$-módulos. Entonces, $\displaystyle\prod_{j\in J}M_j$ tiene la siguiente propiedad universal: Si $\pi_i:\displaystyle\prod_{j\in J}M_j\to M_i$ son los morfismos proyección, entonces para todo $A$-módulo $N$ y familia de morfismos $A$-lineales $\{f_j\}_{j\in J}$, $f_j:N\to M_i$, existe un único $f:N\to\displaystyle\prod_{j\in J}M_j$ morfismo $A$-lineal tal que $\pi_i\circ f = f_i$ para todo $i\in J$. Es decir, para todo $i\in J$, el siguiente diagrama conmuta: \begin{tikzcd}[row sep=3em,column sep=4em,minimum width=3em] N \arrow{d}[font=\normalsize, left]{f_i} \arrow[dashed]{r}[font=\normalsize]{f} & \displaystyle\prod_{j\in J}\arrow{ld}[font=\normalsize, right]{\pi_i}M_j \\
M_i
\end{tikzcd}
\begin{proof}
Es claro, que si alguno existe debe cumplir que $f(x) = (f_j(x))_{j\in J}$. Pero es fácil ver que esto cumple. Entonces, existe $f$ y es único, y listo.
\end{proof}
\end{prop}

\begin{defn}
Sea $\{M_j\}_{j\in J}$ una familia de $A$-módulos y $\displaystyle\prod_{j\in J}M_j$ el producto. Para cada $j\in J$ se tiene una inclusión $\iota_j:M_j\to\displaystyle\prod_{i\in J}M_i$ con $x\mapsto (x_i)_{i\in J}$ donde $x_i=\begin{cases}x\text{ si }i=j \\ 0 \text{ si no}\end{cases}$. Vía $\iota_j$ identifico a $M_j$ con $\iota_j(M_j)\subseteq \displaystyle\prod_{i\in J}M_i$. Defino la \textbf{suma directa} como $\displaystyle\bigoplus_{j\in J}M_j = \left\langle \displaystyle\bigcup_{j\in J} M_j \right\rangle \subseteq \displaystyle\prod_{i\in J}M_i$. Entonces, $\displaystyle\bigoplus_{j\in J}M_j = \left\{ (x_j)_{j\in J}\in\displaystyle\prod_{j\in J}M_j : x_j=0 \text{ salvo finitos}\right\}$.
\end{defn}

\begin{prop}[Propiedad Universal de la Suma Directa o Coproducto]
Para cada $j\in J$ tenemos la inclusión $\iota_j:M_j\to \displaystyle\bigoplus_{i\in J}M_i$, que son morfismos $A$-lineales. Para cualquier $A$-módulo $N$ y familia de morfismos $A$-lineales $f_j:M_j\to N$, existe un único $f:\displaystyle\bigoplus_{j\in J}M_j\to N$ morfismo $A$-lineal tal que $f\circ \iota_j = f_j$ para todo $j\in J$. O sea, para todo $i\in J$, el siguiente diagrama conmuta:
\begin{tikzcd}[row sep=3em,column sep=4em,minimum width=3em] \displaystyle\bigoplus_{i\in J}M_i \arrow[dashed]{r}[font=\normalsize]{f} & N  \\
M_j\arrow{u}[font=\normalsize, left]{\iota_j} \arrow{ru}[font=\normalsize, right]{f_j}
\end{tikzcd}
\begin{proof}
Es similar a la propiedad universal anterior, sólo que ahora vamos a tener que $f((x_j)_{j\in J})=\displaystyle\sum_{j\in J}f_j(x_j)$ (esa suma tiene sentido pues sólo se suman finitos términos al ser $(x_j)_{j\in J}$ de soporte finito). Como antes, es fácil ver que es un morfismo $A$-lineal.
\end{proof}
\end{prop}

\begin{obs}
Si $J$ es finito, entonces $\displaystyle\prod_{j\in J}M_j=\displaystyle\bigoplus_{j\in J}M_j$.

Si $M_j=M$ para todo $j\in J$, entonces $\displaystyle\prod_{j\in J}M_j=M^J$ y $\displaystyle\bigoplus_{j\in J}M_j =  M^{(J)}$
\end{obs}

Tenemos la siguiente proposición, que es muy importante.

\begin{prop}
Sea $\{M_i\}_{i\in I}$ una familia de $A$-módulos y $N$ un $A$-módulo. Entonces, $\hom_A\left(\displaystyle\bigoplus_{i\in I} M_i, N\right) \simeq \displaystyle\prod_{i\in I} \hom_A(M_i, N)$ y $\hom_A\left(N,\displaystyle\prod_{i\in I}M_i\right)\simeq \displaystyle\prod_{i\in I}\hom_A(N,M_i)$.
\end{prop}

Como antes en el caso de grupos, podemos considerar ahora una suma directa interna.

Sea $M$ un $A$-módulo y $\{M_i\}_{i\in I}$ una familia de $A$-submódulos. Consideramos el $A$-submódulo generado por la unión de esta familia: $$\displaystyle\sum_{i\in I}M_i = \left\langle \displaystyle\bigcup_{i\in I}M_i \right\rangle = \left\{ \displaystyle\sum_{i\in I} x_i : x_i\in M_i, x_i=0 \text{ salvo finitos}\right\}$$ Para cada $j\in I$, tenemos la inclusión $M_j\hookrightarrow M$, y así por la propiedad universal de la suma directa, tenemos $\varphi:\displaystyle\bigoplus_{i\in I} M_i\to M$ con $\varphi\left((x_i)_{i\in I}\right) = \displaystyle\sum_{i\in I}x_i$. Es claro que $\im\varphi = \displaystyle\sum_{i\in I} M_i$. 

Notemos que $\varphi$ es epimorfismo si y sólo si $\displaystyle\sum_{i\in I}M_i = M$, es decir, si y sólo si para todo $x\in M$ existen $x_i\in M_i$ tales que $x=\displaystyle\sum_{i\in I}x_i$. 

De manera análoga, $\varphi$ es monomorfismo si y sólo si $\displaystyle\sum_{i\in I}x_i = 0$ implica que $x_i=0 \; \forall \, i\in I$, y equivalentemente, $\displaystyle\sum_{i\in I}x_i = \displaystyle\sum_{i \in I}y_i$ implica que $x_i=y_i \;\forall\, i\in I$.

Por lo tanto, $\varphi$ es isomorfismo si y sólo si para todo $x\in M$ existe un único $(x_i)_{i\in I}$ de soporte finito tal que $\displaystyle\sum_{i\in I}x_i = x$.

En este caso, cuando $\varphi$ es isomorfismo decimos que $M$ es la suma directa interna de los $\{M_i\}_{i\in I}$ y lo denotamos $M=\displaystyle\bigoplus_{i\in I} M_i$.

\section{Secciones, Retracciones, Sucesiones Exactas}

\begin{obs}
Sean $X,Y$ conjuntos no vacíos y $f:X\to Y$ una función. $f$ es inyectiva si y sólo si existe $g:Y\to X$ tal que $g\circ f = 1_X$, y $f$ es sobreyectiva si y sólo si existe $h:Y\to X$ tal que $f\circ h = 1_Y$. Es decir, si tiene una inversa a izquierda es inyectiva y a derecha sobreyectiva. Sin embargo, estas condiciones pueden no ser equivalentes en categorías que no sean las de los conjuntos.
\end{obs}

\begin{obs}
Si $f:M\to N$ es un monomorfismo $A$-lineal, esto no implica que exista $g:N\to M$ tal que $g\circ f=1_M$. Por ejemplo, $f:\ZZ_2\to\ZZ_4$ definido por $f(x)=2x$. No existe $g:\ZZ_4\to\ZZ_2$ tal que $g\circ f=1_{\ZZ_2}$ pues si existiera, tendríamos que $g\circ f(1)=g(2)=g(2\cdot 1)=2g(1)=0$. Absurdo.
\end{obs}

\begin{defn}
Sea $f:M\to N$ un morfismo $A$-lineal. Decimos que $f$ es \textbf{sección} si existe $g:N\to M$ morfismo $A$-lineal tal que $g\circ f = 1_M$.
\end{defn}

\begin{obs}
Ya vimos que sección implica monomorfismo pero la recíproca no vale.
\end{obs}

\begin{defn}
Sea $f:M\to N$ morfismo $A$-lineal. Se dice que $f$ es \textbf{retracción} si existe $g:N\to M$ morfismo $A$-lineal tal que $f\circ g = 1_N$.
\end{defn}

\begin{obs}
Como antes, notemos que retracción implica epimorfismo pero que epimorfismo no implica retracción. Por ejemplo $\ZZ\stackrel{\pi}{\longrightarrow}\ZZ_n\longrightarrow 0$ no es retracción pues hay un único morfismo de $\ZZ_n$ en $\ZZ$, el trivial.
\end{obs}

\begin{obs}
Sea $f:M\to N$ morfismo $A$-lineal. Entonces $f$ es isomorfismo si y sólo si es sección y retracción.
\end{obs}

\begin{obs}
Si $f,g$ son secciones (resp. retracciones) tales que tiene sentido componer, entonces $g\circ f$ es una sección (resp. retracción).
\end{obs}

\begin{teo}
Sea $0\longrightarrow M \stackrel{f}{\longrightarrow} N \stackrel{g}{\longrightarrow} T\longrightarrow 0$ una sucesión exacta corta de $A$-módulos. Son equivalentes:
\begin{enumerate}
\item $f$ es sección.
\item $g$ es retracción.
\item Existe $\varphi:N\to M\oplus T$ isomorfismo $A$-lineal tal que $\varphi\circ f = \iota_1$ y $\pi_2\circ \varphi = g$ con $\iota_1:M\to M\oplus T$ la inclusión de la primer coordenada y $\pi_2:M\oplus T\to T$ la proyección en la segunda coordenada. O sea, el siguiente diagrama tiene sus filas exactas y conmuta:
\begin{center}\begin{tikzcd}[row sep=3em,column sep=2.5em,minimum width=3em]
0\arrow[rightarrow]{r} & M \arrow[equals]{d}\arrow[rightarrow]{r}[font=\normalsize]{f} & N\arrow[rightarrow]{r}[font=\normalsize]{g}\arrow[rightarrow]{d}[font=\normalsize, right]{\varphi} & T\arrow[equals]{d}\arrow[rightarrow]{r} & 0 \\
0\arrow[rightarrow]{r} & M\arrow[rightarrow]{r}[font=\normalsize]{\iota_1} & M\oplus T\arrow[rightarrow]{r}[font=\normalsize]{\pi_2} & T\arrow[rightarrow]{r} & 0
\end{tikzcd}
\end{center}
\end{enumerate}
\begin{proof}
(\textit{3})$\Longrightarrow (\textit{1}),(\textit{2})$: Notemos que $\pi_1:M\oplus T\to M$ es una inversa de $\iota_1$ a izquierda, y así $\iota_1$ es sección. De forma análoga, $\iota_2:T\to M\oplus T$ es inversa a derecha de $\pi_2$ y así es retracción. Entonces, $f = \varphi^{-1}\circ \iota_1$, y como $\varphi$ es isomorfismo, tenemos una composición de secciones y así $f$ es sección. De forma análoga, vemos que $g$ es retracción.

(\textit{1})$\Longrightarrow$(\textit{3}): Sea $f$ sección. Entonces existe $h:N\to M$ tal que $h\circ f = 1_M$. Defino $\varphi : N\to M\oplus T$ como $\varphi(x)=(h(x),g(x))$. Sea $y\in M$. Luego, $\varphi\circ f(y)=(h(f(y)),g(f(y))) = (y,0) = \iota_1(y)$, donde la segunda igualdad se da, en la primer coordenada por la condición de $f$ sección y en la segunda por la exactitud de la secuencia corta. Además, $\pi_2\circ\varphi(y)=\pi_2(h(y),g(y)) = g(y)$. Entonces, el diagrama conmuta. Resta ver que $\varphi$ es un isomorfismo.

Veamos que $\varphi$ es monomorfismo y epimorfismo. En efecto, si $\varphi(x)=0$ entonces $h(x)=g(x)=0$. Pero entonces, $x\in \ker g = \im f$, y así tenemos un $y\in M$ tal que $x=f(y)$. Luego, $h(x) = h\circ f(y) = 0$ implica que $y=0$ y así $x=g(y)=g(0)=0$, lo que quiere decir que $\varphi$ es monomorfismo.

Ahora, sea $(y,z)\in M\oplus T$. Queremos hallar un $x$ tal que $\varphi(x)=(y,z)$, es decir, $h(x)=y$ y $g(x)=z$. Pero sabemos que $g$ es epimorfismo (pues estamos en una sucesión exacta corta), lo que quiere decir que existe un $\tilde{x}$ tal que $g(\tilde{x})=z$. Pero entonces, definamos $x = f(y-h(\tilde{x})) + \tilde{x}$. Notemos que $h(x)=h(f(y-h(\tilde{x}))) + h(\tilde{x}) = y - h(\tilde{x})+h(\tilde{x}) = y$ y además $g(x) = g(f(y-h(\tilde{x}))) + g(\tilde{x}) = g(\tilde{x}) = z$.

(\textit{2})$\Longrightarrow$(\textit{3}): Sale de forma similar, simplemente siguiendo las flechas.
\end{proof}
\end{teo}

\begin{ex}\begin{itemize}
\item Es claro que $0\longrightarrow \ZZ_2 \stackrel{\cdot 2}{\longrightarrow} \ZZ_4 \stackrel{\pi}{\longrightarrow} \ZZ_2\longrightarrow 0$ no se parte (por cualquiera de las tres razones).

\item Notemos que $0\longrightarrow \ZZ\stackrel{\cdot n}{\longrightarrow}\ZZ \stackrel{\pi}{\longrightarrow} \ZZ/n\ZZ \longrightarrow 0$ no se parte pues $\pi$ no es retracción al tener un único morfismo $\ZZ_n\to \ZZ$.

\item Si $A=k$ es un cuerpo, toda sucesión exacta corta de $k$-espacios vectoriales $$0\longrightarrow V\stackrel{f}{\longrightarrow}T\stackrel{g}{\longrightarrow}W\longrightarrow 0$$ se parte pues $f$ monomorfismo de $k$-espacios vectoriales si y sólo si $f$ sección. En efecto, si $B=\{v_i\}_{i\in I}$ una base entonces $\{f(v_i)\}_{i\in I}$ es linealmente independiente en $T$, entonces completo a una base y vuelvo, mandando $f(v_i)\mapsto v_i$ y el resto a cualquier cosa.

\item Sea $k$ un cuerpo y $a\in k$. Consideremos $\mathrm{ev}_a:k[x]\to k$ dada por $f\mapsto f(a)$ (la evaluación en $a$). Entonces, tenemos un isomorfismo como anillos entre $k[x]/\langle x-a\rangle \simeq k$. Pero $k[x]/\langle x-a\rangle$ también se puede pensar como un $k[x]$-módulo vía $f\cdot \overline{g} = \overline{fg}$. 

Es decir, si yo tengo $M,N$ grupos abelianos y $\varphi:M\to N$ isomorfismo de grupos abelianos entonces si $M$ es un $B$-módulo puedo considerar a $N$ como un $B$-módulo vía $b\cdot n = \varphi(b\varphi^{-1}(n))$.

Entonces, podemos ver a $k[x]/\langle x-a\rangle$ como $k[x]$-módulo y así $k$ es un $k[x]$-módulo vía $f\cdot b = f(a)b$. Esto quiere decir que tengo la sucesión exacta corta de $k[x]$-módulos $$0\longrightarrow \langle x-a\rangle \stackrel{\iota}{\longrightarrow} k[x]\stackrel{\mathrm{ev}_a}{\longrightarrow} k \longrightarrow 0$$ que no se parte pues $\mathrm{ev}_a:k[x]\to k$ no es retracción ya que el único morfismo $\psi:k\to k[x]$ que es $k[x]$-lineal es el morfismo trivial. En efecto, sean $p,q\in k[x]$ tales que $p(a)=q(a)$, esto implica que $p(a) = \psi(p\cdot 1) = p\psi(1)$ y así $p\psi(1) = q\psi(1)$ que implica $\psi(1)=0$ pues $k[x]$ es íntegro.

Sin embargo, podemos ver a esa sucesión exacta corta como de $k$-espacios vectoriales, y ahí sí se parte (pues toda sucesión exacta corta de $k$-espacios vectoriales se parte).

\end{itemize}
\end{ex}

\section{Módulos Libres}

\begin{defn}
Sea $M$ un $A$-módulo y sea $\{x_i\}_{i\in I}$ con $x_i\in M \; \forall \, i\in I$. Una \textbf{combinación lineal} de $\{x_i\}_{i\in I}$ es $\displaystyle\sum_{i\in I} a_i x_i$ con $a_i\in A$ y $\{a_i\}_{i\in I}$ de soporte finito.

Decimos que $\{x_i\}_{i\in I}$ es \textbf{linealmente independiente} si $\displaystyle\sum_{i\in I}a_ix_i = 0$ implica que $a_i=0\;\forall\, i\in I$. Es decir, si la única combinación lineal que da $0$ es la trivial. En otro caso decimos que son \textbf{linealmente dependientes}.

Finalmente, una \textbf{base} $B=\{x_i\}_{i\in I}$ de $M$ es un conjunto linealmente independiente que genera a $M$. Equivalentemente, todo elemento de $M$ se escribe de forma única como combinación lineal de los elementos de $B$.
\end{defn}

\begin{ex}
\begin{itemize}
\item Sea $A=k$ un cuerpo. De Álgebra Lineal, sabemos que todo $k$-espacio vectorial $V$ tiene una base, todo conjunto linealmente independiente de $V$ se puede extender a una base, de todo conjunto de generadores de $V$ podemos extraer una base y todas las bases tienen el mismo cardinal.

\item Sea $A=M$, entonces $B=\{1\}$ es una base de $M=A$ como $A$-módulo.

\item Consideremos el caso de $\ZZ$. Es claro que $\{2,3\}$ genera a $\ZZ$ ¡pero no puedo extraer una base de ahí! O sea, $\langle 2\rangle \neq \ZZ \neq \langle 3\rangle$ y $3\cdot 2 - 2\cdot 3 = 0$ y así $\{2,3\}$ tampoco es base.

\item Sea $M$ un $A$-módulo y $x\in M$. Entonces $\{x\}$ será linealmente independiente si y sólo si $x$ no es de torsión.

\item Sea $A=\ZZ$ y $M=\ZZ_n$. Entonces $\ZZ_n$ no tiene base pues todo elemento es de torsión.
\end{itemize}
\end{ex}

\begin{defn}
Un $A$-módulo $M$ se dice \textbf{libre} (sobre $A$) si admite alguna base.
\end{defn}

\begin{ex}
\begin{itemize}
\item Todo $k$-espacio vectorial es libre sobre $k$.
\item $M=A$ es libre como $A$-módulo.
\item $\ZZ_n$ no es un $\ZZ$-módulo libre.
\item $\QQ$ no es libre como $\ZZ$-módulo: tomando $a=\dfrac{m}{n}$ y $b=\dfrac{r}{s}$ entonces $\{a,b\}$ es linealmente dependiente pues $(nr)a - (sm)b = 0$, y así, si fuera libre estaría generado por un único elemento, lo que quiere decir que sería cíclico. Pero sabemos que $\QQ\not\simeq \ZZ, \ZZ_n$. Absurdo, entonces no es libre.
\item Cocientes de módulos libres no necesariamente resultan libres. El ejemplo más claro es $\ZZ$ visto como $\ZZ$-módulo, pero $\ZZ_n = \ZZ/n\ZZ$ no es libre como $\ZZ$-módulo.
\item Submódulos de módulos libres no necesariamente resultan libres. Consideremos el anillo de matrices $A=\ZZ^{2\times 2} = \left\{  \begin{pmatrix}a & b \\ c & d\end{pmatrix}: a,b,c,d\in \ZZ\right\}$. $M=A$ es un $A$-módulo libre con base $\left\{\begin{pmatrix}1&0\\0&1\end{pmatrix}\right\}$. Sea $N = \left\{ \begin{pmatrix}\alpha & 0 \\ \beta & 0 \end{pmatrix} : \alpha,\beta\in\ZZ\right\}$. Entonces $N$ es un $A$-submódulo de $M$. Pero $N$ no es un $A$-submódulo libre, pues todo elemento resulta de torsión ya que $\begin{pmatrix}\beta & -\alpha \\ 0 & 0\end{pmatrix}\begin{pmatrix}\alpha & 0 \\ \beta & 0 \end{pmatrix} = 0$.
\end{itemize}
\end{ex}

\begin{obs}
Sea $S$ un conjunto. Entonces $A^{(S)} = \displaystyle\bigoplus_{s\in S} A$ es un $A$-módulo libre con base $\{e_s\}_{s\in S}$ donde $e_s\in A^{(S)}$ con $e_s:S\to A$ y $e_s(t) = \delta_{st}$ (Es decir, la base canónica).
\end{obs}

\begin{prop}
Si $M$ es un $A$-módulo libre con base $\{x_i\}_{i\in I}$ entonces $M\simeq A^{(I)}$.
\end{prop}

Ahora, caractericemos a las bases de forma categórica:

\begin{prop}
Supongamos que $M$ es un $A$-módulo libre con base $B$. Entonces se tiene la siguiente propiedad universal:

Para todo $A$-módulo $N$ y $f:B\to N$ función de conjuntos, existe una única $\overline{f}:M\to N$ transformación $A$-lineal tal que $\left.\overline{f}\right|_{B} = f$. Es decir, el siguiente diagrama conmuta:

\begin{tikzcd}[row sep=3.3em,column sep=4em,minimum width=2em]
 B\arrow[hookrightarrow]{d}[left,font=\normalsize]{\iota}\arrow[]{r}[above, font=\normalsize]{f}& G' \\
M\arrow[dashed]{ur}[right, font=\normalsize]{\overline{f}} & \\
\end{tikzcd}
\begin{proof}

Si $B=\{x_i\}_{i\in I}$ entonces podemos escribir a $x=\displaystyle\sum_{i\in I}a_ix_i$ de forma única. Entonces, definimos $\overline{f}(x) = \displaystyle\sum_{i\in I}a_if(x_i)$. Es claro ver que si hay alguna, debe cumplir eso por la $A$-linealidad y que esta lo cumple.

\end{proof}
\end{prop}

\begin{prop}
Sea $M$ un $A$-módulo, $B\subseteq M$ un subconjunto que cumple la propiedad universal de la proposición anterior. Entonces $M$ es un $A$-módulo libre y $B$ es base de $M$.
\end{prop}

\begin{prop}
Todo $A$-módulo $M$ es cociente de un $A$-módulo libre.
\begin{proof}
Sea $S\subseteq M$ un conjunto de generadores de $M$. Entonces $A^{(S)}$ es un $A$-módulo libre. Consideremos $\pi:A^{(S)}\to M$ el único morfismo qeu cumple que $\pi(e_s)=s$. Entonces $\pi$ es un epimorfismo pues $S$ genera a $M$. Por el Primer Teorema de Isomorfismo, $M\simeq A^{(S)}/\ker \pi$. Y listo.
\end{proof}
\end{prop}

\begin{obs}
Si $M$ es libre y $B$ base con cardinal $\sharp B=n$ entonces $M\simeq A^n$.
\end{obs}

\begin{obs}
Sean $M,N$ dos $A$-módulos libres y $B_1$ base de $M$, $B_2$ base de $N$ con $\sharp B_1 = \sharp B_2$. Entonces $M\simeq N$. Esto se ve usando la propiedad universal de la base.
\end{obs}

\begin{obs}
Si $M$ y $N$ son libres y $M\simeq N$ con $B_1,B_2$ bases de $M$ y $N$ respectivamente. Esto no implica que $\sharp B_1 = \sharp B_2$. Más aún, un mismo módulo puede tener bases de distinto cardinal.
\end{obs}

\begin{obs}
Sea $A$ un anillo de división y $M$ un $A$-módulo. Tomo $x\in M$, $x\neq 0$. Entonces $\{x\}$ es linealmente independiente.
\end{obs}

\begin{teo}
Si $A$ es un anillo de división y $M$ un $A$-módulo, entonces $M$ es un $A$-módulo libre. Más aún, si $L\subseteq M$ es linealmente independiente, existe una base $B$ de $M$ tal que $L\subseteq B$.
\begin{proof}
Es exactamente la misma demostración que para espacios vectoriales que se hace en Álgebra Lineal.

Sea $\mathcal{C} = \{L'\subseteq M : L\subseteq L' \text{ y es LI}\}$. Notemos que $\mathcal{C}\neq \emptyset$ pues $L\in\mathcal{C}$. Consideremoslo un poset con la inclusión. Sea $\mathcal{D}$ una cadena de $\mathcal{C}$. Tomemos entonces $D=\displaystyle\bigcup_{L'\in\mathcal{D}}L'$. Notemos que $D\in\mathcal{C}$ pues $L\subseteq D$ ya que $L\subseteq L'$ para cualquier $L'\in\mathcal{D}$ y además es LI ya que $\displaystyle\sum_{i\in I} a_ix_i = 0$ al ser una combinación lineal finita y $\mathcal{D}$ una cadena, tenemos que todos los $x_i$ están en algún $L'\in\mathcal{D}$, que es LI y así $a_i=0 \;\forall\, i\in I$. Entonces $D$ es una cota superior de la cadena y así por el Lema de Zorn, $\mathcal{C}$ tiene elementos maximales.

Sea $B\in\mathcal{C}$ algún elemento maximal. Veamos que $B$ es de hecho una base. $B$ es LI y $L\subseteq B$ pues $B\in\mathcal{C}$. Resta probar que $B$ genera a $M$. Supongamos que $\langle B\rangle \neq M$, entonces existe $x\in M$ tal que $x\notin \langle B\rangle$. Consideremos $\tilde{L}=B\cup \{x\}$. Veamos que $\tilde{L}$ resulta LI, lo que nos dará un absurdo por ser $B$ maximal en $\mathcal{C}$. Consideremos una combinación lineal $\left(\displaystyle\sum_{i\in I} a_i x_i\right) + ax = 0$. Si $a=0$ entonces al ser $B$ base, todos los $a_i$ deben ser $0$. Ahora, si $a\neq 0$, al ser $A$ anillo de división, tenemos que $\displaystyle\sum_{i\in I}(-a_i a^{-1})x_i = x$ y así $x\in \langle B\rangle$. Absurdo por suposición. El teorema sigue.
\end{proof}
\end{teo}

\begin{prop}
Sea $0\longrightarrow M\stackrel{f}{\longrightarrow} N\stackrel{g}{\longrightarrow} T\longrightarrow 0$ una sucesión exacta corta de $A$-módulos. Si $T$ es un $A$-módulo libre entonces $g$ resulta retracción y así $N\simeq M\oplus T$. Es decir, la sucesión exacta corta se escinde.
\begin{proof}
Sea $B=\{x_i\}_{i\in I}$ una base de $T$. Defino $h:B\to N$ función vía $h(x_i)=y_i$ donde $y_i\in N$ cumple que $g(y_i)=x_i$ (esto se puede pues $g$ es epimorfismo). Por la propiedad universal de la base, existe una única $\overline{h}:T\to N$ transformación $A$-lineal que extiende a $h$. Pero notemos que $g\circ \overline{h}=1$ pues $g(\overline{h}(x_i)) = g(y_i) = x_i$ y como es la identidad en la base, por la propiedad universal debe ser la identidad, y así $g$ es retracción.
\end{proof}
\end{prop}

\begin{obs}
Si $S\subseteq M$ submódulo y $M/S$ es libre, entonces $M\simeq S\oplus M/S$ pues $0\longrightarrow S\stackrel{\iota}{\longrightarrow} M\stackrel{\pi}{\longrightarrow} M/S\longrightarrow 0$ es una sucesión exacta corta que se parte por la proposición anterior.
\end{obs}

\begin{defn}
Sea $M$ un $A$-módulo. Un conjunto minimal de generadores de $M$ es un subconjunto $\{x_i\}_{i\in I}\subseteq M$ tal que $\langle x_i\rangle_{i\in I} = M$ pero $\forall \, j\in I$ tal que $\langle x_i\rangle_{i\neq j} \neq M$. 
\end{defn}

\begin{obs}
Si $B\subseteq M$ es una base entonces $B$ es un conjunto minimal de generadores.
\end{obs}

\begin{obs}
Ser un conjunto minimal no implica ser base, como ya vimos con $\{2,3\}$ en $\ZZ$.
\end{obs}

\begin{teo}
Sea $M$ un $A$-módulo. Entonces:
\begin{enumerate}
\item Si $M$ es finitamente generado entonces todo conjunto minimal de generadores de $M$ es finito. En particular, si $M$ es libre y finitamente generado, toda base de $M$ es finita.
\item Si $M$ no es finitamente generado entonces todos los conjuntos minimales de generadores de $M$ tienen el mismo cardinal. En particular si $M$ es libre y no es finitamente generado, todas las bases de $M$ tienen el mismo cardinal.
\end{enumerate}
\begin{proof}
Sea $X$ un conjunto minimal de generadores de $M$ e $Y$ un conjunto de generadores de $M$. Para cada $y\in Y$ existe un $X_y\subseteq M$ finito tal que $y\in \langle X_y\rangle$ pues $X$ genera a $M$. Considero $\displaystyle\bigcup_{y\in Y}X_y \subseteq X$. Como $y\in\langle X_y\rangle$ para todo $y$, y además $\langle Y\rangle =M$, tenemos que $\left\langle \displaystyle\bigcup_{y\in Y}X_y\right\rangle = M$, y así como $X$ es un conjunto minimal de generadores, tenemos la otra inclusión y entonces $\displaystyle\bigcup_{y\in Y}X_y = X$.

Para probar el primer ítem, si $M$ es finitamente generado, existe $Y$ conjunto de generadores finito y así, dado $X$ conjunto minimal de generadores $X = \displaystyle\bigcup_{y\in Y} X_y$ que es una unión finita de conjuntos finitos. Entonces $X$ es finito.

Para probar el segundo ítem, si $M$ no es finitamente generado, sean $X,Y$ conjuntos minimales de generadores. Entonces $X=\displaystyle\bigcup_{y\in Y} X_y$, y como los $X_y$ son finitos, tenemos que $\sharp X \leq \sharp Y$. Pero invirtiendo los roles de $X$ e $Y$, se ve que $\sharp Y\leq \sharp X$. Por Cantor-Bernstein, tenemos que $\sharp X=\sharp Y$. Y estamos.
\end{proof}
\end{teo}

\begin{defn}
Sea $A$ un anillo. Decimos que $A$ tiene \textbf{noción de rango} si para todo $A$-módulo libre $M$, todas las bases tienen el mismo cardinal.
\end{defn}

\begin{obs}
Son equivalentes \begin{itemize}\item $A$ tiene noción de rango \item Para todo $A$-módulo de tipo finito y $B_1,B_2$ bases de $M$, $n = \sharp B_1 = \sharp B_2 = m$ \item Para todos $A$-módulos $M,N$ tenemos que $A^m\simeq A^n$ implica $m=n$ \item Para todo $C\in A^{n\times m}$, $D\in A^{m\times n}$ tales que $CD=I_n$ y $DC=I_m$ entonces $n=m$. \end{itemize}
\end{obs}

\begin{obs}
Si $A$ es un anillo de división entonces tiene noción de rango.
\end{obs}

\begin{obs}
Sea $B$ un anillo y consideremos el $B$-módulo libre $B^{(\NN)}$. Sea $A=\End_B(B^{(\NN)})$. Entonces $(A,+,\circ)$ es un anillo. $A$ no tiene noción de rango pues $A\simeq A\oplus A$ vía $\psi :A\to A\oplus A$ dado por $\psi(f)=(f_1,f_2)$ con $f_1(e_n)=f(e_{2n-1})$ y $f_2(e_n)=f(e_{2n})$.
\end{obs}

\begin{prop}
Si $f:A\to B$ es un morfismo de anillos y $B$ tiene noción de rango, entonces $A$ también.
\begin{proof}
Sea $C\in A^{n\times m}$ y $D\in A^{m\times n}$ tal que $CD=I_n$ y $DC=I_m$. Entonces, consideremos $C'\in B^{n\times m}$ y $D'\in B^{m\times n}$ definidas por $C'_{ij} = f(C_{ij})$ y $D'_{ij}=f(D_{ij})$. Como $f$ es morfismo de anillos, tenemos que $C'D' = I_n$ y $D'C'=I_m$. Pero como $B$ tiene noción de rango, $m=n$. Entonces $A$ tiene noción de rango. Y estamos.
\end{proof}
\end{prop}
\begin{cor}
Todo anillo conmutativo tiene noción de rango.
\begin{proof}
Sea $\mathfrak{m}$ un ideal maximal de $A$ anillo conmutativo. Entonces $A\stackrel{\pi}{\longrightarrow} A/\mathfrak{m}$, y los cuerpos tienen noción de rango (por Álgebra Lineal!), entonces por la proposición anterior, $A$ tiene noción de rango.
\end{proof}
\end{cor}

\begin{defn}
Si $A$ tiene noción de rango y $M$ es un $A$-módulo libre, se define $\dim_A(M)=\sharp B$ para cualquier base $B$ de $M$.
\end{defn}

\section{Localización de Módulos}

Recordemos que si $A$ es un anillo y $S\subseteq Z(A)$ es multiplicativamente cerrado y central, entonces $S^{-1}A = \left\{\dfrac{a}{s}:a\in A, s\in S\right\}$ con $\dfrac{a}{s}$ la clase de $(a,s)\in A\times S$ bajo la relación $(a,s)\sim (a',s')$ si $\exists t\in S$ tal que $t(as'-sa')=0$.

\begin{defn}
Sea $M$ un $A$-módulo a izquierda y sea $S\subseteq Z(A)$ multiplicativamente cerrado. Defino una relación de equivalencia en $M\times S$ por $(m,s)\sim (m',s')$ si $\exists t\in S$ tal que $t(s'm - sm')=0$.

Definimos entonces la localización de $M$ por $S$ como $S^{-1}M = \left\{ \dfrac{m}{s} : m\in M, s\in S\right\}$, con la suma usual. Entonces $S^{-1}M$ es un $A$-módulo vía $a\cdot \dfrac{m}{s} = \dfrac{am}{s}$. Más aún, es un $S^{-1}A$-módulo vía $\dfrac{a}{s}\cdot \dfrac{m}{t} = \dfrac{am}{st}$.
\end{defn}

\begin{obs}
Hay que chequear que la relación de equivalencia $\sim$ de la definición anterior de hecho lo es. La simetría y reflexividad son obvias. Veamos la transitividad.	En efecto, supongamos que $(m_1,s_1)\sim (m_2,s_2)$ y $(m_2,s_2)\sim (m_3,s_3)$. Entonces, existen $t_1,t_2\in S$ tales que $t_1(s_2m_1 - s_1m_2)=0$ y $t_2(s_3m_2 - s_2m_3)=0$. Entonces $t_1s_2m_1=t_1s_2m_2$ y $t_2s_3m_3 = t_2s_2m_3$. Multiplicando la primera por $t_2s_3$ y la segunda por $t_1s_1$ y usando que los elementos de $S$ son centrales, obtenemos que $(t_1t_2s_2)(s_3m_1) = (t_1t_2s_2)(s_1m_3)$ y listo.
\end{obs}

\begin{defn}
Sea $S\subseteq Z(A)$ multiplicativamente cerrado y central, y $M,N$ dos $A$-módulos. Si $f:M\to N$ es un morfismo $A$-lineal, definimos $S^{-1}f:S^{-1}M\to S^{-1}N$ por $S^{-1}f\left(\dfrac{m}{s}\right) = \dfrac{f(m)}{s}$. Este $S^{-1}f$ es un morfismo de $S^{-1}A$-módulos.
\end{defn}

\begin{prop}
Sea $S\subseteq Z(A)$ multiplicativamente cerrado y central y tomemos una sucesión exacta corta de $A$-módulos $0\longrightarrow N\stackrel{f}{\longrightarrow} M\stackrel{g}{\longrightarrow} T\longrightarrow 0$. Entonces $$0\longrightarrow S^{-1}N\stackrel{S^{-1}f}{\longrightarrow} S^{-1}M\stackrel{S^{-1}g}{\longrightarrow}S^{-1}T\longrightarrow 0$$ es una sucesión exacta corta de $S^{-1}A$-módulos. Es decir, localizar es exacto.
\begin{proof}
Tenemos que probar varias cosas. Primero, veamos que $S^{-1}f$ es inyectivo. En efecto, sea $\dfrac{n}{s}\in S^{-1}N$ tal que $S^{-1}f\left(\dfrac{n}{s}\right)=0$. Es decir, $\dfrac{f(n)}{s}=0$ y así existe un $s'\in S$ tal que $s'f(n)=f(s'n)=0$. Pero como $f$ es inyectiva, $s'n\in \ker f =0$. Entonces $s'n=0$ y así $\dfrac{n}{s}=0$. Por lo tanto $S^{-1}f$ es inyectivo.

Ahora, $S^{-1}g$ es sobreyectivo, pues para cualquier $\dfrac{t}{s}\in S^{-1}T$, al ser $g$ sobrectivo, existe $m\in M$ tal que $g(m)=t$, y así $S^{-1}g\left(\dfrac{m}{s}\right) = \dfrac{g(m)}{s} = \dfrac{t}{s}$.

Finalmente, debemos ver que $\im S^{-1}f = \ker S^{-1}g$. Veamos la doble inclusión. Sabemos que $g\circ f(n)=0$ por la condición de exactitud de la sucesión. Entonces, $S^{-1}g\circ S^{-1}f \left(\dfrac{n}{s}\right) = \dfrac{g\circ f(n)}{s} = 0$, y así $\im S^{-1}f\subseteq \ker S^{-1}g$. Para ver la otra inclusión, sea $\dfrac{m}{s}\in \ker S^{-1}g$. Entonces, existe un $s'\in S$ tal que $s'g(m)=0$. O sea, $g(s'm)=0$ y así $s'm\in \ker g =\im f$. Entonces, existe un $n\in N$ tal que $f(n)=s'm$. Esto quiere decir que $S^{-1}f\left(\dfrac{n}{ss'}\right) = \dfrac{f(n)}{ss'} = \dfrac{s'm}{ss'} = \dfrac{m}{s}$. Y estamos.

\end{proof}
\end{prop}

\section{Módulos Noetherianos}

\begin{obs}
Sea $A$ un anillo y $M$ un $A$-módulo de tipo finito. Si $M=\langle x_i\rangle_{i\in I}$ entonces existen $\{x_{i_1},\ldots , x_{i_r}\}\subseteq\{x_i\}_{i\in I}$ tales que $M=\langle x_{i_1},\ldots , x_{i_r}\rangle$. En efecto, si $M=\langle Y\rangle$ con $Y$ finito cualquier $y\in Y$ se puede escribir como combinación lineal (finita) de los $x_i$ y unimos esos para obtener un conjunto finito.
\end{obs}

\begin{prop}
Si $f:M\to N$ es un epimorfismo de $A$-módulos y $M$ es finitamente generado entonces $N$ también lo es. Más aún, si tenemos una sucesión exacta corta de $A$-módulos $0\longrightarrow T\stackrel{g}{\longrightarrow} M\stackrel{f}{\longrightarrow} N\longrightarrow 0$ entonces $M$ finitamente generado implica $N$ finitamente generado y si $T$ y $N$ son finitamente generados entonces $M$ lo es.

\begin{proof}
Supongamos que $M=\langle x_1,\ldots , x_n\rangle$. Veamos que $\langle f(x_1),\ldots , f(x_n)\rangle$ genera a $N$. En efecto, como $f$ es epimorfismo, para cualquier $y\in N$ existe $x\in M$ tal que $f(x)=y$, y además, como los $x_i$ generan $M$, tenemos que existen $a_1,\ldots ,a_n\in A$ tales que $a_1x_1+\ldots +a_nx_n = x$, y aplicando $f$, $a_1f(x_1)+\ldots + a_nf(x_n) = f(x)=y$. Y listo.

Ahora, para ver la segunda afirmación, notemos que $\im g$ es finitamente generado pues $T$ lo es. Ahora, sea $\langle y_1,\ldots , y_n\rangle$ un conjunto finito de generadores de $N$. Como $f$ es sobreyectiva, podemos tomar $x_1,\ldots , x_n$ tales que $f(x_i)=y_i$. Ahora, notemos que si juntamos los generadores de $\im g = \ker f$ y los $x_i$ esto genera a $M$. En efecto, si tomamos un $x\in M$ cualquiera, podemos escribir $f(x) = a_1y_1 + \ldots + a_ny_n$ y así $x - a_1x_1 - \ldots - a_n x_n \in \ker f$, que es finitamente generado. Y listo.
 
\end{proof}
\end{prop}

\begin{obs}
Submódulos de finitamente generados no necesariamente son finitamente generados, como ya vimos con $k[x_i]_{i\in\NN}$.
\end{obs}

\begin{obs}
Sea $A$ anillo, $S$ subconjunto infinito y $B=A^S$ como anillo con la suma y producto coordenada a coordenada. Entonces $B=\langle 1\rangle$ como $B$-módulo. En cambio $A^{(S)}\subseteq A^S$ es un ideal de $B$ que no es finitamente generado.
\end{obs}

\begin{defn}
Sea $M$ un $A$-módulo. Decimos que $M$ es \textbf{Noetheriano} si todos sus submódulos son de tipo finito.
\end{defn}

\begin{obs}
\begin{itemize}
\item Por definición, si $M$ es noetheriano entonces es finitamente generado.
\item $A=k[x_i]_{i\in \NN}=M$ no es noetheriano.
\item Si $A$ es DIP, entonces $A$ es un $A$-módulo noetheriano pues todo ideal (=submódulo) es cíclico.
\item Si $k$ es un cuerpo y $V$ un $k$-espacio vectorial entonces es noetheriano si y sólo si $\dim_k V < \infty$.
\item Si $M$ tiene una cantidad finita de submódulos, entonces es noetheriano. En efecto, supongamos que $M$ no lo es. Entonces existe $N\subseteq M$ que no es finitamente generado. Sea $x_1\in N$. Entonces $\langle x_1\rangle \subset N$, pero no vale la igualdad pues $N$ no es finitamente generado. Entonces, tomamos $x_2\in N-\langle x_1\rangle$ y consideramos $\langle x_1,x_2\rangle\subset N$. Nuevamente, no vale la igualdad pues $N$ no es finitamente generado. Por lo tanto, podemos seguir así sucesivamente. Esto es absurdo pues teníamos una cantidad finita de submódulos.
\item Si $M$ es noetheriano y $N\subseteq M$ es un submódulo, entonces también es noetheriano.
\end{itemize}
\end{obs}

\begin{prop}
Sea $M$ un $A$-módulo. Son equivalentes:
\begin{enumerate}
\item $M$ es noetheriano.
\item Toda cadena creciente de submódulos de $M$, $N_1\subseteq N_2\subseteq \ldots \subseteq N_r\subseteq N_{r+1}\subseteq \ldots$ se estanca. Es decir, existe un $r_0$ tal que $N_r = N_{r_0}$ para todo $r\geq r_0$.
\item Todo conjunto no vacío de submódulos de $M$ tiene elementos maximales respecto de la inclusión.
\end{enumerate}
\begin{proof}
(1)$\Longrightarrow$(2): Sea $N_1\subseteq N_2\subseteq \ldots$ una cadena ascendente de submódulos. Entonces $N=\displaystyle\bigcup_{i\in\NN}N_i$ es un submódulo de $M$. Como $M$ es noetheriano, $N=\langle x_1,\ldots , x_s\rangle$. Como son finitos y $x_i\in N_{r_i}$, existen algún $r_0$ tal que $x_i\in N_{r_0}$ para todo $i$. Entonces $N\subseteq N_{r_0}$ y así la cadena se estanca.

(2)$\Longrightarrow$(3): Sea $\mathcal{C}\neq \emptyset$ un conjunto de submódulos de $M$. Supongamos que $\mathcal{C}$ no tiene elementos maximales. Sea $N_1\in\mathcal{C}$. Como $N_1$ no es maximal, hay un $N_2\in\mathcal{C}$ tal que $N_1\subset N_2$. Siguiendo así inductivamente, puedo construir una cadena $N_1\subset N_2\subset\ldots $ que nunca se estaciona. Absurdo.

(3)$\Longrightarrow$(1): Sea $N\subseteq M$ un submódulo. Debo ver que es finitamente generado. Sea $\mathcal{C}=\{T\subseteq N : T \text{ submódulo finitamente generado}\}$. Notemos que $0\in\mathcal{C}$ y así es no vacío. Por hipótesis, existe un $\overline{N}\in\mathcal{C}$ maximal. Veamos que $\overline{N}=N$ y así resultará finitamente generado. Supongamos que $\overline{N}\subset N$, entonces hay un $x\in\overline{N}$ tal que $x\notin N$, y así $\overline{N} + \langle x\rangle\subseteq N$ y es finitamente generado. Pero entonces $\overline{N}$ no es maximal. Y estamos. 

\end{proof}
\end{prop}

\begin{prop}
Sea $0\longrightarrow T\stackrel{g}{\longrightarrow} N\stackrel{f}{\longrightarrow} M\longrightarrow 0$ sucesión exacta corta de $A$-módulos. Entonces $N$ es noetheriano si y sólo si $T$ y $M$ lo son.
\begin{proof}
($\Longrightarrow$) $T$ es noetheriano pues $T\simeq \im g\subseteq M$ submódulo de noetheriano. Además, si $S\subseteq M$, tenemos que $S=f(f^{-1}(S))$ y así $S$ es finitamente generado.

($\Longleftarrow$) Sea $S\subseteq M$ un submódulo. Entonces, tenemos la sucesión exacta corta de $A$-módulos $0\longrightarrow g^{-1}(S)\stackrel{g}{\longrightarrow} S\stackrel{f}{\longrightarrow} f(S)\longrightarrow 0$. Como $T$ y $M$ son noetherianos entonces $g^{-1}(S)$ y $f(S)$ son finitamente generados y así $S$ lo es por una de las proposiciones que vimos.

\end{proof}
\end{prop}

\begin{obs}
Se deduce de la proposición que si $f:N\to M$ es un epimorfismo de $A$-módulos y $N$ es noetheriano entonces $M$ lo es.
\end{obs}

\begin{cor}
Si $M_1,\ldots , M_n$ son $A$-módulos noetherianos, entonces $\displaystyle\bigoplus_{i=1}^{n} M_i$ es noetheriano.
\begin{proof}
Simplemente es proceder por inducción, considerando que la sucesión exacta corta $0\longrightarrow M_n \stackrel{\iota}{\longrightarrow}\displaystyle\bigoplus_{i=1}^{n} M_i \stackrel{\pi}{\longrightarrow} \displaystyle\bigoplus_{i=1}^{n-1} M_i \longrightarrow 0$ y las puntas son noetherianas.
\end{proof}
\end{cor}

\begin{obs}
Notemos que necesitamos que sean finitos en el corolario anterior pues $\ZZ^{(\NN)} = \displaystyle\bigoplus_{n\in\NN}\ZZ$ no es un $\ZZ$-módulo noetheriano.
\end{obs}

\begin{defn}
Un anillo $A$ se dice \textbf{noetheriano} (a izquierda) si $A$ como $A$-módulo (a izquierda) es noetheriano. Es decir, si todos los ideales (a izquierda) de $A$ son finitamente generados.
\end{defn}

\begin{ex}
\begin{itemize}
\item $A$ DIP es noetheriano.
\item $A$ anillo de división es noetheriano.
\item $A=k[x_i]_{i\in\NN}$ no es noetheriano.
\end{itemize}
\end{ex}

\begin{prop}
Si $A$ es un anillo noetheriano y $M$ un $A$-módulo de tipo finito entonces $M$ es un $A$-módulo noetheriano.
\begin{proof}
Como $A$ es un $A$-módulo noetheriano entonces $A^n$ es un $A$-módulo noetheriano. Como $M$ es finitamente generado, existe $\varphi:A^n\to M$ epimorfismo (pues es cociente de un libre). Entonces $M$ es noetheriano.
\end{proof}
\end{prop}

\begin{prop}
Sea $A$ anillo noetheriano y $S\subseteq Z(A)$ un subconjunto multiplicativamente cerrado y central. Entonces $S^{-1}A$ es noetheriano.
\begin{proof}
Sea $\iota:A\to S^{-1}A$ la aplicación canónica, y sea $\mathfrak{a}$ un ideal de $S^{-1}A$. Entonces $\langle \iota(\iota^{-1}(\mathfrak{a}))\rangle = \mathfrak{a}$. En notación de ideales contraídos y extendidos, $\mathfrak{a}=\mathfrak{a}^{ce}$. Pero como $\mathfrak{a}^c = \iota^{-1}(\mathfrak{a})$ es un ideal de $A$, debe ser finitamente generado. Entonces $\mathfrak{a} = \langle \iota(\iota^{-1}(\mathfrak{a})) \rangle$ es finitamente generado. Y listo.
\end{proof}
\end{prop}

\begin{teo}[Teorema de la Base de Hilbert]
Sea $A$ un anillo noetheriano. Entonces $A[x]$ es un anillo noetheriano.
\begin{proof}
Sea $I\subseteq A[x]$ un ideal. Queremos ver que es finitamente generado. Supongamos que no lo es. Sea $p_1\in I$ de grado mínimo. Como $I$ no es finitamente generado, entonces $\langle p_1\rangle \subset I$ y no vale la igualdad. Sea $p_2\in I-\langle p_1\rangle$ de grado mínimo. Entonces, $\langle p_1\rangle \subset \langle p_1,p_2\rangle \subset I$ y nuevamente no vale la igualdad pues $I$ no es finitamente generado. Esto quiere decir que podemos construirnos, inductivamente, una sucesión de polinomios con $p_n \in I - \langle p_1,\ldots , p_{n-1}\rangle$ de grado mínimo. Pero notemos que $\gr p_n\leq \gr p_{n+1}$ pues al tener $p_{n+1}\in I - \langle p_1,\ldots , p_n\rangle\subseteq I - \langle p_1,\ldots , p_{n-1}\rangle$, y en el paso anterior elegimos a $p_n$ en vez de a $p_{n+1}$. 

Ahora, sea $a_n$ el coeficiente principal de $p_n$. Consideremos entonces $\mathfrak{a}=\langle a_n\rangle_{n\in\NN}$ es un ideal, finitamente generado, por ser $A$ noetheriano. Entonces, existe un $m\in\NN$ tal que $\mathfrak{a} = \langle a_1,\ldots , a_m\rangle$. Entonces, si $a_{m+1}$ es el coeficiente principal de $p_{m+1}$ entonces $a_{m+1} = \displaystyle\sum_{i=1}^{m}b_i a_i$. Consideremos el polinomio $p=\displaystyle\sum_{i=1}^m b_ix^{\gr p_{m+1} - \gr p_i}p_i$. Es fácil chequear que el coeficiente principal de este polinomio es $a_{m+1}$. Por como lo definimos, es claro que $p\in\langle p_1,\ldots , p_m\rangle$. Definamos $q = p - p_{m+1}$. Este polinomio tiene grado más chico que el de $p_{m+1}$ pues $p$ y $p_{m+1}$ comparten grado y coeficiente principal. Pero además, $q\in I - \langle p_1,\ldots , p_m\rangle$ pues si $q$ estuviera, como $p$ está entonces $p_{m+1}$ estaría. Pero esto es absurdo, pues conseguimos un polinomio de grado más chico que $p_{m+1}$ que está en $I-\langle p_1,\ldots , p_{m}\rangle$. Y estamos.
\end{proof}
\end{teo}

\begin{cor}
Si $A$ es noetheriano entonces $A[x_1,\ldots , x_n]$ es noetheriano.
\end{cor}

\begin{prop}
Si $\varphi:A\to B$ es un morfismo de anillos sobreyectivo y $A$ es anillo noetheriano, entonces $B$ es anillo noetheriano.
\begin{proof}
Sea $\mathfrak{b}_1\subseteq \mathfrak{b}_2\subseteq \ldots$ una cadena ascendente de ideales de $B$. Entonces $f^{-1}(\mathfrak{b}_1)\subseteq f^{-1}(\mathfrak{b}_2)\subseteq \ldots$ es una cadena ascendente de ideales de $A$, y como es noetheriano se estanca. Pero al ser $f$ sobreyectiva, $f(f^{-1}(\mathfrak{b}_i)) = \mathfrak{b}_i$ y así resulta que la cadena original se estancaba. Luego, $B$ es noetheriano y listo.
\end{proof}
\end{prop}

\begin{cor}
Sea $\alpha\in\CC$. Entonces $\ZZ[\alpha]$ es noetheriano. En efecto, $\ZZ[x]$ lo es por el Teorema de la Base de Hilbert y tenemos el morfismo evaluación $\ev_{\alpha}:\ZZ[x]\to \ZZ[\alpha]$ y es sobreyectivo por definición.
\end{cor}

\section{Módulos Artinianos}

La noción de módulo artiniano va a ser, en algún sentido, dual a la noción de módulo noetheriano. Tratemos de precisar esto.

\begin{lem} Sea $A$ anillo y $M$ un $A$-módulo.
Son equivalentes: \begin{enumerate}\item $M$ es finitamente generado.\item Para toda familia $(N_i)_{i\in I}$ de $A$-módulos y $f:\displaystyle\bigoplus_{i\in I}N_i\to M$ epimorfismo, existe $I'\subseteq I$ finito tal que la restricción $\overline{f}:\displaystyle\bigoplus_{i\in I'}N_i\to M$ es epimorfismo.\end{enumerate}
\begin{proof}
$(1)\Longrightarrow (2)$: Supongamos que $M=\langle x_1,\ldots , x_n\rangle$. Como $f$ es epimorfismo existen $a_1,\ldots ,a_n$ tales que $f(a_i)=x_i$ para cada $1\leq i\leq n$. Notemos que para cada $a_j$ hay un subconjunto finito $I_j\subseteq I$ de modo tal que $a_j\in\displaystyle\bigoplus_{i\in I_j}N_i$ (En efecto, esto sigue de escribir a $a_j = n_{j_1}+\ldots + n_{j_k}$ y tomar $I_j = \{j_1,\ldots , j_k\}$). Tomando entonces $I' = \displaystyle\bigcup_{j=1}^n I_j$ obtenemos lo deseado.

$(2)\Longrightarrow (1)$: Notemos que hay un epimorfismo $f:\displaystyle\bigoplus_{m\in M}A\to M$, y por enunciado, un conjunto finito $I$ tal que $\overline{f}:\displaystyle\bigoplus_{i\in I} A \to M$ también es epimorfismo. Pero si $\sharp I = n$ esto es una suryección $A^n\twoheadrightarrow M$, que implica que $M$ es finitamente generado (simplemente es empujar la base canónica y ver que generan). Y estamos.
\end{proof}
\end{lem}

Usando esta equivalencia para finitamente generado, motivamos la siguiente definición:

\begin{defn}
Sea $M$ un $A$-módulo. Decimos que $M$ es \textbf{finitamente cogenerado} si todo $f:M\to\displaystyle\prod_{i\in I}N_i$ monomorfismo de correstringe a un monomorfismo $\overline{f}:M\to\displaystyle\prod_{i\in I'}N_i$ para $I'\subseteq I$ finito.
\end{defn}

\begin{ex}
Notemos que $\ZZ$ no es un $\ZZ$-módulo finitamente cogenerado. En efecto, consideremos un $p$ primo. Entonces la proyección $\ZZ\to\displaystyle\prod_{n\in\mathbb{N}} \ZZ/p^n \ZZ$ es un monomorfismo, por el Teorema Fundamental de la Aritmética. Pero no se correstringe finitamente a un monomorfismo.
\end{ex}

Ahora, dualicemos también la definición de módulo noetheriano:

\begin{defn}
Sea $A$ un anillo y $M$ un $A$-módulo. Decimos que $M$ es \textbf{artiniano} si todo cociente de $M$ es finitamente cogenerado.
\end{defn}

\begin{prop}
Sea $A$ un anillo y $M$ un $A$-módulo. Son equivalentes:
\begin{enumerate}
\item $M$ es artiniano.
\item $M$ satisface la condición de cadena descendiente. Es decir, toda cadena descendiente $N_1\supseteq N_2\supseteq\ldots \supseteq N_r\supseteq \ldots$ de $A$-submódulos de $M$ se estaciona.
\item Todo conjunto no vacío de submódulos de $M$ tiene un elemento minimal.
\end{enumerate}
\begin{proof}
$(1)\Longrightarrow (2)$: Supongamos que $M$ es artiniano y que tenemos una cadena descendiente $L_1\supseteq L_2\supseteq\ldots$ de $A$-submódulos de $M$. Consideremos entonces $K=\displaystyle\bigcap_{n\in\mathbb{N}}L_n$. Esto es precisamente el núcleo de la aplicación $\displaystyle\prod_{n\in\mathbb{N}}\pi_n : M\to\displaystyle\prod_{n\in\NN}M/L_n$. Pero entonces la aplicación inducida $M/K\to\displaystyle \prod_{n\in\NN}M/L_n$ es monomorfismo. Como $M/K$ es cociente de $M$ que es artiniano, debe ser finitamente cogenerado y así hay un $m\in\NN$ tal que la restricción $M/K\to \displaystyle\prod_{n<m}M/L_n$ resulta monomorfismo. Pero entonces $K=\displaystyle\bigcap_{n\in\NN}L_n = \displaystyle\bigcap_{n<m}L_n = L_{m-1}$. Esto nos dice que la cadena se estaciona a partir de $m-1$.

$(2)\Longrightarrow (3)$: Supongamos que $M$ satisface la condición de cadena descendente. Sea $\mathcal{S}$ un subconjunto no vacío de submódulos de $M$. Supongamos que $\mathcal{S}$ no tiene ningun elemento minimal. Entonces, si $L_1\in\mathcal{S}$ existe un $L_2\in \mathcal{S}$ tal que $L_1\supset L_2$, pero no es igual. Ahora, existe un $L_3\in\mathcal{S}$ tal que $L_2\supset L_3$, pero no es igual. Procediendo inductivamente obtenemos una cadena descendiente que no se estaciona. Absurdo. Entonces $\mathcal{S}$ debe tener un elemento minimal.

$(3)\Longrightarrow (1)$: Supongamos que todo conjunto no vacío de submódulos de $M$ tiene un submódulo minimal. Sea $K\subseteq M$ un submódulo, $\{N_i\}_{i\in I}$ una familia de $A$-módulos y además $f:M/K\to \displaystyle\prod_{i\in I}N_i$ un monomorfismo. Sea $i_0\in I$ y $M_{i_0}$ el núcleo de la composición dada por $M\longrightarrow M/K\longrightarrow \displaystyle\prod_{i\in I}N_i \longrightarrow N_{i_0}$. Es fácil ver que $K=\displaystyle\bigcap_{i\in I}M_i$. Para ver que $M$ es artiniano, bastará probar entonces que si $K\subseteq M$ es un submódulo y $S$ una colección de submódulos de $M$ con $K=\displaystyle\bigcap_{M'\in S}M'$, entonces existe $S'\subseteq S$ finito tal que $K=\displaystyle\bigcap_{M'\in S'}M'$. Consideremos entonces la familia de submódulos de $M$ dada por $\mathcal{P}=\left\{ \displaystyle\bigcap_{M'\in F}M' : F\subseteq S \text{ es finito}\right\}$. Pero por hipótesis, sabemos que $\mathcal{P}$ tiene algún elemento minimal. Esto es, existe $F_0\subseteq S$ finito tal que $K' = \displaystyle\bigcap_{M'\in F_0}M'$ es un elemento minimal de $\mathcal{P}$. Se ve fácilmente por construcción que $K'=K$. Y listo.
 
\end{proof}
\end{prop}

\begin{ex}Se puede chequear fácilmente con la condición de cadena descendiente los siguientes ejemplos:
\begin{itemize}\item Sea $G$ un grupo abeliano finito. Entonces $G$ es un $\ZZ$-módulo artiniano.
\item Consideremos $G_{p^{\infty}} = \displaystyle\bigcup_{n\in\mathbb{N}}G_{p^n}$. Entonces $G_{p^{\infty}}$ es un $\ZZ$-módulo artiniano.\end{itemize}
\end{ex}

Gran parte de las proposiciones que valen para módulos noetherianos valen para artinianos, con demostraciones muy similares. Algunos ejemplos:

\begin{prop}
Sea $0\longrightarrow L\stackrel{f}{\longrightarrow} M\stackrel{g}{\longrightarrow} N\longrightarrow 0$ una sucesión exacta de $A$-módulos. Entonces $M$ es artiniano si y sólo si $L$ y $N$ lo son.
\end{prop}

\begin{prop}
Sea $(M_i)_{i=1,\ldots , n}$ una familia finita de $A$-módulos. Entonces $\displaystyle\bigoplus_{i=1}^n M_i$ es artiniano si y sólo si cada sumando $M_i$ es artiniano para todo $1\leq i\leq n$.
\end{prop}

\begin{prop}
Si $A$ es un anillo artiniano a izquierda y $M$ es un $A$-módulo finitamente generado, entonces $M$ es un $A$-módulo artiniano.
\begin{proof}
Notemos que como $M$ es finitamente generado, existe un $n$ y un epimorfismo $\varphi:A^n\to M$. Pero $A^n=\displaystyle\bigoplus_{i=1}^n A$ es suma directa de $n$ copias de $A$, que es artiniano. Por la proposición anterior, $M$ debe ser artiniano.
\end{proof}
\end{prop}

Para terminar con esta sección, veamos un teorema de estructura de módulos artinianos y noetherianos.

\begin{defn}
Decimos que un $A$-módulo $M$ es \textbf{indescomponible} si no admite sumandos directos propios.
\end{defn}

\begin{teo}
Sea $M$ un $A$-módulo noetheriano o artiniano no nulo. Entonces existen submódulos indescomponibles $M_1,\ldots , M_n$ tales que $M\simeq \displaystyle\bigoplus_{i=1}^n M_i$.
\begin{proof}

Decimos que un submódulo $X\subseteq M$ es \textit{malo} si no es suma directa de submódulos indescomponibles. Entonces, queremos ver que si $M$ es artiniano o noetheriano entonces $M$ no es malo.

Notemos que si $X\subseteq M$ es un submódulo malo entonces tiene algún sumando directo propio $Y\subseteq X$ que es malo. En efecto, si $X$ es malo no puede ser indescomponible, así que existen $Y,Y'\subseteq X$ tales que $X=Y\oplus Y'$. Pero entonces alguno de $Y$ o $Y'$ debe ser malo.

Supongamos que $M$ es malo. Entonces, deben existir $Y_0,Z_0\subseteq M$ tales que $M=Y_0\oplus Z_0$ y $Z_0$ es malo. Ahora, supongamos que $n\geq 0$ y tenemos construídos $Y_0,\ldots , Y_n$ y $Z_0,\ldots ,Z_n$ submódulos de $M$ tales que $Z_i = Y_{i+1}\oplus Z_{i+1}$ si $0\leq i<n$ y $Z_i$ es malo si $0\leq i\leq n$. Pero entonces, existen $Y_{n+1},Z_{n+1}$ submódulos de $Z_n$ tales que $Z_n = Y_{n+1}\oplus Z_{n+1}$ y $Z_{n+1}$ es malo.

Entonces, tenemos construídos $Z_0\supset Z_1\supset \ldots Z_n\supset\ldots$ una cadena estrictamente descendente y $Y_0\subset Y_0\oplus Y_1\subset \ldots \displaystyle\bigoplus_{i=0}^{n}Y_i\subset\ldots$ una cadena estrictamente ascendente. Entonces $M$ no puede ser noetheriano ni artiniano. Y estamos.

\end{proof}
\end{teo}

\section{Un Teorema sobre módulos libres sobre DIPs}

\begin{obs}
Sea $A$ DIP y $\mathfrak{a}\subseteq A$ un ideal no trivial. Entonces $\mathfrak{a}$ es un $A$-módulo libre. En efecto, $\mathfrak{a}=\langle a\rangle$ por ser $A$ DIP y $\{a\}$ es linealmente independiente pues $A$ es íntegro.
\end{obs}

\begin{obs}
Sea $A$ dominio íntegro y $M$ un $A$-módulo libre. Entonces $T(M)=0$. Es decir, es libre de torsión. En efecto, si $ax=0$ con $a\neq 0$ y $B=\{x_i\}_{i\in I}$ una base, entonces $x=\displaystyle\sum_{i\in I}a_i x_i$ y así $ax=0$ implica que $\displaystyle\sum_{i\in I} (aa_i)x_i=0$, y por ser base $aa_i=0$ para todo $i$. Pero como es íntegro y $a\neq 0$ todos los $a_i$ deben ser $0$. Entonces $x=0$.
\end{obs}

\begin{teo}[Principio de Buena Ordenación]
Sea $X$ un conjunto no vacío. Un \textbf{buen orden} en $X$ es un orden total $\leq$ tal que para todo $A\subseteq X$ no vacío existe $a\in A$ elemento mínimo. En particular, al elemento mínimo de $X$ lo denotamos $1$.

El principio de buena ordenación dice que todo conjunto $X$ no vacío admite un buen orden.
\end{teo}

El Principio de Buena Ordenación es equivalente al Lema de Zorn, que a su vez es equivalente al Axioma de Elección.

\begin{teo}[Inducción Transfinita]
Sea $X$ un conjunto bien ordenado y $S\subseteq X$ un subconjunto que cumple que \begin{itemize} \item $1\in S$ \item Dado $i\in X$, si $j\in S$ para todo $j<i$ entonces $i\in S$\end{itemize} Entonces $S=X$.
\begin{proof}
Si $S\neq X$ entonces $X-S\neq \emptyset$ y entonces tiene un elemento mínimo $b$. Entonces, para todo $a< b$, tenemos que $a\in S$. Pero entonces por hipótesis $b\in S$. Absurdo!
\end{proof}
\end{teo}

Ahora estamos en condiciones de probar el siguiente teorema:

\begin{teo}
Sea $A$ DIP y $M$ un $A$-módulo libre. Entonces todo submódulo $N\subseteq M$ es un $A$-módulo libre.
\begin{proof}
Sea $B=\{x_i\}_{i\in I}$ una base de $M$. Como $I$ es no vacío, existen un buen orden de $I$. Sea $N\subseteq M$ un submódulo. Para cada $i\in I$ defino $\overline{N}_i = N\cap \langle x_j\rangle_{j\leq i}$, y defino $N_1=0$ y $N_i=N\cap \langle x_j\rangle_{j<i}$ si $i>1$. Notemos que $N_i = \displaystyle\bigcup_{j<i}\overline{N}_j$.

Como $x_1$ es linealmente por estar en la base, $\langle x_1\rangle$ es libre y así $\langle x_1\rangle\simeq A$. Como tenemos $\overline{N}_1=N\cap\langle x_1\rangle \subseteq \langle x_1\rangle$ entonces $N\cap \langle x_1\rangle$ debe ser isomorfo a un ideal de $A$. Pero como $A$ es un DIP, debemos tener que $\overline{N}_1 = \langle \omega_1\rangle$ para un $\omega_1\in\overline{N}_1$.

Dado $i\in I$ supongo construídos $\omega_j$ para todo $j<i$ y elijo $\omega_i$ de la siguiente forma:

Sea $\mathcal{A}_i = \{a\in A : \exists x\in N, \, x=\displaystyle\sum_{j<i}b_jx_j + ax_i\}$. Es fácil ver que $\mathcal{A}_i$ es un ideal de $A$. Pero como $A$ es DIP entonces $\mathcal{A}_i = \langle a_i\rangle$ para algún $a_i\in A$. Luego, si $x\in \overline{N}_i$, entonces $x=\displaystyle\sum_{j<i} b_jx_j + ca_ix_i$. Si $a_i=0$ entonces tomo $\omega_i=0$. Si no, tomo $\omega_i = \displaystyle\sum_{j<i}b_jx_j + a_ix_i$. Notemos entonces que $\overline{N}_i = N_i + \langle \omega_i\rangle$ ya que si $x\in \overline{N}_i$ entonces $x-c\omega_i\in N_i$ (La otra inclusión es obvia).

Pero además, $N_i\cap \langle \omega_i\rangle = 0$. Si $\omega_i =0$ no hay nada que decir. Si $\langle \omega_i\rangle\neq 0$ entonces $a_i\neq 0$, $y\in N_i\cap\langle \omega_i\rangle$ implica que $y=\displaystyle\sum_{j<i} c_jx_j = d\omega_i = d\displaystyle\sum_{j<i} b_jx_j + da_ix_i$. Pero al ser $\{x_i\}_{i\in I}$ base, debemos tener que $da_i=0$ y así $d=0$ por una de las observaciones que hicimos antes de probar el teorema. Entonces la intersección es el $0$ como afirmamos.

Definamos entonces $\overline{B}=\{\omega_i : \omega_i\neq 0\}_{i\in I}$. Veamos que esto es base de $N$.

En efecto, primero veamos que es linealmente independiente. Tomemos una combinación lineal $c_{i_0}\omega_{i_0}+\ldots + c_{i_r}\omega_{i_r} = 0$ con $i_0<\ldots < i_r$ y $c_{i_k}\neq 0$. Entonces $-c_{i_r}\omega_{i_r} = c_{i_0}\omega_{i_0}+\ldots + c_{i_{r-1}}\omega_{i_{r-1}}$ y así $-c_{i_r}\omega_{i_r}\in N_{i_r}\cap \langle\omega_{i_r}\rangle$ y así $-c_{i_r}\omega_{i_r}=0$. Pero $M$ es libre de torsión por ser libre. Entonces $c_{i_r}=0$. Absurdo.

Para concluir, veamos que generan. Claramente $\langle\overline{B}\rangle\subseteq N$ pues $\omega_i \in \overline{N}_i\subseteq N$. Supongamos que $\langle\overline{B}\rangle \subset N$. Entonces existe un $x\in N$ tal que $x\neq \displaystyle\sum_{\text{finita}} c_j\omega_j$. Es claro que $N=\displaystyle\bigcup_{i\in I}\overline{N}_i$. Sea $j = \min\{i\in I : \exists x\in \overline{N}_i \; x\neq \displaystyle\sum c_k\omega_k\}$. Sea $x\in\overline{N}_j$ que no se puede escribir. Entonces como $x\in\overline{N}_j = N_j + \langle \omega_i\rangle$, tenemos que $x=x'+d\omega_i$ con $x'\in N_j=\displaystyle\bigcup_{k<j}\overline{N}_k$, entonces $x'\in\overline{N}_k$ con $k<j$. Pero como a $x$ no lo puedo escribir, entonces a $x'$ tampoco (pues si puediera, podría escribir a $x$). Pero entonces $x'\in\overline{N}_k$ con $k<j$, que contradice la minimalidad de $j$. Absurdo!

Entonces construímos una base y así todo submódulo de un módulo libre sobre un DIP es libre.

\end{proof}
\end{teo}

\section{Módulos Proyectivos e Inyectivos}

\begin{defn}
Sea $A$ un dominio íntegro. $M$ se dice de \textbf{torsión} si $T(M)=M$. Por ejemplo $\ZZ_n$ es un $\ZZ$-módulo de torsión.

Un módulo $M$ se dice \textbf{sin torsión} si $T(M)=0$. Por ejemplo, $\ZZ,\QQ,\ZZ^n,\ZZ^\NN,\ZZ^{(\NN)}$ son $\ZZ$-módulos sin torsión.

Un módulo $M$ se dice \textbf{con torsión} si $T(M)\neq 0,M$. Por ejemplo $\ZZ\oplus\ZZ_n$ es un $\ZZ$-módulo con torsión.
\end{defn}

\begin{defn}
Sea $A$ un dominio íntegro y $M$ un $A$-módulo. $M$ se dice \textbf{divisible} si $\forall\, m\in M\; \forall\, a\in A-\{0\}$ existe $m'\in M$ tal que $am'=m$.
\end{defn}

\begin{obs}
Notemos que el $m'$ no tiene por qué ser único. Este $m'$ es único si y sólo si $M$ es divisible y sin torsión. En efecto, si $m',m''$ cumplen que $am' = m = am''$ entonces $a(m'-m'') = 0$ y así, como es libre de torsión, $m'-m''=0$. La vuelta es similar.
\end{obs}

\begin{teo}
Sea $A$ un dominio íntegro, $k$ su cuerpo de fracciones y $M$ un $A$-módulo divisible y sin torsión. Entonces, existe una única estructura de $k$-espacio vectorial en $M$ que extiende a la multiplicación de $A$.
\begin{proof}
Si hubiera una estructura, tendría que cumplirse que $m' = \dfrac{a}{a}m' = \dfrac{1}{a} am' = \dfrac{1}{a}m$. Pero es trivial que definir $\dfrac{1}{a}m = m'$ da una estructura de $k$-módulo, pues el $m'$ es único al ser $M$ divisible y sin torsión. Entonces el teorema sigue.
\end{proof}
\end{teo}

\begin{obs}
Submódulos de módulos divisibles no son necesariamente divisibles. Por ejemplo $\QQ$ es $\ZZ$-módulo divisible pero $\ZZ$ no lo es (pues no puedo "`invertir"' todos los elementos).

Lo que sí es cierto es que si $M$ es un $A$-módulo divisible y $N\subseteq M$ un submódulo, entonces $M/N$ es divisible. En efecto, si $am'=m$, tomando clase $a\overline{m'} = \overline{m}$ y listo.
\end{obs}

Recordemos que el funtor $\hom_A(M,-)$ es covariante y exacto a izquierda, pero no a derecha (De forma análoga, el funtor contravariante $\hom_A(-,M)$ es exacto a derecha pero no a izquierda). Es decir, si tengo un epimorfismo, $g:N'\to N''$ entonces $g_*:\hom_A(M,N')\to\hom_A(M,N'')$ no necesariamente es epimorfismo. 

\begin{defn}
Sea $M$ un $A$-módulo. Decimos que es un \textbf{módulo proyectivo} si el funtor $\hom_A(M,-)$ es exacto. Es decir, para cualquier $g:N\to T$ epimorfismo entonces $g_*:\hom_A(M,N)\to \hom_A(M,T)$ es epimorfismo de grupos abelianos.

Equivalentemente, para todo $g:N\to T$ epimorfismo y $h:M\to T$ morfismo de $A$-módulos existe un $\overline{h}:M\to N$ morfismo de $A$-módulos tal que $g\circ \overline{h} = h$ (se dice que $\overline{h}$ \textbf{levanta} a $h$ por $g$). Es decir, el siguiente diagrama conmuta:

\begin{tikzcd}[row sep=3.3em,column sep=4em,minimum width=2em]
 & M\arrow{d}[right, font=\normalsize]{h}\arrow[dashed]{dl}[left, font=\normalsize]{\overline{h}} \\
N\arrow{r}[font=\normalsize]{g} & T\arrow{r} & 0 \\
\end{tikzcd}
\end{defn}

\begin{obs}
Notemos que $\overline{h}$ no necesariamente es única, pues $g_*$ es epimorfismo y no necesariamente isomorfismo.
\end{obs}

\begin{prop}
Si $M$ es un $A$-módulo libre entonces es proyectivo.
\begin{proof}
Sea $B=\{x_i\}_{i\in I}$ una base de $M$. Defino $\overline{h}:B\to N$ como $\overline{h}(x_i)=y_i$ donde $y_i\in N$ cumple que $g(y_i)=h(x_i)$ (existen pues $g$ es epimorfismo). Por la propiedad universal de la base, me definí un $\overline{h}$ que levanta a $h$ (pues lo hace en la base) y ya estamos.
\end{proof}
\end{prop}

\begin{ex}
Notemos que $\ZZ_n$ no es un $\ZZ$-módulo proyectivo pues $\ZZ\stackrel{\pi}{\longrightarrow} \ZZ_n\longrightarrow 0$ es epimorfismo pero el único morfismo $\ZZ_n\to\ZZ$ de $\ZZ$-módulos es el trivial y no levanta a $\pi$. Este ejemplo muestra que cocientes de módulos proyectivos no son necesariamente proyectivos.

Además, submódulos de proyectivos tampoco son necesariamente proyectivos. En efecto, tomando $A=M=\ZZ_4$ y $N=\{0,2\}\simeq \ZZ_2\subseteq \ZZ_4$ es un $\ZZ_4$-submódulo que no es proyectivo. Nuevamente es fácil de chequear pues los únicos morfismos $N\to\ZZ_4$ son el trivial y la inclusión, y ninguno de estos levanta al morfismo que es multiplicar por $2$.
\end{ex}

\begin{prop}
Sea $M$ un $A$-módulo. Son equivalentes:\begin{enumerate}\item $M$ es proyectivo. \item Toda sucesión exacta corta $0\longrightarrow N\stackrel{f}{\longrightarrow} T\stackrel{g}{\longrightarrow} M\longrightarrow 0$ se parte. \item $M$ es sumando directo de un $A$-módulo libre. O sea, existe un $A$-módulo $N$ tal que $M\oplus N$ es libre.\end{enumerate}
\begin{proof}
(1)$\Longrightarrow$(2): Sea $0\longrightarrow N\stackrel{f}{\longrightarrow} T\stackrel{g}{\longrightarrow} M\longrightarrow 0$ una sucesión exacta corta. Basta ver que $g$ es retracción. Pero existe $h:M\to T$ morfismo $A$-lineal tal que $g\circ h=\id$ por la condición de proyectivo (es decir, hay un morfismo $h$ que levanta a $g$ por la identidad). Entonces $g$ es retracción como queríamos.

(2)$\Longrightarrow$(3): Sabemos que todo módulo es cociente de un libre. Entonces, tenemos la sucesión exacta corta $0\longrightarrow \ker g\longrightarrow A^{(I)}\stackrel{g}{\longrightarrow} M \longrightarrow 0$. Por hipótesis, esta sucesión exacta corta se parte y así $A^{(I)}\simeq M\oplus \ker g$. Como queríamos.

(3)$\Longrightarrow$(1): Sabemos que $M\oplus N=L$ libre para algún $N$. Sea $g:S\to T$ epimorfismo de $A$-módulos y $h:M\to T$. Debo hallar $\overline{h}:M\to S$ tal que $g\circ \overline{h}=h$. Sea $\iota:M\to M\oplus N=L$ la inclusión en la primer coordenada $\iota(m)=(m,0)$ y $r:M\oplus N\to M$ la proyección a la primer coordenada $r(m,n)=m$. Entonces $r\circ \iota = 1_M$. Como $L$ es libre, es proyectivo. 

\begin{tikzcd}[row sep=3.3em,column sep=4em,minimum width=2em]
& M\oplus N \arrow[dashed]{ddl}[left, font=\normalsize]{f}\arrow{d}[left,font=\normalsize]{r}\\
& M\arrow{d}[right, font=\normalsize]{h}\arrow[bend right, dashed]{u}[right, font=\normalsize]{\iota} \\
N\arrow{r}[font=\normalsize]{g} & T\arrow{r} & 0 \\
\end{tikzcd}

Entonces, existe $f:L\to S$ tal que $g\circ f = h\circ r$. Por lo tanto, $g\circ (f\circ \iota) = h\circ (r\circ\iota) = h$. Entonces $\overline{h}=f\circ\iota$ es el morfismo que busco. ¡Y listo!



\end{proof}
\end{prop}

\begin{obs}
Si $A$ es DIP y $M$ es un $A$-módulo proyectivo entonces $M$ es $A$-libre. En efecto, como $M\oplus N=L$ libre entonces $M\subseteq L$ es submódulo de un libre y así es libre pues los submódulos de libres son libres sobre DIPs.
\end{obs}

\begin{ex}
Que un módulo sea proyectivo \textbf{NO} implica que sea libre. En efecto, sea $k$ un cuerpo y $A=\left\{\begin{pmatrix}a & 0 \\ b & c\end{pmatrix} : a,b,c\in k\right\}$. Es fácil ver que $(A,+,\cdot)$ es un anillo. Consideremos $e_{11} = \begin{pmatrix}1 & 0 \\ 0 & 0\end{pmatrix}$ y $e_{22} = \begin{pmatrix}0 & 0 \\ 0 & 1\end{pmatrix}$. Notemos que como $\begin{pmatrix}a & 0 \\ b & c\end{pmatrix} \begin{pmatrix}1 & 0 \\ 0 & 0\end{pmatrix} = \begin{pmatrix}a & 0 \\ b & 0\end{pmatrix}$ y $\begin{pmatrix}a & 0 \\ b & c\end{pmatrix}\begin{pmatrix} 0 & 0 \\ 0 & 1\end{pmatrix}\begin{pmatrix}a & 0 \\ b & c\end{pmatrix} = \begin{pmatrix}0 & 0 \\ 0 & c\end{pmatrix}$. Esto nos dice que $A=\langle e_{11}\rangle\oplus\langle e_{22}\rangle$ como $A$-módulos, y así $\langle e_{11}\rangle$ es proyectivo por ser sumando directo de un libre.

Pero notemos que $\langle e_{11}\rangle$ no es libre. Para esto tenemos dos formas de hacerlo, podemos ver que todo elemento es de torsión, como ya lo hicimos antes. Otra forma es notar que si fuese libre, sería isomorfo a un $A^n$ para algún $n\in\NN$ y así $\dim_k\langle e_{11}\rangle = \dim_k A^n$ como $k$-espacios vectoriales. Pero $\dim_k\langle e_{11}\rangle =2$ y $\dim_k A^n = 3n$. Absurdo.
\end{ex}

\begin{defn}
Sea $A$ un anillo y $P$ un $A$-módulo a izquierda. Una \textbf{base dual} para $P$ es un par $((x_i)_{i\in I},(f_i)_{i\in I})$ tal que $x_i\in P$ y $f_i\in \hom_A(P,A)$ para todo $i\in I$ y además \begin{itemize}\item Para todo $x\in P$, $\sharp\{i\in I : f_i(x)\neq 0\}<\infty$. \item Para todo $x\in P$, $x=\displaystyle\sum_{i\in I}f_i(x)x_i$ (Notar que esta condición tiene sentido por la condición anterior). \end{itemize}
\end{defn}

\begin{prop}
Sea $A$ un anillo y $P$ un $A$-módulo a izquierda. Entonces $P$ es proyectivo si y sólo si posee una base dual. Más aún, $P$ es proyectivo y finitamente generado si y sólo si posee una base dual finita.
\begin{proof}
($\Longrightarrow$) Supongamos que $P$ es $A$-proyectivo. Entonces, como todo módulo es cociente de un libre, tenemos un epimorfismo $f:\displaystyle\bigoplus_{i\in I}A \twoheadrightarrow P$, dado por $(r_i)_{i\in I}\mapsto\displaystyle\sum_{i\in I}r_if(e_i)$, y al ser $P$ proyectivo, es retracción. Entonces, existe $g:P\to \displaystyle\bigoplus_{i\in I}A$ tal que $f\circ g = \id$. Si tomamos $f_i$ como la $i$-ésima coordenada de $g$, tenemos una base dual $((f(e_i))_{i\in I}, (f_i)_{i\in I})$. Si $P$ es finitamente generado, $I$ resulta finito y así la base dual finita.

($\Longleftarrow$) Supongamos que tenemos una base dual $((x_i)_{i\in I},(f_i)_{i\in I})$. Entonces, tenemos un $r:\displaystyle\bigoplus_{i\in I}A\twoheadrightarrow P$ dada por $(r_i)_{i\in I}\mapsto \displaystyle\sum_{i\in I}r_ix_i$. Es epimorfismo pues $(f_i(x))_{i\in I}$ va a parar a $x$ por la condición de base dual. Pero notemos que $\overline{f}=(f_i)_{i\in I}$ cumple que $r\circ \overline{f}=\id$.

Ahora, veamos que todo epimorfismo $f:M\twoheadrightarrow P$ es retracción. Como $\displaystyle\bigoplus_{i\in I}A$ es libre, es proyectivo, y así existe $h:\displaystyle\bigoplus_{i\in I}A\to M$ tal que $f\circ h = r$. Entonces $f\circ (h\circ \overline{f}) = r\circ \overline{f} = \id$, y entonces $P$ es proyectivo. Nuevamente, si la base dual es finita, $I$ es finito y así $P$ es finitamente generado. Y estamos.
\end{proof}
\end{prop}

\begin{prop}
Sea $\{M_i\}_{i\in I}$ una familia de $A$-módulos. Entonces $\displaystyle\bigoplus_{i\in I}M_i$ es proyectivo si y sólo si cada $M_i$ es proyectivo $\forall\, i\in I$.
\begin{proof}
El truco consiste en notar que $\hom_A\left(\displaystyle\bigoplus_{i\in I}M_i,N\right)\simeq \displaystyle\prod_{i\in I}\hom_A(M_i,N)$. En efecto, sabemos que $\displaystyle\bigoplus_{i\in I} M_i$ es proyectivo si y sólo si el funtor $\hom_A\left(\displaystyle\bigoplus_{i\in I}M_i,-\right)$ es exacto. Si la suma directa es proyectiva, sea $0\longrightarrow N'\longrightarrow N\longrightarrow N''\longrightarrow 0$ una sucesión exacta corta. Tenemos que $$0\longrightarrow\hom_A\left(\displaystyle\bigoplus_{i\in I}M_i,N'\right)\longrightarrow \hom_A\left(\bigoplus_{i\in I}M_i,N\right)\longrightarrow \hom_A\left(\bigoplus_{i\in I}M_i, N''\right)\longrightarrow 0$$ es una sucesión exacta corta, y por el isomorfismo que remarcamos al principio, esto induce una sucesión exacta corta $$0\longrightarrow\displaystyle\prod_{i\in I}\hom_A(M_i,N')\longrightarrow\prod_{i\in I}\hom_A(M_i,N)\longrightarrow\prod_{i\in I}\hom_A(M_i,N'')\longrightarrow 0$$ Pero los morfismos estos están definidos lugar a lugar, entonces para cada $i\in I$ inducen una sucesión exacta corta $$0\longrightarrow \hom_A(M_i,N')\longrightarrow \hom_A(M_i,N)\longrightarrow\hom_A(M_i,N'')\longrightarrow 0$$ Esto quiere decir que $\hom_A(M_i,-)$ es exacto para cada $i$, y así los $M_i$ son proyectivos. La vuelta es simplemente a partir de la sucesión exacta corta de la exactitud de los $\hom_A(M_i,-)$ construirse la sucesión exacta corta del producto y con el isomorfismo construimos la sucesión exacta corta que muestra que $\hom_A\left(\displaystyle\bigoplus_{i\in I}M_i,-\right)$ es exacto. Y listo.
\end{proof}
\end{prop}

\begin{prop}
Sea $A$ anillo conmutativo y $S\subseteq A$ multiplicativamente cerrado. Si $M$ es un $A$-módulo proyectivo, entonces $S^{-1}M$ es un $S^{-1}A$-módulo proyectivo.
\begin{proof}
Notemos que $S^{-1}(M\oplus N) \simeq S^{-1}M\oplus S^{-1}N$ como $S^{-1}A$-módulos. En efecto, tenemos una sucesión exacta corta $0\longrightarrow M\stackrel{\iota_1}{\longrightarrow}M\oplus N\stackrel{\pi_2}{\longrightarrow} N\longrightarrow 0$ y sabemos que localizar es exacto, entonces tenemos $0\longrightarrow S^{-1}M\stackrel{S^{-1}\iota_1}{\longrightarrow} S^{-1}(M\oplus N)\stackrel{S^{-1}\pi_2}{\longrightarrow} S^{-1}N\longrightarrow 0$. Queremos ver que esa sucesión exacta corta se parte. Basta con ver que $S^{-1}\pi_2$ es retracción. Pero esto es obvio pues al ser $\pi_2$ retracción tenemos que existe $h:N\to M\oplus N$ tal que $\pi_2\circ h = 1_N$. Entonces, $S^{-1}\pi_2 \circ S^{-1}h \left(\dfrac{m}{s}\right) = \dfrac{\pi_2\circ h(m)}{s} = \dfrac{m}{s}$ y así $S^{-1}\pi_2$ es retracción.

Pero además, notemos que si $M$ es un $A$-módulo libre, $S^{-1}M$ es un $S^{-1}A$-módulo libre. En efecto, es fácil ver que si $\{x_i\}_{i\in I}$ es una base de $M$ como $A$-módulo, entonces $\left\{\dfrac{x_i}{1}\right\}_{i\in I}$ genera y es linealmente independiente, y así una base de $S^{-1}M$ como $S^{-1}A$-módulo.

Sabiendo esto, la proposición es trivial: Como $M$ es $A$-proyectivo, tenemos que existe un $A$-módulo $N$ tal que $M\oplus N$ es libre. Pero entonces $S^{-1}(M\oplus N)\simeq S^{-1}M\oplus S^{-1}N$ es $S^{-1}A$-libre y así $S^{-1}M$ es sumando directo de un libre. Entonces es proyectivo.
\end{proof}
\end{prop}

\begin{defn}
Sea $A$ un anillo y $M$ un $A$-módulo. Decimos que $M$ es un \textbf{módulo inyectivo} si el funtor contravariante $\hom_A(-,M)$ es exacto. Es decir, si $f:N\to T$ monomorfismo implica que $f^*:\hom_A(T,M)\to\hom_A(N,M)$ es epimorfismo. Concretamente, para todo $f:N\to T$ monomorfismo y para todo $h:N\to M$ existe $\overline{h}:T\to M$ tal que $\overline{h}\circ f=h$ (se dice que $\overline{h}$ \textbf{extiende} $f$ a $h$). Es decir, el siguiente diagrama conmuta:

\begin{tikzcd}[row sep=3.3em,column sep=4em,minimum width=2em]
0 \arrow{r} & N\arrow{d}[left, font=\normalsize]{h}\arrow{r}[font=\normalsize]{f} & T\arrow[dashed]{ld}[right, font=\normalsize]{\overline{h}} \\
 & M
\end{tikzcd}
\end{defn}

\begin{ex}
$\ZZ_n$ no es un $\ZZ$-módulo inyectivo. En efecto, si miramos la proyección $\pi:\ZZ\to\ZZ_n$ y $\varphi:\ZZ\to\ZZ$ la multiplicación por $n$, entonces no hay ningún $h:\ZZ\to\ZZ_n$ que extienda a $\varphi$ por la proyección (la composición daría siempre trivial).

$\ZZ$ no es un $\ZZ$-módulo inyectivo. Esto nos muestra que módulo libre no implica módulo inyectivo.

$A=k$ cuerpo y $V$ un $k$-espacio vectorial. Entonces $V$ es inyectivo. En efecto, si tenemos $f:T\to N$ monomorfismo de $k$-espacios vectoriales y $\varphi:T\to V$, podemos extender $f$ a $\varphi$ definiendo $h$ en $f(T)$ y extendiendo ese conjunto linealmente independiente a una base y usar la propiedad universal de la base.
\end{ex}

\begin{prop}
Sea $A$ dominio íntegro y $k$ su cuerpo de fracciones. Entonces $k$ es un $A$-módulo inyectivo.
\begin{proof}
Supongamos que tenemos $f:M\to N$ monomorfismo y $g:M\to k$ morfismo. Quiero ver que existe un $\overline{g}:N\to k$ tal que $\overline{g}\circ f = g$.

Sea $S=A-\{0\}$. Sabemos que $k=S^{-1}A$. Considero $S^{-1}M$ t $S^{-1}N$ como $S^{-1}A$-módulos, o sea, $k$-espacios vectoriales. Como $f$ es monomorfismo, $S^{-1}f$ es monomorfismo de $k$-espacios vectoriales. Por la propiedad universal de la localización, tenemos un $h:S^{-1}M\to k$ tal que $h\circ\iota_M = g$, con $\iota_M:M\to S^{-1}M$ el morfismo canónico. Además, tenemos un $\tilde{h}:S^{-1}N\to k$ tal que $\tilde{h}\circ S^{-1}f = h$ pues los $k$-espacios vectoriales son inyectivos y $S^{-1}f$ es monomorfismo. Definamos entonces $\tilde{\tilde{h}}:N\to k$ por $\tilde{\tilde{h}} = \tilde{h}\circ \iota_N$. Veamos que $\overline{g} = \tilde{\tilde{h}}$ es el morfismo que buscamos.

En efecto, $\tilde{\tilde{h}}\circ f = \tilde{h}\circ\iota_N\circ f = \tilde{h}\circ S^{-1}f\circ \iota_M = h\circ \iota_M = g$ (Usamos la conmutatividad $S^{-1}f\circ \iota_M = f\circ \iota_N$). Y estamos.

\end{proof}
\end{prop}

\begin{cor}
Como $\QQ$ es el cuerpo de fracciones de $\ZZ$ que es íntegro, $\QQ$ es $\ZZ$-inyectivo.
\end{cor}

\begin{obs}
Esto nos dice que submódulos de inyectivos no necesariamente lo son, pues $\ZZ\subseteq \QQ$ es un contraejemplo.
\end{obs}

\begin{teo}[Baer]
Sea $A$ un anillo. Un $A$-módulo $M$ es inyectivo si y sólo si para todo $I\subseteq A$ ideal y $f:I\to M$ morfismo $A$-lineal, existe $\overline{f}:A\to M$ que extienda a $f$. Es decir, el siguiente diagrama conmuta: \begin{tikzcd}[row sep=3.3em,column sep=4em,minimum width=2em]
0 \arrow{r} & I\arrow{d}[left, font=\normalsize]{f}\arrow[hookrightarrow]{r}[font=\normalsize]{\mathrm{inc}} & A\arrow[dashed]{ld}[right, font=\normalsize]{\overline{f}} \\
 & M
\end{tikzcd}
\begin{proof}
($\Longrightarrow$) No hay nada que probar.

($\Longleftarrow$) Veamos que $M$ es inyectivo. Supongamos que $g:N\to T$ es un monomorfismo y $f:N\to M$ un morfismo. Debo ver que existe $\overline{f}:T\to M$ tal que $\overline{f}\circ g=f$. Sin pérdida de la generalidad, puedo suponer que $g$ es una inclusión de $N$ en $T$. Es decir, $N\subseteq T$ y $g=\mathrm{inc}$ pues si no considero $g(N)$ con la inclusión en $T$ y $fg^{-1}:g(N)\to M$ y después al ser $g$ un isomorfismo de $N$ con la imagen ya estamos.

Tenemos entonces la inclusión de $N$ en $T$ y un morfismo $f:N\to M$. Consideremos el conjunto $$\mathcal{C}=\{(N',f') : N\subseteq N'\subseteq T \text{ submódulo y } f':N'\to M \text{ con }\left. f'\right|_{N}=f\}$$ Notemos que $\mathcal{C}$ es no vacío pues $(N,f)\in\mathcal{C}$. Pero además, podemos definir un orden parcial $\leq$ en $\mathcal{C}$ por $(N',f')\leq(N'',f'')$ si $N'\subseteq N''$ submódulo y $\left. f''\right|_{N'}=f'$. Pero entonces $(\mathcal{C},\leq)$ cumple las hipótesis del Lema de Zorn (tomando la unión en una cadena, como siempre, obtenemos una cota superior). Entonces, tenemos algún $(N_0,f_0)\in\mathcal{C}$ maximal.

Supongamos que $N_0\neq T$. Luego, existe $y\in T$ tal que $y\notin N_0$. Consideremos $N_0+\langle y\rangle$. Notemos que $N_0\subset N_0+\langle y\rangle \subset T$. Si pruebo que puedo extender la $f_0$ a $\overline{f}:N_0+\langle y\rangle \to M$ contradirá la maximalidad de $N_0$. Sea $I=\{a\in A : ay\in N_0\}$. Es fácil ver que $I$ es un ideal. Defino $g:I\to M$ por $g(a)=f_0(ay)$. Claramente es $A$-lineal. Por hipótesis, existe $\tilde{g}:A\to M$ que extiende a $g$.

Definimos entonces $\tilde{f}:N_0 + \langle y\rangle\to M$ por $\tilde{f}(x+ay) = f_0(x)+\tilde{g}(a)$. Es claro que $\tilde{f}$ extiende a $f_0$. Veamos que $\tilde{f}$ está bien definida. En efecto, si $x+ay=x'+a'y$ tenemos que $x-x' = (a'-a)y\in I$. Entonces, tenemos que $\tilde{g}(a')-\tilde{g}(a) = \tilde{g}(a'-a) = g(a'-a) = f_0((a'-a)y) = f_0(x-x') = f_0(x)-f_0(x')$. Y listo.

\end{proof}
\end{teo}

\begin{prop}
Sea $A$ un dominio integro. Si $M$ es $A$-módulo inyectivo entonces $M$ es $A$-módulo divisible.
\begin{proof}

Consideremos $\mu_a:A\to A$, $\mu_a(b)=ab$. Es decir, $\mu_a$ es la multiplicación a izquierda por $a$. Notemos que $\mu_a$ es monomorfismo pues $A$ es dominio íntegro. Definamos $h_m:A\to M$ por $h_m(b)=bm$ la multiplicación a derecha por un elemento de $M$ fijo. Como $M$ es un $A$-módulo inyectivo, existe un $\overline{h}_m:A\to M$ tal que $\overline{h}_m\circ\mu_a = h$. Entonces, para cualquier $b\in A$ tenemos que $\overline{h}_m(ab)=bm$, y así $a\overline{h}_m(b)=bm$. Tomando $b=1$ obtenemos $a\overline{h}_m(1)=m$, y así $M$ es divisible.

\end{proof}
\end{prop}

\begin{prop}
Sea $A$ un DIP y $M$ un $A$-módulo. Entonces $M$ es divisible si y sólo si es inyectivo.
\begin{proof}
($\Longleftarrow$) Vale siempre por la proposición anterior.

($\Longrightarrow$) Vamos a usar el Teorema de Baer. Debo ver que todo morfismo $f:I\to M$ se puede extender a un morfismo $f:A\to M$ para $I$ ideal de $A$. Si $I=0$ entonces $f=0$ y se extiende por $\overline{f}=0$. Si $I\neq 0$, como $A$ es DIP, $I=\langle a\rangle$ para un $a\neq 0$. Entonces, $f(a)=m\in M$. Como $M$ es divisible, existe un $m'\in M$ tal que $am'=m$. Defino $\tilde{f}(1)=m'$ y así $\tilde{f}(b)=bm'$. Entonces, $\tilde{f}(a)=am'=m=f(a)$. Entonces $\tilde{f}$ exitende a $f$ y listo.
\end{proof}
\end{prop}

\section{Producto Tensorial}

\begin{defn}
Sea $A$ un anillo. $M$ un $A$-módulo a derecha (que denotaremos $M_A$) y $N$ un $A$-módulo a izquierda (que denotaremos $_AN$). Sea $T$ un grupo abeliano. Consideremos $f: M\times N\to T$ función de conjuntos. Decimos que $f$ es: \begin{itemize}\item $\ZZ$-bilineal si $f(m+m',n) = f(m,n)+f(m',n)$ y $f(m,n+n')=f(m,n)+f(m,n')$ para todos $m,m'\in M$, $n,n'\in N$. \item $A$-balanceada si $f(ma,n)=f(m,an)$ para todos $m\in M, n\in N, a\in A$. \item $A$-biaditiva si es $\ZZ$-bilineal y $A$-balanceada. \item Si $A$ es conmutativo y $T$ es un $A$-módulo, es $A$-bilineal si es $A$-biaditiva y $f(ma,n)=f(m,an)=af(m,n)$ para todos $m\in M,n\in N, a\in A$.\end{itemize}
\end{defn}

\begin{ex}\begin{itemize}
\item Si $A$ es anillo y consideramos el producto $\mu:A\times A\to A$, $\mu(a,b)=ab$, entonces $\mu$ es $A$-biaditiva. Si $A$ es conmutativo entonces $\mu$ es $A$-bilineal.

\item Consideremos $_AM$ y la acción $\mu:A\times M\to M$, $\mu(a,m)=am$, entonces $\mu$ es $A$-biaditiva. Si $A$ es conmutativo entonces $\mu$ es $A$-bilineal.

\item Si $M,N$ son $\ZZ$-módulos entonces $\ev:M\times \hom_\ZZ(M,N)\to N$, $(m,f)\mapsto f(m)$ es $\ZZ$-bilineal.\end{itemize}
\end{ex}

\begin{obs}
Sea $S$ un conjunto y $\ZZ^{(S)}=\displaystyle\bigoplus_{s\in S}\ZZ = \left\{ \displaystyle\sum_{\mathrm{finita}} \pm s : s\in S\right\}$. La motivación para lo siguiete va a ser que si tengo $M_A$ y $_AN$ quiero un objeto que factorice de forma única a todas las bilineales. Es decir, quiero linealizar a las bilineales.
\end{obs}

\begin{defn}
Sea $A$ anillo y $M_A$, $_AN$ módulos. Se define el \textbf{producto tensorial} $M\otimes_A N$ como el grupo abeliano $F/P$ donde $F=\ZZ^{(M\times N)} = \left\{ \displaystyle\sum_{\mathrm{finita}} \pm (m,n) : m\in M, n\in N \right\}$ y $P\subseteq F$ el subgrupo generado por los elementos (las relaciones): $$\begin{cases}(m+m',n) - (m,n)-(m',n) \\ (m,n+n') - (m,n) - (m,n') \\ (ma,n) - (m,an) \end{cases}$$ $\forall\, m,m'\in M\;\forall\; n,n'\in N\;\forall\, a\in A$. Denotamos por $m\otimes_A n$ a la clase de $(m,n)$ en $F/P$. A los elementos de esta forma se los denomina \textit{tensores elementales}. Entonces $$M\otimes_A N = \left\{ \displaystyle\sum_{\mathrm{finita}} \pm m\otimes_A n : m\in M, n\in N\right\}$$ Sea $\varphi:M\times N\to M\otimes_A N$ definida por $\varphi(m,n) = m\otimes_A n$. Entonces $\varphi$ es $A$-biaditiva. En efecto, esto se cumple pues cocientamos para que se cumplan estas relaciones.
\end{defn}

\begin{prop}[Propiedad Universal de $\otimes_A$]
Sea $A$ anillo, $M_A$ y $_AN$ $A$-módulos. Sea $M\otimes_A N$ producto tensorial y $\varphi:M\times N\to M\otimes_A N$, $\varphi(m,n)=m\otimes_A n$. Entonces \begin{enumerate}\item $\varphi$ es $A$-biaditiva \item Para todo grupo abeliano $T$ y $\psi:M\times N\to T$ que es $A$-biaditiva existe un único $\overline{\psi}:M\otimes_A N\to T$ morfismo de grupos abelianos tal que $\overline{\psi}\circ \varphi = \psi$. Es decir, el siguiente diagrama conmuta: \begin{tikzcd}[row sep=3.3em,column sep=4em,minimum width=2em]
M\times N \arrow{d}[left, font=\normalsize]{\varphi}\arrow{r}[font=\normalsize]{\psi} & T \\
M\otimes_A N \arrow[dashed]{ru}[right, font=\normalsize]{\overline{\psi}}
\end{tikzcd}\end{enumerate}

Bajo estas condiciones se dice que $\varphi$ es \textbf{universalmente} $A$-\textbf{biaditiva}.
\begin{proof}
El punto \textit{1} ya lo vimos en la definición anterior. Notemos que por la propiedad universal de la base existe un único $\tilde{\psi}:F=\ZZ^{(M\times N)}\to T$ tal que $\tilde{\psi}\circ \iota = \psi$ donde $\iota:M\times N\to F$ la inclusión. Pero ahora, por la propiedad universal del cociente, existe una única $\overline{\psi}:M\otimes_A N = F/P \to T$ tal que $\overline{\psi}\circ \pi = \tilde{\psi}$, pues $\tilde{\psi}$ es $A$-biaditiva y así $\left.\tilde{\psi}\right|_P = 0$. Es decir, la demostración se resume en el siguiente diagrama:
\begin{tikzcd}[row sep=3.3em,column sep=4em,minimum width=2em]
M\times N \arrow{d}[left, font=\normalsize]{\iota}\arrow{r}[font=\normalsize]{\psi} & T \\
F=\ZZ^{(M\times N)}\arrow{d}[left, font=\normalsize]{\pi}\arrow[dashed]{ru}[right, font=\normalsize]{\exists ! \tilde{\psi}}&  \\
M\otimes_A N \arrow[bend right, dashed]{ruu}[right, font=\normalsize]{\exists ! \overline{\psi}}
\end{tikzcd}
Como $\varphi = \pi\circ\iota$, la proposición sigue.

\end{proof}
\end{prop}

\begin{cor}
Por la Propiedad Universal, $M\otimes_A N$ es el único grupo abeliano $H$, a menos de isomorfismo, tal que $\varphi:M\times N\to H$ es universalmente $A$-biaditiva.
\end{cor}

\begin{ex}
\begin{itemize}
\item En $M\otimes_A N$ se tiene que $0\otimes_A n=0$ pues $(0+0)\otimes_A n = 0\otimes_A n  + 0\otimes_a n$. De la misma forma, $m\otimes_A 0=0$.
\item $\QQ\otimes_\ZZ \ZZ_n = 0$ pues $x\otimes_\ZZ y = \dfrac{x}{n}n\otimes_\ZZ y = \dfrac{x}{n}\otimes_\ZZ ny = \dfrac{x}{n}\otimes_\ZZ 0 = 0$.
\item Tenemos que $\ZZ_n\otimes_\ZZ \ZZ_m = \ZZ_{(m:n)}$. En particular, si $m$ y $n$ son coprimos, $\ZZ_n\otimes_\ZZ \ZZ_m = 0$.
\item $_AN$ un $A$-módulo a izquierda. Entonces $A\otimes_A N \simeq N$. En efecto, construyamos un isomorfismo. Sea $\mu:A\times N\to N$, $\mu(a,n)=an$. Sabemos que $\mu$ es $A$-biaditiva. Entonces, existe un único $\overline{\mu}:A\otimes_N N\to N$ con $\overline{\mu}(a\otimes_A n)=an$. Defino $\gamma: N\to A\otimes_A N$ como $\gamma(n)=1\otimes_A n$. Entonces, $\overline{\mu}(\gamma(n)) = \overline{\mu}(1\otimes_A n)=n$ y $\gamma(\overline{\mu}(a\otimes_A n)) = \gamma(an) = 1\otimes_A an = a\otimes_A n$. Y listo.
\end{itemize}
\end{ex}

Ahora, agreguémosle más estructura a $M\otimes_A N$. Supongamos que $_BM_A$ es un $(B,A)$-bimódulo y $_AN_C$ es un $(A,C)$-bimódulo. Entonces $M\otimes_A N$ tiene estructura de $B$-módulo a izquierda y $C$-módulo a derecha. Esto se hace definiendo $b(m\otimes_A n) = (bm)\otimes_A n$ y $(m\otimes_A n)c = m\otimes_A (nc)$. En particular, si $A$ es conmutativo, $M_A= _AM_A$ y $_AN=_AN_A$ y así $M\otimes_A N$ es un $A$-bimódulo, con $a(m\otimes_A n) = (am)\otimes_A n = (ma)\otimes_A n = m\otimes_A (an) = m\otimes_A (na) = (m\otimes_A n)a$.

\begin{obs}
Si $A$ es un anillo conmutativo, el $A$-módulo $M\otimes_A N$ tiene la propiedad universal $\varphi:M\times N\to M\otimes_A N$, $(m,n)\mapsto m\otimes_A n$ es universalmente $A$-bilineal.
\end{obs}

\begin{defn}
Sea $A$ un anillo y $f:M\to M'$ morfismo de $A$-módulos a derecha y $g:N\to N'$ morfismo de $A$-módulos a izquierda. Se define entonces $f\otimes g:M\otimes_A N\to M'\otimes_A N'$ como $(f\otimes g)(m\otimes n) = f(m)\otimes g(n)$.

Es importante notar que esto está bien definido. Para eso, vamos a usar la propiedad universal. Notemos que $f\times g:M\times N\to M'\otimes_A N'$ dada por $(f\times g)(m,n) = f(m)\otimes g(n)$, si fuera $A$-biaditiva se extiende de forma única a un morfismo $\ZZ$-lineal $f\otimes g$. Pero es fácil chequear que es $A$-biaditiva: en efecto, $f\times g(m+m',n) = f(m+m')\otimes g(n) = (f(m)+f(m'))\otimes g(n) = f(m)\otimes g(n) + f(m')\otimes g(n) = f\times g(m,n) + f\times g (m',n)$, y las otras relaciones son similares.
\end{defn}

\begin{prop}
Sea $M$ un $A$-módulo a derecha y $\{N_i\}_{i\in I}$ una familia de $A$-módulos a izquierda. Entonces $M\otimes_A\left(\displaystyle\bigoplus_{i\in I} N_i\right) \simeq \displaystyle\bigoplus_{i\in I}(M\otimes_A N_i)$.
\begin{proof}
Defino $\varphi:M\times \left(\displaystyle\bigoplus_{i\in I}N_i\right)\to \displaystyle\bigoplus_{i\in I}(M\otimes_A N_i)$ por $\varphi(m,(n_i)_{i\in I}) = (m\otimes n_i)_{i\in I}$. Esto está bien definido pues casi todos los $n_i$ son $0$, y así casi todos los $m\otimes n_i$ son $0$. Es claro que $\varphi$ es $A$-biaditiva. Entonces, por la propiedad universal hay un único morfismo $\ZZ$-lineal $\overline{\varphi}:M\otimes_A\left(\displaystyle\bigoplus_{i\in I}N_i\right) \to \displaystyle\bigoplus_{i\in I}M\otimes_A N_i$, con $\overline{\varphi}(m\otimes_A (n_i)_{i\in I})=(m\otimes n_i)_{i\in I}$.

Defino $\overline{\psi}:\displaystyle\bigoplus_{i\in I}(M\otimes_A N_i)\to M\otimes_A \left(\displaystyle\bigoplus_{i\in I}N_i\right)$ inducida por los morfismos $\overline{\psi}_i:M\otimes_A N_i\to M\otimes_A \left(\displaystyle\bigoplus_{i\in I} N_i\right)$ por la propiedad universal de la suma directa. Pero a su vez, estos morfismos $\overline{\psi}_i$ están inducidos por $\psi_i:M\times N_i\to M\otimes_A \left(\bigoplus_{i\in I}N_i\right)$ con $\psi_i(m,x) = m\otimes j_i(x)$ donde $(j_i(x))_k=x\delta_{ik}$, que es trivialmente biaditiva.

Ahora, es trivial verificar que $\overline{\varphi}\circ\overline{\psi}=1$ y $\overline{\psi}\circ\overline{\varphi}=1$, entonces son isomorfos como queríamos.
\end{proof}
\end{prop}

\begin{prop}
Sea $A$ un anillo conmutativo y $M$ un $A$-módulo libre con base $\{v_i\}_{i\in I}$ y $N$ un $A$-módulo libre con base $\{w_j\}_{j\in J}$. Entonces $M\otimes_A N$ es un $A$-módulo libre con base $\{v_i\otimes w_j\}_{i\in I,\; j\in J}$. En particular, si $\sharp I,\sharp J <\infty$ entonces $\dim_A(M\otimes_A N)=\dim_A(M)\dim_A(N)$.
\begin{proof}

\end{proof}
\end{prop}

\begin{ex}
Sea $k$ un cuerpo y $V$ un $k$-espacio vectorial con $\dim_k V = 2$, con base $\{v_1,v_2\}$. Entonces $V\otimes_k V$ tiene dimensión $4$ y base $\{v_1\otimes v_1, v_1\otimes v_2, v_2\otimes v_1, v_2\otimes v_2\}$. Considero $v_1\otimes v_2 + v_2\otimes v_1$. Afirmo que no es un tensor elemental. En efecto, si existieran $v,w\in V$ tales que $v_1\otimes v_2 + v_2\otimes v_1 = v\otimes w$. Entonces, escribiendo a $v=av_1+bv_2$ y $w=cv_1 + dv_2$, tendríamos que $v\otimes w = (ac)v_1\otimes v_1 + (ad)v_1\otimes v_2 + (bc)v_2\otimes v_1 + (bd)v_2\otimes v_2$. Por unicidad de la escritura en base, tendríamos que $ac=0$, $ad=1$, $bc=1$ y $bd=0$. Que es absurdo.
\end{ex}

\begin{ex}
Sea $k$ un cuerpo. Entonces $k[x]\otimes_k k[y]\simeq k[x,y]$ como $k$-espacio vectorial. Busco alguna función $k$-bilineal $\gamma:k[x]\times k[y]\to k[x,y]$. Tomemos $\gamma(p(x),q(y)) = p(x)q(y)$. Es trivial verificar que es $k$-bilineal. Entonces, induce un morfismo $\overline{\gamma}(p(x)\otimes q(y)) = p(x)q(y)$. Definimos $\beta:k[x,y]\to k[x]\otimes_k k[y]$ por $\beta\left(\displaystyle\sum_{i,j}a_{ij}x^i y^j\right) = \displaystyle\sum_{i,j}a_{ij}(x^i\otimes_k y^j)$, que es trivialmente $k$-lineal. Es trivial verificar que $\overline{\gamma}\circ \beta=1$ y que $\beta\circ\overline{\gamma}=1$.

Notar que no todo elemento de $k[x]\otimes_k k[y]$ es un tensor elemental (sólamente los que puedo separar sus variables).
\end{ex}

\begin{prop}
Sea $A$ anillo conmutativo, $S\subseteq A$ multiplicativamente cerrado y $M$ un $A$-módulo a izquierda. Entonces $S^{-1}A\otimes_A M\simeq S^{-1}M$ como $S^{-1}A$-módulos (en particular como $A$-módulos).
\end{prop}

\begin{defn}
Sea $B$ un anillo y $A\subseteq B$ un subanillo. $M$ un $A$-módulo a izquierda. $B$ es un $(B,A)$-bimódulo. Entonces $B\otimes_A M$ es un $B$-módulo a izquierda y se llama la \textbf{extensión a escalares} de $M$ como $B$-módulo.
\end{defn}

\begin{ex}
Sea $A \subseteq B$ un subanillo y $M=A$. Entonces la extensión es $B\otimes_A A \simeq B$. En general, si $M=A^{(I)}$ la extensión es $B^{(I)}\simeq \displaystyle\bigoplus_{i\in I}B\otimes_A A \simeq B\otimes_A \left(\displaystyle\bigoplus_{i\in I}A\right) = B\otimes_A A^{(I)}$.

Si $A=\ZZ\subseteq B=\QQ$ y $M=\ZZ_n$, entonces $\QQ\otimes_{\ZZ} \ZZ_n = 0$ es la extensión de $\ZZ_n$ como $\QQ$-espacio vectorial.

Si $A=\RR$ y $B=\CC$ y $V$ es un $\RR$-espacio vectorial, la extensión de $V$ como $\CC$-espacio vectorial es $\CC\otimes_\RR V$ y se llama la \textit{complexificación} de V. Podemos pensar a $\CC\otimes_\RR V \simeq (\RR\oplus i\RR)\otimes_\RR V\simeq V\oplus Vi = \{v+wi : v,w\in V\}$ como $\CC$-espacio vectorial.
\end{ex}

\section{El funtor \texorpdfstring{$M\otimes_A -$}{MxA -}}

Sea $M$ un $A$-módulo a izquierda fijo. Para cada $A$-módulo a izquierda $N$, le asigno el grupo abeliano $M\otimes_A N$ y si $f:N\to N'$ es morfismo $A$-lineal, entonces le asigno $1\otimes f:M\otimes_A N\to M\otimes_A N'$. Se cumple que si $f:N\to N'$ y $g:N'\to N''$ son morfismos $A$-lineales, entonces $(1\otimes g)\circ (1\otimes f) = 1\otimes (g\circ f)$ (De hecho, es muy fácil de verificar por las definiciones). Además, se cumple que a $1_N:N\to N$ la identidad se le asigna $1\otimes_A 1_N :M\otimes_A N\to M\otimes_A N$, con $1\otimes 1_N = 1_{M\otimes_A N}$, también fácilmente chequeable. Estas consideraciones dicen que $M\otimes_A -:\mathrm{_AMod}\to\mathrm{Ab}$ es un funtor.

Notemos ademas que si $f:N\to N'$ es un isomorfismo entonces $1\otimes f:M\otimes_A N\to M\otimes_A N$ es un isomorfismo también (es fácil de chequear usando las propiedades funtoriales de $M\otimes_A -$).

\begin{teo}
El funtor $M\otimes_A -$ es exacto a derecha. Es decir, si $N\stackrel{f}{\longrightarrow} N'\stackrel{g}{\longrightarrow} N''\longrightarrow 0$ es una sucesión exacta corta de $A$-módulos, luego $M\otimes_A N\stackrel{1\otimes f}{\longrightarrow} M\otimes_A N\stackrel{1\otimes g}{\longrightarrow} M\otimes_A N''\longrightarrow 0$ es una sucesión exacta corta de grupos abelianos.
\begin{proof}
Primero veamos que $1\otimes g$ es epimorfismo. Esto es fácil pues dado un tensor elemental $m\otimes n''\in M\otimes_A N''$, al ser $g$ epimorfismo, existe un $n'\in N'$ tal que $g(n')=n''$. Entonces $(1\otimes g)(m,n')=(m,g(n'))=(m,n'')$. Como los tensores elementales son generadores, ya estamos.

Ahora, tenemos que ver la exactitud en $M\times_A N'$. Notemos que $\im 1\otimes f\subseteq \ker 1\otimes g$ es sencillo. En efecto, $(1\otimes g)\circ (1\otimes f)(m,n) = m\otimes (g\circ f)(n) = m\otimes 0 = 0$. Para ver la inclusión reversa, como $\im 1\otimes f\subseteq \ker 1\otimes g$, podemos considerar el cociente $M\otimes_A N'/\im 1\otimes f$ y así $M\otimes_A N'/\im 1\otimes f \stackrel{\overline{1\otimes g}}{\longrightarrow} M\otimes_A N''$ con $\overline{1\otimes g}(\overline{m\otimes n'}) = m\otimes g(n')$ por la propiedad universal del cociente.

Defino una función $\gamma:M\times N''\to M\otimes_A N'/\im 1\otimes f$ por $\gamma(m,n'')=\overline{m\otimes n'}$ donde $n'\in N'$ cumple que $g(n')=n''$ (existe pues $g$ es epimorfismo). $\gamma$ está bien definido pues si $n_1', n_2'\in N'$ son tales que $g(n_1')=g(n_2')=n''$ entonces $n_1'-n_2'\in \ker g = \im f$, y así $\overline{m\otimes n_1'}-\overline{m\otimes n_2'} = \overline{m\otimes (n_1' - n_2')} = \overline{m\otimes f(n)} = \overline{0}$.

Es fácil chequear que $\gamma$ es $A$-biaditiva. Entonces, existe un único morfismo de grupos $\overline{\gamma}:M\otimes_A N''\to M\otimes_A N'/\im 1\otimes f$ con $\overline{\gamma}\circ\overline{1\otimes g}(\overline{m\otimes n'}) = \gamma(m\otimes g(n')) = \overline{m\otimes n'}$ y así $\overline{\gamma}\circ \overline{1\otimes g} = 1$. Sea $\omega\in\ker 1\otimes g$. Basta ver que $\overline{w}=\overline{0}\in M\otimes N'/\im 1\otimes f$. Entonces, por definición $\omega = \overline{\gamma}\circ (\overline{1\otimes g})(\overline{\omega}) = \overline{\gamma}((1\otimes g)(\omega))=\overline{0}$. Y listo.

\end{proof}
\end{teo}

\begin{obs}
El funtor $M\otimes_A -$ no es exacto a izquierda. Quiero ver que hay monomorfismos que no van a parar a monomorfismos. Un ejemplo de esto es $0\longrightarrow \ZZ\hookrightarrow \QQ$ monomorfismo de $\ZZ$-módulos. Tomando $M=\ZZ_n$ tenemos $0\longrightarrow \ZZ_n\otimes_\ZZ \ZZ \stackrel{1\otimes\mathrm{inc}}{\longrightarrow}\ZZ_n\otimes_{\ZZ}\QQ = 0$, que no puede ser monomorfismo.
\end{obs}

\begin{prop}[Adjunción entre $\hom_A(M,-)$ y $M\otimes_A -$, Ley exponencial]
Sean $X,Y,Z$ grupos abelianos. Entonces $\hom_\ZZ(X\otimes_\ZZ Y,Z)\simeq \hom_\ZZ(X,\hom_\ZZ(Y,Z))$ (Si agregamos estructura de $A$-módulo en vez de $\ZZ$-módulo se tiene que el isomorfismo es de $A$-módulos).
\begin{proof}
El truco va a ser el mismo de siempre: definimos un morfismo $$\varphi:\hom(X\otimes Y,Z)\to \hom(X,\hom(Y,Z))$$ de modo tal que $(\varphi(g)(x))(y) = g(x\otimes y)$ y $$\psi:\hom(X,\hom(Y,Z))\to \hom(X\otimes Y,Z)$$ de modo tal que $\psi(f)(x\otimes y) = (f(x))(y)$. Queda como ejercicio verificar que $\varphi,\psi$ están bien definidos, son $\ZZ$-lineales, $\varphi\circ\psi = 1$ y $\psi\circ\varphi=1$.
\end{proof}
\end{prop}

\begin{defn}
Sea $A$ un anillo y $M$ un $A$-módulo a derecha. Decimos que $M$ es un $A$-módulo \textbf{playo} si $M\otimes_A -$ es un funtor exacto. Es decir, si para todo $f:N\to T$ monomorfismo de $A$-módulos izquierdos se tiene que $1\otimes f:M\otimes_A N\to M\otimes_A T$ es monomorfismo (de grupos abelianos).
\end{defn}

\begin{ex}
Sea $A$ un anillo y $S\subseteq A$ un conjunto multiplicativamente cerrado. Entonces $S^{-1}A$ es un $A$-módulo playo. En particular, si $A$ es un dominio íntegro y $k$ su cuerpo de fracciones, entonces $k$ es un $A$-módulo playo.

En efecto, sea $f:N\to T$ un monomorfismo de $A$-módulos. Entonces, quiero ver que $1\otimes f:S^{-1}A\otimes_A N\to S^{-1}A\otimes_A T$ es un monomorfismo. Notemos que tenemos el siguiente diagrama conmutativo: \begin{center}\begin{tikzcd}[row sep=3.3em,column sep=4em,minimum width=2em]
S^{-1}A\otimes_A N \arrow{d}[left, font=\normalsize]{\stackrel{\simeq}{\varphi_N}}\arrow{r}[font=\normalsize]{\psi} & S^{-1}A\otimes_A T\arrow{d}[right, font=\normalsize]{\stackrel{\simeq}{\varphi_T}} \\
S^{-1}N\arrow{r}[font=\normalsize]{S^{-1}f} & S^{-1}T
\end{tikzcd}\end{center}
Es decir, el isomorfismo entre $S^{-1}A\otimes_A N$ y $S^{-1}N$ es natural. Pero localizar es exacto, entonces si $f:N\to T$ es monomorfismo, $S^{-1}f$ es monomorfismo y así, por el diagrama natural, tenemos que $1\otimes f$ es monomorfismo y estamos.
\end{ex}

\begin{prop}
Sea $A$ un dominio íntegro y $0\longrightarrow M\longrightarrow N\longrightarrow T\longrightarrow 0$ una sucesión exacta corta de $A$-módulos y $M\simeq A^m\oplus M'$, $N\simeq A^n \oplus N'$ y $T\simeq A^t\oplus T'$ con $M',N',T'$ de torsión. Entonces $n=m+t$.
\begin{proof}
Sea $k$ el cuerpo de fracciones de $A$. Como $k$ es $A$-playo, entonces tenemos que $0\longrightarrow k\otimes_A M\longrightarrow k\otimes_A N\longrightarrow k\otimes_A T\longrightarrow 0$ es una sucesión exacta corta. Como $k\otimes_A M = k\otimes_A (A^m\oplus M') = (k\otimes_A A^m)\oplus (k\otimes_A M') = k\otimes_A A^m$, ya que $k\otimes_A M'=0$ pues $k$ es divisible y $M'$ es de torsión. Entonces si $V=k\otimes_A A^m$, $W=k\otimes_A A^n$ y $Z=k\otimes_A A^t$, tenemos que $0\longrightarrow V\longrightarrow W\longrightarrow Z\longrightarrow 0$ es una sucesión exacta corta. Pero esta sucesión se parte, pues los $k$-espacios vectoriales son proyectivos. Esto quiere decir que $V\oplus Z \simeq W$, y así $\dim_k V + \dim_k Z = \dim_k W$. Es decir, $n = m+t$. Como queríamos.
\end{proof}
\end{prop}

\begin{obs}
Sea $M$ un $A$-módulo playo y $N\simeq M$. Entonces $N$ es un $A$-módulo playo.

En efecto, si $f:T\to Z$ es un monomorfismo de $A$-módulos, veamos que se tiene que $1\otimes f:N\otimes_A T\to N\otimes_A Z$ es monomorfismo. Pero tenemos un diagrama natural: \begin{center}\begin{tikzcd}[row sep=3.3em,column sep=4em,minimum width=2em]
N\otimes_A T \arrow{d}[left, font=\normalsize]{\simeq}\arrow{r}[font=\normalsize]{1_N\otimes f} & N\otimes_A Z\arrow{d}[right, font=\normalsize]{\simeq} \\
M \otimes_A T \arrow{r}[font=\normalsize]{1_M\otimes f} & M\otimes_A Z
\end{tikzcd}\end{center}
Entonces, si $f:T\to Z$ es monomorfismo, al ser $M$ un $A$-módulo playo, tenemos que $1_M\otimes f$ es monomorfismo, y por la conmutatividad del diagrama anterior, tenemos que $1_N\otimes f$ es monomorfismo. Y listo.
\end{obs}

\begin{teo}
Sea $A$ un anillo y $M$ un $A$-módulo a derecha proyectivo. Entonces $M$ es $A$-playo.
\begin{proof}
Notemos primero que si $L$ es un $A$-módulo libre entonces $L$ es playo. Por la observación anterior, bastará probar la afirmación para $A^{(X)}$ donde $A^{(X)}\simeq L$. Sea $f:N\to T$ un monomorfismo. Quiero ver que $1\otimes f:A^{(X)}\otimes_A N\to A^{(X)}\otimes_A T$ es un monomorfismo. Pero notemos que tenemos el siguiente diagrama natural:
\begin{center}\begin{tikzcd}[row sep=3.3em,column sep=4em,minimum width=2em]
A^{(X)}\otimes_A N \arrow{d}[left, font=\normalsize]{\simeq}\arrow{r}[font=\normalsize]{1\otimes f} & A^{(X)}\otimes_A T\arrow{d}[right, font=\normalsize]{\simeq} \\
\bigoplus_{x\in X}N \arrow{r}[font=\normalsize]{\bigoplus_{x\in X} f} & \bigoplus_{x\in X} T
\end{tikzcd}\end{center} Como $f$ es monomorfismo, $\displaystyle\bigoplus_{x\in X}f$ es monomorfismo y así $1\otimes f$ es monomorfismo por la conmutatividad del diagrama. Como queríamos ver.

Ahora, sea $M$ un $A$-módulo proyectivo. Entonces $M$ es sumando directo de un libre: $M\oplus R = L$. Sea $f:N\to T$ monomorfismo de $A$-módulos. Tenemos el siguiente diagrama conmutativo:
\begin{center}\begin{tikzcd}[row sep=3.3em,column sep=4em,minimum width=2em]
M\otimes_A N\arrow[hookrightarrow]{d}[left, font=\normalsize]{\mathrm{inc}_1}\arrow{r}[font=\normalsize]{1\otimes f} & M\otimes_A T\arrow[hookrightarrow]{d}[left, font=\normalsize]{\mathrm{inc}_1} \\
(M\otimes_A N)\oplus (R\otimes_A N) \arrow{d}[left, font=\normalsize]{\simeq} & (M\otimes_A T)\oplus (R\otimes_A T)\arrow{d}[right, font=\normalsize]{\simeq} \\
L \otimes_A T \arrow{r}[font=\normalsize]{1_L\otimes f} & L\otimes_A T
\end{tikzcd}\end{center}

Como $f$ es monomorfismo, $1_L\otimes f$ es monomorfismo. Como la flecha de $M\otimes_A N\to L\otimes_A N$ es monomorfismo y el diagrama conmuta, debemos tener que $1\otimes f$ es monomorfismo, y así $M$ debe ser playo. Y el teorema sigue.
\end{proof}
\end{teo}

\begin{prop}
Sea $0\longrightarrow N\longrightarrow T\longrightarrow R\longrightarrow 0$ una sucesión exacta corta de $A$-módulos que se parte. Entonces $0\longrightarrow M\otimes_A N\longrightarrow M\otimes_A T\longrightarrow M\otimes_A R\longrightarrow 0$ es una sucesión exacta corta de $A$-módulos que se parte para cualquier $A$-módulo $M$.
\begin{proof}
Es fácil notando que $N\longrightarrow T$ es sección (pues la sucesión exacta corta se parte) y que $M\otimes_A -$ es un funtor.
\end{proof}
\end{prop}

\begin{prop}[Criterio sobre Ideales de Platitud]
Sea $M$ un $A$-módulo a derecha. Son equivalentes:
\begin{enumerate}
\item $M$ es playo.
\item Para todo ideal $\mathfrak{a}\subseteq A$, la aplicación $M\otimes_A \mathfrak{a} \to M\otimes_A A$ inducida por la inclusión $\mathfrak{a}\hookrightarrow A$ es inyectiva.
\item Para todo ideal finitamente generado $\mathfrak{a}\subseteq A$, la aplicación $M\otimes_A \mathfrak{a}\to M\otimes_A A$ inducida por la inclusión $\mathfrak{a}\hookrightarrow A$ es inyectiva.
\end{enumerate}
\begin{proof}
$(1)\Longrightarrow (2)$ es claro por la definición de platitud y para $(2)\Longrightarrow (3)$ no hay nada que probar.

$(3)\Longrightarrow (2)$: Supongamos que $I\subseteq A$ es un ideal arbitrario y consideremos el morfismo inducido $1\otimes\iota :M\otimes_A I\to M\otimes_A A$ y supongamos que $(1\otimes \iota)\left(\displaystyle\sum_{i=1}^n a_i\otimes r_i\right) = 0$. Pero entonces si consideramos el ideal $J$ generado por los $r_i$, $i=1,\ldots , n$, tenemos que $1\otimes \iota$ restringido a $M\otimes_A J$ es inyectivo por hipótesis. Entonces $\displaystyle\sum_{i=1}^n a_i\otimes r_i = 0$ y así $1\otimes \iota$ es inyectivo.

$(2)\Longrightarrow (1)$: Veamos primero que si $F$ es un $A$-módulo libre y $K\subseteq F$ un submódulo, entonces el morfismo inducido por la inclusión $M\otimes_A K\to M\otimes_A F$ es inyectivo. En efecto, primero veamos esto en el caso en que $F$ es finitamente generado. Si $F=\displaystyle\bigoplus_{t\in T}A$ y $K = \displaystyle\bigoplus_{t\in T}I_t$ con $I_t\subseteq A$ un ideal a izquierda, entonces $$M\otimes_A \left(\displaystyle\bigoplus_{t\in T} I_t\right) \simeq \bigoplus_{t\in T}(M\otimes_A I_t) \hookrightarrow \bigoplus_{t\in T}M\otimes_A A\simeq M\otimes_A \bigoplus_{t\in T}A = M\otimes_A F$$ Y así el morfismo inducido $1\otimes \iota:M\otimes_A K\to M\otimes_A F$ es inyectivo. Ahora, si $F$ es un $A$-módulo libre arbitrario, considero el morfismo inducido $1\otimes\iota:M\otimes_A K\to M\otimes_A F$ y si $(1\otimes \iota)\left(\displaystyle\sum_{i=1}^n a_i\otimes k_i\right) = 0$ entonces hay un submódulo finitamente generado de $F$ que contiene a los $k_i$ y por lo que probamos recién, $1\otimes \iota$ restringido ahí es inyectivo y así $1\otimes \iota$ debe ser inyectivo. 

Ahora, sea $\varphi:L\to N$ un morfismo inyectivo de $A$-módulos a izquierda. Notemos que $N$ es cociente de un módulo libre. Entonces, existe $F$ libre y $K\subseteq F$ tal que $N\simeq F/K$, y así tenemos la sucesión exacta corta $0\longrightarrow K\longrightarrow F\stackrel{\pi}{\longrightarrow}N\longrightarrow 0$. Consideremos el $A$-submódulo $J=\{x\in F : \pi(x) \in \im \varphi\}$. Definamos $\theta:J\to L$ por $\theta(x)=y$ donde $y$ cumple que $\varphi(y) = \pi(x)$. Por definición, $\theta$ es epimorfismo. Veamos que $\theta$ es de hecho morfismo. En efecto, sean $x,y\in J$ y $r\in A$. Entonces, existen $a,b\in L$ tales que $\theta(x)=a$ y $\theta(y)=b$. Entonces, $\pi(x) = \varphi(a)$ y $\pi(y)= \varphi(b)$. Entonces $\pi(x) + r\pi(y) = \varphi(a) + r\varphi(b)$ y así $\pi(x+ry) = \varphi(a+rb)$. Entonces, $\theta(x+ry)=a+rb = \theta(x) + r\theta(y)$, y es morfismo. Además, tenemos $K\subseteq J$ pues $K=\ker \pi$. Esto quiere decir que tenemos el siguiente diagrama:

\begin{center}\begin{tikzcd}[row sep=3.3em,column sep=4em,minimum width=2em]
0\arrow{r} & K\arrow[equals]{d}\arrow[hookrightarrow]{r}[font=\normalsize]{\iota}   & J\arrow[hookrightarrow]{d}[left,font=\normalsize]{\iota}\arrow{r}[font=\normalsize]{\theta}  & L\arrow{r}\arrow{d}[left,font=\normalsize]{\varphi} & 0 \\
0\arrow{r} & K\arrow[hookrightarrow]{r}[font=\normalsize]{\iota} & F\arrow{r}[font=\normalsize]{\pi} & N\arrow{r} & 0
\end{tikzcd}\end{center}

Notemos que las filas son exactas y el diagrama conmuta. La conmutatividad del primer cuadrado es trivial y la del segundo sigue por cómo definimos las cosas. La exactitud de la fila de abajo ya la sabíamos. Sólo resta ver que la fila de arriba es exacta. Veamos la doble inclusión para ver que $\ker \theta = \im \iota$. Es fácil ver que si $x\in \im\iota = \ker \pi$ entonces  $\varphi(\theta(x)) = \pi(x) = 0$ y como $\varphi$ es inyectivo, $\theta(x)=0$. Ahora, si $x\in\ker\theta$, tennemos $\theta(x)=0$ y así $\varphi(\theta(x))=\pi(x) = 0$ y así $x\in\ker\pi = \im \iota$, y así la fila es exacta.

Pero recordemos que el funtor $M\otimes_A -$ es exacto a derecha. Entonces, tenemos el siguiente diagrama conmutativo con filas exactas:

\begin{center}\begin{tikzcd}[row sep=3.3em,column sep=4em,minimum width=2em]
M\otimes_A K\arrow[equals]{d}\arrow[hookrightarrow]{r}[font=\normalsize]{1\otimes\iota}   & M\otimes_A J\arrow[hookrightarrow]{d}[left,font=\normalsize]{1\otimes\iota}\arrow{r}[font=\normalsize]{1\otimes \theta}  & M\otimes_A L\arrow{r}\arrow{d}[left,font=\normalsize]{1\otimes \varphi} & 0 \\
M\otimes_A K\arrow[hookrightarrow]{r}[font=\normalsize]{1\otimes\iota} & M\otimes_A F\arrow{r}[font=\normalsize]{1\otimes\pi} & M\otimes_A N\arrow{r} & 0
\end{tikzcd}\end{center}

Sabemos que $\id:M\otimes_A K\to M\otimes_A K$ es claramente sobreyectivo y también sabemos que $1\otimes\theta:M\otimes_A J\to M\otimes_A L$ lo es pues tensorizar preserva suryecciones. Además, $1\otimes \iota: M\otimes_A J\to M\otimes_A F$ es inyectivo pues $J$ es un submódulo de $F$ libre. Entonces, simplemente siguiendo las flechas podemos probar que $1\otimes\varphi$ es inyectivo. Entonces tensorizar preserva inyecciones y así $M$ es playo.

\end{proof}
\end{prop}

\begin{cor}
Sea $A$ un anillo íntegro y $M$ un $A$-módulo playo. Entonces $M$ es libre de torsión.
\begin{proof}
Si existe algún $m\in T(M)$ entonces existe $a\in A$, $a\neq 0$ tal que $am=0$. Entonces, el morfismo inducido por la inclusión $\langle a\rangle\hookrightarrow A$ no es isomorfismo ya que tenemos $(\iota\otimes 1)(a\otimes m) = am = 0$. Pero esto es absurdo pues $M$ es playo. Entonces $T(M)=0$ y así es libre de torsión.
\end{proof}
\end{cor}

\begin{defn}
Sea $A$ un anillo y $M$ un $A$-módulo. Una relación $\displaystyle\sum_{i=1}^n f_ix_i = 0$ con $f_i\in A$ para todo $1\leq i\leq n$ y $x_i\in M$ para todo $1\leq i\leq n$ se dice \textbf{trivial} si existen $m\in\mathbb{N}_0$, $y_j\in M$ para $1\leq j\leq m$ y $a_{ij}\in A$ para $1\leq i \leq n, 1\leq j\leq m$ tales que $x_i = \displaystyle\sum_{j=1}^m a_{ij}y_j \;\forall 1\leq i\leq n$ y $0 = \displaystyle\sum_{i=1}^n f_ia_{ij}\;\forall 1\leq j\leq m$.
\end{defn}

\begin{prop}[Criterio Ecuacional de Platitud]
Sea $M$ un $A$-módulo a izquierda. $M$ es playo si y sólo si toda relación es trivial.
\begin{proof}
$(\Longrightarrow)$ Supongamos que $M$ es playo y sea $\displaystyle\sum_{i=1}^n f_ix_i =0$ una relación. Sea $I=\langle f_1,\ldots , f_n\rangle$ y consideremos el morfismo canónico $\varphi: A^n\to I$ definido por $e_i\mapsto f_i$ donde $\{e_i\}_{i=1,\ldots ,n}$ la base canónica de $A^n$. Tenemos entonces la siguiente sucesión exacta corta $0\longrightarrow \ker \varphi\longrightarrow A^n \stackrel{\varphi}{\longrightarrow} I\longrightarrow 0$. Notemos que, bajo el morfismo inducido por la inclusión de $I\hookrightarrow A$, tenemos que $1\otimes \iota \left(\displaystyle\sum_{i=1}^n f_i\otimes x_i\right) = 0$. Como $M$ es playo y por el anterior criterio, debemos tener que $\displaystyle\sum_{i=1}^n f_i\otimes x_i = 0\in I\otimes_A M$. Entonces, por la tensorización de la sucesión exacta corta, tenemos algún elemento de $\ker\varphi\otimes_A M$ que va a parar a $\displaystyle\sum_{i=1}^n e_i \otimes x_i\in A^n\otimes_A M$, llamémoslo $\displaystyle\sum_{j=1}^m k_j\otimes y_j$, y la imagen de los $k_j$ en $A^n$ está dada por $\displaystyle\sum_{i=1}^n a_{ij}e_i$. Es fácil ver que estos elementos que nombramos sugerentemente cumplen lo pedido.

$(\Longleftarrow)$ Supongamos que toda relación es trivial. Sea $I$ un ideal finitamente generado y $x=\displaystyle\sum_{i=1}^n f_i\otimes x_i \in I\otimes_A M$ que va a parar a $0$ bajo el morfismo inducido por la inclusión $I\hookrightarrow A$ bajo tensorizar (O sea, $\displaystyle\sum_{i=1}^n f_ix_i = 0$ es una relación). Pero entonces $$x =\displaystyle\sum_{i=1}^n f_i\otimes x_i = \sum_{i=1}^n f_i\otimes \left( \sum_{j=1}^m a_{ij}y_j\right) = \displaystyle\sum_{j=1}^m \left( \sum_{i=1}^n f_i\otimes a_{ij}\right)\otimes y_j = 0$$ Por el criterio sobre ideales de platitud, esto nos dice que $M$ debe ser playo. Y estamos.
\end{proof}
\end{prop}

\section{Módulos sobre Dominios de Ideales Principales}

Recordemos que si $A$ es un dominio íntegro y $M$ un $A$-módulo, entonces el submódulo de torsión es $T(M) = \{m\in M : \exists a\neq 0, \; am=0\}\subseteq M$. Si $A$ es íntegro y $M$ es un $A$-módulo libre, entonces es libre de torsión. Es decir, $T(M)=0$. Además, si $A$ es un DIP y $M$ es un $A$-módulo libre con $N\subseteq M$ submódulo, entonces $N$ es $A$-libre. En particular, para $A$ DIP, las nociones de libre y proyectivo son lo mismo.

\begin{lem}
Sea $A$ un dominio íntegro y $M$ un $A$-módulo finitamente generado libre de torsión ($T(M)=0$). Entonces existen $L$ un $A$-módulo libre de tipo finito y un morfismo $\iota:M\to L$.
\begin{proof}
Sean $\{x_1,\ldots , x_n\}\subseteq M$ tales que $M=\langle x_1,\ldots , x_n\rangle$ con $x_i\neq 0$ para todo $1\leq i\leq n$. Tomemos un conjunto linealmente independiente maximal $\{x_{i_1},\ldots , x_{i_r}\}\subseteq \{x_1,\ldots , x_n\}$. Esto existe pues cada $\{x_i\}$ es linealmente independiente (pues $T(M)=0$) y  $\{x_1,\ldots , x_n\}$ es finito. Sea $L=\langle x_{i_1},\ldots , x_{i_r}\rangle$. Por construcción, $L$ es libre (es el generado por un conjunto linealmente independiente). Tomemos $x_j\in \{x_1,\ldots , x_n\}-\{x_{i_1},\ldots , x_{i_r}\}$ Entonces, por la maximalidad tenemos que $\{x_{i_1},\ldots , x_{i_r},x_j\}$ es linealmente dependiente. El coeficiente de $x_j$ en la combinación lineal no es $0$, pues si no $\{x_{i_1},\ldots , x_{i_r}\}$ no serían linealmente independientes. Entonces existen $\lambda_j\neq 0$ tales que $\lambda_j x_j\in L$. Tomo $\lambda = \displaystyle\prod_{x_j\in \{x_1,\ldots , x_n\}-\{x_{i_1},\ldots , x_{i_r}\}} \lambda_j $. Como $A$ es dominio íntegro, $\lambda \neq 0$. Usando la conmutatividad, como $\lambda_j x_j\in L$ entonces $\lambda x_j\in L$ para todo $j$. Pero entonces $\lambda x \in L$ para cualquier $x\in M$ pues $x=\displaystyle\sum_{j=1}^n a_jx_j$. Finalmente defino $\iota:M\to L$ por $\iota(x) = \lambda x$, que es monomorfismo pues $T(M)=0$. Y estamos.
\end{proof}
\end{lem}

\begin{cor}
Sea $A$ DIP y $M$ un $A$-módulo finitamente generado y sin torsión. Entonces $M$ es libre.
\begin{proof}
Por lo anterior, existe un $A$-módulo libre $L$ y un monomorfismo $\iota:M\to L$. Entonces $M\simeq \iota(M)\subseteq L$. Como $L$ es libre y $A$ es DIP, entonces $\iota(M)$ es libre (por ser submódulo de libre). Esto quiere decir que $M$ es libre y listo.
\end{proof}
\end{cor}

\begin{obs}
Si $M$ es un $A$-módulo de tipo finito, como $A$ es un dominio de ideales principales, es noetheriano (pues los módulos sobre anillos noetherianos son noetherianos). Tenemos la siguiente sucesión exacta corta: $0\longrightarrow T(M)\stackrel{\iota}{\longrightarrow} M\stackrel{\pi}{\longrightarrow} M/T(M)\longrightarrow 0$. Pero como $M$ es de tipo finito y noetheriano y a $T(M)$ lo vemos como un submódulo de $M$ vía la inclusión, tenemos que $T(M)$ debe ser de tipo finito. Además, aplicandole $\pi$ a los generadores se ve que $M/T(M)$ también es de tipo finito. Pero $M/T(M)$ no tiene torsión. En efecto, si $a\overline{x}=\overline{0}$, tendríamos que $\overline{ax}=\overline{0}$ y así $ax\in T(M)$, por lo tanto habría un $b\in A$ tal que $bax = 0$, y como $A$ es íntegro, $ba\neq 0$ pues $a,b\neq 0$. Entonces $x\in T(M)$, y así $\overline{x}=\overline{0}$. Como $M/T(M)$ es de tipo finito y no tiene torsión, por el corolario anterior debe ser libre. Pero entonces es proyectivo y así la sucesión exacta corta que teníamos se parte. Esto quiere decir que $M\simeq T(M)\oplus M/T(M)$. Si $L=M/T(M)$ libre, como es de tipo finito, tenemos que $L\simeq A^n$, con $n\in\mathbb{N}_0$. Además, como $A$ es conmutativo, tiene noción de rango y este $n$ está determinado (por el $M$). Decimos que el \textbf{rango} $\mathrm{rk}{M}$ es la dimensión como $A$-módulo de la parte libre de $M$.
\end{obs}

Vamos a querer estudiar la torsión de $M$, pues la parte libre ya sabemos cómo es. Es decir, buscamos caracterizar los módulos de torsión de tipo finito sobre $A$ para $A$ un dominio de ideales principales.

\begin{defn}
Sea $A$ un dominio de ideales principales y $p\in A$ un elemento irreducible. Sea $M$ un $A$-módulo. Entonces, se define la \textbf{componente} $p$-\textbf{primaria} de $M$ como $$M_p=\{m\in M : \;\exists n\in\mathbb{N}_0 \; \text{tal que } p^n m = 0 \}$$
\end{defn}
\begin{defn}
Sea $A$ un anillo y $M$ un $A$-módulo a izquierda. Se define el anulador de $M$ como $\ann(M)=\{a\in A : ax=0 \;\forall x\in M\}$. Es fácil ver que $\ann(M)$ es un ideal a izquierda.
\end{defn}

\begin{lem}
Sea $A$ un dominio íntegro y $M$ un $A$-módulo de torsión de tipo finito. Entonces $\ann(M)\neq 0$.
\begin{proof}
Supongamos que $M=\langle x_1,\ldots , x_n\rangle$. Para cada $i=1,\ldots ,n$ existe $a_i\neq 0$ tal que $a_i x_i=0$. Tomo entonces $a=\displaystyle\prod_{i=1}^n a_i\neq 0$ pues $A$ es dominio íntegro. Entonces $a\in\ann(M)$ pues anula a todos los generadores. Y listo.
\end{proof}
\end{lem}

\begin{obs}
Si $A$ es un dominio de ideales principales y $M$ un $A$-módulo de torsión de tipo finito entonces $\ann(M)=\langle a\rangle$ para algún $a\neq 0$. Además, $\mathcal{P}$ representará al conjunto de clases de irreducibles de $A$ módulo asociados.
\end{obs}

\begin{teo}
Sea $A$ un dominio de ideales principales y $M$ un $A$-módulo de torsión. Entonces $M = \displaystyle\bigoplus_{p\in\mathcal{P}}M_p$.
\begin{proof}

Primero probemos que $M=\displaystyle\sum_{p\in\mathcal{P}}M_p$. Para ello, veamos que todo $m\in M$ se puede escribir como suma de elementos de los $M_p$. Como $M$ es de torsión, existe $a\in A$, $a\neq 0$ tal que $am=0$. Escribimos $a=\mu p_1^{r_1}\cdots p_s^{r_s}$ con $p_i\in \mathcal{P}$ para cada $1\leq i\leq s$ y $\mu\in\mathcal{U}(A)$. Para cada $1\leq i\leq s$ tomo $b_i = a/p_i^{r_i}$. Notemos que $\mcd{b_1,\ldots , b_s}=1$. Entonces, existen $c_1,\ldots , c_s\in A$ tales que $1 = c_1b_1 + \ldots + c_sb_s$ pues $A$ es un dominio de ideales principales. Pero entonces $m = c_1(b_1m) + \ldots + c_s(b_sm)$. Como $b_im\in M_{p_i}$ pues $p_i^{r_i}b_im = am=0$, tenemos que $m\in\displaystyle\sum_{i=1}^r M_{p_i}$. Entonces $M=\displaystyle\sum_{p\in\mathcal{P}}M_p$.

Ahora veamos que la suma es directa. Para eso, debo ver que si $m_{p_1}+\ldots + m_{p_k}=0$ para $m_{p_i}\in M_{p_i}$ entonces $m_{p_i}=0$ para todo $i$. Esto lo probaremos por inducción sobre $k$. Para $k=1$ no hay nada que probar. Supongamos que vale para $k$ y probemos que vale para $k+1$. Para cada $1\leq i\leq k+1$ existe un $r_i$ tal que $p_i^{r_i}m_{p_i}=0$. Notemos que se tiene, multiplicando por el producto de los primeros $k$ irreducibles que $p_1^{r_1}\cdots p_k^{r_k}(m_{p_1}+\ldots + m_{p_k}+m_{p_{k+1}})=0$. Pero como ese producto anula a los primeros $k$ términos, tenemos que $p_1^{r_1}\cdots p_k^{r_k}m_{p_{k+1}}=0$. Como $p_1^{r_1}\cdots p_k^{r_k}$ y $p_{k+1}^{r_{k+1}}$ son coprimos, existen $a,b\in A$ tales que $1=ap_1^{r_1}\cdots p_k^{r_k} + b p_{k+1}^{r_{k+1}}$. Pero entonces $m_{p_{k+1}} = ap_1^{r_1}\cdots p_k^{r_k}m_{p_{k+1}} + bp_{k+1}^{r_{k+1}}m_{p_{k+1}} = 0$. Luego, $m_{p_1}+\ldots + m_{p_k}=0$ y por hipótesis inductiva, los $m_{p_i}$ deben ser todos nulos. Luego la suma debe ser directa y estamos.

\end{proof}
\end{teo}

\begin{obs}
Si $M$ es un $A$-módulo de torsión de tipo finito, luego $\ann(M)=\langle a\rangle$ con $a\neq 0$. Si nos fijamos con cuidado en la demostración del teorema anterior, se ve que para todo $m\in M$ tenemos que $am=0$ y factorizando en primos $a=\mu p_1^{r_1}\cdots p_s^{r_s}$ tenemos que $M=\displaystyle\bigoplus_{i=1}^s M_{p_i}$. Además, tenemos que $\ann(M_{p_i}) = \langle p_i^{r_i}\rangle$. En efecto, esto se ve fácil mirando la doble contención. $(\supseteq)$ sigue de la demostración del teorema anterior y $(\subseteq)$ sale de que si $\ann(M)=a$ y factorizamos a $a$ en primos y $q\in\ann(M_{p_i})$ pero $p_i^{r_i}\nmid q$, multiplicando a $q$ por los otros $p_j^{r_j}$ tenemos que $\left(\displaystyle\prod_{j\neq i}p_j^{r_j}\right)q \in\ann(M)$, que es absurdo pues $p_i^{r_i}\nmid\left(\displaystyle\prod_{j\neq i}p_j^{r_j}\right)q$.
\end{obs}

Ahora nos concentraremos en estudiar los $M_p$.

\begin{obs}
Sea $A$ un DIP y $p\in A$ irreducible. Entonces $A/\langle p\rangle$ es un cuerpo, pues $\langle p\rangle$ es un ideal maximal. Además, si $M$ es un $A$-módulo, podemos considerar $M/pM$ como $A$-módulo y como $A/\langle p\rangle$-espacio vectorial donde $\overline{a}\overline{m} = \overline{am}$ (¡hay que chequear la buena definición de esto!).
\end{obs}
\begin{obs}
Sea $A$ un DIP, $M$ un $A$-módulo y $x\in M$. Entonces, se tiene que $A/\ann(\langle x\rangle)\simeq\langle x\rangle$. En efecto, consideramos $\varphi:A\to \langle x\rangle$ de modo que $\varphi(a)=ax$ con $\ker \varphi = \ann(\langle x\rangle)$, y por el primer teorema de isomorfismo sigue lo que queremos.
\end{obs}

\begin{teo}
Sea $M$ un $A$-módulo de tipo finito. $A$ dominio de ideales principales tal que $\ann(M)=\langle p^n\rangle$. Entonces, para todo $1\leq i\leq n$ existen familias $\{x_j^{(i)}\}_{j\in J_i}$ de elementos de $M$ (eventualmente $J_i$ puede ser vacío para algunos $i$) de modo tal que $M=\displaystyle\bigoplus_{i=1}^n \displaystyle\bigoplus_{j\in J_i} \langle x_j^{(i)}\rangle$ y $\ann(\langle x_j^{(i)}\rangle) = \langle p^i\rangle$ para todo $j\in J_{i}$. Además, $\sharp J_i$ es independiente de la factorización (depende sólo del $M$).
\begin{proof}
Vamos a dividir la demostración en tres partes.

Para la primera parte, veamos que $M=\displaystyle\sum_{i=1}^n \sum_{j\in J_i} \langle x_j^{(i)}\rangle$ para ciertos $\{x_j^{(i)}\}_{j\in J_i}$. Consideremos $M_i=\{x\in M : p^i m = 0\}$. En particular $M_0=0$. Notemos que tenemos una filtración $0=M_0\subseteq M_1\subseteq\ldots\subseteq M_n=M$, y consideremos entonces la filtración $0=\dfrac{M_0+pM}{pM}\subseteq \dfrac{M_1+pM}{pM}\subseteq \ldots \subseteq \dfrac{M_n + pM}{pM}=\dfrac{M}{pM}$ de $A/\langle p\rangle$-subespacios vectoriales. Sea $\{\overline{x}_j^{(1)}\}_{j\in J_1}$ una base de $(M_1+pM)/pM$. Completamos a esta base a una base de $(M_2 + pM)/pM$ con $\{\overline{x}_j^{(2)}\}_{j\in J_2}$. Siguiendo recursivamente, obtenemos $\{\overline{x}_j^{(i)}\}_{1\leq i\leq n; j\in J_i}$ una base de $M/pM$ como $A/\langle p\rangle$-espacio vectorial. Para cada $\overline{x}_j^{(i)}\in (M_i + pM)/pM$ tomemos un representante de esa clase $x_j^{(i)}\in M_i- M_{i-1}$ (pues si estuviera en $M_{i-1}$ no completaba ninguna base al pasar al cociente). Notemos que por cómo elegimos a estos $x_j^{(i)}$ tenemos que $\ann(x_j^{(i)}) = \langle p^i\rangle$. Veamos que $M=\displaystyle\sum_{i=1}^n \sum_{j\in J_i} \langle x_j^{(i)}\rangle$. Consideremos entonces $N=\langle x_j^{(i)}\rangle_{i,j} = \displaystyle\sum_{i=1}^n \sum_{j\in J_i} \langle x_j^{(i)}\rangle$. Quiero ver que $M=N$. Notemos que si $m\in M$ entonces $\overline{m}\in M/pM$ y así $\overline{m} = \displaystyle\sum_{i,j} \overline{a}_{j}^{(i)} \overline{x}_j^{(i)}$ y así $m - \displaystyle\sum_{i,j}a_j^{(i)}x_j^{(i)}\in pM$. Entonces $M=N+pM$. Pero esto quiere decir que $M = N+pM = N + p(N+pM) = N+pN + p^2M$. Pero $pN\subseteq N$ y así $N+pN=N$. O sea, $M = N + p^2M$. Continuando inductivamente, obtenemos $M = N + p^n M$ pero $p^nM=0$ por hipótesis. Entonces $M=N$ como queríamos.

Para la segunda parte, vamos a ver que esta suma es de hecho directa. Supongamos que tenemos una suma $\displaystyle\sum_{i,j}a_j^{(i)}x_j^{(i)}=0$. Debo ver que $a_j^{(i)}x_j^{(i)}=0$ para cada $i,j$. Para eso, vamos a probar que $a_j^{(i)}x_j^{(i)}\in p^k \langle x_j^{(i)}\rangle$ para todo $k\geq 0$ y para todos $i,j$. Como $p^n\langle x_j^{(i)}\rangle=0$ esto probará que $a_j^{(i)}x_j^{(i)}=0$. Procederemos por inducción en $k$. Para $k=0$ no hay nada que probar. Supongamos que $a_j^{(i)}x_j^{(i)}\in p^k \langle x_j^{(i)}\rangle$ para todos $i,j$. Entonces, existe $b_j^{(i)}\in A$ tal que $a_j^{(i)}x_j^{(i)} = p^k b_j^{(i)}x_j^{(i)}$. Si $k\geq i$ entonces $a_j^{(i)}x_j^{(i)}=0$ pues $\ann(x_k^{(i)}) = \langle p^i\rangle$. Además, $$0 = \displaystyle\sum_{i,j}a_j^{(i)}x_j^{(i)} = \displaystyle\sum_{i=k+1}^n \sum_{j\in J_i} a_j^{(i)}x_j^{(i)} = \displaystyle\sum_{i=k+1}^n \sum_{j\in J_i} p^k b_j^{(i)}x_j^{(i)} = p^k \displaystyle\sum_{i=k+1}^n\sum_{j\in J_i} b_j^{(i)}x_j^{(i)} $$ Es decir, $\displaystyle\sum_{i=k+1}^n \sum_{j\in J_i}b_j^{(i)}x_j^{(i)}\in M_k$. Miremos su clase en $(M_k + pM)/pM$. Pero notemos que una base de este subespacio es, por la construcción que hicimos en la primera parte, $\{\overline{x}_j^{(i)}\}_{i\leq k; j\in J_i}$. Esto quiere decir que $\displaystyle\sum_{i=k+1}^n \sum_{j\in J_i} \overline{b}_j^{(i)}\overline{x}_j^{(i)}=\overline{0}$. Y por la escritura única en base de $M/pM$, tenemos que $\overline{b}_{j}^{(i)}=\overline{0}$ para todo $i\geq k+1$ y $j\in J_i$. Esto quiere decir que existen $c_j^{(i)}\in A$ tales que $b_j^{(i)}=pc_j^{(i)}$. Esto quiere decir que $a_j^{(i)}x_j^{(i)}\in p^{k+1}\langle x_j^{(i)}\rangle$ para todo $i\geq k+1$ y $j\in J_i$. Para los índices más chicos ya sabíamos que daba $0$ y así trivialmente están. Entonces la inducción está completa.

Para la tercera parte, veamos que $\sharp J_i$ no depende de la factorización que elijamos, sino sólamente de $M$. Para esto, procederemos por inducción en $n$ con $\ann(M)=\langle p^n\rangle$. En efecto, sabemos que $M=\displaystyle\bigoplus_{i=1}^n\bigoplus_{j\in J_i}\langle x_j^{(i)}\rangle$ con $\ann (x_j^{(i)})=\langle p^i\rangle$. Cuando $n=1$ notemos que $pM=0$ y así $M/pM\simeq M$ y entonces $\sharp J_1 = \dim_{A/\langle p\rangle}(M)$. Supongamos que vale para $n-1$ y veamos que vale para $n$. Consideremos $pM = \displaystyle\bigoplus_{i=2}^n \displaystyle\bigoplus_{j\in J_i} \langle px_j^{(i)}\rangle$ (esto vale pues cuando $i=1$ son las cosas de orden $p$ y desaparecen). Entonces $\ann pM = \langle p^{n-1}\rangle$ y $\sharp J_2, \ldots , \sharp J_n$ quedan determinados por $pM$ por hipótesis inductiva, y así por $M$. Consideremos ahora $p^2M = \displaystyle\bigoplus_{i=3}^n \displaystyle\bigoplus_{j\in J_i} \langle p^2 x_j^{(i)}\rangle$ y $\{m\in M : pm \in p^2 M\} = \left(\displaystyle\bigoplus_{j\in J_1} \langle x_j^{(1)}\rangle\right)\displaystyle\bigoplus_{i=2}^n \bigoplus_{j\in J_i}\langle px_j^{(i)}\rangle$. Pero esto quiere decir que $\{m\in M : pm \in p^2 M\} / pM = \displaystyle\bigoplus_{j\in J_1}\langle x_j^{(1)}\rangle$ es un $A/\langle p\rangle$-espacio vectorial, y así $\sharp J_1$ queda fijo como su dimensión, que no depende de la factorización. Y estamos. 
\end{proof}
\end{teo}

\begin{cor}
Sea $A$ un dominio de ideales principales y $M$ un $A$-módulo de tipo finito y de torsión. Entonces $M\simeq A/\langle d_1\rangle \oplus\ldots \oplus A/\langle d_n\rangle$ con $d_1\mid d_2\mid \ldots \mid d_n$ y es única salvo asociados.
\begin{proof}
Sabemos que $M$ es de torsión de tipo finito, con $\ann(M) = \langle a\rangle$. Si factorizamos $a=\mu p_1^{r_1}\cdots p_s^{r_s}$, tenemos que $M\simeq M_{p_1}\oplus\ldots\oplus M_{p_s}$ donde cada $M_{p_i}$ es la componente $p_i$-primaria. Pero además sabemos que cada $M_{p_i}$ se puede escribir como $M_{p_i} = \langle x_1^i\rangle \oplus\ldots\langle x_{n_i}^i \rangle$ con $\ann(x_j^i)=\langle p^{m_{ij}}\rangle$ y $m_{ij}\leq m_{i(j-1)}$ para $i=1,\ldots , s$. Sea $n=\max_{i=1,\ldots , n}\{n_i\}$. Entonces, completando los primeros lugares con ceros si hace falta, tenemos que \begin{align*}M_{p_1} &= \langle x_1^1 \rangle \oplus \ldots \oplus \langle x_n^1 \rangle \\ M_{p_2} &= \langle x_1^2\rangle \oplus \ldots \oplus\langle x_n^2\rangle \\ \vdots\quad& \\ M_{p_s} &= \langle x_1^s\rangle\oplus\ldots\oplus \langle x_n^s\rangle \end{align*}
En esa escritura tenemos $\ann(x_j^i) = \langle p_i^{m_{ij}}\rangle$ (eventualmente $m_{ij}=0$ y eso dice que $\langle x_i^j\rangle =0$).
Pero además, $\langle x_j^i\rangle \simeq A/\langle p_i^{m_{ij}}\rangle$. La idea ahora es sumar por columnas y hacer Teorema Chino del Resto:
$$\langle x_j^1\rangle\oplus\ldots\oplus\langle x_j^s\rangle\simeq A/\langle p_1^{m_{1j}}\rangle \oplus\ldots \oplus A/\langle p_s^{m_{sj}}\rangle \stackrel{TCR}{\simeq} A/\langle p_1^{m_{1j}}\cdots p_s^{m_{sj}}\rangle$$ Si llamo entonces $d_j = p_1^{m_{1j}}\cdots p_s^{m_{sj}}$ obtenemos lo deseado.

\end{proof}
\end{cor}

\begin{cor}
En el caso de $M$ grupo abeliano finitamente generados (es decir, $\ZZ$-módulo de tipo finito), la conclusión del corolario anterior se puede expresar como que $M\simeq \ZZ^m \oplus \ZZ_{d_1}\oplus\ldots \oplus \ZZ_{d_n}$ con $d_1\mid d_2\mid\ldots \mid d_n$ en forma única.
\end{cor}

\begin{ex}
Notemos que $\ZZ_{2}\oplus \ZZ_3\oplus\ZZ_{10}$ no está en la forma de la factorización del corolario. Notemos que $\ZZ_{10}\simeq \ZZ_{2}\oplus\ZZ_5$. Entonces tengo dos factores $\ZZ_2$ y un $\ZZ_3$ y un $\ZZ_5$. La factorización del corolario es $\ZZ_2 \oplus (\ZZ_2\oplus \ZZ_3\oplus \ZZ_5) \simeq \ZZ_2 \oplus \ZZ_{30}$.
\end{ex}

\section{Módulos Semisimples}

\begin{defn}
Sea $A$ un anillo y $M$ un $A$-módulo. Decimos que $M$ es \textbf{semisimple} si es suma de submódulos simples. Un anillo $A$ es semisimple si es semisimple como $A$-módulo a izquierda.
\end{defn}

\begin{lem}
Sea $M=\displaystyle\sum_{i\in I}S_i$ con $S_i$ simple para cada $i\in I$. Si $N\subseteq M$ es un submódulo, entonces existe $J\subseteq I$ tal que $M=N \oplus \displaystyle\bigoplus_{i\in J}S_i$.
\begin{proof}[Idea de la demostración]
Consideremos $\mathcal{J}=\{J\subseteq I : N+\displaystyle\sum_{i\in J}S_i \text{ la suma es directa}\}$. Notemos que $\emptyset\in \mathcal{J}$, y así $J\neq \emptyset$. Aplicando el Lema de Zorn, vemos que $M' = N + \displaystyle\sum_{i\in J}S_i$ con $J$ un elemento maximal de $\mathcal{J}$. Ahora, veamos que $M'=M$. Para esto, basta ver que para cada $i\in I$, $S_i\subseteq M'$. En efecto, si $S_{i_0}\not\subseteq M'$ para algún $i_0\in I$, tenemos que $S_{i_0}\cap M'=0$ por la simplicidad de $S_{i_0}$. Pero entonces $J\cup \{i_0\}\in\mathcal{J}$. Esto contradice la maximalidad de $J$. Entonces $M'=M$, como queríamos probar.
\end{proof}
\end{lem}

\begin{teo}
Sea $M$ un $A$-módulo. Son equivalentes:
\begin{enumerate}
\item $M$ es semisimple.
\item $M$ es suma directa de submódulos simples.
\item Todo submódulo de $M$ es un sumando directo.
\end{enumerate}
\begin{proof}
$(1)\Longrightarrow (2)$: Sigue de tomar $N=0$ en el lema anterior.

$(2)\Longrightarrow (1)$: Es trivial por definición.

$(1)\Longrightarrow (3)$: Es consecuencia directa del lema anterior.

$(3)\Longrightarrow (1)$: Supongamos que todo submódulo de $M$ es un sumando directo. Consideremos $M' = \displaystyle\bigoplus_{S\in\mathcal{S}}S$ con $\mathcal{S}=\{S\subseteq M \text{ submódulo simple}\}$. Supongamos que $M'\subset M$ está contenido de forma propia. Por hipótesis, existe $N\subseteq M$ tal que $M'\oplus N=M$. Pero si $S\subseteq N$ es un submódulo simple de $N$, como $M'\cap S=0$ tenemos que $S$ es simple y que no está en $M'$. Absurdo. Entonces $M'=M$. Resta ver que, bajo las hipótesis, todo submódulo de $M$ posee un submódulo simple. En efecto, basta verlo para los submódulos cíclicos $Am$, $m\in M - \{0\}$. Si $\mathfrak{a}$ es un ideal a izquierda maximal que contiene a $\ann(m)$ entonces $\mathfrak{a}m$ es un submódulo maximal de $Am$. Entonces $Am/\mathfrak{a}m$ es simple. Pero entonces, existe $L\subseteq M$ tal que $M=\mathfrak{a}m\oplus L$ y así $Am = (\mathfrak{a}\oplus L)\cap Am = \mathfrak{a}m\oplus (L\cap Am)$. Entonces, $L\cap Am \simeq Am/\mathfrak{a}m$ es un submódulo simple de $Am$. Esto concluye la demostración.
\end{proof}
\end{teo}

\begin{cor}
Todo submódulo de un módulo semisimple es semisimple.
\begin{proof}
Sea $N$ un submódulo de $M = \displaystyle\sum_{i\in I}S_i$. Entonces, existe $P\subseteq M$ tal que $N\oplus P = M$. Pero por el lema, existe $J\subseteq I$ tal que $P\oplus \displaystyle\bigoplus_{j\in J}S_j = M$. Pasando al cociente tenemos que $N\simeq M/P \simeq \displaystyle\bigoplus_{j\in J} S_j$, y así $N$ es semisimple. 
\end{proof}
\end{cor}

\begin{cor}
Si $0\longrightarrow M'\stackrel{f}{\longrightarrow} M\stackrel{g}{\longrightarrow}M''\longrightarrow 0$ es una sucesión exacta corta de $A$-módulos y $M$ es semisimple entonces la sucesión se parte y $M'$ y $M''$ son semisimples.
\begin{proof}
Veamos que $f$ es sección. En efecto, $\im f\subseteq M$ es sumando directo y así se puede definir una inversa como $f^{-1}(x) = y$ para algún $y$ tal que $f(y)=x$ en $\im f$ y $0$ en el complemento. Entonces la sucesión exacta corta se parte y así $M\simeq M' \oplus M''$. Esto quiere decir que $M'$ y $M''$ son submódulos de $M$ y, por el corolario anterior, semisimples.
\end{proof}
\end{cor}

\begin{prop}
Sean $A,B$ anillos. $A\times B$ es semisimple si y sólo si $A$ y $B$ lo son.
\begin{proof}
$(\Longrightarrow)$ Es trivial mirando a $A\hookrightarrow A\times B$ y $B\hookrightarrow A\times B$ como submódulos.

$(\Longleftarrow)$ Si $A=\displaystyle\sum_{i\in I}S_i$ y $B=\displaystyle\sum_{j\in J}T_j$ y $\pi_1,\pi_2$ las proyecciones canónicas, entonces se tiene que $A\times B = \displaystyle\sum_{i\in I}\pi_1^*(S_i) + \sum_{j\in J}\pi_2^*(T_j)$. Y listo.
\end{proof}
\end{prop}

\begin{lem}[Lema de Schur]
Sean $S$ y $S'$ dos $A$-módulos simples. Entonces: \begin{enumerate}\item Si $f:S\to M$ es un morfismo no trivial, entonces $f$ es inyectivo. \item Si $g:M\to S'$ es un morfismo no trivial, entonces $g$ es sobreyectivo. \item Todo morfismo no trivial $f:S\to S'$ es un isomorfismo.\end{enumerate} En particular, $\End_A(S)$ es un anillo de división.
\begin{proof}
Es trivial simplemente notando que $\ker$ e $\im$ son submódulos.
\end{proof}
\end{lem}

\begin{prop}
Sea $A$ un anillo. Todo ideal bilátero de $M_n(A)$ es de la forma $M_n(I)$ donde $I$ es un ideal bilátero de $A$.
\begin{proof}
Si $I$ es un ideal bilátero de $A$ es muy fácil chequear que $M_n(I)$ es un ideal bilátero de $M_n(A)$. Ahora, sea $J$ un ideal bilátero de $M_n(A)$. Si $M\in J$ entonces tenemos $E^{ij}ME^{k\ell} = (M)_{jk}E^{i\ell}\in J$. Tomando $i=\ell=1$ tenemos que $J\subseteq M_n(I)$ si $I$ es el conjunto de valores que toman las matrices de $J$ en el lugar $(1,1)$. Pero si $r\in I$ entonces $r=C_{11}$ para alguna $C\in J$. Entonces $rE^{ij}=E^{i1}CE^{1j}\in J$ para todos $i,j$. Luego si $M=(M)_{ij}\in M_n(I)$ se tiene que  $M=\displaystyle\sum_{i,j}(M)_{ij}E^{ij}\in J$. Y listo.
\end{proof}
\end{prop}

\begin{cor}
Si $A=D$ es un anillo de división, entonces $M_n(D)$ es simple.
\begin{proof}
En efecto, los ideales de $M_n(D)$ se corresponden con ideales de $D$, que no hay (más allá de los triviales) por ser anillo de división.
\end{proof}
\end{cor}

\begin{teo}[Artin-Wedderburn]
Sea $A$ un anillo artiniano. Las siguientes afirmaciones son equivalentes:
\begin{enumerate}
\item $A$ es semisimple.
\item Todo $A$-módulo es semisimple.
\item Existen $r,n_1,\ldots , n_r\in\mathbb{N}$ y anillos de división $D_1,\ldots , D_r$ tales que hay un isomorfismo de anillos $A\simeq M_{n_1}(D_1)\times \ldots \times M_{n_r}(D_r)$.
\end{enumerate}
\begin{proof}

$(1)\Longrightarrow(2)$ Como suma de semisimples es semisimple. todo $A$-módulo libre es semisimple pues $A$ lo es. Además, como todo módulo es cociente de un libre, existe $\pi:L\to M$ suryectiva y tenemos $0\longrightarrow \ker\pi\hookrightarrow L\stackrel{\pi}{\longrightarrow}M\longrightarrow 0$. Como $L$ es semisimple, $M$ lo es por uno de los corolarios de sucesiones exactas cortas de recién. Entonces todo $A$-módulo es semisimple.

$(2)\Longrightarrow (3)$ Como todo $A$-módulo es semisimple, en particular $A$ lo es. Luego $A=\displaystyle\bigoplus_{i\in I}S_i$ con $S_i\subseteq A$ submódulos (ideales) simples. Como $A$ es finitamente generado, $I$ es finito. Pero entonces $A\simeq \End_A(A) \simeq \End_A\left(\displaystyle\bigoplus_{i\in I}S_i\right)\simeq M_{n_1}(\End_A(S_1))\times \ldots \times M_{n_k}(\End_A(S_k))$. Pero vimos que $\End_A(S_i)=D_i$ son anillos de división, y así la implicación sigue.

$(3)\Longrightarrow (1)$ De forma similar a lo que hicimos en la proposición anterior, vemos que $M_n(D)$ es semisimple si $D$ es un anillo de división y que si $A$ y $B$ son semisimples entonces $A\times B$ lo es. Por inducción, se ve que $M_{n_1}(D_1)\times \ldots \times M_{n_k}(D_k)$ es semisimple. Y estamos.

\end{proof}
\end{teo}

\begin{cor}
Un anillo $A$ es semisimple si y sólo si el $A$-módulo a derecha $A$ es semisimple.
\begin{proof}
Es claro que la tercera condición es simétrica.
\end{proof}
\end{cor}

\begin{cor}
Un anillo semisimple es artiniano y noetheriano.
\begin{proof}
Como $M_n(D)$ es simple si $D$ es anillo de división, se tiene que $M_n(D)$ es artiniano y noetheriano. Como el producto directo finito de artinianos/noetherianos es artiniano/noetheriano, el corolario sigue.
\end{proof}
\end{cor}

\begin{prop}
Sea $A$ un anillo. Las siguientes afirmaciones son equivalentes:
\begin{enumerate}
\item $A$ es semisimple.
\item Toda sucesión exacta corta de $A$-módulos se parte.
\item Todo $A$-módulo es proyectivo.
\item Todo $A$-módulo es inyectivo.
\end{enumerate}
\begin{proof}
$(1)\Longleftrightarrow (2)$ Por Artin-Wedderburn, $A$ es semisimple si y sólo si todo $A$-módulo es semisimple y así cualquier sucesión exacta corta $0\longrightarrow M'\longrightarrow M\longrightarrow M''\longrightarrow 0$ al ser $M$ semisimple, se parte.

$(2)\Longleftrightarrow(3)$ y $(2)\Longleftrightarrow (4)$ son triviales.
\end{proof}
\end{prop}
\begin{obs}
Sea $k$ un anillo conmutativo y $A$ una $k$-álgebra. Si $A$ es semisimple en tanto anillo, entonces, por Artin-Wedderburn, $A\simeq M_{n_1}(D_1)\times\ldots\times M_{n_r}(D_r)$ como anillos. Pero mirando con cuidado la demostración, los $M_{n_i}(D_i)$ son $k$-álgebras y el isomorfismo es un isomorfismo de $k$-álgebras.
\end{obs}
\begin{defn}
Dado un grupo $G$ y $k$ un anillo de base, la $k$-álgebra de grupo $k[G]$ es el conjunto de combinaciones lineales finitas con coeficientes en $k$ de elementos de $G$ dotado de la suma y producto obvios.
\end{defn}

\begin{teo}[Maschke]
Sea $G$ un grupo finito y $k$ un cuerpo cuya característica no divide al orden de $G$. Entonces el álgebra $k[G]$ es semisimple.
\begin{proof}
Veamos que si $M$ es un $k[G]$-módulo y $N\subseteq M$ un submódulo, entonces $N$ posee un complemento en $M$. Sea $\iota:N\to M$ la inclusión y $s:M\to N$ un morfismo de $k$-espacios vectoriales tal que $s\circ \iota =1$ (existe pues los $k$-espacios vectoriales son proyectivos). Definamos $\tilde{s}:M\to N$ poniendo $\tilde{s}(m) = \dfrac{1}{|G|}\displaystyle\sum_{g\in G}g s(g^{-1}m)$ para todo $m\in M$ (Acá usamos que $|G|\in k^\times$). Afirmamos que $\tilde{s}$ es $k[G]$-lineal. Basta probar que $\tilde{s}(hm)=h\tilde{s}(m)$ para todos $m\in M, h\in G$ pues claramente $\tilde{s}$ es $k$-lineal y $G$ genera a $k[G]$ como $k$-álgebra. Pero si $m\in M$ y $h\in G$, entonces $$\tilde{s}(hm) = \dfrac{1}{|G|}\displaystyle\sum_{g\in G}gs(g^{-1}hm) = \dfrac{1}{|G|}\sum_{g\in G}hgs(g^{-1}m) = h\left(\dfrac{1}{|G|}\sum_{g\in G}gs(g^{-1}m)\right)=h\tilde{s}(m)$$ Para finalizar, veamos que $\tilde{s}\circ \iota =1$. Esto implicará que $N$ es sumando directo de $M$. Si $n\in N$, luego $(\tilde{s}\circ\iota)(n) = \dfrac{1}{|G|}\displaystyle\sum_{g\in G}gs(g^{-1}\iota(n)) = \dfrac{1}{|G|}\sum_{g\in G}g(s\circ\iota)(g^{-1}n)$ y como $s\circ \iota =1$ tenemos que $(\tilde{s}\circ \iota)(n) = \dfrac{1}{|G|}\displaystyle\sum_{g\in G}gg^{-1}n = n$. Y estamos.
\end{proof}
\end{teo}

\section{Introducción al Álgebra Homológica}

\begin{obs}
Sea $A$ un anillo y consideremos el siguiente diagrama conmutativo de $A$-módulos: \begin{center}\begin{tikzcd}[row sep=3.3em,column sep=4em,minimum width=2em]
M\arrow{d}[left,font=\normalsize]{d}\arrow{r}[font=\normalsize]{f} & M'\arrow{d}[right, font=\normalsize]{d'} \\
N\arrow{r}[font=\normalsize]{g} & N' 
\end{tikzcd}\end{center}

Notemos que $f(\ker d)\subseteq \ker d'$. En efecto, si $x\in \ker d$ entonces $(d'\circ f)(x) = (g\circ d)(x)$, y así $d'(f(x)) = g(d(x))=g(0)=0$. Luego, consideramos $\left. f\right|_{\ker d}:\ker d\to \ker d'$. Por abuso de notación, la vamos a seguir denotando $f$.

Notemos también que $g$ induce un morfismo $\overline{g}:N/\im d = \coker d\to N'/\im d' = \coker d'$ por $\overline{g}(\overline{y})=\overline{g(y)}$. Esta $\overline{g}$ está bien definida: en efecto, por la linealidad basta ver que si $\overline{y}=\overline{0}$ entonces $\overline{g(y)}=\overline{0}$. Como $\overline{y}=\overline{0}$, tenemos que $y=d(x)$, y  así $g(y)=g(d(x)) = d'(f(x))$. Entonces $\overline{g(y)}=\overline{0}$, como queríamos. Por el mismo abuso de notación de antes, a $\overline{g}$ la vamos a seguir llamando $g$.
\end{obs}

Probemos entonces un resultado básico que vamos a necesitar:
\newpage
\begin{lem}[Lema de la Serpiente]
Supongamos que tenemos el siguiente diagrama conmutativo de $A$-módulos donde las filas son exactas:
\begin{center}\begin{tikzcd}[row sep=3.3em,column sep=4em,minimum width=2em]
& M' \arrow{d}[left,font=\normalsize]{d'}\arrow{r}[font=\normalsize]{f} & M \arrow{d}[left, font=\normalsize]{d}\arrow{r}[font=\normalsize]{g} & M'' \arrow{d}[left,font=\normalsize]{d''}\arrow{r} & 0 \\
0\arrow{r}& N'\arrow{r}[font=\normalsize]{h} & N\arrow{r}[font=\normalsize]{k} & N'' & 
\end{tikzcd}\end{center}
Se tiene entonces la siguiente sucesion exacta: $$ \ker d'\stackrel{f}{\longrightarrow} \ker d \stackrel{g}{\longrightarrow} \ker d'' \stackrel{\partial}{\longrightarrow} \coker d' \stackrel{h}{\longrightarrow} \coker d \stackrel{k}{\longrightarrow} \coker d''$$
(A $\partial$ se lo llama \textbf{morfismo de conexión} o de \textbf{frontera})

\begin{proof}

Primero definamos el morfismo de frontera $\partial$. Tomemos un $m''\in \ker d''\subseteq M''$. Como $g$ es epimorfismo, existe algún $m\in M$ tal que $g(m)=m''$. Entonces $(k\circ d)(m) = (d''\circ g)(m) = d''(m'')=0$ y así $d(m)\in \ker k=\im h$. Entonces existe un $n'\in N'$ tal que $h(n')=d(m)$ y es único pues $h$ es monomorfismo. Defino entonces $\partial(m'') = \overline{n'}\in\coker d'$. Veamos que la definición de $\partial$ no depende del $m\in M$ elegido, es decir, que $\overline{n'}$ queda unívocamente determinado por $m''$. Sean $m_1,m_2\in M$ tales que $g(m_1)=g(m_2)=m''$. Entonces, existen únicos $n_1', n_2'\in N'$ tales que $h(n_1') = d(m_1)$ y $h(n_2') = d(m_2)$. Pero notemos que $m_1-m_2\in \ker g = \im f$ por la exactitud de la primera fila. Entonces, existe un $m'\in M'$ tal que $f(m')=m_1-m_2$. Por lo tanto, $h(n_1' - n_2') = d(m_1-m_2) = d(f(m')) = h(d'(m'))$. Como $h$ es monomorfismo, debemos tener que $n_1' - n_2' = d'(m')$ y así $\overline{n_1'}=\overline{n_2'}$.

Además, es claro que $\partial$ es morfismo de $A$-módulos por la linealidad de $g,d$ y $h$ y usando que $\partial(m'')$ no depende de la elección de $m$.

La exactitud de la sucesión en $\ker d$ y en $\coker d$ es obvia. Veamoslo en $\ker d''$ (para $\coker d'$ será análogo). Quiero ver entonces que $\im g = \ker \partial$. Veamos la doble inclusión. $(\im g\subseteq \ker \partial)$: Sea $m''\in \ker d''$ tal que $m''=g(m)$ con $m\in \ker d$. Entonces $d(m)=0$ y así el único $n'\in N'$ tal que $h(n')=d(m)=0$ es $n'=0$. Entonces $\partial(m'')=\overline{n'}=\overline{0}$. $(\im g\supseteq \ker\partial)$: Sea $m''\in\ker \partial\subseteq M''$. Quiero ver que $m''\in\im g$. O sea, que $\exists m\in\ker d$ tal que $g(m)=m''$. Sabemos que existe un $\overline{m}\in M$ tal que $g(\overline{m})=m''$ pues $g$ es epimorfismo. Entonces $d(\overline{m})=h(n')$ con $\partial(m'')=\overline{n'}=\overline{0}$. Entonces, $n'\in \im d'$ y así $n'=d'(m')$ para algún $m'\in M'$. Luego, $d(\overline{m}) = h(d'(m')) = d(f(m'))$, y así $d(\overline{m}-f(m'))=0$. Tomo entonces $m=\overline{m}-f(m')$. Notemos que $g(m)=g(\overline{m}) - (g\circ f)(m) = m''$ y además $m\in\ker d$. Y estamos.

\end{proof}
\end{lem}

\begin{defn}
Un conjunto $\{v_0,\ldots , v_n\}\subseteq \RR^k$ se dice \textbf{afinmente independiente} si $\{v_1-v_0,\ldots , v_n-v_0\}$ es linealmente independiente. Es decir, generan una variedad lineal de dimensión $n$.
\end{defn}

\begin{defn}
Sea $n\in\NN_0$. Un $n$-simplex es la cápsula convexa de $n+1$ puntos afinmente independientes en $\RR^{n+1}$. El $n$-simplex estándar $\Delta^n$ está dado por $$\Delta^n = \{(t_0,\ldots , t_n) \in \RR^{n+1} : \displaystyle\sum_{i=0}^n t_i = 1 \text{ y } t_i\geq 0 \;\forall 0\leq i\leq n\}$$ Entonces, si $\{v_0,\ldots ,v_n\}$ es un conjunto afinmente independiente, tenemos una transformación afín canónica $(t_0,\ldots ,t_n)\mapsto \displaystyle\sum_{i=1}^n t_i v_i$, que son las \textbf{coordenadas baricéntricas} de un punto del $n$-simplex generado por $\{v_0,\ldots , v_n\}$.
\end{defn}

\begin{defn}
Se dice que $\eta$ es cara de un simplex $\sigma$ si $\eta$ es un simplex generado por algunos vértices de $\sigma$. Lo denotamos $\eta \leq \sigma$.
\end{defn}

\begin{defn}
Un poliedro es un subconjunto $X\subseteq \RR^m$ que es unión de simplices que se pegan en las caras. Es decir, $X=\displaystyle\bigcup_{\sigma\in K}\sigma$ tal que $\forall\;\sigma,\tau\in K$ con $\sigma\cap\tau\leq \sigma,\tau$.
\end{defn}

La motivación para definir homología va a ser contar agujeros $n$-dimensionales en un poliedro. Primero intentemos ver cuándo un poliedro es arcoconexo. Basta ver que puedo unir $0$-simplices con caminos de $1$-simplices.

Sea $C_0$ el grupo abeliano libre generado por los vértices $[v_0]$ de $X$. Sea $C_1$ el grupo abeliano libre generado por las aristas orientadas $[v_0,v_1]$. Tenemos entonces un morfismo $d_1:C_1\to C_0$ definido en la base por $d_1([v_0,v_1]) = v_1-v_0$. Entonces, un camino entre dos vértices $v,w$ es tener $c=\displaystyle\sum_{\text{finita}} \sigma\in C_1$ tal que $d_1(c)=w-v$. Entonces, $X$ es arcoconexo si y sólo si $C_0/\im d_1 \simeq \ZZ$. Definimos entonces $H_0(X) = C_0/\im d_1$.

Ahora quiero contar agujeros de dimensión mayor. Un $1$-ciclo, o sea, un ciclo de aristas de dimensión $1$ es un elemento de $\ker d_1$. Un $1$-ciclo es un posible agujero de dimensión $1$: podría estar lleno como no estarlo. Sea ahora $C_2$ el grupo abeliano libre generado por los $2$-simplex orientados $[v_0,v_1,v_2]$. El borde de $[v_0,v_1,v_2]$ es $d_2([v_0,v_1,v_2])=[v_0,v_1]+[v_1,v_2]+[v_2,v_0]$. Notemos además que $d_1(d_2([v_0,v_1,v_2])) = 0$. Entonces $\im d_2\subseteq \ker d_1$. Entonces, para separar los agujeros de verdad de los que simplemente son ciclos, miramos $H_1(X) = \ker d_1/\im d_2$.

Inductivamente, podemos ver que tenemos $\ldots\longrightarrow C_n\stackrel{d_n}{\longrightarrow}\ldots \stackrel{d_2}{\longrightarrow}C_1\stackrel{d_1}{\longrightarrow}C_0$ con $C_n$ el grupo abeliano libre generado por los $n$-simplices orientados $[v_0,v_1,\ldots , v_n]$ y $d_n:C_n\to C_{n-1}$ definida por $d_n([v_0,\ldots , v_n])=\displaystyle\sum_{k=0}^n (-1)^k [v_0,\ldots , \hat{v_k},\ldots , v_n]$ el borde del $n$-simplex. Entonces, $H_n(X) = \ker d_n/\im d_{n+1}$ es el cociente entre los $n$-ciclos y los $n$-bordes.

Esto motiva las siguientes definiciones:

\begin{defn}
Sea $A$ un anillo. Un \textbf{complejo de cadenas} $(C_*,d)$ es una familia de $A$-módulos $\{C_n\}_{n\in\NN_0}$ y morfismos de $A$-módulos $d_n:C_n\to C_{n-1}$ que se denominan morfismos de bordes o diferenciales tales que $\im d_{n+1}\subseteq \ker d_n$. Por abuso de notación, se suele escribir $d_n=d$ y así la condición como $d^2 = 0$.
\end{defn}

\begin{defn}
Sea $(C_*,d)$ un complejo de cadenas de $A$-módulos. Se define la $n$-ésima \textbf{homología} del complejo $C$ como $H_n(C) = \ker d_n/\im d_{n+1}$. Notar que es un $A$-módulo.
\end{defn}

\begin{defn}
Un complejo de cadenas $(C_*,d)$ se dice \textbf{acíclico} si $H_n(C)=0$ para todo $n\in\NN_0$. O sea, si $(C_*,d)$ es una sucesión exacta.
\end{defn}

\begin{ex}
Supongamos $A=\ZZ$. Entonces, $0\longrightarrow \ZZ\stackrel{\times 4}{\longrightarrow} \ZZ\stackrel{0}{\longrightarrow}\ZZ\stackrel{\times 2}{\longrightarrow} \ZZ\longrightarrow 0$ es un complejo de cadenas. Se puede calcular fácilmente su homología: $H_0(C)=\ZZ_2$, $H_1(C)=0$, $H_2(C)=\ZZ_4$ y $H_3(C)=0$.

Ahora, si $A$ es un anillo cualquiera, $\ldots\longrightarrow A\stackrel{\id}{\longrightarrow} A\stackrel{0}{\longrightarrow} A\stackrel{\id}{\longrightarrow} A\longrightarrow 0$. Es decir, el complejo consiste en $A$ y alternando el morfismo identidad con el trivial. Se puede ver que es un complejo acíclico, es decir $H_n(C)=0$ para todo $n\in\NN_0$.
\end{ex}

\begin{defn}
Sean $(C_*,d)$ y $(C_*',d')$ dos complejos de cadenas de $A$-módulos. Un \textbf{morfismo de complejos} $\varphi:C_*\to C_*'$ es una familia de morfismos de $A$-módulos $\varphi_n:C_n\to C_n'$ tal que el siguiente diagrama conmuta para todo $n\in\NN$:
\begin{center}\begin{tikzcd}[row sep=3.3em,column sep=4em,minimum width=2em]
C_n\arrow{d}[left,font=\normalsize]{\varphi_n}\arrow{r}[font=\normalsize]{d_n} & C_{n-1}\arrow{d}[right, font=\normalsize]{\varphi_{n-1}} \\
C_n'\arrow{r}[font=\normalsize]{d_n'} & C_{n-1}' 
\end{tikzcd}\end{center} Por abuso de notación, esta condición se suele denotar $d'\circ \varphi = \varphi\circ d$.
\end{defn}

\begin{prop}
Sea $\varphi:C_*\to C_*'$ un morfismo de complejos. Entonces, $\varphi$ induce un morfismo entre las homologías, que por abuso de notación, denotamos $\varphi_n:H_n(C)\to H_n(C')$.
\begin{proof}
Se define $\varphi_n(\overline{x}) = \overline{\varphi(x)}$. En efecto, veamos la buena definición. Si $\overline{x}=\overline{y}$, quiere decir que $x-y\in \im d_{n+1}$ y así existe un $z$ tal que $x-y = d_{n+1}(z)$. Luego, vemos que $\varphi_n(x)-\varphi_n(y)=\varphi_n(d_{n+1}(z)) = d_{n+1}'(\varphi_n(z))$ y así $\overline{\varphi_n(x)} = \overline{\varphi_n(y)}$ en $H_n(C)'$. Y listo.
\end{proof}
\end{prop}

\begin{defn}
Sea $(C_*,d)$ un complejo de $A$-módulos. Decimos que $(C_*',d')$ es un \textbf{subcomplejo} si $C_n'\subseteq C_n$ es un $A$-submódulo y $d_n' = \left. d_n\right|_{C_n'}:C_n'\to C_{n-1}'$ para cada $n\in\NN$.
\end{defn}

\begin{obs}
La inclusión $\mathrm{inc}:(C',d')\to (C,d)$ es un morfismo de complejos de cadena:
\begin{center}\begin{tikzcd}[row sep=3.3em,column sep=4em,minimum width=2em]
\ldots\arrow{r}&C_n'\arrow{d}[left,font=\normalsize]{\mathrm{inc}_n}\arrow{r}[font=\normalsize]{d_n'} & C_{n-1}'\arrow{d}[right, font=\normalsize]{\mathrm{inc}_{n-1}}\arrow{r}&\ldots \\
\ldots\arrow{r}&C_n\arrow{r}[font=\normalsize]{d_n} & C_{n-1}\arrow{r}&\ldots 
\end{tikzcd}\end{center}
\end{obs}

\begin{defn}
Sean $(C_*,d)$ y $(C_*',d')$ dos complejos de cadenas. Definimos la suma directa de el complejo como el complejo $(C\oplus C',d\oplus d')$: $$\ldots\longrightarrow C_n\oplus C_n'\stackrel{d_n\oplus d_n'}{\longrightarrow} C_{n-1}\oplus C_{n-1}' \longrightarrow\ldots $$
\end{defn}

\begin{prop}
Si $(C_*,d)$ y $(C_*',d')$ son dos complejos de cadenas, tenemos que $H_n(C\oplus C') \simeq H_n(C)\oplus H_n(C')$.
\begin{proof}
Consideremos el morfismo $f: \ker d_n\oplus d_n'\to H_n(C)\oplus H_n(C')$ dado por $f(x,y)=(\overline{x},\overline{y})$. La buena definición está porque $x\in \ker d_n$ e $y\in\ker d_n'$. Es claro que $\ker f = \{(x,y) : x\in\im d_{n+1}, y\in \im d_{n+1}'\} = \im d_{n+1}\oplus d_{n+1}'$. Por el primer teorema de isomorfismo sigue lo deseado.
\end{proof}
\end{prop}

\begin{defn}
Sea $(C_*',d')\subseteq (C_*,d)$ un subcomplejo. Se define el cociente $(C_*/C_*' , d)$ como el complejo $$ \ldots \longrightarrow C_n/C_n' \stackrel{\overline{d}_n}{\longrightarrow} C_{n-1}/C_{n-1}'\longrightarrow \ldots$$ con $\overline{d}_n(\overline{x}) = \overline{d_n(x)}$ (Queda como ejercicio verificar la buena definición de $\overline{d}_n$).
\end{defn}

\begin{defn}
Una \textbf{sucesión exacta corta de complejos} es un diagrama de complejos $0\longrightarrow C_*'\stackrel{f}{\longrightarrow}C_*\stackrel{g}{\longrightarrow}C_*''\longrightarrow 0$ tal que $0\longrightarrow C_n'\stackrel{f_n}{\longrightarrow} C_n\stackrel{g_n}{\longrightarrow} C_n''\longrightarrow 0$ es una sucesión exacta corta de $A$-módulos para todo $n\in\NN_0$.
\end{defn}

\begin{teo}
Toda sucesión exacta corta de complejos $0\longrightarrow C_*'\stackrel{f}{\longrightarrow}C_*\stackrel{g}{\longrightarrow} C_*''\longrightarrow 0$ induce una sucesión exacta larga en $H_*$: $$\ldots \stackrel{\partial}{\longrightarrow} H_n(C')\stackrel{f}{\longrightarrow}H_n(C)\stackrel{g}{\longrightarrow} H_n(C'') \stackrel{\partial}{\longrightarrow} H_{n-1}(C')\stackrel{f}{\longrightarrow} H_{n-1}(C)\stackrel{g}{\longrightarrow} H_{n-1}(C'')\stackrel{\partial}{\longrightarrow}\ldots $$
\begin{proof}
Simplemente es aplicar el Lema de la Serpiente en el siguiente diagrama:

\begin{center}\begin{tikzcd}[row sep=2.8em,column sep=2em,minimum width=2em]
& H_n(C')\arrow{r}[font=\normalsize]{f}\arrow[dashed,black!30]{d} & H_n(C)\arrow{r}[font=\normalsize]{g}\arrow[dashed,black!30]{d} & H_n(C'')\arrow[dashed,black!30]{d} \ar[out=0, in=180, dashed, red]{dddll}[font=\normalsize]{\partial} \\
& C_n'/\im d_{n+1}' \arrow{d}[left,font=\normalsize]{d_n'}\arrow{r}[font=\normalsize]{f} & C_n/\im d_{n+1} \arrow{d}[left, font=\normalsize]{d_n}\arrow{r}[font=\normalsize]{g} & C_n''/\im d_{n+1}'' \arrow{d}[left,font=\normalsize]{d_n''}\arrow{r} & 0 \\
0\arrow{r}& \ker d_{n-1}'\arrow{r}[font=\normalsize]{f}\arrow[dashed,black!30]{d} & \ker d_{n-1}\arrow{r}[font=\normalsize]{g}\arrow[dashed,black!30]{d} & \ker d_{n-1}''\arrow[dashed,black!30]{d} & \\
& H_{n-1}(C')\arrow{r}[font=\normalsize]{f} & H_{n-1}(C)\arrow{r}[font=\normalsize]{g} & H_{n-1}(C'')
\end{tikzcd}\end{center}

\end{proof}
\end{teo}

\begin{defn}
Sean $(C_*,d)$ y $(C_*',d')$ dos complejos de $A$-módulos y $f,g:C_*\to C_*'$ morfismos de complejos. Una \textbf{homotopía} $H:f\Longrightarrow g$ es una familia $\{H_n\}_{n\in\NN_0}$ de morfismos $H_n:C_n\to C_{n+1}'$ para todo $n$ de modo tal que $d_{n+1}'\circ H_n + H_{n-1}\circ d_n = f_n-g_n$. Decimos que dos morfismos de complejos son homotópicos si existe $H:f\Longrightarrow g$ homotopía, y lo notamos $f\simeq g$.
\end{defn}

\begin{prop}
"`Ser homotópico a"' es una relación de equivalencia.
\begin{proof}
La reflexividad es obvia tomando $H=0$. Para la simetría, si tengo $f\simeq g$, es decir, $H:f\Longrightarrow g$, me construyo $-H:g\Longrightarrow f$. Finalmente, para la transitividad, si $H_1:f\Longrightarrow g$ y $H_2:g\Longrightarrow r$ tomo $H=H_1+H_2:f\Longrightarrow r$ y listo.
\end{proof}
\end{prop}

\begin{prop}
Si $f,g:C_*\to C_*'$ son dos morfismos de cadena y son homotópicos, entonces a nivel homología inducen los mismos morfismos. Es decir, si $f\simeq g$ tenemos que $f_* = g_* : H_*(C)\to H_*(C')$.

\begin{proof}

Notemos que $f_*:H_n(C)\to H_n(C')$ está definido como $f(\overline{x})=\overline{f(x)}$ donde $\overline{x}\in\ker d_n/\im d_{n+1}$. Debo ver que $f_*(\overline{x})=g_*(\overline{x})$. Esto pasa si y sólo si $f_*(\overline{x})-g_*(\overline{x})=\overline{0}$, que a su vez, pasa si y sólo si $\overline{f(x)-g(x)} = \overline{0}$, si y sólo si $\overline{d'\circ H(x) + H\circ d(x)}=\overline{0}$. Pero notemos que $\overline{H\circ d(x)}=\overline{0}$ pues $x\in \ker d$ y $\overline{d'(H(x))}=\overline{0}$ pues está en la imagen de $d'$ por definición. Y listo.

\end{proof}
\end{prop}

\begin{defn}
Un morfismo de complejos $f:C_*\to C_*'$ se dice una \textbf{equivalencia homotópica} si existe un morfismo de complejos $g:C_*'\to C_*$ tal que $f\circ g\simeq 1_{C'}$ y $g\circ f\simeq 1_C$.
\end{defn}

\begin{obs}
Si $f:C_*\to C_*'$ es una equivalencia homotópica, entonces inducen $f_n:H_n(C)\to H_n(C')$ es isomorfismo para todo $n$ con inversa $g$ (a nivel homología) si $g$ era el que $f\circ g\simeq 1_{C'}$ y $g\circ f\simeq 1_C$. En efecto, esto sigue de la definición de equivalencia homotópica y la proposición anterior.
\end{obs}

\begin{defn}
Un morfismo de complejos $f:C_*\to C_*'$ se dice \textbf{quasi-isomorfismo} si $f_*:H_n(C)\to H_n(C')$ es un isomorfismo para todo $n$.
\end{defn}

\begin{obs}
La observación anterior nos dice que equivalencia homotópica implica quasi-isomorfismo. Veamos con un ejemplo que la vuelta es falsa.
\end{obs}

\begin{ex}
Un quasi-isomorfismo que no es equivalencia homotópica es el siguiente morfismo de complejos de $\ZZ$-módulos:

\begin{center}\begin{tikzcd}[row sep=3.3em,column sep=4em,minimum width=2em]
0 \arrow{r}&\ZZ_2 \arrow{d}[left,font=\normalsize]{0}\arrow{r}[font=\normalsize]{\times 2} &\ZZ_4 \arrow{d}[right, font=\normalsize]{0}\arrow{r}[font=\normalsize]{\pi}& \ZZ_2\arrow{r}\arrow{d}[right,font=\normalsize]{0} & 0 \\
0\arrow{r}&\ZZ \arrow{r}[font=\normalsize]{1} & \ZZ\arrow{r}&0 \arrow{r}& 0
\end{tikzcd}\end{center}

El morfismo $0$ entre esos dos complejos es un quasi-isomorfismo pues ambos complejos tienen homología $0$ (es decir, son sucesiones exactas cortas). Veamos que no es una equivalencia homotópica. Si lo fuera, existiría $g:C_*' \to C_*$ tal que $0\circ g\simeq 1_{C'}$ y $g\circ 0\simeq 1_C$. O sea, $1_C\simeq 0$. Pero esto implicaría que existe un $H:C_*\to C_*$ de modo tal que $\pi\circ H = 1$. O sea que $\pi$ debería ser retracción. Absurdo pues la sucesión exacta corta no se parte.

\end{ex}

\begin{defn}
Sea $M$ un $A$-módulo. Una \textbf{resolución} de $M$ es un complejo de cadenas $(C_*,d)$ y un morfismo $\varepsilon:C_0\to M$ que se denomina \textbf{morfismo de aumentación} tal que $\ldots\longrightarrow C_2\stackrel{d_2}{\longrightarrow}C_1\stackrel{d_1}{\longrightarrow} C_0\stackrel{\varepsilon}{\longrightarrow} M\longrightarrow 0$ es exacta. Lo notamos $C_*\stackrel{\varepsilon}{\longrightarrow}M$.

Una resolución $C_*\stackrel{\varepsilon}{\longrightarrow}M$ se dice \textbf{libre} (\textit{resp}. \textbf{proyectivo}) si cada $C_n$ es \textbf{libre} (\textit{resp}. \textbf{proyectivo}).
\end{defn}

\begin{prop}
Si $M$ es un $A$-módulo entonces siempre existe una resolución libre $L_*\stackrel{\varepsilon}{\longrightarrow} M$.
\begin{proof}

Sabemos que todo módulo es cociente de un libre, entonces existe un $L_0$ y un morfismo $\pi_0$ tal que $L_0\stackrel{\pi_0}{\longrightarrow} M$ es sobreyectivo. Miro $\ker \pi_0$ y también es cociente de un libre entonces existe $L_1$ libre y $\pi_1$ tal que $L_1\stackrel{\pi_1}{\longrightarrow} \ker \pi_0$ es sobreyectivo. Es fácil ver que $L_1 \stackrel{\iota\circ \pi_1}{\longrightarrow} L_0\stackrel{\pi_0}{\longrightarrow} M\longrightarrow 0$ es exacto. Ahora miramos $\ker\iota\circ\pi_1$ que es cociente de un libre $L_2$ y así seguimos inductivamente.

\end{proof}
\end{prop}

\begin{prop}
Sea $C_*\stackrel{\varepsilon}{\longrightarrow}M$ una resolución proyectiva de $M$ y $C_*'\stackrel{\varepsilon'}{\longrightarrow} M'$ un morfismo de complejos que extiende a $f:M\to M'$. Es decir, el siguiente diagrama conmuta:

\begin{center}\begin{tikzcd}[row sep=3.3em,column sep=4em,minimum width=2em]
\ldots \arrow{r} &C_2\arrow{d}[left,font=\normalsize]{f_2} \arrow{r}[font=\normalsize]{d_2}& C_1 \arrow{d}[left,font=\normalsize]{f_1}\arrow{r}[font=\normalsize]{d_1} & C_0 \arrow{d}[right, font=\normalsize]{f_0}\arrow{r}[font=\normalsize]{\varepsilon}& M\arrow{r}\arrow{d}[right,font=\normalsize]{f} & 0 \\

\ldots \arrow{r} & C_2' \arrow{r}[font=\normalsize]{d_2'}& C_1' \arrow{r}[font=\normalsize]{d_1'} & C_0'\arrow{r}[font=\normalsize]{\varepsilon'}& M' \arrow{r}& 0
\end{tikzcd}\end{center}

Más aún, la extensión es única salvo homotopía. Es decir, si $g:C_*\to C_*'$ es otra extensión de $f:M\to M'$ entonces $f\simeq g$.

\begin{proof}

Notemos que existe un $f_0:C_0\to C_0'$ tal que $\varepsilon'\circ f_0 = f\circ\varepsilon$. En efecto, esto sigue pues $C_0$ es proyectivo y queremos un $f_0$ que levante a $f\circ\varepsilon$ por $\varepsilon'$.

Ahora, notemos que $\im f_0\circ d_1 \subseteq \ker \varepsilon'$ pues $\varepsilon'\circ f_0\circ d_1 = f\circ\varepsilon\circ d_1 = 0$. Pero por la exactitud de la fila de abajo, tenemos $\ker \varepsilon' = \im d_1'$, entonces $\im f_0\circ d_1 \subseteq \im d_1'$. Vamos a volver a usar un razonamiento como el anterior: miramos la correstricción a la imagen $C_1'\stackrel{d_1'}{\longrightarrow}\im d_1'$, que es un epimorfismo. Además, podemos correstringir a $f_0\circ d_1$ a $\im d_1'$ también, pues su imagen queda contenida ahí. Como $C_1$ es proyectivo, existe $f_1:C_1\to C_1'$ que levanta a $f_0\circ d_1$ por $d_1'$. Es decir, el siguiente diagrama conmuta:

\begin{center}
\begin{tikzcd}[row sep=3.3em,column sep=4em,minimum width=2em]
 & C_1\arrow{d}[right, font=\normalsize]{f_0\circ d_1}\arrow[dashed]{dl}[left, font=\normalsize]{f_1} \\
C_1'\arrow{r}[font=\normalsize]{d_1'} & \im d_1'\arrow{r} & 0
\end{tikzcd}
\end{center}

De la misma forma que definimos $f_1$, suponiendo definido a $f_n$ se puede definir $f_{n+1}$ haciendo el mismo razonamiento.

Veamos ahora que el morfismo de complejos $f$ que se induce es único a menos de homotopía. En efecto, supongamos que tenemos dos extensiones $f,g:C_*\to C_*'$, construyamos una homotopía entre estos dos morfismos. Considero $g_0-f_0:C_0\to C_0'$. Entonces, $\varepsilon'\circ (g_0-f_0) = \varepsilon'\circ g_0 - \varepsilon'\circ f_0 = f\circ\varepsilon - f\circ \varepsilon = 0$ pues $f_0$ y $g_0$ son extensiones de $f$. Por lo tanto, $\im (g_0-f_0) \subseteq \ker \varepsilon' = \im d_1'$. Nuevamente vamos a usar que $C_0$ es proyectivo: existe un $H_0:C_0\to C_1'$ que hace conmutar el siguiente diagrama:

\begin{center}
\begin{tikzcd}[row sep=3.3em,column sep=4em,minimum width=2em]
 & C_0\arrow{d}[right, font=\normalsize]{g_0 - f_0}\arrow[dashed]{dl}[left, font=\normalsize]{H_0} \\
C_1'\arrow{r}[font=\normalsize]{d_1'} & \im d_1'\arrow{r} & 0
\end{tikzcd}
\end{center}

Ahora, supongamos construídos $H_{n-1}$ y $H_{n-2}$. Buscamos un morfismo $H_n:C_n\to C_{n+1}'$ tal que $d_{n+1}'\circ H_n + H_{n-1}\circ d_n = g_n-f_n$. Pero notemos que $\im (g_n - f_n - H_{n-1}\circ d_n) \subseteq \im d_{n+1}'$. En efecto, $d_n'(g_n - f_n - H_{n-1}\circ d_n) = d_n'\circ (g_n-f_n) - d_n'\circ H_{n-1}\circ d_n = (g_{n-1}-f_{n-1})\circ d_n - d_n'\circ H_{n-1}\circ d_n$. Pero $g_{n-1} - f_{n-1} = d_n' \circ H_{n-1} + H_{n-2}\circ d_{n-1}$. Reemplazando eso en la identidad anterior, obtenemos $0$. Es decir, $\im g_n-f_n-H_{n-1}\circ d_n\subseteq \ker d_n'=\im d_{n+1}'$, como queríamos. Entonces, usando que $C_n$ es proyectivo, tenemos un $H_n$ que hace conmutar el siguiente diagrama:

\begin{center}
\begin{tikzcd}[row sep=3.3em,column sep=4em,minimum width=2em]
 & C_n\arrow{d}[right, font=\normalsize]{g_n - f_n - H_{n-1}\circ d_n}\arrow[dashed]{dl}[left, font=\normalsize]{H_n} \\
C_{n+1}'\arrow{r}[font=\normalsize]{d_{n+1}'} & \im d_{n+1}'\arrow{r} & 0
\end{tikzcd}
\end{center}
Y estamos.
\end{proof}
\end{prop}

\begin{cor}
Sean $P_*\stackrel{\varepsilon}{\longrightarrow} M$ y $P_*'\stackrel{\varepsilon'}{\longrightarrow} M$ dos resoluciones proyectivas de $M$. Entonces, $P_*$ y $P_*'$ son homotópicamente equivalentes.
\begin{proof}

Tomemos $1_M:M\to M$ la identidad. Entonces, existe $f:P_*\to P_*'$ que extiende a $1_M$ pues $P_*$ es una resolución proyectiva. Pero también existe $g:P_*'\to P_*$ que extiende a la identidad, pues $P_*'$ también es proyectivo. Consideremos entonces el siguiente diagrama:

\begin{center}\begin{tikzcd}[row sep=3.3em,column sep=4em,minimum width=2em]
\ldots \arrow{r} &P_2\arrow{d}[left,font=\normalsize]{g_2\circ f_2} \arrow{r}[font=\normalsize]{d_2}& P_1 \arrow{d}[left,font=\normalsize]{g_1\circ f_1}\arrow{r}[font=\normalsize]{d_1} & P_0 \arrow{d}[left, font=\normalsize]{g_0\circ f_0}\arrow{r}[font=\normalsize]{\varepsilon}& M\arrow{r}\arrow[equals]{d} & 0 \\

\ldots \arrow{r} & P_2 \arrow{r}[font=\normalsize]{d_2}& P_1 \arrow{r}[font=\normalsize]{d_1} & P_0\arrow{r}[font=\normalsize]{\varepsilon}& M \arrow{r}& 0
\end{tikzcd}\end{center}

Como la identidad $1_M$ también extiende al diagrama, por la proposición anterior, tenemos que $g\circ f \simeq 1$. Pero de forma análoga, tenemos que $f\circ g\simeq 1$, y así $P_*$ y $P_*'$ son homotópicamente equivalentes. Como queríamos.

\end{proof}
\end{cor}

\begin{defn}
Sea $A$ un anillo, $M$ un $A$-módulo a derecha y $N$ un $A$-módulo a izquierda. Sea $P_*\stackrel{\varepsilon}{\longrightarrow} M$ una resolución proyectiva de $M$ (como $A$-módulos a derecha). Consideramos el complejo $(P_*\otimes_A N,d\otimes 1_N)$, es decir, $$ \ldots \stackrel{d_3\otimes 1}{\longrightarrow} P_2\otimes_A N\stackrel{d_2\otimes 1}{\longrightarrow} P_1\otimes_A N\stackrel{d_1\otimes 1}{\longrightarrow} P_0\otimes_A N\longrightarrow 0$$ Definimos $\mathrm{Tor}_n^A(M,N) = H_n(P_*\otimes_A N)$.
\end{defn}

\begin{obs}
Notemos que esto está bien definido, es decir, no depende de la resolución proyectiva que tomemos. En efecto, sabemos que si $P_*'\stackrel{\varepsilon'}{\longrightarrow} M$ es otra resolución proyectiva existe una equivalencia homotópica $f:P_*\to P_*'$. Luego, $f\otimes 1:P_*\otimes_A N\to P_*'\otimes_A N$ es una equivalencia homotópica (en efecto, es tomarse la misma "`inversa homotópica"' para $f$ y tensorizarla contra la identidad). Entonces, como $P_*\otimes_A N$ y $P_*'\otimes_A N$ son homotópicamente equivalentes inducen la misma homología, y así el $\mathrm{Tor}_n^A$ está bien definido (a menos de isomorfismo).
\end{obs}

\begin{ex}
Sea $M$ un $A$-módulo proyectivo. Entonces $\mathrm{Tor}_n^A(M,N)=0$ para todo $n\geq 1$. En efecto, tomo la resolución proyectiva $\ldots\longrightarrow 0\longrightarrow M\stackrel{1}{\longrightarrow} M \longrightarrow 0$, y así le queremos calcular la homología al complejo $\ldots\longrightarrow 0\longrightarrow 0\longrightarrow M\otimes_A N\longrightarrow 0$. Entonces, $\mathrm{Tor}_n^A(M,N)=0 \; \forall n\geq 1$.

Notemos que en general, tenemos que $\mathrm{Tor}_0^A(M,N)\simeq M\otimes_A N$. Sea $P_*\stackrel{\varepsilon}{\longrightarrow} M$ una resolución proyectiva de $M$. Entonces, $P_1\stackrel{d_1}{\longrightarrow} P_0\stackrel{\varepsilon}{\longrightarrow} M\longrightarrow 0$ es una sucesión exacta, y como que tensorizar es exacto a derecha, $P_1\otimes_A N\stackrel{d_1\otimes 1}{\longrightarrow} P_0\otimes_A N\stackrel{\varepsilon\otimes 1}{\longrightarrow} M\otimes_A N\longrightarrow 0$ es exacto. Pero esto por el Primer Teorema de Isomorfismo $M\otimes_A N\simeq P_0\otimes_A N/\im d_1\otimes 1$, y esto es precisamente $H_0(P_*\otimes_A N)$.

El $\mathrm{Tor}$ se lleva bien con las sumas directas. En efecto, \begin{align*}\mathrm{Tor}_n^A\left(M,\bigoplus_{i\in I}N_i\right)&\simeq H_n\left(P_*\otimes\left(\bigoplus_{i\in I}N_i\right)\right)\simeq H_n\left( \bigoplus_{i\in I}P_*\otimes N_i\right) \simeq \bigoplus_{i\in I}H_n(P_*\otimes_A N_i) \\ &\simeq \bigoplus_{i\in I}\mathrm{Tor}_n^A (M,N_i)\end{align*}

En el caso en que el anillo sea $A=\ZZ$, simplemente notamos $\mathrm{Tor}_n(M,N) = \mathrm{Tor}_n^\ZZ(M,N)$.
\end{ex}

\begin{prop}
Sean $M,N$ dos $\ZZ$-módulos. Entonces $\mathrm{Tor}_n(M,N)=0$ para todo $n\geq 2$.
\begin{proof}
Como $\ZZ$ es un dominio de ideales principales, tenemos una resolución libre corta: $0\longrightarrow L_1\stackrel{d_1}{\longrightarrow}L_0\stackrel{\varepsilon}{\longrightarrow}M\longrightarrow 0$ (en efecto, $M$ es cociente de un libre $L_0$ y $L_1$ es su núcleo, que es libre por ser submódulo de libre en un dominio de ideales principales). Entonces, le tenemos que mirar la homología a $0\longrightarrow L_1\otimes N\stackrel{d_1\otimes 1}{L_0\otimes N}\longrightarrow 0$, de donde se ve claro que $\mathrm{Tor}_n(M,N)=0$ para todo $n\geq 2$.
\end{proof}
\end{prop}

\begin{obs}
La proposición anterior nos dice que el único caso interesante del $\mathrm{Tor}$ sobre $\ZZ$-módulos es el $\mathrm{Tor}_1$, pues en efecto ya sabemos que $\mathrm{Tor}_0(M,N)=M\otimes N$ y los otros son $0$. Pero si miramos la homología del complejo que nos quedaba en la demostración, obtenemos que $\mathrm{Tor}_1(M,N) = \ker d_1\otimes 1$. Entonces, en cierta forma, el $\mathrm{Tor}_1$ nos mide cuánto falla la exactitud de tensorizar, pues tenemos la sucesión exacta $$0\longrightarrow \mathrm{Tor}_1(M,N) \hookrightarrow L_1\otimes N\stackrel{d_1\otimes 1}{\longrightarrow} L_0\otimes N\stackrel{\varepsilon\otimes 1}{\longrightarrow} M\otimes N\longrightarrow 0$$
\end{obs}

\begin{obs}
Si $N$ es un $\ZZ$-módulo playo, entonces $\mathrm{Tor}_1(M,N)=0$ pues tensorizar contra $N$ es exacto.

Ahora, supongamos que queremos calcular $\mathrm{Tor}_1(\ZZ_n,N)$. En efecto, tomando la resolución $0\longrightarrow \ZZ\stackrel{\times n}{\longrightarrow}\ZZ\stackrel{\pi}{\longrightarrow} \ZZ_n\longrightarrow 0$, miramos $0\longrightarrow \ZZ\otimes N\stackrel{(\times n)\otimes 1}{\longrightarrow} \ZZ\otimes N\longrightarrow 0$. Como el isomorfismo entre $\ZZ\otimes N$ y $N$ es natural, se sigue fácil que $\mathrm{Tor}_1(\ZZ_n,N) = \ker(N\stackrel{\times n}{\longrightarrow} N)$. En particular, $\mathrm{Tor}_1(\ZZ_n,\ZZ_m)=\ZZ_{(n:m)}$.

Entonces, usando el Teorema de Estructura y que el $\mathrm{Tor}$ se comporta bien con las sumas directas, puedo calcular el $\mathrm{Tor}$ contra cualquier grupo abeliano libre de tipo finito.
\end{obs}

\begin{lem}[Lema de la Herradura]
Supongamos que tenemos una sucesión exacta corta de $A$-módulos $0\longrightarrow M'\stackrel{f}{\longrightarrow}M\stackrel{g}{\longrightarrow}M''\longrightarrow 0$. Si $P_*'\stackrel{\varepsilon'}{\longrightarrow}M'$ y $P_*''\stackrel{\varepsilon''}{\longrightarrow}M''$ son dos resoluciones proyectivas, entonces existe una resolución proyectiva $P_*\stackrel{\varepsilon}{\longrightarrow}M$ de $M$ y morfismos de complejos $f:P_*''\to P_*$, $g:P_*\to P_*''$ que extienden a $f$ y $g$. Es decir, el siguiente diagrama conmuta:  

\begin{center}\begin{tikzcd}[row sep=3.3em,column sep=4em,minimum width=2em]
  & 0 \arrow{d} & 0 \arrow{d} & 0\arrow{d} &  \\

\ldots \arrow{r} & P_1' \arrow{d}[left,font=\normalsize]{f_1}\arrow{r}[font=\normalsize]{d_1'} & P_0' \arrow{d}[right, font=\normalsize]{f_0}\arrow{r}[font=\normalsize]{\varepsilon'}& M'\arrow{r}\arrow{d}[right,font=\normalsize]{f} & 0 \\

\ldots \arrow{r} & P_1 \arrow{d}[left,font=\normalsize]{g_1}\arrow{r}[font=\normalsize]{d_1} & P_0 \arrow{d}[right, font=\normalsize]{g_0}\arrow{r}[font=\normalsize]{\varepsilon}& M\arrow{r}\arrow{d}[right,font=\normalsize]{g} & 0 \\

\ldots \arrow{r} & P_1'' \arrow{d}\arrow{r}[font=\normalsize]{d_1''} & P_0'' \arrow{d}\arrow{r}[font=\normalsize]{\varepsilon''}& M''\arrow{r}\arrow{d} & 0 \\


 & 0 & 0 & 0  & 
\end{tikzcd}\end{center}
\begin{proof}

Vamos a tomar $P_n = P_n'\oplus P_n''$ para todo $n\geq 0$, cada $f_n$ va a ser la inclusión a la primer coordenada y cada $g_n$ la inclusión a la segunda coordenada. Vamos a definir $\varepsilon:P_0\to M$ y después definir los $d_n$ se hará de manera similar y por inducción.

Como $P_0''$ es proyectivo y $g:M\to M''$ epimorfismo, tenemos un morfismo $\alpha:P_0''\to M$ tal que $g\circ \alpha = \varepsilon''$. Si $\pi_1:P_0'\oplus P_0''$ es la proyección canónica a la primer coordenada, definimos $\varepsilon = \alpha\circ g_0 + f\circ \varepsilon'\circ \pi_1$.

La conmutatividad de los dos cuadrados es obvia: \begin{align*} g\circ\varepsilon &= g\circ(\alpha\circ g_0 + f\circ\varepsilon'\circ\pi_1) = g\circ\alpha\circ g_0 + (g\circ f)\circ \varepsilon'\circ \pi_1 = \varepsilon''\circ g_0 \\ \varepsilon \circ f_0 &= (\alpha\circ g_0 + f\circ \varepsilon'\circ\pi_1)\circ f_0 = \alpha\circ(g_0\circ f_0) + f\circ\varepsilon' \circ (\pi_1\circ f_0) = f\circ \varepsilon'\end{align*}

Para ver que $\varepsilon$ es epimorfismo, simplemente debemos seguir las flechas: Sea $x\in M$. Al ser $\varepsilon''$ epimorfismo, existe $p_0''\in P_0''$ tal que $\varepsilon''(p_0'') = g(x)$. Entonces, para cualquier $m\in P_0'$, tenemos que $g\circ\varepsilon(m,p_0'') = \alpha''\circ g_0(m,p_0'') = \varepsilon''(p_0'') = g(x)$, y así $\varepsilon(m,p_0'')-x \in \ker g = \im f$. Entonces, existe $z\in M'$ tal que $f(z) = \varepsilon(m,p_0'') - x$. Pero como $\varepsilon'$ es epimorfismo, existe un $p_0'\in P_0'$ tal que $z = \varepsilon'(p_0')$. Luego, es fácil comprobar que $\varepsilon(p_0',p_0'') = x$, y así es epimorfismo. Y el lema sigue.

\end{proof}
\end{lem}

\begin{prop}[Sucesión Exacta Larga del Tor]
Sea $0\longrightarrow M'\stackrel{f}{\longrightarrow} M\stackrel{g}{\longrightarrow} M''\longrightarrow 0$ una sucesión exacta corta de $A$-módulos. Entonces, para cualquier $A$-módulo $N$, se induce una sucesión exacta larga de $\mathrm{Tor}$: 
\begin{center}\begin{tikzcd}[row sep=2em,column sep=1.8em,minimum width=1.6em]
\ldots\arrow[dashed]{r}[font=\normalsize]{\partial}& \mathrm{Tor}_n(M',N)\arrow{r}[font=\normalsize]{f} & \mathrm{Tor}_n(M,N)\arrow{r}[font=\normalsize]{g} & \mathrm{Tor}_n(M'',N)\ar[out=0, in=180]{dll}[font=\normalsize]{\partial} \\
& \mathrm{Tor}_{n-1}(M',N)\arrow{r}[font=\normalsize]{f} & \mathrm{Tor}_{n-1}(M,N)\arrow{r}[font=\normalsize]{g} & \mathrm{Tor}_{n-1}(M'',N)\ar[out=0, in=180, dashed]{ddll} \\
& \vdots & \vdots & \vdots \\
& \mathrm{Tor}_{1}(M',N)\arrow{r}[font=\normalsize]{f} & \mathrm{Tor}_{1}(M,N)\arrow{r}[font=\normalsize]{g} & \mathrm{Tor}_{1}(M'',N)\ar[out=0, in=180]{dll}[font=\normalsize]{\partial} \\
& M'\otimes_A N\arrow{r}[font=\normalsize]{f\otimes 1} & M\otimes_A N\arrow{r}[font=\normalsize]{g\otimes 1} & M''\otimes_A N\arrow{r} & 0
\end{tikzcd}\end{center}
\begin{proof}
Simplemente es tomar una resolución proyectiva de $M'$ y de $M''$, usar el Lema de la Herradura para obtener una sucesión exacta corta de complejos y ahí usar la sucesión exacta larga de homología.
\end{proof}
\end{prop}

\begin{prop}
Sea $M$ un $A$-módulo a derecha. Son equivalentes: \begin{enumerate}\item $M$ es playo. \item $\mathrm{Tor}_n^A(M,N)=0$ para todo $n\in\mathbb{N}$ y $A$-módulo $N$ a izquierda. \item $\mathrm{Tor}_1^A(M,N)=0$ para todo $A$-módulo $N$ a izquierda.  \end{enumerate}
\end{prop}

\begin{prop}
Sea $0\longrightarrow M'\longrightarrow M\longrightarrow M''\longrightarrow 0$ una sucesión exacta corta de $A$-módulos tal que $M''$ es playo. Entonces, $M$ es playo si y sólo si $M'$ es playo.
\end{prop}

\begin{prop}
Sea $M$ un $A$-módulo a derecha. Entonces $M$ es playo si y sólo si $\mathrm{Tor}_1^A(A/I,M)=0$ para todo ideal a derecha $I$ de $A$.
\end{prop}

\begin{prop}
Sea $0\longrightarrow M'\longrightarrow M\longrightarrow M''\longrightarrow 0$ una sucesión exacta corta de $A$-módulos a derecha. Si $M$ es playo, entonces $\mathrm{Tor}_n^A(M',N)\simeq \mathrm{Tor}_{n+1}^A(M'',N)$ para todo $A$-módulo a izquierda $N$ y $n\geq 1$.
\end{prop}

\end{document}


